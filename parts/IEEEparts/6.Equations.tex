\section{Mathematical Typography \\ and Why It Matters}

Typographical conventions for mathematical formulas have been developed to {\bf provide uniformity and clarity of presentation across mathematical texts}. This enables the readers of those texts to both understand the author's ideas and to grasp new concepts quickly. While software such as \LaTeX \ and MathType\textsuperscript{\textregistered} can produce aesthetically pleasing math when used properly, it is also very easy to misuse the software, potentially resulting in incorrect math display.

IEEE aims to provide authors with the proper guidance on mathematical typesetting style and assist them in writing the best possible article. As such, IEEE has assembled a set of examples of good and bad mathematical typesetting \cite{ref1,ref2,ref3,ref4,ref5}. 

Further examples can be found at \url{http://journals.ieeeauthorcenter.ieee.org/wp-content/uploads/sites/7/IEEE-Math-Typesetting-Guide-for-LaTeX-Users.pdf}

\subsection{Display Equations}
The simple display equation example shown below uses the ``equation'' environment. To number the equations, use the $\backslash${\tt{label}} macro to create an identifier for the equation. LaTeX will automatically number the equation for you.
\begin{equation}
\label{deqn_ex1}
x = \sum_{i=0}^{n} 2{i} Q.
\end{equation}

\noindent is coded as follows:
\begin{verbatim}
\begin{equation}
\label{deqn_ex1}
x = \sum_{i=0}^{n} 2{i} Q.
\end{equation}
\end{verbatim}

To reference this equation in the text use the $\backslash${\tt{ref}} macro. 
Please see (\ref{deqn_ex1})\\
\noindent is coded as follows:
\begin{verbatim}
Please see (\ref{deqn_ex1})\end{verbatim}

\subsection{Equation Numbering}
{\bf{Consecutive Numbering:}} Equations within an article are numbered consecutively from the beginning of the
article to the end, i.e., (1), (2), (3), (4), (5), etc. Do not use roman numerals or section numbers for equation numbering.

\noindent {\bf{Appendix Equations:}} The continuation of consecutively numbered equations is best in the Appendix, but numbering
 as (A1), (A2), etc., is permissible.\\

\noindent {\bf{Hyphens and Periods}}: Hyphens and periods should not be used in equation numbers, i.e., use (1a) rather than
(1-a) and (2a) rather than (2.a) for subequations. This should be consistent throughout the article.

\subsection{Multi-Line Equations and Alignment}
Here we show several examples of multi-line equations and proper alignments.

\noindent {\bf{A single equation that must break over multiple lines due to length with no specific alignment.}}
\begin{multline}
\text{The first line of this example}\\
\text{The second line of this example}\\
\text{The third line of this example}
\end{multline}

\noindent is coded as:
\begin{verbatim}
\begin{multline}
\text{The first line of this example}\\
\text{The second line of this example}\\
\text{The third line of this example}
\end{multline}
\end{verbatim}

\noindent {\bf{A single equation with multiple lines aligned at the = signs}}
\begin{align}
a &= c+d \\
b &= e+f
\end{align}
\noindent is coded as:
\begin{verbatim}
\begin{align}
a &= c+d \\
b &= e+f
\end{align}
\end{verbatim}

The {\tt{align}} environment can align on multiple  points as shown in the following example:
\begin{align}
x &= y & X & =Y & a &=bc\\
x' &= y' & X' &=Y' &a' &=bz
\end{align}
\noindent is coded as:
\begin{verbatim}
\begin{align}
x &= y & X & =Y & a &=bc\\
x' &= y' & X' &=Y' &a' &=bz
\end{align}
\end{verbatim}





\subsection{Subnumbering}
The amsmath package provides a {\tt{subequations}} environment to facilitate subnumbering. An example:

\begin{subequations}\label{eq:2}
\begin{align}
f&=g \label{eq:2A}\\
f' &=g' \label{eq:2B}\\
\mathcal{L}f &= \mathcal{L}g \label{eq:2c}
\end{align}
\end{subequations}

\noindent is coded as:
\begin{verbatim}
\begin{subequations}\label{eq:2}
\begin{align}
f&=g \label{eq:2A}\\
f' &=g' \label{eq:2B}\\
\mathcal{L}f &= \mathcal{L}g \label{eq:2c}
\end{align}
\end{subequations}

\end{verbatim}

\subsection{Matrices}
There are several useful matrix environments that can save you some keystrokes. See the example coding below and the output.

\noindent {\bf{A simple matrix:}}
\begin{equation}
\begin{matrix}  0 &  1 \\ 
1 &  0 \end{matrix}
\end{equation}
is coded as:
\begin{verbatim}
\begin{equation}
\begin{matrix}  0 &  1 \\ 
1 &  0 \end{matrix}
\end{equation}
\end{verbatim}

\noindent {\bf{A matrix with parenthesis}}
\begin{equation}
\begin{pmatrix} 0 & -i \\
 i &  0 \end{pmatrix}
\end{equation}
is coded as:
\begin{verbatim}
\begin{equation}
\begin{pmatrix} 0 & -i \\
 i &  0 \end{pmatrix}
\end{equation}
\end{verbatim}

\noindent {\bf{A matrix with square brackets}}
\begin{equation}
\begin{bmatrix} 0 & -1 \\ 
1 &  0 \end{bmatrix}
\end{equation}
is coded as:
\begin{verbatim}
\begin{equation}
\begin{bmatrix} 0 & -1 \\ 
1 &  0 \end{bmatrix}
\end{equation}
\end{verbatim}

\noindent {\bf{A matrix with curly braces}}
\begin{equation}
\begin{Bmatrix} 1 &  0 \\ 
0 & -1 \end{Bmatrix}
\end{equation}
is coded as:
\begin{verbatim}
\begin{equation}
\begin{Bmatrix} 1 &  0 \\ 
0 & -1 \end{Bmatrix}
\end{equation}\end{verbatim}

\noindent {\bf{A matrix with single verticals}}
\begin{equation}
\begin{vmatrix} a &  b \\ 
c &  d \end{vmatrix}
\end{equation}
is coded as:
\begin{verbatim}
\begin{equation}
\begin{vmatrix} a &  b \\ 
c &  d \end{vmatrix}
\end{equation}\end{verbatim}

\noindent {\bf{A matrix with double verticals}}
\begin{equation}
\begin{Vmatrix} i &  0 \\ 
0 & -i \end{Vmatrix}
\end{equation}
is coded as:
\begin{verbatim}
\begin{equation}
\begin{Vmatrix} i &  0 \\ 
0 & -i \end{Vmatrix}
\end{equation}\end{verbatim}

\subsection{Arrays}
The {\tt{array}} environment allows you some options for matrix-like equations. You will have to manually key the fences, but there are other options for alignment of the columns and for setting horizontal and vertical rules. The argument to {\tt{array}} controls alignment and placement of vertical rules.

A simple array
\begin{equation}
\left(
\begin{array}{cccc}
a+b+c & uv & x-y & 27\\
a+b & u+v & z & 134
\end{array}\right)
\end{equation}
is coded as:
\begin{verbatim}
\begin{equation}
\left(
\begin{array}{cccc}
a+b+c & uv & x-y & 27\\
a+b & u+v & z & 134
\end{array} \right)
\end{equation}
\end{verbatim}

A slight variation on this to better align the numbers in the last column
\begin{equation}
\left(
\begin{array}{cccr}
a+b+c & uv & x-y & 27\\
a+b & u+v & z & 134
\end{array}\right)
\end{equation}
is coded as:
\begin{verbatim}
\begin{equation}
\left(
\begin{array}{cccr}
a+b+c & uv & x-y & 27\\
a+b & u+v & z & 134
\end{array} \right)
\end{equation}
\end{verbatim}

An array with vertical and horizontal rules
\begin{equation}
\left( \begin{array}{c|c|c|r}
a+b+c & uv & x-y & 27\\ \hline
a+b & u+v & z & 134
\end{array}\right)
\end{equation}
is coded as:
\begin{verbatim}
\begin{equation}
\left(
\begin{array}{c|c|c|r}
a+b+c & uv & x-y & 27\\
a+b & u+v & z & 134
\end{array} \right)
\end{equation}
\end{verbatim}
Note the argument now has the pipe "$\vert$" included to indicate the placement of the vertical rules.


\subsection{Cases Structures}
Many times cases can be miscoded using the wrong environment, i.e., {\tt{array}}. Using the {\tt{cases}} environment will save keystrokes (from not having to type the $\backslash${\tt{left}}$\backslash${\tt{lbrace}}) and automatically provide the correct column alignment.
\begin{equation*}
{z_m(t)} = \begin{cases}
1,&{\text{if}}\ {\beta }_m(t) \\ 
{0,}&{\text{otherwise.}} 
\end{cases}
\end{equation*}
\noindent is coded as follows:
\begin{verbatim}
\begin{equation*}
{z_m(t)} = 
\begin{cases}
1,&{\text{if}}\ {\beta }_m(t),\\ 
{0,}&{\text{otherwise.}} 
\end{cases}
\end{equation*}
\end{verbatim}
\noindent Note that the ``\&'' is used to mark the tabular alignment. This is important to get  proper column alignment. Do not use $\backslash${\tt{quad}} or other fixed spaces to try and align the columns. Also, note the use of the $\backslash${\tt{text}} macro for text elements such as ``if'' and ``otherwise.''

\subsection{Function Formatting in Equations}
Often, there is an easy way to properly format most common functions. Use of the $\backslash$ in front of the function name will in most cases, provide the correct formatting. When this does not work, the following example provides a solution using the $\backslash${\tt{text}} macro:

\begin{equation*} 
  d_{R}^{KM} = \underset {d_{l}^{KM}} {\text{arg min}} \{ d_{1}^{KM},\ldots,d_{6}^{KM}\}.
\end{equation*}

\noindent is coded as follows:
\begin{verbatim}
\begin{equation*} 
 d_{R}^{KM} = \underset {d_{l}^{KM}} 
 {\text{arg min}} \{ d_{1}^{KM},
 \ldots,d_{6}^{KM}\}.
\end{equation*}
\end{verbatim}

\subsection{ Text Acronyms Inside Equations}
This example shows where the acronym ``MSE" is coded using $\backslash${\tt{text\{\}}} to match how it appears in the text.

\begin{equation*}
 \text{MSE} = \frac {1}{n}\sum _{i=1}^{n}(Y_{i} - \hat {Y_{i}})^{2}
\end{equation*}

\begin{verbatim}
\begin{equation*}
 \text{MSE} = \frac {1}{n}\sum _{i=1}^{n}
(Y_{i} - \hat {Y_{i}})^{2}
\end{equation*}
\end{verbatim}