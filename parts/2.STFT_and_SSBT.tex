\section{Frequency and Time-Frequency Transforms}
\label{sec:stft_and_ssbt}

When studying signals and systems, frequency and time-frequency transforms are often used to change the signal domain \cite{demuth_frequency_1977} from its naturally obtained time domain, allowing the exploitation of different patterns and information inherent to the signal. From now on, we assume that all time-domain signals are real-valued, which allows us access to some shortcuts for some transforms and enables the use of others.

For continuous time and frequency domains, the Fourier Transform (FT) is defined as
\begin{equations}{eq:sec2:def_fourier_transform}
	X_{\F}(f)
	& \equiv \FT{x(t)} \\
	& = \int\limits_{-\infty}^{\infty} x(t) e^{-\j 2\pi f t}\dd t
\end{equations}
where, for a real-valued $x(t)$, it is known that $X_{\F}(f) = X_{\F}^*(\-f)$, with $(\cdot)^*$ is the complex-conjugate operation.

We define the Real Fourier Transform (RFT) in a similar way to the FT, being cleverly constructed such that its frequency spectrum is real-valued without loss of information. The RFT is given by
\begin{equations}{eq:sec2:def_real_fourier_transform}
	X_{\R}(f)
	& \equiv \RFT{x(t)} \\
	& = \Sqrt{2} \real{\int\limits_{-\infty}^{\infty} x(t) e^{-\j 2\pi f t + \j\frac{3\pi}{4}} \dd t} \\
	& = \int\limits_{-\infty}^{\infty} x(t) \bts{-\cos\pts{2\pi f t} + \sin\pts{2\pi f t}} \dd t
\end{equations}
and the Inverse Real Fourier Transform (IRFT) as (see \cref{prop:RFT_IRFT_inverses} in \cref{app:properties_rft})
\begin{equations}
	x(t)
	& \equiv \IRFT{X_{\R}(f)} \\
	& = \Sqrt{2} \real{\int\limits_{-\infty}^{\infty} X_{\R}(f) e^{\j 2\pi f t - \j\frac{3\pi}{4}} \dd f}
\end{equations}

We can also define the RFT in terms of the FT through a simple substitution of \cref{eq:sec2:def_fourier_transform} in \cref{eq:sec2:def_real_fourier_transform} (see \cref{prop:FT_RFT_equivalence} in \cref{app:properties_rft}), such that
\begin{equations}{eq:sec2:equivalence_ft_rft}
	X_{\R}(f)
	& = -\Re{X}_{\F}(f) - \Im{X}_{\F}(f)
\end{equations}

One can write the RFT in terms of the FT and the FT in terms of the RFT, these being given by (\cref{prop:FT_RFT_equivalence}, \cref{app:properties_rft})
\begin{subgather}{eqs:sec2:equivalence_pair_ft_and_rft}
    X_{\R}(f) = \frac{1}{\Sqrt{2}} \pts{e^{\j\frac{3\pi}{4}} X_{\F}(f) + e^{-\j\frac{3\pi}{4}} X_{\F}(\-f) }\label{eq:sec2:equivalence_pair_ft_to_rft} \\
    X_{\F}(f) = \frac{1}{\Sqrt{2}} \pts{e^{-\j\frac{3\pi}{4}} X_{\R}(f) + e^{\j\frac{3\pi}{4}} X_{\R}(\-f) } \label{eq:sec2:equivalence_pair_rft_to_ft}
\end{subgather}
Trivially this forms a bijective relationship between the two transforms, where one of them can be uniquely obtained from the other, and vice-versa. Trivially, this is only true if the original time-domain signal is real-valued, which is a requirement for the RFT to be invertible.

\subsection{Convolution}
\label{subsec:sec2:convolution}
Given an LIT system with an impulse response $h(t)$, the convolution theorem for the Fourier transform states that
\begin{equation}
	h(t) \ast x(t) \hrel{\F} H_{\F}(f) X_{\F}(f)
\end{equation}
where $\hrel{\F}$ indicates a Fourier transform pair. This theorem isn't strictly valid for the RFT (see \cref{prop:conv_theorem_RFT} in \cref{app:properties_rft}). However, it is possible to prove that there is an equivalent of the convolution theorem for the RFT (see \cref{prop:equivalent_conv_theorem_RFT} in \cref{app:properties_rft}), with
\begin{equation}
    \label{eq:sec2:convolution_equiv_rft}
	h(t) \ast x(t) \hrel{\R} X_{\R}(f) \Re{H}_{\F}(f) + X_{\R}(\-f) \Im{H}_{\F}(f)
\end{equation}
where, $\hrel{\R}$ indicates an RFT pair. We see that, for a given frequency $f$, the convolution's output on the RFT domain depends on both it and its dual frequency $\-f$. This makes intuitive sense since, given that the RFT is a real-valued transform, it is impossible to correctly encompass the phase information of each frequency only through it and therefore, if one wants to encompass the phase information after the convolution correctly, both frequencies are necessary.

\subsection{Relative frequency responses}
\label{subsec:sec2:relative_transfer_functions}

Given two systems that share an input $x(t)$, each with an impulse response $h_1(t)$ and $h_2(t)$, on the FT domain $H_1(f)$ and $H_2(f)$, we can calculate their relative frequency responses (RFRs), respective to the output of one of the systems (assumed to be the first without compromise). We denote these RFRs $A_1(f)$ and $A_2(f)$, respective to each system.

Let us denote $Y_1(f)$ as the first system's output, given by
\begin{equation}
	\label{eq:output_sys1_ft}
	Y_1(f) = H_1(f) X(f)
\end{equation}
and similarly for $Y_2(f)$. We write $X_1(f) = H_1(f) X(f)$, and thus $Y_1(f) = A_1(f) X_1(f)$ with $A_1(f) = 1$. We can obtain $A_2(f)$ as
\begin{equation}
    \label{eq:sec2:calc_RFR_ft_ideal}
	A_2(f) = \frac{H_2(f)}{H_1(f)}
\end{equation}
which trivially satisfies that $A_2(f) X_1(f) = H_2(f) X(f)$. These RFRs can be calculated as
\begin{equation}
	\label{eq:sec2:calc_RFR_ft_expec}
	A_m(f) = \frac{\expec{X_m(f) X_1^*(f)}}{\expec{X_1(f) X_1^*(f)}}
\end{equation}
where $\expec{\cdot}$ is the expectation operator. It is trivial to see that \cref{eq:sec2:calc_RFR_ft_ideal} and \cref{eq:sec2:calc_RFR_ft_expec} are equivalent, at least in an ideal scenario. %Continue from here

Since after the convolution in the RFT each frequency depends on its conjugate as well, the RFR isn't as straightforward and needs a more careful examination. In this case, we will have two frequency responses $H_m'(f)$ and $H_m''(f)$ for each system, as well as two inputs $X'(f) = X(f)$ and $X''(f) = X(\-f)$. Then, our outputs can be described as
\begin{subgather}{eqs:sec2:system_eqs_X1f_X2f_H}
	X_1(f) = H_1'(f) X'(f) + H_1''(f) X''(f) \label{eqs:sec2:system_eqs_X1f_X2f_H:subeq1}\\
	X_2(f) = H_2'(f) X'(f) + H_2''(f) X''(f) \label{eqs:sec2:system_eqs_X1f_X2f_H:subeq2}
\end{subgather}

From \cref{eq:sec2:convolution_equiv_rft}, we easily see that
\begin{subgather}{eqs:sec2:def_Hm'_Hm''_rft}
	H_m'(f) = H_m'(\-f) = \Re{H}_{\sF;m}(f) \\
	H_m''(f) = - H_m''(\-f) = \Im{H}_{\sF;m}(f)
\end{subgather}

We now let $X_1'(f) = X_1(f)$ and $X_1''(f) = X_1(\-f)$, as our new system inputs. We write
\begin{subgather}{eqs:sec2:system_eqs_X1f_X2f_G}
	X_1(f) = A_1'(f) X_1'(f) + A_1''(f) X_1''(f) \label{eq:sec2:system_eqs_X1f_X2f_G:subeq1}\\
	X_2(f) = A_2'(f) X_1'(f) + A_2''(f) X_1''(f)
\end{subgather}
where $A_m'(f)$ and $A_m''(f)$ are the relative frequency responses between the new inputs $X_1'(f)$ and $X_1''(f)$ and the output $X_m(f)$. In a similar way to the FT, here we have that $A_1'(f) = 1$ and $A_1''(f) = 0$. Using \cref{eqs:sec2:def_Hm'_Hm''_rft,eqs:sec2:system_eqs_X1f_X2f_H:subeq1} on \cref{eq:sec2:system_eqs_X1f_X2f_G:subeq1} we get
\begin{equation}
	X_2(f) = \bts{A_2'(f) H_1'(f) - A_2''(f) H_1''(f)} X'(f) + \bts{A_2'(f) H_1''(f) + A_2''(f) H_1'(f)} X''(f)
\end{equation}
and, by comparing with $X_2(f)$ from \cref{eqs:sec2:system_eqs_X1f_X2f_H:subeq2}, we have a system of equations (omitting the frequency index for simplicity, as now all signals are on positive frequencies)
\begin{subgather}
	A_2' H_1' - A_2'' H_1'' = H_2' \\
	A_2' H_1'' + A_2'' H_1' = H_2''
\end{subgather}
whose solution is
\begin{subgather}
	A_2' = \frac{H_1' H_2' + H_1'' H_2''}{{H_1'}^2 + {H_1''}^2} \\
	A_2'' = \frac{H_1' H_2'' - H_1'' H_2'}{{H_1'}^2 + {H_1''}^2}
\end{subgather}

Through the properties of the RFT exposed in \cref{app:properties_rft}, we have that
\begin{subalign}{eqs:sec2:system_equations}
	\expec{X_{1}(f) X_{1}(f)}   & = \pts{ {H_1'}^2(f)      + {H_1''}^2(f)     } \sigma_{X}^2(f) \label{eq:sec2:system_equations:subeq1} \\
	\expec{X_{1}(f) X_{2}(f)}   & = \pts{ H_1'(f) H_2'(f)  +  H_1'(f) H_2'(f) } \sigma_{X}^2(f) \label{eq:sec2:system_equations:subeq2} \\
	\expec{X_{1}(\-f) X_{2}(f)} & = \pts{ H_1'(f) H_2''(f) -  H_1''(f) H_2'(f)} \sigma_{X}^2(f) \label{eq:sec2:system_equations:subeq3}
\end{subalign}
where $\sigma_X^2(f) = \expec{X_{\F}(f)^2} = \expec{X_{\R}(f)^2}$. With this, it is trivial to see that
\begin{subgather}{eqs:sec2:calc_RFR_rft_expec}
	A_2'(f) = \frac{\expec{X_1(f) X_2(f)}}{\expec{X_1^2(f)}} \label{eq:sec2:calc_RFR_rft_expec:subeq1} \\
	A_2''(f) = \frac{\expec{X_1(\-f) X_2(f)}}{\expec{X_1^2(f)}} \label{eq:sec2:calc_RFR_rft_expec:subeq2}
\end{subgather}
These formulations are the same as those in \cref{eq:sec2:calc_RFR_ft_expec}, but without the complex-conjugate operation (given that now all signals are real-valued), and for \cref{eq:sec2:calc_RFR_rft_expec:subeq2} we calculate the correlation with the cross-frequency portion $X_{1}(\-f)$. \cref{prop:RFT_same-variance_FT} was also used, such that $\expec{X_1(f)^2} = \expec{X_1(\-f)^2} = 2\sigma_X^2(f)$. Again, the results in \cref{eqs:sec2:calc_RFR_rft_expec} have an intuitive sense to them, where $A_2'(f)$ is the RFR between $X_2(f)$ and $X_1'(f)$ (same frequency), and $A_2''(f)$ is the RFR between $X_2(f)$ and $X_1'(\-f) = X_1''(f)$ (conjugate frequency), which is in line with \cref{eqs:sec2:system_eqs_X1f_X2f_G}.

Generalizing to a situation with $M$ sensors, one can follow the same steps and achieve that, for each $m$-th sensor, we have
\begin{subgather}{eqs:sec2:calc_RFR_rft_expec_generic}
	A_m'(f) = \frac{\expec{X_1(f) X_m(f)}}{\expec{X_1^2(f)}} \label{eqs:sec2:calc_RFR_rft_expec_generic:subeq1} \\
	A_m''(f) = \frac{\expec{X_1(\-f) X_m(f)}}{\expec{X_1^2(f)}} \label{eqs:sec2:calc_RFR_rft_expec_generic:subeq2} 
\end{subgather}
this also working for $m = 1$. Trivially, \cref{eqs:sec2:calc_RFR_rft_expec_generic:subeq1} reduces to $A_1'(f) = 1$, and using that $X_1(f)$ and $X_1(\-f)$ are uncorrelated (see \cref{prop:conjugate-freqs_independent_RFT}), then $A_1''(f) = 0$, as expected.

Observing strictly the mathematical structures of \cref{eq:output_sys1_ft} and \cref{eqs:sec2:system_eqs_X1f_X2f_H}, the FT formulation from \cref{eq:output_sys1_ft} can be considered a particular case of the RFT, where $A_{m}'(f) = A_{m;\F}(f)$, and $A_{m}''(f) = 0$. Knowing this, we will use the SSBT formulation from \cref{eq:sec2:system_ctf_ssbt_output} for both the SSBT and the STFT, as it is a less restricting model. The necessary considerations will be taken when particularizing the equations for the STFT.

\subsection{Discrete time-frequency transforms}

Given a discrete-time-domain signal $x[n]$, its Short-time Fourier Transform (STFT) \cite{kiymik_comparison_2005,pan_microphone_2021} is
\begin{equation}
	\label{eq:sec2:def_stft_xn}
	X_{\sF}[l,k] = \sum_{n=0}^{K-1} w[n] x[n + l\cdot O] e^{-\j 2\pi k \frac{(n + l\cdot O)}{K}}
\end{equation}
where $w[n]$ is an analysis window of length $K$; and $O$ is the overlap between windows of the transform, usually $O = \floor{\nicefrac{K}{2}}$. The STFT can be seen as a discretization of the FT while also being applied over different time ``snippets''.

The Single-Sideband Transform (SSBT) \cite{crochiere_multirate_1983} is similarly defined, being the RFT's windowed discrete-time adaptation. The SSB transform of $x[n]$ is defined as
\begin{equation}
	\label{eq:sec2:def_ssbt_xn}
	X_{\sS}[l,k] = \Sqrt{2} \real{\sum_{n=0}^{K-1} w[n] x[n + l\cdot O] e^{-\j 2\pi k \frac{(n + l\cdot O)}{K} + \j\frac{3\pi}{4} } }
\end{equation}

One advantage of using the STFT is that we only need to work with $\floor{\nicefrac{(K+1)}{2}}+1$ frequency bins, given its complex-conjugate behavior for real time-domain signals. Meanwhile, the SSBT requires all $K$ bins to correctly capture all information of $x[n]$; however, it is real-valued.

Assuming that all $K$ bins of the STFT are available (even though not all of them are strictly necessary), similarly to \cref{eq:sec2:equivalence_ft_rft,eq:sec2:equivalence_pair_ft_to_rft,eq:sec2:equivalence_pair_rft_to_ft} we now have\footnote{For the abuse of notation, we let $X_{\sS}[l,K] \equiv X_{\sS}[l,0]$, and equally for $X_{\sF}[l,K]$.}
\begin{equations}
	\label{eq:sec2:equivalence_stft_ssbt}
	X_{\sS}[l,k]
	& = \Sqrt{2} \real{X_{\sF}[l,k] e^{\j\frac{3\pi}{4}}} \\
	& = - \real{X_{\sF}[l,k]} - \imag{X_{\sF}[l,k]}
\end{equations}
\begin{equation}
	X_{\sS}[l,k] = \frac{1}{\Sqrt{2}} \pts{ e^{\j\frac{3\pi}{4}} X_{\sF}[l,k] + e^{-\j\frac{3\pi}{4}} X_{\sF}[l,K-k] }
\end{equation}
\begin{equation}
	\label{eq:sec2:equivalence_ssbt_stft}
	X_{\sF}[l,k] = \frac{1}{\Sqrt{2}} \pts{ e^{-\j\frac{3\pi}{4}} X_{\sS}[l,k] + e^{\j\frac{3\pi}{4}} X_{\sS}[l,K-k] }
\end{equation}

As was the case for the RFT, the SSBT also doesn't hold the convolution theorem the same way as the STFT. However, similarly to what was shown in \cref{eq:sec2:convolution_equiv_rft} (and in \cref{prop:equivalent_conv_theorem_RFT}), we can write the convolution on the SSBT domain through the MTF model \cite{talmon_relative_2009} as
\begin{equation}
	\label{eq:sec2:convolution_on_ssbt_mtf}
	h[n] \ast x[n] \hrel{\sS} X_{\sS}[l,k] H_{\sS}'[k] + X_{\sS}[l,K-k] H_{\sS}''[k]
\end{equation}
or, with the CTF model,
\begin{equation}
	\label{eq:sec2:convolution_on_ssbt_ctf}
	h[n] \ast x[n] \hrel{\sS} X_{\sS}[l,k] \ast H_{\sS}'[l,k] + X_{\sS}[l,K-k] \ast H_{\sS}''[l,k]
\end{equation}
in which this convolution is done over the frames $l$, with $H_{\sS}'[l,k] = \Re{H}_{\sF}[l,k]$ and $H_{\sS}''[l,k] = \Im{H}_{\sF}[l,k]$. Note that this is an approximation, as cross-band interference \cite{avargel_system_2007} is necessary to model the convolution perfectly; however, this effect was not considered here for either transform. Also, note that the presence of the conjugate frequency is a byproduct distinct to the cross-band interference, the latter happening due to aliasing and the windowing processes, and the former coming from the continuous-time convolution theorem for the SSBT (see \cref{subsec:sec2:convolution,prop:conv_theorem_RFT}).

\subsection{RFR estimation for time-frequency transforms}
\label{subsec:sec2:rfr_estimation_time-freq_transforms}
Under the same assumptions as in \cref{subsec:sec2:relative_transfer_functions}, we can write the output of our systems with the CTF model in the SSBT domain as
\begin{equation}
	\label{eq:sec2:system_ctf_ssbt_output}
	X_{m;\sS}[l,k] = X_{1;\sS}[l,k] \ast A_{m;\sS}'[l,k] + X_{1;\sS}[l,K-k] \ast A_{m;\sS}''[l,k]
\end{equation}
where $A_{m;\sS}'[l,k]$ and $A_{m;\sS}''[l,k]$ are calculated in a similar way to $A_m'(f)$ and $A_m''(f)$ from \cref{subsec:sec2:relative_transfer_functions}. While it was shown in \cref{subsec:sec2:relative_transfer_functions} that, for the FT and RFT, the relative transfer functions could be estimated from the frequency signals, this also needs to be shown for the time-frequency transforms. In particular, under the STFT, given $X_{m;\sF}[l,k]$ and $X_{1;\sF}[l,k]$, we have that
\begin{equations}
	\expec{X_{m;\sF}[l+\lambda,k] X_{1;\sF}^*[l,k]}
	& = \expec{\sum_{\tau}A_{m;\sF}[\tau,k] X_{1;\sF}[l+\lambda-\tau,k] X_{1;\sF}^*[l,k]} \\
	& = \sum_{\tau} A_{m;\sF}[\tau,k] \expec{X_{1;\sF}[l+\lambda-\tau,k] X_{1;\sF}^*[l,k]} \\
	& = \sum_{\tau} A_{m;\sF}[\tau,k] \sigma_{X,\lambda-\tau}^2[k]
\end{equations}
which is only equal to $A_{m;\sF}[\lambda] \sigma_{X}^2$ (as in \cref{eq:sec2:system_equations:subeq2}) if $\sigma_{X,\gamma-\tau} = 0$ for $\lambda \neq \tau$; that is, different samples of our desired signal are uncorrelated. This is untrue for time-frequency transforms given their overlap and windowing processes, but also since our desired signal is the signal observed on the reference sensor $X_{1;\sF}[l,k] = H_{1;\sF}[l,k] \ast X_{\sF}[l,k]$, with $H_{1;\sF}[l,k]$ being different to a Dirac impulse for an echoic medium.

Applying this same process for the SSBT, we have that
\begin{equation}
    \label{eq:sec2:expec_Xms-llk_X1s-lk}
	\expec{X_{m;\sS}[l+\lambda,k] X_{1;\sS}[l,k]} = \sum_{\tau} A_{m;\sS}'[\tau,k] \sigma_{X',\lambda-\tau}^2[k] + \sum_{\gamma} A_{m;\sS}''[\gamma,k] \sigma_{X'X'',\lambda-\gamma}^2[k]
\end{equation}
where we not only have additional terms regarding different windows on the same-frequency portion (similar to the STFT), but also on the cross-frequency part of the RFRs. Through a similar development as done before, but now on $\expec{X_{1;\sS}[l+\lambda-\gamma,k] X_{1;\sS}[l,k]}$, we get that
\begin{equations}{eq:sigmaX'X''_ssbt}
	\sigma_{X'X'',\lambda-\gamma}^2[k]
	& = \expec{X'_{1;\sS}[l+\gamma - \lambda,k] X''_{1;\sS}[l,k]} \\
	& = \sigma_{X}^2[k] \sum_{\mu \geq 0} (H'_{1,\mu} - H'_{1,-\mu})(H''_{1,\mu + (\lambda - \gamma)} - H''_{1,\mu - (\lambda - \gamma)}) 
\end{equations}
with $\sigma_{X}^2[k]$ being the desired signal's variance at source. We here have 3 variables: the frame delay between different sensors $\lambda$, the RFR index $\gamma$, and the delay between same- and cross-frequency $\mu$. If $\mu = 0$, we are comparing same-frequency and cross-frequency portions on the same window, which is trivially $0$ since in this case $H_{1,\mu}' = H_{1,-\mu}' = H_{1,0}'$. If $\gamma = \lambda$ the delay in the observed signal $\lambda$ and the RFR delay $\gamma$ cancel out, and we again have that in this case the summation's term is $0$ since $H_{1,\mu+0}'' = H_{1,\mu-0}''$. Both these scenarios are equivalent to comparing same- and cross-frequency portions of the signals on the same window, which are known to be zero. In all other cases, the summation terms of \cref{eq:sigmaX'X''_ssbt} aren't trivially null, and therefore the summation over $\gamma$ in \cref{eq:sec2:expec_Xms-llk_X1s-lk} also isn't identically zero.

With this, we reach the conclusion that even though we have independence for conjugate frequencies on the same frame with the SSBT, this isn't true for different frames. This notably is not a relevant effect with the STFT up to the degree of approximation we are bearing for both. Therefore, compared to the STFT, this estimation through the SSBT adds error terms for the conjugate frequencies' correlation. This decreases the beamformer's robustness through this transform, making it more prone to output distortion due to a mismatch in estimating the relative frequency response.