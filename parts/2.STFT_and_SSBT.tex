\section{Frequency and Time-Frequency Transforms}
\label{sec:stft_and_ssbt}

When studying signals and systems, often frequency and time-frequency transforms are used in order to change the signal domain \cite{demuth_frequency_1977}, allowing the exploitation of different patterns and informations inherent to the signal. We from now on assume that all time-domain signals are real-valued.

For continuous time and frequency domains, the Fourier Transform (FT) is defined as
\begin{equations}{eq:sec2:def_fourier_transform}
	X_{\F}(f)
	& \equiv \FT{x(t)} \\
	& = \int\limits_{-\infty}^{\infty} x(t) e^{-\j 2\pi f t}\dd t
\end{equations}

We define the Real Fourier Transform (RFT) similarly, being cleverly constructed such that its frequency spectrum is real-valued without loss of information, as
\begin{equations}{eq:sec2:def_real_fourier_transform}
	X_{\R}(f)
	& \equiv \RFT{x(t)} \\
	& = \Sqrt{2} \real{\int\limits_{-\infty}^{\infty} x(t) e^{-\j 2\pi f t + \j\frac{3\pi}{4}} \dd t} \\
	& = \int\limits_{-\infty}^{\infty} x(t) \bts{-\cos\pts{2\pi f t} + \sin\pts{2\pi f t}} \dd t
\end{equations}
and the Inverse Real Fourier Transform (IRFT) as (see \cref{prop:RFT_IRFT_inverses} in \cref{app:properties_rft})
\begin{equations}
	x(t)
	& \equiv \IRFT{X_{\R}(f)} \\
	& = \Sqrt{2} \real{\int\limits_{-\infty}^{\infty} X_{\R}(f) e^{\j 2\pi f t - \j\frac{3\pi}{4}} \dd f}
\end{equations}

We can also define the RFT in terms of the FT through a simple substitution of \cref{eq:sec2:def_fourier_transform} in \cref{eq:sec2:def_real_fourier_transform} (see \cref{prop:FT_RFT_equivalence} in \cref{app:properties_rft}), such that
\begin{equations}{eq:sec2:equivalence_ft_rft}
	X_{\R}(f)
	& = \Sqrt{2} \real{X_{\F}(f) e^{\j\frac{3\pi}{4}}} \\
	& = -\Re{X}_{\F}(f) - \Im{X}_{\F}(f)
\end{equations}

One can also write the RFT in terms of the FT as
\begin{equation}\label{eq:sec2:equivalence_pair_ft_to_rft}
	X_{\R}(f) = \frac{1}{\Sqrt{2}} \pts{e^{\j\frac{3\pi}{4}} X_{\F}(f) + e^{-\j\frac{3\pi}{4}} X_{\F}(\-f) }
\end{equation}
from which we deduce that
\begin{equation}\label{eq:sec2:equivalence_pair_rft_to_ft}
	X_{\F}(f) = \frac{1}{\Sqrt{2}} \pts{e^{-\j\frac{3\pi}{4}} X_{\R}(f) + e^{\j\frac{3\pi}{4}} X_{\R}(\-f) }
\end{equation}

\subsection{Convolution}

Given an impulse response $h(t)$ for a LIT system, the FT the convolution theorem states that
\begin{equation}
	h(t) \ast x(t) \hrel{\F} H_{\F}(f) X_{\F}(f)
\end{equation}
where $\hrel{\F}$ indicates a Fourier transform pair. It can be shown that this theorem isn't strictly valid for the RFT (see \cref{prop:conv_theorem_RFT} in \cref{app:properties_rft}). That is, if $H_{\R}(f)$ and $X_{\R}(f)$ are the RFT's of $h(t)$ and $x(t)$ respectively, then
\begin{equation}
	h(t) \ast x(t) \centernot{\hrel{\R}} H_{\R}(f) X_{\R}(f)
\end{equation}

However, it is possible to prove that there is an equivalent of the convolution theorem for the RFT (see \cref{prop:equivalent_conv_theorem_RFT} in \cref{app:properties_rft}), where
\begin{equation}
	\label{eq:sec2:convolution_equiv_rft}
	h(t) \ast x(t) \hrel{\R} X_{\R}(f) \Re{H}_{\F}(f) + X_{\R}(\-f) \Im{H}_{\F}(f)
\end{equation}
where, for a given frequency $f$, the convolution's output on the RFT domain depends on both it and its dual frequency $\-f$.

\subsection{Relative transfer functions}
\label{subsec:sec2:relative_transfer_functions}

Given two systems that share an input $x(t)$, each with an impulse response $h_1(t)$ and $h_2(t)$, on the FT domain $H_1(f)$ and $H_2(f)$, we can calculate their relative transfer functions (RTF's), respective to a common input. We denote these RTF's $A_1(f)$ and $A_2(f)$, respective for each system.

Let us denote $Y_1(f)$ as the first system's output, given by
\begin{equation}
	\label{eq:output_sys1_ft}
	Y_1(f) = H_1(f) X(f)
\end{equation}
and similarly for $Y_2(f)$. With the FT, we write $X_1(f) = H_1(f) X(f)$, and thus $Y_1(f) = A_1(f) X_1(f)$ with $A_1(f) = 1$. We can obtain $A_2(f)$ as
\begin{equation}
	A_2(f) = \frac{H_2(f)}{H_1(f)}
\end{equation}
which trivially satisfies that $A_2(f) X_1(f) = H_2(f) X(f)$. These RTF's can be calculated as
\begin{equation}
	\label{eq:sec2:calc_RTF_ft_expec}
	A_m(f) = \frac{\expec{X_m(f) X_1^*(f)}}{\expec{X_1(f) X_1^*(f)}}
\end{equation}
where $\expec{\cdot}$ is the expectation operator.

This isn't as straight-forward with the RFT, since after the convolution each frequency depends on its conjugate as well.  However, by considering each system to have two inputs $X'(f) = X(f)$ and $X''(f) = X[l,K-k]$ and two transfer functions $H_m'(f)$ and $H_m''(f)$ (where $m$ represents the system's index), then our outputs can be described as
\begin{equation}
	\label{eq:output_sys1_rft}
	Y_1(f) = H_1'(f) X'(f) + H_1''(f) X''(f)
\end{equation}
and in the same way for $Y_2(f)$ under the RFT. From \cref{eq:sec2:convolution_equiv_rft}, we easily see that
\begin{subgather}{eqs:sec2:def_Hm'_Hm''_rft}
	H_m'(f) = H_m'(\-f) = \Re{H}_{\sF;m}(f) \\
	H_m''(f) = - H_m''(\-f) = \Im{H}_{\sF;m}(f)
\end{subgather}
We let
\begin{subgather}{eqs:sec2:def_X1'_and_X1''_rft}
	X_1'(f) = H_1'(f)X'(f) \\X_1''(f) = H_1''(f) X''(f)
\end{subgather}
and therefore
\begin{equations}{eq:sec2:definition_X1_rft}
	X_1(f)
	& = H_1'(f) X'(f) + H_1''(f) X''(f) \\
	& = X_1'(f) + X_1''(f)
\end{equations}
in which $X_1'(f)$ and $X_1''(f)$ are the inputs processed by the first system, and $X_1(f)$ is the observable input signal.
Through these, we get that
\begin{subgather}{eqs:sec2:rtf_systems_rft}
	Y_1(f) = A_1'(f) X_1'(f) + A_1''(f) X_1''(f) \\
	Y_2(f) = A_2'(f) X_1'(f) + A_2''(f) X_1''(f)
\end{subgather}
where
\begin{equation}
	\label{eq:sec2:rtf_form_rft}
	A_m^{n}(f) = \frac{H_m^{n}(f)}{H_1^{n}(f)}
\end{equation}
are the RTF's for the $n$-th input, between the $m$-th system and the reference (assumed to be $m=1$). Trivially, $A_1'(f) = A_1''(f) = 1$. Through the properties of the RFT exposed in \cref{app:properties_rft}, we can build the following system of equations:
\begin{subalign}{eqs:sec2:system_equations}
	\expec{X_{1}(f) X_{1}(f)}   & = \pts{  {\Re{H_{1\F}}}^2(f)             + {\Im{H_{1\F}}}^2(f)         } \sigma_{X}^2(f) \label{eq:sec2:system_equations:subeq1} \\
	\expec{X_{1}(f) X_{2}(f)}   & = \pts{   \Re{H_{1\F}}(f) H_{2\F}^\re(f) +  \Im{H_{1\F}} H_{2\F}^\im(f)} \sigma_{X}^2(f) \label{eq:sec2:system_equations:subeq2} \\
	\expec{X_{1}(f) X_{2}(\-f)} & = \pts{- \Re{H_{1\F}}(f) H_{2\F}^\im(f) +  \Im{H_{1\F}} H_{2\F}^\re(f)} \sigma_{X}^2(f) \label{eq:sec2:system_equations:subeq3}
\end{subalign}
where $H_{m\F}(f) = \Re{H_{m\F}}(f) + \j\Im{H_{m\F}}(f)$ is the FT of $h_m(t)$, and $\sigma_X^2(f) = \expec{X_{\F}(f)^2} = \expec{X_{\R}(f)^2}$.

Note that, we have 3 equations and 5 variables. We define the following variables
\begin{equations}
	\ar & 	= \Re{H_{1\F}}(f) \sigma_X(f) \\
	\ai & 	= \Im{H_{1\F}}(f) \sigma_X(f) \\
	\gr & 	= \frac{\Re{H_{2\F}}(f) \sigma_X(f)}{\ar} \\
	\gi & 	= \frac{\Im{H_{2\F}}(f) \sigma_X(f)}{\ai}
\end{equations}
\begin{equations}
	\E{11} 	& 	= \expec{X_{1}(f) X_{1}(f)} \\
	\E{12}	& 	= \expec{X_{1}(f) X_{2}(f)} \\
	\E*{12} &   = \expec{X_{1}(f) X_{2}(\-f)}
\end{equations}
where we will be omitting the $(f)$ index for clarity in these newly defined variables, however they are frequency-dependent. With this, \cref{eqs:sec2:system_equations} becomes	
\begin{subalign}{eqs:sec2:system_equations_gamma}
	\ar^2 + \ai^2 & = \E{11} \label{eq:sec2:system_equations_gamma:subeq1} \\
	\ar^2 \gr + \ai^2 \gi & = \E{12} \label{eq:sec2:system_equations_gamma:subeq2} \\
	\ar\ai\pts{\gr - \gi} & = \E*{12} \label{eq:sec2:system_equations_gamma:subeq3}
\end{subalign}

In this new scenario our objective is to calculate $\gr$ and $\gi$, since
\begin{subgather}
	A_{2,\R}'(f) = \gr \\
	A_{2,\R}''(f) = \gi
\end{subgather}

Firstly, from \cref{eq:sec2:system_equations_gamma:subeq2,eq:sec2:system_equations_gamma:subeq2} we can find $\gr$ and $\gi$ in terms of the expectations, $\ar$ and $\ai$. Solving it gives us
\begin{subalign}{eq:sec2:system_equations_gamma:sol_gr_gi}
	\gr & = \frac{\E{12}}{\E{11}} + \frac{\frac{\ai}{\ar} \E*{12}}{\E{11}} \\
	\gi & = \frac{\E{12}}{\E{11}} - \frac{\frac{\ar}{\ai} \E*{12}}{\E{11}}
\end{subalign}

It is easy to verify that
\begin{subgather}
	\expec{X_2(f) X_1(\-f)} = -\E*{12} \\
	\begin{split}
		\expec{X_2(f)^2}
		& = \frac{\E*{12}^2 + \E{12}^2}{\E{11}} \\
		& = \ar^2 \gr^2 + \ai^2 \gi^2
	\end{split}
\end{subgather}
not being useful to solve this problem since they're dependent equations. Other possibilities include trying to use the FT's RTF, however it is also easy to check that 
\begin{subgather}
	\abs{A_{m;\F}(f)}^2 = \frac{\E{22}}{\E{11}} \\
	\angle{A_{m;\F}(f)} = \atan{\frac{\E*{12}}{\E{12}}}
\end{subgather}
thus again not being independent relationships between the variables at hand. This is an under-determined system with $3$ equations and $4$ variables, which has infinite solutions. Given any choice of $\ar$ such that $\abs{\ar} \leq \E{11}$, we can find $\ai$ through \cref{eq:sec2:system_equations_gamma:subeq1}, and $\gr$ and $\gi$ (which are equivalent to $A_{2,\R}'$ and $A_{2,\R}''$) via \cref{eq:sec2:system_equations_gamma:sol_gr_gi}.

The FT formulation can be seen as a particular case of the RFT (comparing \cref{eq:output_sys1_ft} with \cref{eq:output_sys1_rft}), where $A_{m;\F}'(f) = A_{m;\F}(f)$ and $A_{m;\F}''(f) = 0$. Therefore, from now on we will use the SSBT formulation from \cref{eq:sec2:system_ctf_ssbt_output}, as it is a less restricting model.

\subsection{Discrete time-frequency transforms}

Given a time-domain signal $x[n]$, its Short-time Fourier Transform (STFT) \cite{kiymik_comparison_2005,pan_microphone_2021} is
\begin{equation}
	\label{eq:sec2:def_stft_xn}
	X_{\sF}[l,k] = \sum_{n=0}^{K-1} w[n] x[n + l\cdot O] e^{-\j 2\pi k \frac{(n + l\cdot O)}{K}}
\end{equation}
where $w[n]$ is an analysis window of length $K$; and $O$ is the overlap between windows of the transform, usually $O = \floor{\nicefrac{K}{2}}$. The STFT can be seen as a discretization of the FT, while also applying it over different ``snippets'' of time.

%Even though the STFT is the most traditionally used time-frequency transform, it isn't the only one available. Thus, exploring different possibilities for such an operation can be useful and lead to interesting results.

The Single-Sideband Transform (SSBT) \cite{crochiere_multirate_1983} is similarly defined, being the RFT's windowed discrete-time adaptation. The SSB transform of $x[n]$ is defined as
\begin{equation}
	\label{eq:sec2:def_ssbt_xn}
	X_{\sS}[l,k] = \Sqrt{2} \real{\sum_{n=0}^{K-1} w[n] x[n + l\cdot O] e^{-\j 2\pi k \frac{(n + l\cdot O)}{K} + \j\frac{3\pi}{4} } }
\end{equation}

One advantage of using the STFT is that we only need to work with $\floor{\nicefrac{(K+1)}{2}}+1$ frequency bins, given its complex-conjugate behavior. Meanwhile, the SSBT requires all $K$ bins to correctly capture all information of $x[n]$, however it is real-valued.

Assuming that all $K$ bins of the STFT are available, like with \cref{eq:sec2:equivalence_ft_rft,eq:sec2:equivalence_pair_ft_to_rft,eq:sec2:equivalence_pair_rft_to_ft} we have\footnote{For the abuse of notation, we let $X_{\sS}[l,K] \equiv X_{\sS}[l,0]$, and equally for $X_{\sF}[l,K]$.}
\begin{equations}
	\label{eq:sec2:equivalence_stft_ssbt}
	X_{\sS}[l,k]
	& = \Sqrt{2} \real{X_{\sF}[l,k] e^{\j\frac{3\pi}{4}}} \\
	& = - \real{X_{\sF}[l,k]} - \imag{X_{\sF}[l,k]}
\end{equations}
\begin{equation}
	X_{\sS}[l,k] = \frac{1}{\Sqrt{2}} \pts{ e^{\j\frac{3\pi}{4}} X_{\sF}[l,k] + e^{-\j\frac{3\pi}{4}} X_{\sF}[l,K-k] }
\end{equation}
\begin{equation}
	\label{eq:sec2:equivalence_ssbt_stft}
	X_{\sF}[l,k] = \frac{1}{\Sqrt{2}} \pts{ e^{-\j\frac{3\pi}{4}} X_{\sS}[l,k] + e^{\j\frac{3\pi}{4}} X_{\sS}[l,K-k] }
\end{equation}

As was the case for the RFT, the SSBT also doesn't hold the convolution theorem the same way as the STFT does. However, similarly to what was shown in \cref{eq:sec2:convolution_equiv_rft} (and in \cref{prop:equivalent_conv_theorem_RFT}), we can write the convolution on the SSBT domain as
\begin{equation}
	\label{eq:sec2:convolution_on_ssbt_mtf}
	h[n] \ast x[n] \hrel{\sS} X_{\sS}[l,k] H_{\sS}'[k] + X_{\sS}[l,K-k] H_{\sS}''[k]
\end{equation}
or, with the convolutive transfer function (CTF) model \cite{talmon_relative_2009},
\begin{equation}
	\label{eq:sec2:convolution_on_ssbt_ctf}
	h[n] \ast x[n] \hrel{\sS} X_{\sS}[l,k] \ast H_{\sS}'[l,k] + X_{\sS}[l,K-k] \ast H_{\sS}''[l,k]
\end{equation}
in which this convolution is done over the frames $l$, with $H_{\sS}'[l,k] = \Re{H}_{\sF}[l,k]$ and $H_{\sS}''[l,k] = \Im{H}_{\sF}[l,k]$.

Under the same assumptions as in \cref{subsec:sec2:relative_transfer_functions}, we can write the output of our systems with the CTF model in the SSBT domain as
\begin{equation}
	\label{eq:sec2:system_ctf_ssbt_output}
	X_{m;\sS}[l,k] = X_{1;\sS}'[l,k] \ast A_{m;\sS}'[l,k] + X_{1;\sS}'' \ast A_{m;\sS}''[l,k]
\end{equation}
where $A_{m;\sS}'[l,k]$ and $A_{m;\sS}''[l,k]$ are defined such that
\begin{equation}
	X_{1;\sS}^n[l,k] \ast A_{m;\sS}^n[l,k] = X_{\sS}^n[l,k] \ast H_{m;\sS}^n[l,k]
\end{equation}

From \cref{eqs:sec2:def_Hm'_Hm''_rft,eqs:sec2:def_X1'_and_X1''_rft,eq:sec2:definition_X1_rft} we have that $X_{1}'[l,k] \neq X_{1}''[l,K-k]$, even though both originate from $X_{1}[l,k]$. Likewise, for the same reason $X_{1}''[l,k] \neq X_{1}'[l,K-k]$.