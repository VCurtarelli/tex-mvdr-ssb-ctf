\section{STFT and the Single-Sideband Transform}
\label{sec:stft_and_ssbt}

When studying signals and systems, often frequency and time-frequency transforms are used in order to change the signal domain \cite{demuth_frequency_1977}, allowing the exploitation of different patterns and informations inherent to the signal.

Given a time-domain signal $x[n]$, its Short-time Fourier Transform (STFT) \cite{kiymik_comparison_2005,pan_microphone_2021} is
\begin{equation}
	\label{eq:sec2:def_stft_xn}
	X_{\F}[l,k] = \sum_{n=0}^{K-1} w[n] x[n + l\cdot O] e^{-\j 2\pi k \frac{(n + l\cdot O)}{K}}
\end{equation}
where $w[n]$ is an analysis window of length $K$; and $O$ is the overlap between windows of the transform, usually $O = \floor{\nicefrac{K}{2}}$. Even though the STFT is the most traditionally used time-frequency transform, it isn't the only one available. Thus, exploring different possibilities for such an operation can be useful and lead to interesting results.

The Single-Sideband Transform (SSBT) \cite{crochiere_multirate_1983} is one such alternative, being cleverly constructed such that its frequency spectrum is real-valued, without loss of information. The SSB transform of $x[n]$ is defined as
\begin{equation}
	\label{eq:sec2:def_ssbt_xn}
	X_{\S}[l,k] = \Sqrt{2} \real{\sum_{n=0}^{K-1} w[n] x[n + l\cdot O] e^{-\j 2\pi k \frac{(n + l\cdot O)}{K} + \j\frac{3\pi}{4} } }
\end{equation}

Assuming that $x[n]$ is real-valued, one advantage of using the STFT is that we only need to work with $\floor{\nicefrac{(K+1)}{2}}+1$ frequency bins, given its complex-conjugate behavior. Meanwhile, the SSBT requires all $K$ bins to correctly capture all information of $x[n]$, however it is real-valued.

Assuming that all $K$ bins of the STFT are available, from \cref{eq:sec2:def_stft_xn,eq:sec2:def_ssbt_xn} we have
\begin{equations}
	\label{eq:sec2:equivalence_stft_ssbt}
	X_{\S}[l,k]
	& = \Sqrt{2} \real{X_{\F}[l,k] e^{\j\frac{3\pi}{4}}} \\
	& = - \real{X_{\F}[l,k]} - \imag{X_{\F}[l,k]}
\end{equations}

It is easy to see that\footnote{For the abuse of notation, we let $X_{\S}[l,K] \equiv X_{\S}[l,0]$, and equally for $X_{\F}[l,K]$.}
\begin{equation}
	X_{\S}[l,k] = \frac{1}{\Sqrt{2}} \pts{ e^{\j\frac{3\pi}{4}} X_{\F}[l,k] + e^{-\j\frac{3\pi}{4}} X_{\F}[l,K-k] }
\end{equation}
from which it we deduce
\begin{equation}
	\label{eq:sec2:equivalence_ssbt_stft}
	X_{\F}[l,k] = \frac{1}{\Sqrt{2}} \pts{ e^{-\j\frac{3\pi}{4}} X_{\S}[l,k] + e^{\j\frac{3\pi}{4}} X_{\S}[l,K-k] }
\end{equation}

One disadvantage of the SSBT is that the convolution theorem does not hold when employing it (see \cref{app:ssbt_convolution}), not even as an approximation. Nonetheless, by converting any result in the SSBT domain to the STFT domain (using \cref{eq:sec2:equivalence_stft_ssbt}) before utilization, it remains feasible to employ the transform to study of the problem at hand.