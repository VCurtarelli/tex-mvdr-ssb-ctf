\section{STFT and Single-Sideband Transform}
\label{sec:stft_and_ssbt}

When studying signals and systems, often frequency and time-frequency transforms are used in order to change the signal domain \cite{demuth_frequency_1977}, allowing the exploitation of different patterns and informations that are inherent to the signal.

Given a time-domain signal $x[n]$, its Short-time Fourier Transform (STFT) \cite{kiymik_comparison_2005,pan_microphone_2021} is given by
\begin{equation}
	X_{\F}[l,k] = \sum_{n=0}^{K-1} w[n] x[n - l\cdot O] e^{-\j 2\pi k \frac{(n - l\cdot O)}{K}}
\end{equation}
where $w[n]$ is an analysis window of length $K$; and $O$ is the overlap between windows of the transform, usually $O = \floor{\nicefrac{K}{2}}$. Even though the STFT is the most traditionally used time-frequency transform, it isn't the only one available. Thus, exploring different possibilities for such an operation can be useful and lead to interesting results.

The Single-Sideband Transform (SSBT) \cite{crochiere_multirate_1983} is one such alternative, in which the frequency values are cleverly calculated such that its spectrum is real-valued, without loss of information. The SSB transform of $x[n]$ is defined as
\begin{equation}
	\label{eq:sec2:def_ssbt_xn}
	X_{\S}[l,k] = \Sqrt{2} \real{\sum_{n=0}^{L-1} w[n] x[n-l\cdot O] e^{\j 2\pi k \frac{(n - l\cdot O)}{K} + \j\frac{3\pi}{4} } }
\end{equation}

Assuming that $x[n]$ is real-valued, one advantage of using the STFT is that we only need to work with $\floor{\frac{K+1}{2}}+1$ frequency bins, given its complex-conjugate behavior. Meanwhile, the SSBT needs to use all $K$ possible bins to correctly capture all information of $x[n]$, however it is real-valued.

From \cref{eq:sec2:def_ssbt_xn} it's easy to see that
\begin{equations}
	\label{eq:sec2:equivalence_stft_ssbt}
	X_{\S}[l,k]
	& = \Sqrt{2} \real{X_{\F}[l,k] e^{\j\frac{3\pi}{4}}} \\
	& = - \real{X_{\F}[l,k]} + \imag{X_{\F}[l,k]}
\end{equations}
assuming that all $K$ bins of the STFT are available.

It is possible to show that, unlike with the STFT, the convolution theorem does not hold when employing the SSBT. In other words, if $y[n] = h[n] \ast x[n]$, then $Y_{\F}[l,k] = H_{\F}[l,k] \ast X_{\F}[l,k]$, but $Y_{\S}[l,k] \neq H_{\S}[l,k] \ast X_{\S}[l,k]$. Nonetheless, by first converting any result into the STFT domain (using \cref{eq:sec2:equivalence_stft_ssbt}) before utilization, it remains feasible to employ the obtained values for estimating matrices and signals.