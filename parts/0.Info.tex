\title{%
%{\smaller[2][DRAFT]}%
Constant-Beamwidth LCMV Beamformer with Rectangular Arrays}

\author{Vitor Probst Curtarelli and Israel Cohen%\,\orcidlink{0000-0002-2556-3972}
        % <-this % stops a space
\thanks{This paper was produced by the IEEE Publication Technology Group. They are in Piscataway, NJ.}% <-this % stops a space
\thanks{Manuscript received April 19, 2021; revised August 16, 2021.}}

% The paper headers
\markboth{Journal of \LaTeX\ Class Files,~Vol.~14, No.~8, August~2021}%
{Shell \MakeLowercase{\textit{et al.}}: A Sample Article Using IEEEtran.cls for IEEE Journals}

% \IEEEpubid{0000--0000/00\$00.00~\copyright~2021 IEEE}
% Remember, if you use this you must call \IEEEpubidadjcol in the second
% column for its text to clear the IEEEpubid mark.

\maketitle

\begin{abstract}
This paper presents a novel approach utilizing rectangular uniform arrays to design a constant-beamwidth linearly constrained beamformer with some gain in white noise gain and directivity. By employing a generalization of the convolutional Kronecker product beamforming technique, we decompose the physical array into virtual subarrays, each tailored to achieve a specific desired feature, and subsequently synthesize the original array's beamformer. Through extensive simulations, we demonstrate that the proposed approach successfully achieves the desired beamforming characteristics while maintaining favorable levels of white noise gain and directivity. Comparative analysis against existing methods from the literature reveals both advantages and disadvantages of our proposed method.
\end{abstract}

\begin{IEEEkeywords}
LCMV, Constant-beamwidth, KP beamformer, CKP beamformer, sensor array, rectangular arrays
\end{IEEEkeywords}