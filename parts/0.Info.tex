\Title{On Beamforming with the Single-Sideband Transform}

% MDPI internal command: Title for citation in the left column
\TitleCitation{On Beamforming with the Single-Sideband Transform}

% Author Orchid ID: enter ID or remove command
\newcommand{\orcidauthorA}{0009-0009-3996-5452} % Add \orcidA{} behind the author's name
\newcommand\autAFN{Vitor Probst}
\newcommand\autALN{Curtarelli}
\newcommand\autAfn{V. P.}
\newcommand{\orcidauthorB}{0000-0002-2556-3972} % Add \orcidB{} behind the author's name
\newcommand\autBFN{Israel}
\newcommand\autBLN{Cohen}
\newcommand\autBfn{I.}

% Authors, for the paper (add full first names)
\Author{\autAFN{} \autALN $^{1,\ast}$\orcidA{}, \autBFN{} \autBLN $^{1}$\orcidB{}}

%\longauthorlist{yes}

% MDPI internal command: Authors, for metadata in PDF
\AuthorNames{\autAFN{} \autALN, \autBFN{} \autBLN}

% MDPI internal command: Authors, for citation in the left column
\AuthorCitation{\autALN, \autAfn; \autBLN, \autBfn}


\address{%\\
$^{1}$ \quad Andrew and Erna Viterbi Faculty of Electrical and Computer Engineering, Technion--Israel Institute of Technology, Technion City, Haifa 3200003, Israel
}

% Contact information of the corresponding author
\corres{Correspondence: vitor.c@campus.technion.ac.il
}
\abstract{%
    In this paper, we explore the application of the Single-Sideband Transform for convolutive beamformers. We investigate the unique properties of the SSBT and their implications for beamformer design. Despite its advantageous real coefficients, we find that the transform's handling of convolution poses challenges, impacting fundamental beamforming premises. Compared to the Short-Time Fourier Transform, the SSBT exhibits lower robustness, particularly in scenarios involving mismatch and modeling noise. We validate our theoretical findings through realistic simulations. These simulations demonstrate that while the real-valued SSBT performs marginally worse than its complex-valued counterpart under ideal conditions, its performance deteriorates significantly in non-ideal scenarios. Notably, we establish a direct equivalence between SSBT and STFT under identical parameters, facilitating seamless interchangeability and allowing their joint application. Our study underscores the nuanced trade-offs between SSBT and STFT in beamforming applications, providing insights into their respective strengths and limitations.
}

% Keywords
\keyword{Single-sideband transform; Time-frequency transforms; Convolutive beamforming; Array signal processing; Signal enhancement.} 