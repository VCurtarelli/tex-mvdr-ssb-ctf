\section{Constant-Beamwidth LCMV Beamformer with Rectangular Arrays}
\label{sec:cblcmv_beamformer}

Both the KP and LCKP beamforming methods have advantages and disadvantages. While the LCKP is only usable over linear arrays, beamformers achieved through it have (virtually) more sensors than there are available in the physical array, generally leading to better performance when combining different techniques. Meanwhile, the KP method can be applied to rectangular arrays, which on its own is beneficial, but it does not have the virtual utilization of more sensors. To benefit from both the lack of symmetry from the rectangular arrays but also be able to have more sensors available for each subarray's beamformer, we propose a generalization of the LCKP to URAs to take advantage of both methods, enabling virtual sensor augmentation while exploiting the rectangular array's symmetry.

\subsection{Rectangular Convolutional Kronecker-Product beamforming}

Let $S_1$ be a subarray of $S$, including the $\sz{\Mx{1}}{\My{1}}$ first sensors of $S$, with a steering vector $\bvd{1}$; and similarly for $S_2$. These arrays sizes are such that $\Mx{1} + \Mx{2} - 1 = \Mx$, and $\My{1} + \My{2} - 1 = \My$. By designing beamformers $\bvh{1}$ and $\bvh{2}$ for $S_1$ and $S_2$, respectively,  we show in \cref{app:RCKP_proof} that it is possible to synthesize the beamformer for the full-array $S$ through
\begin{equations}\label{eq:def_2D_CKP}
	\bvh
	& = \vect{\ivect{\My{1}}{\bvh{1}} \bconv \ivect{\My{2}}{\bvh{2}}} \eqc
\end{equations}
with $\bconv$ representing the 2-D convolution operation. We call this the rectangular convolutional Kronecker product (RCKP) method. A simple implementation of the proposed RCKP method is presented in \cref{alg:RCKP_beamformer} in \cref{app:algorithms}, which is written in a Python-like pseudolanguage.

In the same way as with KP and LCKP methods, with RCKP, one can design beamformers for the subarrays $S_1$ and $S_2$, and synthesize the URA's beamformer $\bvh$ through the 2-D convolution (and vectorization processes). Accordingly, $\bvh$ is an $\sz{M}{1}$ vector, however we have a total of $M' = \Mx{1} \My{1}  \Mx{2}  \My{2}$ virtual sensors. Using that all $M$s are $\geq 1$, it is trivial to see that\footnote{The equality happens if $S_1$ and $S_2$ are perpendicular ULAs, or one of them has only one sensor.} $M' \geq M$. Therefore, we virtually have more sensors than if $S$ was split into two VAs through the KP method.

It is easy to verify that the properties in \eqref{eq:properties_beam_dsdi} are also valid for the proposed RCKP. With this, like for the KP and LCKP, the full-array beamformer obtained through the RCKP inherits the beampattern and distortionless features from the subarray beamformers. Also, since convolution is commutative and associative, one can split the $S$ array into more than two virtual arrays, design a beamformer for each subarray, and combine all the beamformers through the RCKP without loss of generalization. In this case, assuming we are using $K$ beamformers, each with size $\sz{\Mx{k}}{\My{k}}$, then their dimensions must be such that
\begin{subgather}
    \sum_{k=1}^{K} \Mx{k} - (K-1) = \Mx \eqc \\
    \sum_{k=1}^{K} \My{k} - (K-1) = \My \eqp
\end{subgather}
This is easily verifiable by repeating the synthesis operation from \cref{eq:def_2D_CKP} $K-1$ times.

\subsection{CB-LCMV beamformer with RCKP}

Here we propose the use of the RCKP method as derived previously to construct a CB-LCMV beamformer with an increase in white noise gain and directivity measures. For such, the full-array $S$ is separated into four subarrays: $S_1$, $S_{2;\x}'$, $S_{2;\x}''$ and $S_{2;\y}$. Each subarray is used to design one of the desired beamformers presented in \cref{ssec:defs:lcmv_beamformer,ssec:defs:cb_beamformer,ssec:defs:sd_ds_beamformers}.

$S_1$ is used to design the LCMV beamformer, following the steps detailed in \cref{ssec:defs:lcmv_beamformer}. Since the LCMV beamformer does not require the array to be linear, we  use $S_1$ as a rectangular array of size $\sz{\Mx{1}}{\My{1}}$. As explained previously, this choice has the advantage of the rectangular array's lesser symmetry compared to a linear array. We have $M_1 = \Mx{1}\My{1}$ virtual sensors, and at most $M_1 - 1$ nulls to be placed. If $N = 0$, this same array is used for the MVDR beamformer instead.

The subarray $S_{2;\y}$ is used to construct the CB beamformer. Given that the CB is achieved through the window technique (as per \cref{ssec:defs:cb_beamformer}), this subarray must be a ULA, and the desired source direction should be on its broadside direction. Here, this implies that $\td = 0\dg$.

The SD and DS beamformers are built from the $S_{2;\x}'$ and $S_{2;\x}''$ subarrays respectively, based on \cref{ssec:defs:sd_ds_beamformers}. We assume they are constructed from linear arrays, but this condition is flexible and can be changed in other implementations.

Once all the subarray beamformers are designed, we use the RCKP method to combine these beamformers into the full-array beamformer. Thus, we construct a beamformer with many desired features (null-placement $+$ constant-beamwidth $+$ white noise gain $+$ directivity factor gain) that exploit the symmetry of the rectangular array, which implies more spatial information and more performance. Algebraically, the full-array beamformer is given by
\begin{equation}
    \bvh = \vect{ \ivect{\My{1}}{\bvh{\lcmv}} \bconv \ivect{1}{\bvh{\sd}} \bconv \ivect{1}{\bvh{\ds}} \bconv \ivect{\My{2}}{\bvh{\cb}} }\eqp
\end{equation}

An implementation of the proposed CB-LCMV beamformer is shown in \cref{alg:CBLCMV_beamformer} in \cref{app:algorithms}.
The procedure calculates the beamformer $\bvh$ for a single frequency.

For the CB beamformer to be effective, its condition must be valid for all beamformers. That is, $\abs{\beam{\bvh,\bvd}} > 0$ if $\abs{\t - \td} < %\nfrac{\tB\,}{\,2}$, 
\tB\,/\,2$,
where $\bvh \in [\bvh{\lcmv},~\bvh{\cb},~\bvh{\sd},~\bvh{\ds}]$. This is true by definition for the CB beamformer, and by setting $N = M_1 - 1$ for the LCMV, no nulls are free to end up inside the main beam. For the SD and DS beamformers, we assume that their FNBW is sufficient to satisfy the condition.