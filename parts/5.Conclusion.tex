\section{Conclusion}
\label{sec:conclusion}

In this study, we investigated the Single-Sideband Transform, examining and mathematically exploring its properties. We employed it within a reverberant environment utilizing the convolutive transfer function model for the design of a filter bank (i.e., a beamformer). We implemented a Minimum-Power Distortionless Response filter to enhance the input signals of a sensor array in a real-life-like scenario, comparing the results of a filter obtained using the SSB transform for time-frequency signal analysis, with those obtained when this process is done via the Short-Time Fourier Transform.

The theoretical results show that, although interesting from a practical point of view, given its real-valued property, the SSBT is much more prone to errors in relative frequency response estimation, as well as needing twice as many mathematical constraints for each practical one to be achieved. We also showed that, contrary to previous belief, when properly designed the SSBT beamformer is more suitable with longer windows and under the multiplicative transfer function model, instead of shorter ones and the convolutive model, given that in these situations the estimation errors are lessened.

Practical results corroborate the theoretical ones achieved where, when we perfectly have an accurate representation of the relative frequency responses, the SSBT filter performance is slightly worse to that of the STFT filter, which is in line with being twice as constrained; and when we estimate the RFR's from the desired signal's samples, the SSBT beamformer performance drastically worsens when compared to STFT, except for when we consider longer windows and the multiplicative transfer function model, while maintaining a low input Signal-to-Echo Ratio. Therefore, in a real-life application, one should be aware of these limitations and considerations when considering implementing this transform.

Future research avenues may explore the integration of this transform into different beamformers, or undertake further comparisons of the proposed SSBT beamformer (following the considerations and properties exposed in here) against the established and reliable STFT method.