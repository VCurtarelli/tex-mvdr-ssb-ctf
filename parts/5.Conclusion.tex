\section{Conclusions}
\label{sec:conclusion}

We have conducted a comprehensive investigation into the Single-Sideband Transform within the context of beamforming, examining its mathematical properties and interaction with key processes such as convolution, relative frequency response estimation, and the distortionless constraint.
Our theoretical study reveals that despite its interesting real-valued representation, the SSBT exhibits higher susceptibility to errors in RFR estimation, requiring stricter constraints for its proper application. We found that in scenarios with longer time windows and under the multiplicative transfer function model, the estimation errors with the SSBT are reduced. Furthermore, we established that the SSBT and the Short-Time Fourier Transform are interchangeable in the time-frequency domain without the need for their respective inverse transforms, enabling a seamless conversion between the two.

To validate our theoretical findings, we employed both transforms in the design of a convolutive Minimum-Power Distortionless Response beamformer within a reverberant environment across various scenarios. These practical results support our theoretical claims, showing that the SSBT-based filter slightly underperforms the STFT-based one in optimal conditions and significantly underperforms in non-ideal situations.
While these findings highlight challenges in directly applying the SSBT for beamforming, the interchangeability of these transforms allows for filtering in the STFT domain -- even when signals are initially in the SSBT domain. This allows for the combined use of the transformations, taking advantage of their respective strengths as applicable.
Future research could explore integrating the SSBT into more robust beamformers, further comparing it to the STFT, and studying the cooperative integration of the two transforms in greater depth.