\section{Conclusions}
\label{sec:conclusion}

In this study, we conducted a comprehensive investigation into the Single-Sideband Transform within the context of beamforming, examining its mathematical properties and interaction with key processes such as convolution, relative frequency response estimation, and the distortionless constraint.

Our theoretical study reveals that despite its interesting real-valued spectrum, the SSBT exhibits higher susceptibility to errors in RFR estimation, requiring stricter constraints for its proper application. We identified that in scenarios with longer windows and under the multiplicative transfer function model, the estimation errors with the SSBT are lessened. Furthermore, we established that the SSBT and the Short-Time Fourier Transform are interchangeable in the time-frequency domain without the need for their respective inverse transforms, enabling a seamless conversion between the two.

To validate our theoretical findings, we employed both transforms in the design of a convolutive Minimum-Power Distortionless Response beamformer within a reverberant environment across various scenarios. These practical results corroborate our theoretical claims, demonstrating that the SSBT-based filter marginally underperforms the STFT-based one in ideal conditions and significantly underperforms in non-ideal scenarios.

While these findings highlight challenges in direct applying the SSBT for beamforming, the interchangeability of these transforms allow for filtering in the STFT domain -- even when signals are initially in the SSBT domain. This viabilizes the joint use of the transforms, leveraging the strengths of each where applicable.

Future research avenues could explore integrating the SSBT into more robust beamformers, further comparing it to the STFT considering the properties exposed here; and more deeply study the cooperative integration of the two transforms.