\definecolor{ColA}{HTML}{991F3D}
\definecolor{ColB}{HTML}{997A1F}
\definecolor{ColC}{HTML}{3D991F}
\definecolor{ColD}{HTML}{1F997A}
\definecolor{ColE}{HTML}{1F3D99}
\definecolor{ColF}{HTML}{7A1F99}
\NewDocumentCommand{\filename}{m}{%
	\small{\texttt{#1}}
}
\let\mc\multicolumn
\section{Comparisons and simulations}
\label{sec:results}

In the simulations\footnote{Code is available at \url{https://github.com/VCurtarelli/py-ssb-ctf-bf}.}, we employ a sampling frequency of $16\si{\kilo\hertz}$. Room impulse responses were generated using Habets' RIR generator \cite{habets_rir-generator}, and signals were selected from the SMARD database \cite{smard_database}. In all cases we will use $\Ld = \Ly$ and $\Delta = 0$; that is, we don't use any non-causal frames, and we consider as many frame samples as RIR samples.

The room's dimensions are $4\m \times 6\m \times 3\m$ (width $\times$ length $\times$ height), with a reverberation time of $0.3\si{\second}$. The device composed of the loudspeaker + sensors is centered at $(3\m,~4\m,~1\m)$, being comprised of $M=8$ sensors arranged in a circular array with radius of $8\si{\centi\meter}$, and all are omnidirectional of flat frequency response. The positions and signals used for the sources are in \cref{tab:sec4:information_position_sources}. The room's layout is in \cref{fig:room_layout}, where in green we have the desired source (assumed to be omnidirectional), and in red the device, with the $8$ sensors and the loudspeaker on the center.

\begin{table}[H]
	\centering
	\begin{tabular}{rll}
		Source & Position 				& Signal \\
		\hline\vphantom{$\tilde{d}$}
		$x[n]$ & $(2\m,~5\m,~1.8\m)$ 	& \filename{50\_male\_speech\_english\_ch8\_OmniPower4296.flac} \\
		$s[n]$ & $(3\m,~4\m,~1\m)$ 		& \filename{69\_abba\_ch8\_OmniPower4296.flac} \\
		$r[n]$ & \mc{1}{c}{$\sim$}		& \filename{wgn\_48kHz\_ch8\_OmniPower4296.flac}
	\end{tabular}
	\caption{Source information for the simulations.}
	\label{tab:sec4:information_position_sources}
\end{table}\vspace*{-2em}

\begin{figure}[!t]
	\centering
	\includesvg[width=0.7\linewidth]{figures/drawing/room_layout.svg}
	\caption{Room layout for simulations.}
	\label{fig:room_layout}
\end{figure}

All signals were resampled to the desired sampling frequency of $16\si{\kilo\hertz}$. For the transforms, Hamming windows were used, with a length of 32 samples/window and an overlap of $50\%$. The beamformers were calculated once for the whole signal, for faster processing and ease to compare the results. We will compare the two transforms exposed, the STFT and the SSBT, for a varying set of conditions on the signals and variables considered.

In the line plots, STFT results are presented in red lines with circles and SSBT results in green lines with triangles. The output metrics were averaged over 200 frames and presented every 100 windows, for a smoother visualization.
%

\subsection{Metrics of interest}

The main objective of the filters compared is the reduction of the loudspeaker's signal, while preserving the desired signal without distortion. Along this, although our filter isn't designed to reduce the white noise, a minimal enhancement of it would also be of interest. The loudspeaker's signal minimization will be measured through the echo reduction loss enhancement (ERLE, or $\erle$), the maintenance of the desired signal through the desired signal reduction factor (DSRF, or $\dsrf$), and the reduction of the white noise through the white noise gain (WNG, or $\wng$). We are will also observe the directivity index (DI, or $\di$), which measures the behavior of the beamformer when it would be located in a spherical anisotropic noise field. Respectively, these metrics in their time-dependent broadband formulations are given by
\begin{subgather}
	\dsrf[l] = \frac{\sum_{k}\abs{X_{1,k}[l]}^2}{\sum_{k}\abs{X_{f,k}[l]}^2} \\
	\erle[l] = \frac{\sum_{k}\abs{S_{1,k}[l]}^2}{\sum_{k}\abs{S_{f,k}[l]}^2} \\
	\wng[l] = \frac{\sum_{k}\abs{R_{1,k}[l]}^2}{\sum_{k}\abs{R_{f,k}[l]}^2} \\
	\di[l] = \frac{\sum_{k} \abs{\he{\bvf{k}}[l] \bvd{k}[l]}^2}{\sum_{k} \he{\bvf{k}}[l] \bvGa{k}[l] \bvf{k}[l]}
\end{subgather}
in which $\bvGa{k}[l]$ is the spherical anisotropic noise field correlation matrix \cite{habets_generating_2007} and $\bvd{k}[l]$ is the steering vector between the desired source and the sensor array, both assuming a far-end free field environment. We are also interested in a time-average broadband formulation for these metrics, this being given by
\begin{subgather}
	\dsrf = \frac{\sum_{l,k}\abs{X_{1,k}[l]}^2}{\sum_{l,k}\abs{X_{f,k}[l]}^2} \\
	\erle = \frac{\sum_{l,k}\abs{S_{1,k}[l]}^2}{\sum_{l,k}\abs{S_{f,k}[l]}^2} \\
	\wng = \frac{\sum_{l,k}\abs{R_{1,k}[l]}^2}{\sum_{l,k}\abs{R_{f,k}[l]}^2} \\
	\di = \frac{\sum_{l,k} \abs{\he{\bvf{k}}[l] \bvd{k}[l]}^2}{\sum_{l,k} \he{\bvf{k}}[l] \bvGa{k}[l] \bvf{k}[l]}
\end{subgather}

\subsection{Basic comparison}
In this scenario, at the reference sensor (assumed to be the one at $(3\m,~1.92\m,~1\m)$), the SNR for the loudspeaker's and noise signals are respectively $-15\dB$ and $30\dB$. These will be referred as Signal-to-Echo and Signal-to-Noise Ratios (SER and SNR) in that order. We will use $\Ly = 1$ and $\Ly = 16$ to illustrate two different cases in terms of samples used, one where we consider only the latest sample for each sensor, and another where we have sufficient samples for a (empirically) good enough performance.

\def\meshcols{316}
\def\meshrows{15}
\def\tmin{0}
\def\tmax{8.3725}
\def\fmin{0.25}
\def\fmax{7.75}
\begin{figure}[!ht]
	\centering
	\begin{subfigure}{\textwidth}
		\centering
		%% Requires:
% pgfplots.sty
% edit_pgfplots.tex

\pgfplotsset{compat=1.18}
%\begin{subfigure}{\linewidth}
%\centering
%\tikzsetnextfilename{dsrf_lineplot_32_n15_acc_v7}
\begin{tikzpicture}
	\begin{timeplot}{DSRF (dB)}[%ymin=-2.5, ymax=4, ytick={-2.5, 0, 2.5}, 
		legend to name = {timeplot_legend}]
		\addplot [style=resA1]
		table [col sep=comma, y=val] {figures/io_input/STFT/DSRF_l__h__STFT__N_32__iSER_n15__Ly_1__err_0.csv};
		
		\addplot [style=resA2]
		table [col sep=comma, y=val] {figures/io_input/STFT/DSRF_l__h__STFT__N_32__iSER_n15__Ly_15__err_0.csv};
		
		\addplot [style=resC1]
		table [col sep=comma, y=val] {figures/io_input/SSBT/DSRF_l__h__SSBT__N_32__iSER_n15__Ly_1__err_0.csv};
		
		\addplot [style=resC2]
		table [col sep=comma, y=val] {figures/io_input/SSBT/DSRF_l__h__SSBT__N_32__iSER_n15__Ly_15__err_0.csv};
		
		\addlegendentry{STFT - $\Ly = 1$};
		\addlegendentry{STFT - $\Ly = 15$};
		\addlegendentry{SSBT - $\Ly = 1$};
		\addlegendentry{SSBT - $\Ly = 15$};
%		\addlegendentry{S-SSBT};
	\end{timeplot}
\end{tikzpicture}
%	\caption{}
%	\label{subfig:1_gain_lineplot}
%\end{subfigure}
		\caption{Per-window broadband DSRF.}
		\label{subfig:lineplot__DSRF_l__iSER_n15__err_0}
	\end{subfigure}\\[1em]
	\begin{subfigure}{\textwidth}
		\centering
		%% Requires:
% pgfplots.sty
% edit_pgfplots.tex

\pgfplotsset{compat=1.18}
%\begin{subfigure}{\linewidth}
%\centering
%\tikzsetnextfilename{erle_lineplot_32_n15_acc_v7}
\begin{tikzpicture}
	\begin{timeplot}{ERLE (dB)}%[ymin=18, ymax=33, ytick={18, 24, 30}]
		\addplot [style=resA1]
		table [col sep=comma, y=val] {figures/io_input/STFT/ERLE_l__h__STFT__N_32__iSER_n15__Ly_1__err_0.csv};
		
		\addplot [style=resA2]
		table [col sep=comma, y=val] {figures/io_input/STFT/ERLE_l__h__STFT__N_32__iSER_n15__Ly_16__err_0.csv};
		
		\addplot [style=resC1]
		table [col sep=comma, y=val] {figures/io_input/SSBT/ERLE_l__h__SSBT__N_32__iSER_n15__Ly_1__err_0.csv};
		
		\addplot [style=resC2]
		table [col sep=comma, y=val] {figures/io_input/SSBT/ERLE_l__h__SSBT__N_32__iSER_n15__Ly_16__err_0.csv};
	\end{timeplot}
\end{tikzpicture}
%	\caption{}
%	\label{subfig:1_gain_lineplot}
%\end{subfigure}
		\caption{Per-window broadband ERLE.}
		\label{subfig:lineplot__ERLE_l__iSER_n15__err_0}
	\end{subfigure}\\[1em]
	\begin{subfigure}{\textwidth}
		\centering
		%% Requires:
% pgfplots.sty
% edit_pgfplots.tex

\pgfplotsset{compat=1.18}
%\begin{subfigure}{\linewidth}
%\centering
%\tikzsetnextfilename{nsrf_lineplot_32_n15_acc_v7}
\begin{tikzpicture}
	\begin{timeplot}{WNG (dB)}%[ymin=-15, ymax=-2]
		\addplot [style=resA1]
		table [col sep=comma, y=val] {figures/io_input/STFT/WNG_l__h__STFT__N_32__iSER_n15__Ly_1__err_0.csv};
		
		\addplot [style=resA2]
		table [col sep=comma, y=val] {figures/io_input/STFT/WNG_l__h__STFT__N_32__iSER_n15__Ly_16__err_0.csv};
		
		\addplot [style=resC1]
		table [col sep=comma, y=val] {figures/io_input/SSBT/WNG_l__h__SSBT__N_32__iSER_n15__Ly_1__err_0.csv};
		
		\addplot [style=resC2]
		table [col sep=comma, y=val] {figures/io_input/SSBT/WNG_l__h__SSBT__N_32__iSER_n15__Ly_16__err_0.csv};
	\end{timeplot}
\end{tikzpicture}
%	\caption{}
%	\label{subfig:1_gain_lineplot}
%\end{subfigure}
		\caption{Per-window broadband WNG.}
		\label{subfig:lineplot__WNG_l__iSER_n15__err_0}
	\end{subfigure}\\[1em]
	\ref*{timeplot_legend}
	\caption{Output metrics for the beamformers over time, in the base scenario.}
	\label{fig:lineplot__iSER_n15__Ly_1__err_0}
\end{figure}


From \cref{subfig:lineplot__DSRF_l__iSER_n15__err_0}, we see that in the case $\Ly = 1$ both filters led to some distortion on the desired signal, given that $\dsrf[l] \neq 0$, with the SSBT filter having marginally better distortion overall. In this situation, averaging over time we have a DSRF of $2.74 \dB$ for the STFT filter, and of $2.55 \dB$ for the SSBT filter. When $\Ly = 16$, both beamformers led to no distortion whatsoever, indicating the correct formulation and modeling of the problem. This is due to the fact that when we consider only the current sample ($\Ly = 1$), previous samples of $X_{1,k}[l]$ interfere in the results, and increasing the number of samples ($\Ly = 16$) reduces this effect drastically.

From the ERLE results in \cref{subfig:lineplot__ERLE_l__iSER_n15__err_0} it is noticeable that the STFT filter outperformed the SSBT beamformer for both number of samples, marginally for $\Ly = 1$ (by about $1.5\dB$) and a larger margin for $\Ly = 16$ (around $5.5 \dB$). The reason is that, to achieve a distortionless behavior, the SSBT beamformer has to enforce twice as many constraints as the STFT (one for the same-frequency and one for the cross-frequency RTFs), therefore having less ``freedom'' for the noise reduction portion of the enhancement. A similar result can be seen in terms of WNG in \cref{subfig:lineplot__WNG_l__iSER_n15__err_0}, where the STFT beamformer again outperformed the SSBT, by some small margin.

%%%%%%%%%%%%%%%%%%%%%%%%%%%%%%%%%%%%%%%%%%%%%%%%%%%%%%%%%%%%%%%%%%%%%%%%%%%%%%%%%%%%%%%

%\subsection{Comparison for different $L_Y$}

\begin{figure}[!ht]
	\centering
	\begin{subfigure}{\textwidth}
		\centering
		%% Requires:
% pgfplots.sty
% edit_pgfplots.tex

\pgfplotsset{compat=1.18}
%\begin{subfigure}{\linewidth}
%\centering
%\tikzsetnextfilename{dsrf_lineplot_32_n15_acc_v7}
\begin{tikzpicture}
	\begin{lyplot}{DSRF (dB)}[ymin=-24, ymax=2, ytick={-24, -16, ..., 0}]
		\addplot [style=resA]
		table [col sep=comma, y=val] {figures/io_input/STFT/DSRF__STFT__N_32__iSER_n15__Ly_var__err_0.csv};
		
		\addplot [style=resC]
		table [col sep=comma, y=val] {figures/io_input/NSSBT/DSRF__NSSBT__N_32__iSER_n15__Ly_var__err_0.csv};
		
		\addplot [style=resE]
		table [col sep=comma, y=val] {figures/io_input/SSBT/DSRF__h____TSSBT__N_32__iSER_n15__Ly_var__err_0.csv};
	\end{lyplot}
\end{tikzpicture}
%	\caption{}
%	\label{subfig:1_gain_lineplot}
%\end{subfigure}
		\caption{Time-average broadband DSRF for varying $\Ly$.}
		\label{subfig:lineplot__DSRF__iSER_n15__Ly_var__err_0}
	\end{subfigure}\\[1em]
	\begin{subfigure}{\textwidth}
		\centering
		%% Requires:
% pgfplots.sty
% edit_pgfplots.tex

\pgfplotsset{compat=1.18}
%\begin{subfigure}{\linewidth}
%\centering
%\tikzsetnextfilename{erle_lineplot_32_n15_acc_v7}
\begin{tikzpicture}
	\begin{lyplot}{ERLE (dB)}[ymin=-28, ymax=37, ytick={-24, -12, ..., 36}]
		\addplot [style=resA]
		table [col sep=comma, y=val] {figures/io_input/STFT/ERLE__STFT__N_32__iSER_n15__Ly_var__err_0.csv};
		
		\addplot [style=resC]
		table [col sep=comma, y=val] {figures/io_input/NSSBT/ERLE__NSSBT__N_32__iSER_n15__Ly_var__err_0.csv};
		
		\addplot [style=resE]
		table [col sep=comma, y=val] {figures/io_input/SSBT/ERLE__h____TSSBT__N_32__iSER_n15__Ly_var__err_0.csv};
	\end{lyplot}
\end{tikzpicture}
%	\caption{}
%	\label{subfig:1_gain_lineplot}
%\end{subfigure}
		\caption{Time-average broadband ERLE for varying $\Ly$.}
		\label{subfig:lineplot__ERLE__iSER_n15__Ly_var__err_0}
	\end{subfigure}\\[1em]
	\begin{subfigure}{\textwidth}
		\centering
		%% Requires:
% pgfplots.sty
% edit_pgfplots.tex

\pgfplotsset{compat=1.18}
%\begin{subfigure}{\linewidth}
%\centering
%\tikzsetnextfilename{nsrf_lineplot_32_n15_acc_v7}
\begin{tikzpicture}
	\begin{lyplot}{WNG (dB)}%[ymin=-37, ymax=1, ytick={-36, -24, ..., 0}]
		\addplot [style=resA1]
		table [col sep=comma, y=val] {figures/io_input/STFT/WNG__h__STFT__N_32__iSER_n15__Ly_var__err_0.csv};
		
		\addplot [style=resC1]
		table [col sep=comma, y=val] {figures/io_input/SSBT/WNG__h__SSBT__N_32__iSER_n15__Ly_var__err_0.csv};
	\end{lyplot}
\end{tikzpicture}
%	\caption{}
%	\label{subfig:1_gain_lineplot}
%\end{subfigure}
		\caption{Time-average broadband WNG for varying $\Ly$.}
		\label{subfig:lineplot__WNG__iSER_n15__Ly_var__err_0}
	\end{subfigure}\\[1em]
	\begin{subfigure}{\textwidth}
		\centering
		%% Requires:
% pgfplots.sty
% edit_pgfplots.tex

\pgfplotsset{compat=1.18}
%\begin{subfigure}{\linewidth}
%\centering
%\tikzsetnextfilename{erle_lineplot_32_n15_acc_v7}
\begin{tikzpicture}
	\begin{lyplot}{DI (dB)}%[ymin=-28, ymax=37, ytick={-24, -12, ..., 36}]
		\addplot [style=resA1]
		table [col sep=comma, y=val] {figures/io_input/STFT/DI__h__STFT__N_32__iSER_n15__Ly_var__err_0.csv};
		
		\addplot [style=resC1]
		table [col sep=comma, y=val] {figures/io_input/SSBT/DI__h__SSBT__N_32__iSER_n15__Ly_var__err_0.csv};
	\end{lyplot}
\end{tikzpicture}
%	\caption{}
%	\label{subfig:1_gain_lineplot}
%\end{subfigure}
		\caption{Time-average broadband DI for varying $\Ly$.}
		\label{subfig:lineplot__DI__iSER_n15__Ly_var__err_0}
	\end{subfigure}\\[1em]
	\ref*{lyplot_legend}
	\caption{Output metrics for the beamformers for varying input $L_Y$'s.}
	\label{fig:lineplot__iSER_n15__Ly_var__err_0}
\end{figure}
%%%%%%%%%%%%%%%%%%%%%%%%%%%%%%%%%%%%%%%%%%%%%%%%%%%%%%%%%%%%%%%%%%%%%%%%%%%%%%

\subsection{Comparison over different frame samples}
\label{subsec:sec4:comparison_different_Ly}

We also compare the two filters for a range of values of $\Ly$. In all plots of \cref{fig:lineplot__iSER_n15__Ly_var__err_0} we can see a behavior similar to that previously achieved. In the DSRF, we see that for both filters the distortion goes to zero as we increase $\Ly$, with the SSBT beamformer being barely better. We also see in \cref{subfig:lineplot__ERLE__iSER_n15__Ly_var__err_0} that in terms of ERLE the STFT beamformer consistently outperformed the SSBT one for all values of $\Ly$, in line with the result from the previous simulation.

On the other hand, in general we see that for small values of $\Ly$ the SSBT beamformer had a better performance in terms of white noise enhancement, although for an increased number of frames the STFT beamformer regains the lead. Overall, this enhancement in the white noise (which could also be seen in \cref{subfig:lineplot__WNG_l__iSER_n15__err_0}) indicates a lack of regularization, and could be solved by tweaking the constraints in the minimization problem of \cref{eq:sec3:minimization_problem_mpdr_hard} to also ensure a minimum WNG. This wasn't done here, since our focus is on the minimization of the undesired signal, which is majorly comprised of $\bvs{k}[l]$.

In terms of directivity index as exposed in \cref{subfig:lineplot__DI__iSER_n15__Ly_var__err_0}, it is clear that the STFT filter had a better DI than the SSBT filter overall, by about $3 \dB$ on average. This can also be attributed to the SSBT beamformer having two constraints to achieve the non-distortionality, while the STFT only had one. We can also see that the DI didn't increase meaningfully as we increased the number of frames considered, differently to the ERLE and even the WNG.
%%%%%%%%%%%%%%%%%%%%%%%%%%%%%%%%%%%%%%%%%%%%%%%%%%%%%%%%%%%%%%%%%%%%%%%%%%%%%%

\subsection{Comparison over different input SERs}

We now examine the results with a varying input SERs, to assess the beamformer's performances for different loudspeaker signal levels. The results will be again presented for $\Ly = 1$ and $\Ly = 16$, representing two contrasting situations in terms of distortionless-ness and performance for the beamformers.

\def\meshcols{316}
\def\meshrows{15}
\def\tmin{0}
\def\tmax{8.3725}
\def\fmin{0.25}
\def\fmax{7.75}
\begin{figure}[!t]
	\centering
	\begin{subfigure}{\textwidth}
		\centering
		%% Requires:
% pgfplots.sty
% edit_pgfplots.tex

\pgfplotsset{compat=1.18}
%\begin{subfigure}{\linewidth}
%\centering
%\tikzsetnextfilename{dsrf_lineplot_32_var_acc_v7}
\begin{tikzpicture}
	\begin{snrplot}{DSRF (dB)}%[ymin=-3, ymax=0.03, ytick={-3, -2, -1, 0}]
		\addplot [style=resA1]
		table [col sep=comma, y=val] {figures/io_input/STFT/DSRF__h__STFT__N_32__iSER_var__Ly_1__err_0.csv};
		
		\addplot [style=resA2]
		table [col sep=comma, y=val] {figures/io_input/STFT/DSRF__h__STFT__N_32__iSER_var__Ly_15__err_0.csv};
		
		\addplot [style=resC1]
		table [col sep=comma, y=val] {figures/io_input/SSBT/DSRF__h__SSBT__N_32__iSER_var__Ly_1__err_0.csv};
		
		\addplot [style=resC2]
		table [col sep=comma, y=val] {figures/io_input/SSBT/DSRF__h__SSBT__N_32__iSER_var__Ly_15__err_0.csv};
	\end{snrplot}
\end{tikzpicture}
%	\caption{}
%	\label{subfig:1_gain_lineplot}
%\end{subfigure}
		\caption{Time-average broadband DSRF for varying $\iser$.}
		\label{subfig:lineplot__DSRF__iSER_var__err_0}
	\end{subfigure}\\[1em]
	\begin{subfigure}{\textwidth}
		\centering
		%% Requires:
% pgfplots.sty
% edit_pgfplots.tex

\pgfplotsset{compat=1.18}
%\begin{subfigure}{\linewidth}
%\centering
%\tikzsetnextfilename{erle_lineplot_32_var_acc_v7}
\begin{tikzpicture}
	\begin{snrplot}{ERLE (dB)}%[ymin=10, ymax=30, ytick={12, 18, 24, 30}]
		\addplot [style=resA1]
		table [col sep=comma, y=val] {figures/io_input/STFT/ERLE__h__STFT__N_32__iSER_var__Ly_1__err_0.csv};
		
		\addplot [style=resA2]
		table [col sep=comma, y=val] {figures/io_input/STFT/ERLE__h__STFT__N_32__iSER_var__Ly_16__err_0.csv};
		
		\addplot [style=resC1]
		table [col sep=comma, y=val] {figures/io_input/SSBT/ERLE__h__SSBT__N_32__iSER_var__Ly_1__err_0.csv};
		
		\addplot [style=resC2]
		table [col sep=comma, y=val] {figures/io_input/SSBT/ERLE__h__SSBT__N_32__iSER_var__Ly_16__err_0.csv};
	\end{snrplot}
\end{tikzpicture}
%	\caption{}
%	\label{subfig:1_gain_lineplot}
%\end{subfigure}
		\caption{Time-average broadband ERLE for varying $\iser$.}
		\label{subfig:lineplot__ERLE__iSER_var__err_0}
	\end{subfigure}\\[1em]
	\begin{subfigure}{\textwidth}
		\centering
		%% Requires:
% pgfplots.sty
% edit_pgfplots.tex

\pgfplotsset{compat=1.18}
%\begin{subfigure}{\linewidth}
%\centering
%\tikzsetnextfilename{nsrf_lineplot_32_var_acc_v7}
\begin{tikzpicture}
	\begin{snrplot}{WNG (dB)}%[ymin=-13, ymax=-3, ytick={-12, -9, ..., -3}]
		\addplot [style=resA1]
		table [col sep=comma, y=val] {figures/io_input/STFT/WNG__h__STFT__N_32__iSER_var__Ly_1__err_0.csv};
		
		\addplot [style=resA2]
		table [col sep=comma, y=val] {figures/io_input/STFT/WNG__h__STFT__N_32__iSER_var__Ly_16__err_0.csv};
		
		\addplot [style=resC1]
		table [col sep=comma, y=val] {figures/io_input/SSBT/WNG__h__SSBT__N_32__iSER_var__Ly_1__err_0.csv};
		
		\addplot [style=resC2]
		table [col sep=comma, y=val] {figures/io_input/SSBT/WNG__h__SSBT__N_32__iSER_var__Ly_16__err_0.csv};
	\end{snrplot}
\end{tikzpicture}
%	\caption{}
%	\label{subfig:1_gain_lineplot}
%\end{subfigure}
		\caption{Time-average broadband WNG for varying $\iser$.}
		\label{subfig:lineplot__WNG__iSER_var__err_0}
	\end{subfigure}\\[1em]
	\begin{subfigure}{\textwidth}
		\centering
		%% Requires:
% pgfplots.sty
% edit_pgfplots.tex

\pgfplotsset{compat=1.18}
%\begin{subfigure}{\linewidth}
%\centering
%\tikzsetnextfilename{erle_lineplot_32_var_acc_v7}
\begin{tikzpicture}
	\begin{snrplot}{DI (dB)}%[ymin=10, ymax=30, ytick={12, 18, 24, 30}]
		\addplot [style=resA1]
		table [col sep=comma, y=val] {figures/io_input/STFT/DI__h__STFT__N_32__iSER_var__Ly_1__err_0.csv};
		
		\addplot [style=resA2]
		table [col sep=comma, y=val] {figures/io_input/STFT/DI__h__STFT__N_32__iSER_var__Ly_16__err_0.csv};
		
		\addplot [style=resC1]
		table [col sep=comma, y=val] {figures/io_input/SSBT/DI__h__SSBT__N_32__iSER_var__Ly_1__err_0.csv};
		
		\addplot [style=resC2]
		table [col sep=comma, y=val] {figures/io_input/SSBT/DI__h__SSBT__N_32__iSER_var__Ly_16__err_0.csv};
	\end{snrplot}
\end{tikzpicture}
%	\caption{}
%	\label{subfig:1_gain_lineplot}
%\end{subfigure}
		\caption{Time-average broadband DI for varying $\iser$.}
		\label{subfig:lineplot__DI__iSER_var__err_0}
	\end{subfigure}\\[1em]
	\ref*{timeplot_legend}
	\caption{Output metrics for the beamformers for varying input SERs.}
	\label{fig:lineplot__iSER_var__Ly_1__err_0}
\end{figure}

From \cref{subfig:lineplot__DSRF__iSER_var__err_0}, we see that for $\Ly = 1$ the distortion on the desired signal increases as the input SER increases, and that the distortion is null for $\Ly = 16$, which is in line with the results previously obtained, as well as with the explanations found. Also, we see that over the studied range the SSBT filter has less distortion than the STFT one, except when $\iser = -30\dB$.

For the ERLE, we see that it decreases as the input SER increases, which is again on par with what would be expected. However, we see that for higher input SERs the performance gap between the two filters diminishes, leading to practically identical results. We also see that with $\Ly = 1$ and higher input SERs, the SSBT filter has a better performance than the STFT one.

In terms of WNG, for $\Ly = 1$ the STFT filter is mostly better (except for lower input SERs). For $\Ly = 16$, both have an overall similar result, also converging for higher input SERs. We also see that, for higher input SERs, both STFT filters with $\Ly = 1$ and $\Ly = 16$ also converge, leading to a similar white noise enhancement. In the DI we again see that the STFT filter led to better performances, for both $\Ly$ values. With the STFT transform, we see that the multiplicative filter led to better results for lower iSER and the convolutive one for higher iSER, the opposite being true for the SSBT filter.

%%%%%%%%%%%%%%%%%%%%%%%%%%%%%%%%%%%%%%%%%%%%%%%%%%%%%%%%%%%%%%%%%%%%%%%%%%%%%%

\subsection{Comparison over different number of bins}

Previously on \cref{subsec:sec4:comparison_different_Ly} we compared the filters for different $\Ly$, increasing the number of samples considered for each frequency bin. Now, we will compare the filters for different numbers of frequency bins, going from $N = 32$ to $N = 512$, for $\Ly = 1$ only. That is, we will consider only a multiplicative filter, instead of a convolutive one, for an increasing number of time samples used for the transforms.

When working with $N = 32$ and $\Ly = 16$, we are effectively using information regarding the last $272$ time samples, considering the $50\%$ overlap. Therefore, it is also interesting to compare these result to those obtained via varying $\Ly$, which will fare a multiplicative filter ($\Ly = 1$ and varying $N$) and a convolutive one (in this case with $\Ly = 16$ and $N = 32$). On each plot from \cref{fig:lineplot__N_var__iSER_n15__Ly_1__err_0}, the best performance from \cref{fig:lineplot__iSER_n15__Ly_var__err_0} for each filter will be denoted as a dashed gray line for the STFT and a dotted light gray line for the SSBT.

\def\meshcols{316}
\def\meshrows{15}
\def\tmin{0}
\def\tmax{8.3725}
\def\fmin{0.25}
\def\fmax{7.75}
\begin{figure}[!t]
	\centering
	\begin{subfigure}{\textwidth}
		\centering
		%% Requires:
% pgfplots.sty
% edit_pgfplots.tex

\pgfplotsset{compat=1.18}
%\begin{subfigure}{\linewidth}
%\centering
%\tikzsetnextfilename{dsrf_lineplot_32_var_acc_v7}
\begin{tikzpicture}
	\begin{nbinsplot}{DSRF (dB)}%[ymin=-3, ymax=0.03, ytick={-3, -2, -1, 0}]
		\addplot [style=resA1]
		table [col sep=comma, y=val] {figures/io_input/STFT/DSRF__h__STFT__N_var__iSER_n15__Ly_1__err_0.csv};
		
		\addplot [style=resC1]
		table [col sep=comma, y=val] {figures/io_input/SSBT/DSRF__h__SSBT__N_var__iSER_n15__Ly_1__err_0.csv};
	\end{nbinsplot}
\end{tikzpicture}
%	\caption{}
%	\label{subfig:1_gain_lineplot}
%\end{subfigure}
		\caption{Time-average broadband DSRF for varying $N$.}
		\label{subfig:lineplot__DSRF__N_var__iSER_n15}
	\end{subfigure}\\[1em]
	\begin{subfigure}{\textwidth}
		\centering
		%% Requires:
% pgfplots.sty
% edit_pgfplots.tex

\pgfplotsset{compat=1.18}
%\begin{subfigure}{\linewidth}
%\centering
%\tikzsetnextfilename{dsrf_lineplot_32_var_acc_v7}
\begin{tikzpicture}
	\begin{nbinsplot}{ERLE (dB)}%[ymin=-3, ymax=0.03, ytick={-3, -2, -1, 0}]
		\addplot [style=resA1]
		table [col sep=comma, y=val] {figures/io_input/STFT/ERLE__h__STFT__N_var__iSER_n15__Ly_1__err_0.csv};
		
		\addplot [style=resC1]
		table [col sep=comma, y=val] {figures/io_input/SSBT/ERLE__h__SSBT__N_var__iSER_n15__Ly_1__err_0.csv};
		
		\addplot [mark=none, gray, dash pattern={on 3pt off 3pt}] coordinates {(32, 39.661) (512, 39.661)};
		
		\addplot [mark=none, lightgray, dash pattern={on 0.5pt off 1pt}] coordinates {(32, 34.3742) (512, 34.3742)};
	\end{nbinsplot}
\end{tikzpicture}
%	\caption{}
%	\label{subfig:1_gain_lineplot}
%\end{subfigure}
		\caption{Time-average broadband ERLE for varying $N$.}
		\label{subfig:lineplot__ERLE__N_var__iSER_n15}
	\end{subfigure}\\[1em]
	\begin{subfigure}{\textwidth}
		\centering
		%% Requires:
% pgfplots.sty
% edit_pgfplots.tex

\pgfplotsset{compat=1.18}
%\begin{subfigure}{\linewidth}
%\centering
%\tikzsetnextfilename{dsrf_lineplot_32_var_acc_v7}
\begin{tikzpicture}
	\begin{nbinsplot}{WNG (dB)}%[ymin=-3, ymax=0.03, ytick={-3, -2, -1, 0}]
		\addplot [style=resA1]
		table [col sep=comma, y=val] {figures/io_input/STFT/WNG__h__STFT__N_var__iSER_n15__Ly_1__err_0.csv};
		
		\addplot [style=resC1]
		table [col sep=comma, y=val] {figures/io_input/SSBT/WNG__h__SSBT__N_var__iSER_n15__Ly_1__err_0.csv};
		
		\addplot [mark=none, gray, dash pattern={on 3pt off 3pt}] coordinates {(32, -5.827) (512, -5.827)};
		
		\addplot [mark=none, lightgray, dash pattern={on 0.5pt off 1pt}] coordinates {(32, -6.026) (512, -6.026)};
	\end{nbinsplot}
\end{tikzpicture}
%	\caption{}
%	\label{subfig:1_gain_lineplot}
%\end{subfigure}
		\caption{Time-average broadband WNG for varying $N$.}
		\label{subfig:lineplot__WNG__N_var__iSER_n15}
	\end{subfigure}\\[1em]
	\begin{subfigure}{\textwidth}
		\centering
		%% Requires:
% pgfplots.sty
% edit_pgfplots.tex

\pgfplotsset{compat=1.18}
%\begin{subfigure}{\linewidth}
%\centering
%\tikzsetnextfilename{dsrf_lineplot_32_var_acc_v7}
\begin{tikzpicture}
	\begin{nbinsplot}{DI (dB)}%[ymin=-3, ymax=0.03, ytick={-3, -2, -1, 0}]
		\addplot [style=resA1]
		table [col sep=comma, y=val] {figures/io_input/STFT/DI__h__STFT__N_var__iSER_n15__Ly_1__err_0.csv};
		
		\addplot [style=resC1]
		table [col sep=comma, y=val] {figures/io_input/SSBT/DI__h__SSBT__N_var__iSER_n15__Ly_1__err_0.csv};
		
		\addplot [mark=none, gray, dash pattern={on 3pt off 3pt}] coordinates {(32, 15.8249) (512, 15.8249)};
%		
		\addplot [mark=none, lightgray, dash pattern={on 0.5pt off 1pt}] coordinates {(32, 13.0609) (512, 13.0609)};
	\end{nbinsplot}
\end{tikzpicture}
%	\caption{}
%	\label{subfig:1_gain_lineplot}
%\end{subfigure}
		\caption{Time-average broadband DI for varying $N$.}
		\label{subfig:lineplot__DI__N_var__iSER_n15}
	\end{subfigure}\\[1em]
	\ref*{lyplot_legend}
	\caption{Output metrics for the beamformers for varying number of bins.}
	\label{fig:lineplot__N_var__iSER_n15__Ly_1__err_0}
\end{figure}

Overall, we see a similar behavior as previously obtained. From \cref{subfig:lineplot__DSRF__N_var__iSER_n15}, we see that neither filter was able to achieve a truly null distortion, although both got considerably close to it, with the SSBT filter having a small margin over the STFT one. Regarding the ERLE, the STFT filter had the best performance out of the two of them, however we see that while the SSBT beamformer ultimately achieved the same performance by increasing the number of bins as was achieved by increasing the number of frames (although with double the amount of time samples), this wasn't the case for the STFT filter, being worse by about $3\dB$ compared to its convolutive counterpart.

The same can be seen regarding the WNG, where although the SSBT filter had less white noise enhancement than the STFT one, neither of them achieved the same performance as their convolutive equivalents. In terms of the DI, in \cref{subfig:lineplot__DI__iSER_n15__Ly_var__err_0} we see that the filters' performances kept close to their peak performance, while now (as in \cref{subfig:lineplot__DI__N_var__iSER_n15}) the filters' performances are generally much worse than their peaks, and also than the best performance when varying $\Ly$.
%%%%%%%%%%%%%%%%%%%%%%%%%%%%%%%%%%%%%%%%%%%%%%%%%%%%%%%%%%%%%%%%%%%%%%%%%%%%%%%%%%%%%%%%%%%%%%%%%%%%%%%%%%%%%%%%%%%%%%%%%%%%%%%%%%%%%%%%%%%%%
%\subsection{Comparison with perturbation}
%
%\begin{figure}[!t]
%	\centering
%	\begin{subfigure}{\textwidth}
	%		\centering
	%		%% Requires:
% pgfplots.sty
% edit_pgfplots.tex

\pgfplotsset{compat=1.18}
%\begin{subfigure}{\linewidth}
%\centering
%\tikzsetnextfilename{dsrf_lineplot_32_var_acc_v7}
\begin{tikzpicture}
	\begin{errplot}{DSRF (dB)}%[ymin=-0.09, ymax=0.01, ytick={-0.09, -0.06, ..., 0}]
		\addplot [style=resA]
		table [col sep=comma, y=val] {figures/io_input/STFT/v7__DSRF__STFT__N_32__iSER_n15__Ly_1__err_var.csv};
		
		\addplot [style=resC]
		table [col sep=comma, y=val] {figures/io_input/NSSBT/v7__DSRF__NSSBT__N_32__iSER_n15__Ly_1__err_var.csv};
		
		\addplot [style=resE]
		table [col sep=comma, y=val] {figures/io_input/TSSBT/v7__DSRF__TSSBT__N_32__iSER_n15__Ly_1__err_var.csv};
	\end{errplot}
\end{tikzpicture}
%	\caption{}
%	\label{subfig:1_gain_lineplot}
%\end{subfigure}
	%		\caption{Per-window broadband DSRF.}
	%		\label{subfig:lineplot__DSRF__iSER_n15__Ly_1__err_var}
	%	\end{subfigure}\\[1em]
%	\begin{subfigure}{\textwidth}
	%		\centering
	%		%% Requires:
% pgfplots.sty
% edit_pgfplots.tex

\pgfplotsset{compat=1.18}
%\begin{subfigure}{\linewidth}
%\centering
%\tikzsetnextfilename{erle_lineplot_32_var_acc_v7}
\begin{tikzpicture}
	\begin{errplot}{gSER (dB)}[ymin=6, ymax=30, ytick={6, 12, ..., 30}]
		\addplot [style=resA]
		table [col sep=comma, y=val] {figures/io_input/STFT/v7__gSER__STFT__N_32__iSER_n15__Ly_1__err_var.csv};
		
		\addplot [style=resC]
		table [col sep=comma, y=val] {figures/io_input/NSSBT/v7__gSER__NSSBT__N_32__iSER_n15__Ly_1__err_var.csv};
		
		\addplot [style=resE]
		table [col sep=comma, y=val] {figures/io_input/TSSBT/v7__gSER__TSSBT__N_32__iSER_n15__Ly_1__err_var.csv};
	\end{errplot}
\end{tikzpicture}
%	\caption{}
%	\label{subfig:1_gain_lineplot}
%\end{subfigure}
	%		\caption{Per-window broadband ERLE.}
	%		\label{subfig:lineplot__ERLE__iSER_n15__Ly_1__err_var}
	%	\end{subfigure}\\[1em]
%	\begin{subfigure}{\textwidth}
	%		\centering
	%		%% Requires:
% pgfplots.sty
% edit_pgfplots.tex

\pgfplotsset{compat=1.18}
%\begin{subfigure}{\linewidth}
%\centering
%\tikzsetnextfilename{nsrf_lineplot_32_var_acc_v7}
\begin{tikzpicture}
	\begin{errplot}{gSNR (dB)}[ymin=-30, ymax=5, ytick={-30, -20, ..., 0}]
		\addplot [style=resA]
		table [col sep=comma, y=val] {figures/io_input/STFT/v7__gSNR__STFT__N_32__iSER_n15__Ly_1__err_var.csv};
		
		\addplot [style=resC]
		table [col sep=comma, y=val] {figures/io_input/NSSBT/v7__gSNR__NSSBT__N_32__iSER_n15__Ly_1__err_var.csv};
		
		\addplot [style=resE]
		table [col sep=comma, y=val] {figures/io_input/TSSBT/v7__gSNR__TSSBT__N_32__iSER_n15__Ly_1__err_var.csv};
	\end{errplot}
\end{tikzpicture}
%	\caption{}
%	\label{subfig:1_gain_lineplot}
%\end{subfigure}
	%		\caption{Per-window broadband WNG.}
	%		\label{subfig:lineplot__WNG__iSER_n15__Ly_1__err_var}
	%	\end{subfigure}\\[1em]
%	\ref*{timeplot_legend}
%	\caption{Output metrics for the beamformers with error in the steering vectors.}
%	\label{fig:lineplot__v7_iSER_n15__Ly_1__err_var}
%\end{figure}
%
%As exposed in \cref{sec:perturbation_analysis}, it is also of interest to compare how robust the derived beamformers are, when the information regarding the desired signal's RIR isn't accurate. For such, we model the matrix $\bvA[u]{k}$ as
%\begin{equation}
%	\bvA[u]{k} = \bvA[u]{k}^\star + \Delta\bvA[u]{k}
%\end{equation}
%where $\bvA[u]{k}^\star$ is the accurate steering matrix, and $\Delta\bvA[u]{k}$ is a perturbation on it, which we assume is a zero-mean uniform white noise, with an adjustable variance.
%
%Since in this scenario the desired signal can suffer some distortion (given that its steering matrix isn't appropriately estimated), we will use the gain in SER and gain in SNR metrics instead of ERLE and NSRF, to take this distortion into account. These are defined as
%\begin{subgather}
%	\gser = \frac{\erle}{\dsrf} \\
%	\gsnr = \frac{\nsrf}{\dsrf}
%\end{subgather}
%
%The DSRF will still be showed, to give a sense of proportion on how much the beamformer distorts the desired signal. In the results of \cref{fig:lineplot__v7_iSER_n15__Ly_1__err_var}, the x-axis represents the standard deviation of $\Delta\bvA[u]{k}$, as a percentage of the standard deviation of $\bvA[u]{k}^\star$.
%
%Each metric showed a different result: differently than before, the SF-SSBT beamformer led to the least distortion on the desired signal, out of all three beamformers, and the DF-SSBT led to the most distortion. Meanwhile, the gain in SER showed the STFT beamformer to be the (overall) more robust, and the one that led to the best results, with the DF-SSBT again being the worse one. A similar result can be seen for the gain in SNR, with the STFT beamformer being the best, but in this regard the SF-SSBT beamformer led to the worst results, although marginally.