\definecolor{ColA}{HTML}{991F3D}
\definecolor{ColB}{HTML}{997A1F}
\definecolor{ColC}{HTML}{3D991F}
\definecolor{ColD}{HTML}{1F997A}
\definecolor{ColE}{HTML}{1F3D99}
\definecolor{ColF}{HTML}{7A1F99}
\NewDocumentCommand{\filename}{m}{%
	\small{\texttt{#1}}
}
\let\mc\multicolumn
\section{Comparisons and simulations}
\label{sec:results}

In the simulations\footnote{Code is available at \url{https://github.com/VCurtarelli/py-ssb-ctf-bf}.}, we employ a sampling frequency of $16\si{\kilo\hertz}$. Room impulse responses were generated using Habets' RIR generator \cite{habets_rir-generator}, and signals were selected from the SMARD database \cite{smard_database}. In all cases we will use $\Ld = \Ly$ and $\Delta = 0$; that is, we don't use any non-causal frames, and we consider as many frame samples as RIR samples.

The room's dimensions are $4\m \times 6\m \times 3\m$ (width $\times$ length $\times$ height), with a reverberation time of $0.3\si{\second}$. The device composed of the loudspeaker + sensors is centered at $(3\m,~4\m,~1\m)$, being comprised of $M=8$ sensors. They are arranged in a circular array with radius of $8\si{\centi\meter}$, and all are omnidirectional of flat frequency response. The positions and signals used for the sources are in \cref{tab:sec4:information_position_sources}. The room's layout is in \cref{fig:room_layout}, where in green we have the desired source (assumed to be omnidirectional), and in red the device, with the $8$ sensors and the loudspeaker on the center.

\begin{table}[H]
	\centering
	\begin{tabular}{rll}
		Source & Position 				& Signal \\
		\hline\vphantom{$\tilde{d}$}
		$x[n]$ & $(2\m,~1\m,~1.8\m)$ 	& \filename{50\_male\_speech\_english\_ch8\_OmniPower4296.flac} \\
		$s[n]$ & $(3\m,~2\m,~1\m)$ 		& \filename{69\_abba\_ch8\_OmniPower4296.flac} \\
		$r[n]$ & \mc{1}{c}{$\sim$}		& \filename{wgn\_48kHz\_ch8\_OmniPower4296.flac}
	\end{tabular}
	\caption{Source information for the simulations.}
	\label{tab:sec4:information_position_sources}
\end{table}\vspace*{-2em}

\begin{figure}[!t]
	\centering
	\includesvg[width=0.7\linewidth]{figures/drawing/room_layout.svg}
	\caption{Room layout for simulations.}
	\label{fig:room_layout}
\end{figure}

All signals were resampled to the desired sampling frequency of $16\si{\kilo\hertz}$. For the transforms, Hamming windows were used, with a length of 32 samples/window and an overlap of $50\%$. The beamformers were calculated once for the whole signal, for faster processing and ease to compare the results. We will compare the two transforms exposed, the STFT and the SSBT.

In line plots, STFT is presented in red and SSBT in green lines. The output metrics were averaged over 200 frames and presented every 100 windows, for a better visualization.
%

\subsection{Basic comparison}
At the reference sensor (assumed to be the one at $(3\m,~1.92\m,~1\m)$), the SNR for the loudspeaker's and noise signals are respectively $-15\dB$ and $30\dB$. These will be referred as Signal-to-Echo and Signal-to-Noise Ratios (SER and SNR) in that order. We will use $\Ly = 1$ and $\Ly = 15$ to illustrate two different cases in terms of samples used.

\def\meshcols{316}
\def\meshrows{15}
\def\tmin{0}
\def\tmax{8.3725}
\def\fmin{0.25}
\def\fmax{7.75}
\begin{figure}[!ht]
	\centering
	\begin{subfigure}{\textwidth}
		\centering
		%% Requires:
% pgfplots.sty
% edit_pgfplots.tex

\pgfplotsset{compat=1.18}
%\begin{subfigure}{\linewidth}
%\centering
%\tikzsetnextfilename{dsrf_lineplot_32_n15_acc_v7}
\begin{tikzpicture}
	\begin{timeplot}{DSRF (dB)}[%ymin=-2.5, ymax=4, ytick={-2.5, 0, 2.5}, 
		legend to name = {timeplot_legend}]
		\addplot [style=resA1]
		table [col sep=comma, y=val] {figures/io_input/STFT/DSRF_l__h__STFT__N_32__iSER_n15__Ly_1__err_0.csv};
		
		\addplot [style=resA2]
		table [col sep=comma, y=val] {figures/io_input/STFT/DSRF_l__h__STFT__N_32__iSER_n15__Ly_15__err_0.csv};
		
		\addplot [style=resC1]
		table [col sep=comma, y=val] {figures/io_input/SSBT/DSRF_l__h__SSBT__N_32__iSER_n15__Ly_1__err_0.csv};
		
		\addplot [style=resC2]
		table [col sep=comma, y=val] {figures/io_input/SSBT/DSRF_l__h__SSBT__N_32__iSER_n15__Ly_15__err_0.csv};
		
		\addlegendentry{STFT - $\Ly = 1$};
		\addlegendentry{STFT - $\Ly = 15$};
		\addlegendentry{SSBT - $\Ly = 1$};
		\addlegendentry{SSBT - $\Ly = 15$};
%		\addlegendentry{S-SSBT};
	\end{timeplot}
\end{tikzpicture}
%	\caption{}
%	\label{subfig:1_gain_lineplot}
%\end{subfigure}
		\caption{Per-window broadband DSRF.}
		\label{subfig:lineplot__DSRF_l__iSER_n15__err_0}
	\end{subfigure}\\[1em]
	\begin{subfigure}{\textwidth}
		\centering
		%% Requires:
% pgfplots.sty
% edit_pgfplots.tex

\pgfplotsset{compat=1.18}
%\begin{subfigure}{\linewidth}
%\centering
%\tikzsetnextfilename{erle_lineplot_32_n15_acc_v7}
\begin{tikzpicture}
	\begin{timeplot}{gSER (dB)}%[ymin=18, ymax=33, ytick={18, 24, 30}]
		\addplot [style=resA1]
		table [col sep=comma, y=val] {figures/io_input/STFT/gSER_l__h__STFT__N_32__iSER_n15__Ly_1__err_0.csv};
		
		\addplot [style=resA2]
		table [col sep=comma, y=val] {figures/io_input/STFT/gSER_l__h__STFT__N_32__iSER_n15__Ly_16__err_0.csv};
		
		\addplot [style=resC1]
		table [col sep=comma, y=val] {figures/io_input/SSBT/gSER_l__h__SSBT__N_32__iSER_n15__Ly_1__err_0.csv};
		
		\addplot [style=resC2]
		table [col sep=comma, y=val] {figures/io_input/SSBT/gSER_l__h__SSBT__N_32__iSER_n15__Ly_16__err_0.csv};
	\end{timeplot}
\end{tikzpicture}
%	\caption{}
%	\label{subfig:1_gain_lineplot}
%\end{subfigure}
		\caption{Per-window broadband ERLE.}
		\label{subfig:lineplot__gSER_l__iSER_n15__err_0}
	\end{subfigure}\\[1em]
	\begin{subfigure}{\textwidth}
		\centering
		%% Requires:
% pgfplots.sty
% edit_pgfplots.tex

\pgfplotsset{compat=1.18}
%\begin{subfigure}{\linewidth}
%\centering
%\tikzsetnextfilename{nsrf_lineplot_32_n15_acc_v7}
\begin{tikzpicture}
	\begin{timeplot}{gSNR (dB)}%[ymin=-15, ymax=-2]
		\addplot [style=resA1]
		table [col sep=comma, y=val] {figures/io_input/STFT/gSNR_l__h__STFT__N_32__iSER_n15__Ly_1__err_0.csv};
		
		\addplot [style=resA2]
		table [col sep=comma, y=val] {figures/io_input/STFT/gSNR_l__h__STFT__N_32__iSER_n15__Ly_15__err_0.csv};
		
		\addplot [style=resC1]
		table [col sep=comma, y=val] {figures/io_input/SSBT/gSNR_l__h__SSBT__N_32__iSER_n15__Ly_1__err_0.csv};
		
		\addplot [style=resC2]
		table [col sep=comma, y=val] {figures/io_input/SSBT/gSNR_l__h__SSBT__N_32__iSER_n15__Ly_15__err_0.csv};
	\end{timeplot}
\end{tikzpicture}
%	\caption{}
%	\label{subfig:1_gain_lineplot}
%\end{subfigure}
		\caption{Per-window broadband NSRF.}
		\label{subfig:lineplot__gSNR_l__iSER_n15__err_0}
	\end{subfigure}\\[1em]
	\ref*{timeplot_legend}
	\caption{Output metrics for the beamformers over time, in the base scenario.}
	\label{fig:lineplot__iSER_n15__Ly_1__err_0}
\end{figure}

In these simulations, we are interested in three metrics results: maintenance/no-distortion of the desired signal; decrease in the loudspeaker's signal; and reduction/minimal enhancement of the white noise. In order, these will be measured by the desired signal reduction factor (DSRF), SER gain (gSER), and SNR gain (gSNR), the later being used for the white noise given that the only other undesired signal at the sensors is white uncorrelated. Their time-dependent broadband formulations are, respectively,
\begin{subgather}
	\dsrf[l] = \frac{\sum_{k}\abs{X_{1,k}[l]}^2}{\sum_{k}\abs{X_{f,k}[l]}^2} \\
	\gser[l] = \frac{\sum_{k}\abs{S_{1,k}[l]}^2}{\sum_{k}\abs{S_{f,k}[l]}^2} \cdot \frac{1}{\dsrf[l]} \\
	\gsnr[l] = \frac{\sum_{k}\abs{V_{1,k}[l]}^2}{\sum_{k}\abs{V_{f,k}[l]}^2} \cdot \frac{1}{\dsrf[l]}
\end{subgather}

From \cref{subfig:lineplot__DSRF_l__iSER_n15__err_0}, we see that in the case $\Ly = 1$ both filters led to some distortion on the desired signal, given that their DSRF isn't zero. Also, with $\Ly = 10$ the STFT beamformer caused no distortion, while the SSBT beamformer led to some (although much less) distortion. This is due to the fact that when we consider only the current sample ($\Ly = 1$), previous samples of $X_{1,k}[l]$ interfere in the results, and increasing the number of samples ($\Ly = 10$) reduces this effect drastically.

From the gSER results in \cref{subfig:lineplot__gSER_l__iSER_n15__err_0} it is noticeable that the STFT filter outperformed the SSBT beamformer for both number of samples, with a small margin for $\Ly = 1$ and a large margin for $\Ly = 10$. This effect can be explained by the fact that, to achieve a distortionless behavior, the SSBT beamformer has to enforce twice as many constraints as the STFT, therefore leaving less ``space'' for the noise reduction portion of the enhancement. A similar result can be seen in terms of gain in SNR in \cref{subfig:lineplot__gSNR_l__iSER_n15__err_0}, with the SSBT filter being strictly outperformed by the STFT one.

%%%%%%%%%%%%%%%%%%%%%%%%%%%%%%%%%%%%%%%%%%%%%%%%%%%%%%%%%%%%%%%%%%%%%%%%%%%%%%%%%%%%%%%

%\subsection{Comparison for different $L_Y$}

\begin{figure}[!ht]
	\centering
	\begin{subfigure}{\textwidth}
			\centering
			%% Requires:
% pgfplots.sty
% edit_pgfplots.tex

\pgfplotsset{compat=1.18}
%\begin{subfigure}{\linewidth}
%\centering
%\tikzsetnextfilename{dsrf_lineplot_32_n15_acc_v7}
\begin{tikzpicture}
	\begin{lyplot}{DSRF (dB)}[ymin=-24, ymax=2, ytick={-24, -16, ..., 0}]
		\addplot [style=resA]
		table [col sep=comma, y=val] {figures/io_input/STFT/DSRF__STFT__N_32__iSER_n15__Ly_var__err_0.csv};
		
		\addplot [style=resC]
		table [col sep=comma, y=val] {figures/io_input/NSSBT/DSRF__NSSBT__N_32__iSER_n15__Ly_var__err_0.csv};
		
		\addplot [style=resE]
		table [col sep=comma, y=val] {figures/io_input/SSBT/DSRF__h____TSSBT__N_32__iSER_n15__Ly_var__err_0.csv};
	\end{lyplot}
\end{tikzpicture}
%	\caption{}
%	\label{subfig:1_gain_lineplot}
%\end{subfigure}
			\caption{Per-window broadband DSRF.}
			\label{subfig:lineplot__DSRF__iSER_n15__Ly_var__err_0}
		\end{subfigure}\\[1em]
	\begin{subfigure}{\textwidth}
			\centering
			%% Requires:
% pgfplots.sty
% edit_pgfplots.tex

\pgfplotsset{compat=1.18}
%\begin{subfigure}{\linewidth}
%\centering
%\tikzsetnextfilename{erle_lineplot_32_n15_acc_v7}
\begin{tikzpicture}
	\begin{lyplot}{gSER (dB)}%[ymin=-28, ymax=37, ytick={-24, -12, ..., 36}]
		\addplot [style=resA1]
		table [col sep=comma, y=val] {figures/io_input/STFT/gSER__h__STFT__N_32__iSER_n15__Ly_var__err_0.csv};
		
		\addplot [style=resC1]
		table [col sep=comma, y=val] {figures/io_input/SSBT/gSER__h__SSBT__N_32__iSER_n15__Ly_var__err_0.csv};
	\end{lyplot}
\end{tikzpicture}
%	\caption{}
%	\label{subfig:1_gain_lineplot}
%\end{subfigure}
			\caption{Per-window broadband ERLE.}
			\label{subfig:lineplot__gSER__iSER_n15__Ly_var__err_0}
		\end{subfigure}\\[1em]
	\begin{subfigure}{\textwidth}
			\centering
			%% Requires:
% pgfplots.sty
% edit_pgfplots.tex

\pgfplotsset{compat=1.18}
%\begin{subfigure}{\linewidth}
%\centering
%\tikzsetnextfilename{nsrf_lineplot_32_n15_acc_v7}
\begin{tikzpicture}
	\begin{lyplot}{gSNR (dB)}%[ymin=-37, ymax=1, ytick={-36, -24, ..., 0}]
		\addplot [style=resA1]
		table [col sep=comma, y=val] {figures/io_input/STFT/gSNR__h__STFT__N_32__iSER_n15__Ly_var__err_0.csv};
		
		\addplot [style=resC1]
		table [col sep=comma, y=val] {figures/io_input/SSBT/gSNR__h__SSBT__N_32__iSER_n15__Ly_var__err_0.csv};
	\end{lyplot}
\end{tikzpicture}
%	\caption{}
%	\label{subfig:1_gain_lineplot}
%\end{subfigure}
			\caption{Per-window broadband NSRF.}
			\label{subfig:lineplot__gSNR__iSER_n15__Ly_var__err_0}
		\end{subfigure}\\[1em]
	\ref*{lyplot_legend}
	\caption{Output metrics for the beamformers for varying input $L_Y$'s.}
	\label{fig:lineplot__iSER_n15__Ly_var__err_0}
\end{figure}
%%%%%%%%%%%%%%%%%%%%%%%%%%%%%%%%%%%%%%%%%%%%%%%%%%%%%%%%%%%%%%%%%%%%%%%%%%%%%%

\subsection{Comparison over different frame samples}

We also compare the two filters for a range of values of $\Ly$. From now on, the results will be presented averaging over all bins and frames as well, according to \cref{eqs:metrics_avg-bin-frame}.
\begin{subgather}{eqs:metrics_avg-bin-frame}
	\dsrf = \frac{\sum_{l,k}\abs{X_{1,k}[l]}^2}{\sum_{l,k}\abs{X_{f,k}[l]}^2} \\
	\gser = \frac{\sum_{l,k}\abs{S_{1,k}[l]}^2}{\sum_{l,k}\abs{S_{f,k}[l]}^2} \cdot \frac{1}{\dsrf} \\
	\gsnr = \frac{\sum_{l,k}\abs{V_{1,k}[l]}^2}{\sum_{l,k}\abs{V_{f,k}[l]}^2} \cdot \frac{1}{\dsrf}
\end{subgather}

In all figures of \cref{fig:lineplot__iSER_n15__Ly_var__err_0} we can see a behavior similar to that previously achieved. In the DSRF, we see that for both filters the distortion goes to zero as we increase $\Ly$. Although from \cref{subfig:lineplot__DSRF_l__iSER_n15__err_0} we saw that there was some distortion over time with the SSBT, on average the distortion is null. We also see in \cref{subfig:lineplot__gSER__iSER_n15__Ly_var__err_0,subfig:lineplot__gSNR__iSER_n15__Ly_var__err_0} that the STFT beamformer strictly outperformed the SSBT one for all values of $\Ly$, having a much better performance over the whole range, reaching almost $15\dB$ of enhancement capability, as well as increasing the white noise less.
%%%%%%%%%%%%%%%%%%%%%%%%%%%%%%%%%%%%%%%%%%%%%%%%%%%%%%%%%%%%%%%%%%%%%%%%%%%%%%

\subsection{Comparison over different input SERs}

We now examine the results with a varying input SERs, to assess the beamformer's performances for different loudspeaker signal levels. The results will be again presented for $\Ly = 1$ and $\Ly = 15$, representing two contrasting situations in terms of distortionless-ness and performance for the beamformers.

\def\meshcols{316}
\def\meshrows{15}
\def\tmin{0}
\def\tmax{8.3725}
\def\fmin{0.25}
\def\fmax{7.75}
\begin{figure}[!t]
	\centering
	\begin{subfigure}{\textwidth}
		\centering
		%% Requires:
% pgfplots.sty
% edit_pgfplots.tex

\pgfplotsset{compat=1.18}
%\begin{subfigure}{\linewidth}
%\centering
%\tikzsetnextfilename{dsrf_lineplot_32_var_acc_v7}
\begin{tikzpicture}
	\begin{snrplot}{DSRF (dB)}%[ymin=-3, ymax=0.03, ytick={-3, -2, -1, 0}]
		\addplot [style=resA1]
		table [col sep=comma, y=val] {figures/io_input/STFT/DSRF__h__STFT__N_32__iSER_var__Ly_1__err_0.csv};
		
		\addplot [style=resA2]
		table [col sep=comma, y=val] {figures/io_input/STFT/DSRF__h__STFT__N_32__iSER_var__Ly_15__err_0.csv};
		
		\addplot [style=resC1]
		table [col sep=comma, y=val] {figures/io_input/SSBT/DSRF__h__SSBT__N_32__iSER_var__Ly_1__err_0.csv};
		
		\addplot [style=resC2]
		table [col sep=comma, y=val] {figures/io_input/SSBT/DSRF__h__SSBT__N_32__iSER_var__Ly_15__err_0.csv};
	\end{snrplot}
\end{tikzpicture}
%	\caption{}
%	\label{subfig:1_gain_lineplot}
%\end{subfigure}
		\caption{Time-average broadband DSRF.}
		\label{subfig:lineplot__DSRF__iSER_var__err_0}
	\end{subfigure}\\[1em]
	\begin{subfigure}{\textwidth}
		\centering
		%% Requires:
% pgfplots.sty
% edit_pgfplots.tex

\pgfplotsset{compat=1.18}
%\begin{subfigure}{\linewidth}
%\centering
%\tikzsetnextfilename{erle_lineplot_32_var_acc_v7}
\begin{tikzpicture}
	\begin{snrplot}{gSER (dB)}%[ymin=10, ymax=30, ytick={12, 18, 24, 30}]
		\addplot [style=resA1]
		table [col sep=comma, y=val] {figures/io_input/STFT/gSER__h__STFT__N_32__iSER_var__Ly_1__err_0.csv};
		
		\addplot [style=resA2]
		table [col sep=comma, y=val] {figures/io_input/STFT/gSER__h__STFT__N_32__iSER_var__Ly_15__err_0.csv};
		
		\addplot [style=resC1]
		table [col sep=comma, y=val] {figures/io_input/SSBT/gSER__h__SSBT__N_32__iSER_var__Ly_1__err_0.csv};
		
		\addplot [style=resC2]
		table [col sep=comma, y=val] {figures/io_input/SSBT/gSER__h__SSBT__N_32__iSER_var__Ly_15__err_0.csv};
	\end{snrplot}
\end{tikzpicture}
%	\caption{}
%	\label{subfig:1_gain_lineplot}
%\end{subfigure}
		\caption{Time-average broadband ERLE.}
		\label{subfig:lineplot__gSER__iSER_var__err_0}
	\end{subfigure}\\[1em]
	\begin{subfigure}{\textwidth}
		\centering
		%% Requires:
% pgfplots.sty
% edit_pgfplots.tex

\pgfplotsset{compat=1.18}
%\begin{subfigure}{\linewidth}
%\centering
%\tikzsetnextfilename{nsrf_lineplot_32_var_acc_v7}
\begin{tikzpicture}
	\begin{snrplot}{gSNR (dB)}%[ymin=-13, ymax=-3, ytick={-12, -9, ..., -3}]
		\addplot [style=resA1]
		table [col sep=comma, y=val] {figures/io_input/STFT/gSNR__h__STFT__N_32__iSER_var__Ly_1__err_0.csv};
		
		\addplot [style=resA2]
		table [col sep=comma, y=val] {figures/io_input/STFT/gSNR__h__STFT__N_32__iSER_var__Ly_15__err_0.csv};
		
		\addplot [style=resC1]
		table [col sep=comma, y=val] {figures/io_input/SSBT/gSNR__h__SSBT__N_32__iSER_var__Ly_1__err_0.csv};
		
		\addplot [style=resC2]
		table [col sep=comma, y=val] {figures/io_input/SSBT/gSNR__h__SSBT__N_32__iSER_var__Ly_15__err_0.csv};
	\end{snrplot}
\end{tikzpicture}
%	\caption{}
%	\label{subfig:1_gain_lineplot}
%\end{subfigure}
		\caption{Time-average broadband NSRF.}
		\label{subfig:lineplot__gSNR__iSER_var__err_0}
	\end{subfigure}\\[1em]
	\ref*{timeplot_legend}
	\caption{Output metrics for the beamformers for varying input SERs.}
	\label{fig:lineplot__iSER_var__Ly_1__err_0}
\end{figure}

Each figure brings some new relevant information. From \cref{subfig:lineplot__DSRF__iSER_var__err_0}, we see that for $\Ly = 1$ the distortion on the desired signal increases as the input SER increases, and that the distortion is null for $\Ly = 15$, which is in line with the results previously obtained, as well as with the explanations found. For the gain in SER, we see that it decreases as the input SER increases, which is again on par with what would be expected. However, we see that for higher SERs the performance gap between the two filters decreases, leading to practically identical results. The same happens in terms of gain in SNR, at least for $\Ly = 15$; for $\Ly = 1$ the STFT filter is strictly better, in terms of SNR gain.

%%%%%%%%%%%%%%%%%%%%%%%%%%%%%%%%%%%%%%%%%%%%%%%%%%%%%%%%%%%%%%%%%%%%%%%%%%%%%%%%%%%%%%%%%%%%%%%%%%%%%%%%%%%%%%%%%%%%%%%%%%%%%%%%%%%%%%%%%%%%%
%\subsection{Comparison with perturbation}
%
%\begin{figure}[!t]
%	\centering
%	\begin{subfigure}{\textwidth}
%		\centering
%		%% Requires:
% pgfplots.sty
% edit_pgfplots.tex

\pgfplotsset{compat=1.18}
%\begin{subfigure}{\linewidth}
%\centering
%\tikzsetnextfilename{dsrf_lineplot_32_var_acc_v7}
\begin{tikzpicture}
	\begin{errplot}{DSRF (dB)}%[ymin=-0.09, ymax=0.01, ytick={-0.09, -0.06, ..., 0}]
		\addplot [style=resA]
		table [col sep=comma, y=val] {figures/io_input/STFT/v7__DSRF__STFT__N_32__iSER_n15__Ly_1__err_var.csv};
		
		\addplot [style=resC]
		table [col sep=comma, y=val] {figures/io_input/NSSBT/v7__DSRF__NSSBT__N_32__iSER_n15__Ly_1__err_var.csv};
		
		\addplot [style=resE]
		table [col sep=comma, y=val] {figures/io_input/TSSBT/v7__DSRF__TSSBT__N_32__iSER_n15__Ly_1__err_var.csv};
	\end{errplot}
\end{tikzpicture}
%	\caption{}
%	\label{subfig:1_gain_lineplot}
%\end{subfigure}
%		\caption{Per-window broadband DSRF.}
%		\label{subfig:lineplot__DSRF__iSER_n15__Ly_1__err_var}
%	\end{subfigure}\\[1em]
%	\begin{subfigure}{\textwidth}
%		\centering
%		%% Requires:
% pgfplots.sty
% edit_pgfplots.tex

\pgfplotsset{compat=1.18}
%\begin{subfigure}{\linewidth}
%\centering
%\tikzsetnextfilename{erle_lineplot_32_var_acc_v7}
\begin{tikzpicture}
	\begin{errplot}{gSER (dB)}[ymin=6, ymax=30, ytick={6, 12, ..., 30}]
		\addplot [style=resA]
		table [col sep=comma, y=val] {figures/io_input/STFT/v7__gSER__STFT__N_32__iSER_n15__Ly_1__err_var.csv};
		
		\addplot [style=resC]
		table [col sep=comma, y=val] {figures/io_input/NSSBT/v7__gSER__NSSBT__N_32__iSER_n15__Ly_1__err_var.csv};
		
		\addplot [style=resE]
		table [col sep=comma, y=val] {figures/io_input/TSSBT/v7__gSER__TSSBT__N_32__iSER_n15__Ly_1__err_var.csv};
	\end{errplot}
\end{tikzpicture}
%	\caption{}
%	\label{subfig:1_gain_lineplot}
%\end{subfigure}
%		\caption{Per-window broadband gSER.}
%		\label{subfig:lineplot__gSER__iSER_n15__Ly_1__err_var}
%	\end{subfigure}\\[1em]
%	\begin{subfigure}{\textwidth}
%		\centering
%		%% Requires:
% pgfplots.sty
% edit_pgfplots.tex

\pgfplotsset{compat=1.18}
%\begin{subfigure}{\linewidth}
%\centering
%\tikzsetnextfilename{nsrf_lineplot_32_var_acc_v7}
\begin{tikzpicture}
	\begin{errplot}{gSNR (dB)}[ymin=-30, ymax=5, ytick={-30, -20, ..., 0}]
		\addplot [style=resA]
		table [col sep=comma, y=val] {figures/io_input/STFT/v7__gSNR__STFT__N_32__iSER_n15__Ly_1__err_var.csv};
		
		\addplot [style=resC]
		table [col sep=comma, y=val] {figures/io_input/NSSBT/v7__gSNR__NSSBT__N_32__iSER_n15__Ly_1__err_var.csv};
		
		\addplot [style=resE]
		table [col sep=comma, y=val] {figures/io_input/TSSBT/v7__gSNR__TSSBT__N_32__iSER_n15__Ly_1__err_var.csv};
	\end{errplot}
\end{tikzpicture}
%	\caption{}
%	\label{subfig:1_gain_lineplot}
%\end{subfigure}
%		\caption{Per-window broadband gSNR.}
%		\label{subfig:lineplot__gSNR__iSER_n15__Ly_1__err_var}
%	\end{subfigure}\\[1em]
%	\ref*{timeplot_legend}
%	\caption{Output metrics for the beamformers with error in the steering vectors.}
%	\label{fig:lineplot__v7_iSER_n15__Ly_1__err_var}
%\end{figure}
%
%As exposed in \cref{sec:perturbation_analysis}, it is also of interest to compare how robust the derived beamformers are, when the information regarding the desired signal's RIR isn't accurate. For such, we model the matrix $\bvA[u]{k}$ as
%\begin{equation}
%	\bvA[u]{k} = \bvA[u]{k}^\star + \Delta\bvA[u]{k}
%\end{equation}
%where $\bvA[u]{k}^\star$ is the accurate steering matrix, and $\Delta\bvA[u]{k}$ is a perturbation on it, which we assume is a zero-mean uniform white noise, with an adjustable variance.
%
%Since in this scenario the desired signal can suffer some distortion (given that its steering matrix isn't appropriately estimated), we will use the gain in SER and gain in SNR metrics instead of ERLE and NSRF, to take this distortion into account. These are defined as
%\begin{subgather}
%	\gser = \frac{\erle}{\dsrf} \\
%	\gsnr = \frac{\nsrf}{\dsrf}
%\end{subgather}
%
%The DSRF will still be showed, to give a sense of proportion on how much the beamformer distorts the desired signal. In the results of \cref{fig:lineplot__v7_iSER_n15__Ly_1__err_var}, the x-axis represents the standard deviation of $\Delta\bvA[u]{k}$, as a percentage of the standard deviation of $\bvA[u]{k}^\star$.
%
%Each metric showed a different result: differently than before, the SF-SSBT beamformer led to the least distortion on the desired signal, out of all three beamformers, and the DF-SSBT led to the most distortion. Meanwhile, the gain in SER showed the STFT beamformer to be the (overall) more robust, and the one that led to the best results, with the DF-SSBT again being the worse one. A similar result can be seen for the gain in SNR, with the STFT beamformer being the best, but in this regard the SF-SSBT beamformer led to the worst results, although marginally.