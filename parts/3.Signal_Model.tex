\section{Signal Model and Beamforming}
\label{sec:signal_model}

Let there be a device that consists of $M$ sensors and a loudspeaker (LS) in a reverberant environment, in which there also is a desired source, both traveling with a speed $c$. We also assume the presence of undesired noise at each sensor. For simplicity we assume that all sources are spatially stationary, although this condition can be easily removed.

We denote $y_m[n]$ as the signal at the $m$-th sensor, being defined as
\begin{equation}
	\label{eq:sec3:time_model_basic}
	y_m[n] = h_m[n] \ast x[n] + e_m[n] \ast s[n] + r_m[n]
\end{equation}
in which $h_m[n]$ is the impulse response between the desired source and the $m$-th sensor ($1 \leq m \leq M$), with $x[n]$ being the desired source's signal; similarly for  speaker's signal $s[n]$ and its IR $e_m[n]$; and $r_m[n]$ is the uncorrelated noise.

We let $m'$ be the reference sensor's index, for simplicity assume $m' = 1$, and also $x_1[n] = h_1[n] \ast x[n]$ (similarly for $s_1[n]$). We define $a_m[n]$ as the \textit{relative} impulse response between the desired signal (at the reference sensor) and the $m$-th sensor, such that
\begin{equation}
	a_m[n] \ast x_1[n] = h_m[n] \ast x[n]
\end{equation}
We similarly define $c_m[n]$ from $e_m[n]$ and $s[n]$. Therefore, \cref{eq:sec3:time_model_basic} becomes
\begin{equation}
	\label{eq:sec3:time_model_relative}
	y_m[n] = a_m[n] \ast x_1[n] + c_m[n] \ast s_1[n] + r_m[n]
\end{equation}
Here, the impulse responses ($a_m[n]$ and $c_m[n]$) can be non-causal, depending on the direction of arrival and features of the reverberant environment.

We use a time-frequency transform (such as the STFT or SSBT, as in \cref{sec:stft_and_ssbt}) with the convolutive transfer-function (CTF) model \cite{talmon_relative_2009} to obtain our time-frequency signal model,
\begin{equation}
	\label{eq:sec3:time-freq_model_conv}
	Y_m[l,k] = A_m[l,k] \ast X_1[l,k] + C_m[l,k] \ast S_1[l,k] + R_m[l,k]
\end{equation}
where $Y_m[l,k]$ is the transform of $y_m[n]$ (resp. all other signals); $l$ and $k$ are the window (or decimated-time) and bin indexes, with $0 \leq k \leq K-1$; and the convolution is in the window-index axis.

Using that $A_m[l,k]$ is a finite (possibly truncated) response with $L_A$ frames, then
\begin{equation}
	\label{eq:sec3:convert_convol_matmult}
	A_m[l,k] \ast X_1[l,k] = \tr{\bva{m}}[k] \bvx{1}[l,k]
\end{equation}
in which
\begin{subalign}
	\bva{m}[k] & = \tr{\tup{ {A_m[-\Delta,k]} ,, {A_m[0,k]} , , {A_m[L_A-\Delta-k,1]} }} \\
	\bvx{1}[l,k] & = \tr{\tup{ {X_1[l+\Delta,k]} , , { X_1[l,k]} , , {X_1[l-L_A+\Delta+k,1]} }} \label{subeq:sec3:def_bvx1lk}
\end{subalign}
and in the same way we define $\bvc{m}[k]$ and $\bvs{1}[l,k]$. Note that $\bva{m}[k]$ and $\bvb{m}[k]$ don't depend on the index $l$, given the spatial stationarity assumption. Also, $\Delta$ is the number of non-causal frames in the reference sensor necessary to capture the whole signal. With this, \cref{eq:sec3:time-freq_model_conv} becomes
\begin{equation}
	\label{eq:sec3:time-freq_model_mult1}
	Y_m[l,k] = \tr{\bva{m}}[k] \bvx{1}[l,k] + \tr{\bvc{m}}[k] \bvs{1}[l,k] + R_m[l,k]
\end{equation}

We will change the indexing, such that $X_{k,m}[l] \equiv X_m[l,k]$. With this, the problem at hand clearly becomes a time-domain beamforming situation, where the we have a different convolutive time-domain beamformer for each frequency bin $k$.

We take $L_Y$ samples of our observed signal, and define $\bvy[b]{k,m}[l]$ as
\begin{equation}
	\label{eq:sec3:sample_vectorization}
	\bvy[b]{k,m}[l] = \tr{\tup{{Y_{k,m}[l]}, {Y_{k,m}[l-1]} ,, {Y_{k,m}[l-L_Y+1]}}}
\end{equation}

In this new framework, we can write $\bvy{k}[m,l]$ as a $\sz{L_Y}{1}$ vector
\begin{equation}
	\bvy[b]{k,m}[l] = \bvA[b]{k,m} \bvx[b]{k,1}[l] + \bvC[b]{k,m} \bvs[b]{k,1}[l] + \bvr[b]{k,m}[l]
\end{equation}
where $\bvr[b]{k,m}[l]$ is defined similarly to \cref{eq:sec3:sample_vectorization}, $\bvx[b]{k,1}[l]$ is a $\sz{L}{1}$ vector, and $\bvA[b]{k,m}$ is a $\sz{L_Y}{L}$ matrix, both being given by
\begin{subgather}
	\bvx[b]{k,1}[l] = \tr{\tup{{X_{k,1}[l+\Delta]} , {X_{k,1}[l+\Delta-1]} ,, {X_{k,1}[l+\Delta - L]}}} \\
	\bvA[b]{k,m} =
	\begin{bmatrix}
		\tr{\bva{k,m}} 	& 0 			& \cdots 	& 0		 			\\
		0	 			& \tr{\bva{k,m}} & \cdots 	& 0		 			\\
		\vdots 			& \vdots 		& \ddots 	& \vdots 			\\
		0	 			& 0			 	& \cdots	& \tr{\bva{k,m}}	\\
	\end{bmatrix}
\end{subgather}
where $L = L_Y + L_A -1$, and $\bva{k,m} \equiv \bva{m}[k]$. $\bvs[b]{k,1}[l]$ and $\bvC[b]{k,m}$ are defined similarly, them being a $\sz{(L+L_C-1)}{1}$ vector and a $\sz{L_Y}{(L+L_C-1)}$ matrix respectively. We now concatenate the matrices and vectors for the $M$ different sensors, such that
\begin{equation}
	\label{eq:sec3:bvyk_final}
	\bvy{k}[l] = \bvA{k} \bvx[b]{k,1}[l] + \bvC{k} \bvs[b]{k,1}[l] + \bvr{k}[l]
\end{equation}
where
\begin{subgather}
	\bvy{k}[l] = \vtup{\bvy[b]{k,1}[l] ,, \bvy[b]{k,M}[l]} \\
	\bvA{k} = \vtup{\bvA[b]{k,1} ,, \bvA[b]{k,M}}
\end{subgather}
$\bvy{k}[l]$ is a $\sz{ML_Y}{1}$ vector, and $\bvA{k}$ is a $\sz{ML_Y}{L}$ matrix. $\bvr{k}[l]$ is defined in the same way as $\bvy{k}[l]$ and $\bvC{k}$ as $\bvA{k}$. 

\subsection{Filtering and the MPDR beamformer}

We denote $Z_{k}[l]$ as an estimate the desired signal at reference $X_{k,1}[l]$, via a filter $\bvf{k}[l]$ of length $ML_Y$, such that
\begin{equations}{eq:sec3:def_filtering_process}
	Z_{k}[l]
	& \approx X_{k,1}[l] \\
	& = \he{\bvf{k}}[l] \bvy{k}[l]
\end{equations}
with $\he{(\cdot)}$ being the transposed-complex-conjugate operator. This process can also be interpreted as
\begin{equations}{eq:sec3:convolutive_form_filter}
	Z_{k}[l]
	& = \sum_{m} \he{\bvf[b]{k,m}}[l] \bvy[b]{k,m}[l] \\
	& = \sum_{m} F_{k,m}^*[l] \ast Y_{k,m}[l]
\end{equations}
where $\bvf[b]{k,m}[l]$ is the $\sz{L_Y}{1}$ part of $\bvf{k}[l]$ that filters the $m$-th sensor, and $F_{k,m}[l]$ is its signal-form counterpart. In this sense, the filtering process can be interpreted as the sum across all sensors of the convolution between the signal and the observations.

Going back to \cref{eq:sec3:def_filtering_process}, with \cref{eq:sec3:bvyk_final} we can write
\begin{equation}
	Z_{k}[l] = \he{\bvf{k}}[l] \bvA{k} \bvx[b]{k,1}[l] + \he{\bvf{k}}[l] \bvC{k} \bvs[b]{k,1}[l] + \he{\bvf{k}}[l] \bvr{k}[l]
\end{equation}
From this, we easily see that to achieve a distortionless response from the desired signal, we must have that $\he{\bvf{k}}[l] \bvA{k} \bvx[b]{k,1}[l] = X_{k,1}[l]$, and therefore the distortionless constraint is given by
\begin{equation}
	\he{\bvf{k}}[l] \bvA{k} = \tr{\bvi{\Delta}}
\end{equation}
where $\bvi{\Delta}$ is a $\sz{L}{1}$ vector of zeroes, except for the $\Delta$-th entry which is a $1$.

To minimize the variance of the output signal while obeying the distortionless constraint, a Minimum-Power Distortionless Response (MPDR) beamformer will be used, it being defined as
\begin{equation}
	\label{eq:sec3:minimization_problem_mpdr}
	\bvf{\mpdr;k}[l] = \min_{\bvf{k}[l]} \he{\bvf{k}}[l] \Corr{\bvy{k}}[l] \bvf{k}[l]~\text{s.t.}~\he{\bvf{k}}[l] \bvA{k} = \tr{\bvi{\Delta}}
\end{equation}
where $\Corr{\bvy{k}}[l]$ is the correlation matrix of the observed signal $\bvy{k}[l]$. The solution to this minimization problem 
\begin{equation}
	\label{eq:sec3:solution_mpdr_beamformer}
	\bvf{\mpdr;k}[l] = \iCorr{\bvy{k}}[l] \bvA{k} \inv{\bts{ \he{\bvA{k}} \iCorr{\bvy{k}}[l] \bvA{k} }} \bvi{\Delta}
\end{equation}