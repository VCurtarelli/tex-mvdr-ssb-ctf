\section{Signal Model and Beamforming}
\label{sec:signal_model}

%\begin{changingfrom}
%	Let there be a device that consists of $M$ sensors and a loudspeaker (LS) in a reverberant environment, in which there also is a desired source, both traveling with a speed $c$. We also assume the presence of undesired noise at each sensor. For simplicity we assume that all sources are spatially stationary, although this condition can be easily removed.
%	
%	We denote $y_m[n]$ as the signal at the $m$-th sensor, being defined as
%	\begin{equation}
%		\label{eq:sec3:time_model_basic}
%		y_m[n] = h_m[n] \ast x[n] + e_m[n] \ast s[n] + r_m[n]
%	\end{equation}
%	in which $h_m[n]$ is the impulse response between the desired source and the $m$-th sensor ($1 \leq m \leq M$), with $x[n]$ being the desired source's signal; similarly for  speaker's signal $s[n]$ and its IR $e_m[n]$; and $r_m[n]$ is the uncorrelated noise.
%	
%	We let $m'$ be the reference sensor's index, for simplicity assume $m' = 1$, and also $x_1[n] = h_1[n] \ast x[n]$ (similarly for $s_1[n]$). We define $a_m[n]$ as the \textit{relative} impulse response between the desired signal (at the reference sensor) and the $m$-th sensor, such that
%	\begin{equation}
%		a_m[n] \ast x_1[n] = h_m[n] \ast x[n]
%	\end{equation}
%	We similarly define $c_m[n]$ from $e_m[n]$ and $s[n]$. Therefore, \cref{eq:sec3:time_model_basic} becomes
%	\begin{equation}
%		\label{eq:sec3:time_model_relative}
%		y_m[n] = a_m[n] \ast x_1[n] + c_m[n] \ast s_1[n] + r_m[n]
%	\end{equation}
%	Here, the impulse responses ($a_m[n]$ and $c_m[n]$) can be non-causal, depending on the direction of arrival and features of the reverberant environment.
%	
%	We use a time-frequency transform (such as the STFT or SSBT, as in \cref{sec:stft_and_ssbt}) with the convolutive transfer-function (CTF) model \cite{talmon_relative_2009} to obtain our time-frequency signal model,
%	\begin{equation}
%		\label{eq:sec3:time-freq_model_conv}
%		Y_m[l,k] = A_m[l,k] \ast X_1[l,k] + C_m[l,k] \ast S_1[l,k] + R_m[l,k]
%	\end{equation}
%	where $Y_m[l,k]$ is the transform of $y_m[n]$ (resp. all other signals); $l$ and $k$ are the window (or decimated-time) and bin indexes, with $0 \leq k \leq K-1$; and the convolution is in the window-index axis.
%	
%	Using that $A_m[l,k]$ is a finite (possibly truncated) response with $L_A$ frames, then
%	\begin{equation}
%		\label{eq:sec3:convert_convol_matmult}
%		A_m[l,k] \ast X_1[l,k] = \tr{\bva{m}}[k] \bvx{1}[l,k]
%	\end{equation}
%	in which
%	\begin{subalign}
%		\bva{m}[k] & = \tr{\tup{ {A_m[-\Delta,k]} ,, {A_m[0,k]} , , {A_m[L_A-\Delta-k,1]} }} \\
%		\bvx{1}[l,k] & = \tr{\tup{ {X_1[l+\Delta,k]} , , { X_1[l,k]} , , {X_1[l-L_A+\Delta+k,1]} }} \label{subeq:sec3:def_bvx1lk}
%	\end{subalign}
%	and in the same way we define $\bvb{m}[k]$ and $\bvs{1}[l,k]$. Note that $\bva{m}[k]$ and $\bvb{m}[k]$ don't depend on the index $l$, given the spatial stationarity assumption. Also, $\Delta$ is the number of non-causal frames in the reference sensor necessary to capture the whole signal. With this, \cref{eq:sec3:time-freq_model_conv} becomes
%	\begin{equation}
%		\label{eq:sec3:time-freq_model_mult1}
%		Y_m[l,k] = \tr{\bva{m}}[k] \bvx{1}[l,k] + \tr{\bvb{m}}[k] \bvs{1}[l,k] + R_m[l,k]
%	\end{equation}
%	
%	We will change the indexing, such that $X_{m,k}[l] \equiv X_m[l,k]$. With this, the problem at hand clearly becomes a time-domain beamforming situation, where the we have a different convolutive time-domain beamformer for each frequency bin $k$.
%	
%	We take $L_Y$ samples of our observed signal, and define $\bvy[b]{m,k}[l]$ as
%	\begin{equation}
%		\label{eq:sec3:sample_vectorization}
%		\bvy[b]{m,k}[l] = \tr{\tup{{Y_{m,k}[l]}, {Y_{m,k}[l-1]} ,, {Y_{m,k}[l-L_Y+1]}}}
%	\end{equation}
%	
%	In this new framework, we can write $\bvy{k}[m,l]$ as a $\sz{L_Y}{1}$ vector
%	\begin{equation}
%		\bvy[b]{m,k}[l] = \bvA[b]{m,k} \bvx[b]{1,k}[l] + \bvB[b]{m,k} \bvs[b]{1,k}[l] + \bvr[b]{m,k}[l]
%	\end{equation}
%	where $\bvr[b]{m,k}[l]$ is defined similarly to \cref{eq:sec3:sample_vectorization}, $\bvx[b]{1,k}[l]$ is a $\sz{L}{1}$ vector, and $\bvA[b]{m,k}$ is a $\sz{L_Y}{L}$ matrix, both being given by
%	\begin{subgather}
%		\bvx[b]{1,k}[l] = \tr{\tup{{X_{1,k}[l+\Delta]} , {X_{1,k}[l+\Delta-1]} ,, {X_{1,k}[l+\Delta - L]}}} \\
%		\bvA[b]{m,k} =
%		\begin{bmatrix}
%			\tr{\bva{m,k}} 	& 0 			& \cdots 	& 0		 			\\
%			0	 			& \tr{\bva{m,k}} & \cdots 	& 0		 			\\
%			\vdots 			& \vdots 		& \ddots 	& \vdots 			\\
%			0	 			& 0			 	& \cdots	& \tr{\bva{m,k}}	\\
%		\end{bmatrix}
%	\end{subgather}
%	where $L = L_Y + L_A -1$, and $\bva{m,k} \equiv \bva{m}[k]$. $\bvs[b]{1,k}[l]$ and $\bvB[b]{m,k}$ are defined similarly, them being a $\sz{(L+L_C-1)}{1}$ vector and a $\sz{L_Y}{(L+L_C-1)}$ matrix respectively. We now concatenate the matrices and vectors for the $M$ different sensors, such that
%	\begin{equation}
%		\label{eq:sec3:bvyk_final}
%		\bvy{k}[l] = \bvA{k} \bvx[b]{1,k}[l] + \bvB{k} \bvs[b]{1,k}[l] + \bvr{k}[l]
%	\end{equation}
%	where
%	\begin{subgather}
%		\bvy{k}[l] = \vtup{\bvy[b]{1,k}[l] ,, \bvy[b]{k,M}[l]} \\
%		\bvA{k} = \vtup{\bvA[b]{1,k} ,, \bvA[b]{k,M}}
%	\end{subgather}
%	$\bvy{k}[l]$ is a $\sz{ML_Y}{1}$ vector, and $\bvA{k}$ is a $\sz{ML_Y}{L}$ matrix. $\bvr{k}[l]$ is defined in the same way as $\bvy{k}[l]$ and $\bvB{k}$ as $\bvA{k}$. 
%\end{changingfrom}
%
%\begin{changingto}
	Let there be a device that consists of $M$ sensors and a loudspeaker (LS) in a reverberant environment, in which there also is a desired source, both traveling with a speed $c$. We also assume the presence of undesired noise at each sensor. For simplicity we assume that all sources are spatially stationary, although this condition can be easily removed.
	
	We denote $y_m[n]$ as the signal at the $m$-th sensor, being defined as
	\begin{equation}
		\label{eq:sec3:time_model_basic}
		y_m[n] = h_m[n] \ast x[n] + g_m[n] \ast s[n] + r_m[n]
	\end{equation}
	in which $h_m[n]$ is the impulse response between the desired source and the $m$-th sensor ($1 \leq m \leq M$), with $x[n]$ being the desired source's signal; similarly for speaker's signal $s[n]$ and its IR $g_m[n]$; and $r_m[n]$ is the uncorrelated noise.
	
	We use a time-frequency transform (such as the STFT or SSBT, as in \cref{sec:stft_and_ssbt}) with the convolutive transfer-function (CTF) model \cite{talmon_relative_2009} to obtain our time-frequency signal model,
	\begin{equation}
		\label{eq:sec3:time-freq_model_conv}
		Y_{m,k}[l] = H_{m,k}[l] \ast X_{k}[l] + G_{m,k}[l] \ast S_{k}[l] + R_{m,k}[l]
	\end{equation}
	where $Y_{m,k}[l]$ is the transform of $y_m[n]$ (resp. all other signals); $k$ is the frequency bin index, with $0 \leq k \leq K-1$, and $l$ is the window (or decimated-time) index. The convolution is in the window-index axis.
	
	We let $m'$ be the reference sensor's index, for simplicity assume $m' = 1$, and we denote $x_{1,k}[l] = H_{1,k}[l] \ast X_{k}[l]$. For each sensor $m$ and each frequency $k$, we denote $A_{m,k}[l]$ as the relative impulse response for the desired signal (at the reference sensor) and the $m$-th sensor, such that
	\begin{equation}
		A_{m,k}[l] \ast X_{1,k}[l] = H_{m,k}[l] \ast X_{k}[l]
	\end{equation} 
	We similarly define $B_{m,k}[l]$ and $S_{1,k}[l]$ based on $G_{m,k}[l]$ and $S_{k}[l]$. Therefore, \cref{eq:sec3:time-freq_model_conv} becomes
	\begin{equation}
		Y_{m,k}[l] = A_{m,k}[l] \ast X_{1,k}[l] + B_{m,k}[l] \ast S_{1,k}[l] + R_{m,k}[l]
	\end{equation}
	
	Here, the impulse responses $A_{m,k}[l]$ and $B_{m,k}[l]$ can be non-causal, depending on the direction of arrival and features of the reverberant environment, as well as relative delays between the sources at each sensor. They can also be non-causal on account of the windowing process of the time-frequency transforms. We assume that there are $\Delta$ non-causal samples in $A_{m,k}[l]$. It is trivial to see that $A_{1,k}[l] = \delta_{0,l}$, a Kronecker delta at $l = 0$.
	
	We let $A'_{m,k}[l]$ comprise the $L_A$ samples of $A_{m,k}[l]$ of most interest (for example, the $L_A$ causal samples starting on the non-zero value of $A_{1,k}[l]$), and define $Q_{m,k}[l]$ such that
	\begin{equation}
		Q_{m,k}[l] = A_{m,k}[l] \ast X_{1,k}[l] - A'_{m,k}[l] \ast X_{1,k}[l]
	\end{equation}
	We thus have
	\begin{equation}
		A'_{m,k}[l] \ast X_{1,k}[l] = \tr{\bva{m,k}} \bvx{1,k}[l]
	\end{equation}
	in which
	\begin{subalign}
		\bva{m,k} & = \tr{\tup{ {A_{m,k}[\Delta]} ,, {A_{m,k}[0]} ,, {A_{m,k}[L_A-\Delta-k]} }} \\
		\bvx{1,k}[l] & = \tr{\tup{ {X_{1,k}[l+\Delta]} , , { X_{1,k}[l]} ,, {X_{1,k}[l-L_A+\Delta+k]} }} \label{subeq:sec3:def_bvx1lk}
	\end{subalign}
	and in the same way we define $\bvb{m,k}$ and $\bvs{1,k}[l]$. Note that $\bva{m,k}$ and $\bvb{m,k}$ don't depend on the index $l$, given the spatial stationarity assumption. With this, \cref{eq:sec3:time-freq_model_conv} becomes
	\begin{equation}
		\label{eq:sec3:time-freq_model_mult1}
		Y_{m,k}[l] = \tr{\bva{m,k}} \bvx{1,k}[l] + \tr{\bvb{m,k}} \bvs{1,k}[l] + R_{m,k}[l] + Q_{m,k}[l]
	\end{equation}
	
	We take $L_Y$ samples of our observed signal, and define $\bvy[b]{m,k}[l]$ as
	\begin{equation}
		\label{eq:sec3:sample_vectorization}
		\bvy[b]{m,k}[l] = \tr{\tup{{Y_{m,k}[l]}, {Y_{m,k}[l-1]} ,, {Y_{m,k}[l-L_Y+1]}}}
	\end{equation}
	
	In this new framework, we can write $\bvy{m,k}[l]$ as a $\sz{L_Y}{1}$ vector
	\begin{equation}
		\bvy[b]{m,k}[l] = \bvA[b]{m,k} \bvx[b]{1,k}[l] + \bvB[b]{m,k} \bvs[b]{1,k}[l] + \bvr[b]{m,k}[l] + \bvq[b]{m,k}[l]
	\end{equation}
	where $\bvr[b]{m,k}[l]$ and $\bvq[b]{m,k}[l]$ are defined similarly to \cref{eq:sec3:sample_vectorization}, $\bvx[b]{1,k}[l]$ is a $\sz{L}{1}$ vector, and $\bvA[b]{m,k}$ is a $\sz{L_Y}{L}$ matrix, both being given by
	\begin{subgather}
		\bvx[b]{1,k}[l] = \tr{\tup{{X_{1,k}[l+\Delta]} , {X_{1,k}[l+\Delta-1]} ,, {X_{1,k}[l+\Delta - L]}}} \\
		\bvA[b]{m,k} =
		\begin{bmatrix}
			\tr{\bva{m,k}} 	& 0 			& \cdots 	& 0		 			\\
			0	 			& \tr{\bva{m,k}} & \cdots 	& 0		 			\\
			\vdots 			& \vdots 		& \ddots 	& \vdots 			\\
			0	 			& 0			 	& \cdots	& \tr{\bva{m,k}}	\\
		\end{bmatrix}
	\end{subgather}
	in which $L = L_Y + L_A -1$, and $\bva{m,k} \equiv \bva{m}[k]$. $\bvs[b]{1,k}[l]$ and $\bvB[b]{m,k}$ are defined similarly, them being a $\sz{(L+L_C-1)}{1}$ vector and a $\sz{L_Y}{(L+L_C-1)}$ matrix respectively. We now concatenate the matrices and vectors for the $M$ different sensors, such that
	\begin{equation}
		\label{eq:sec3:bvyk_final}
		\bvy{k}[l] = \bvA{k} \bvx[b]{1,k}[l] + \bvB{k} \bvs[b]{1,k}[l] + \bvr{k}[l] + \bvq{k}[l]
	\end{equation}
	where
	\begin{subgather}
		\bvy{k}[l] = \vtup{\bvy[b]{1,k}[l] ,, \bvy[b]{k,M}[l]} \\
		\bvA{k} = \vtup{\bvA[b]{1,k} ,, \bvA[b]{k,M}}
	\end{subgather}
	$\bvy{k}[l]$ is a $\sz{ML_Y}{1}$ vector, and $\bvA{k}$ is a $\sz{ML_Y}{L}$ matrix. $\bvr{k}[l]$ and $\bvq{k}[l]$ are defined in the same way as $\bvy{k}[l]$, and $\bvB{k}$ as $\bvA{k}$. 
%\end{changingto}

\subsection{Filtering and the MPDR beamformer}

We denote $Z_{k}[l]$ as an estimate the desired signal at reference $X_{1,k}[l]$, via a filter $\bvf{k}[l]$ of length $ML_Y$, such that
\begin{equations}{eq:sec3:def_filtering_process}
	Z_{k}[l]
	& \approx X_{1,k}[l] \\
	& = \he{\bvf{k}}[l] \bvy{k}[l]
\end{equations}
with $\he{(\cdot)}$ being the transposed-complex-conjugate operator. This process can also be interpreted as
\begin{equations}{eq:sec3:convolutive_form_filter}
	Z_{k}[l]
	& = \sum_{m} \he{\bvf[b]{m,k}}[l] \bvy[b]{m,k}[l] \\
	& = \sum_{m} F_{m,k}^*[l] \ast Y_{m,k}[l]
\end{equations}
where $\bvf[b]{m,k}[l]$ is the $\sz{L_Y}{1}$ part of $\bvf{k}[l]$ that filters the $m$-th sensor, and $F_{m,k}[l]$ is its signal-form counterpart. In this sense, the filtering process can be interpreted as the sum across all sensors of the convolution between the signal and the observations.

Going back to \cref{eq:sec3:def_filtering_process}, with \cref{eq:sec3:bvyk_final} we can write
\begin{equation}
	Z_{k}[l] = \he{\bvf{k}}[l] \bvA{k} \bvx[b]{1,k}[l] + \he{\bvf{k}}[l] \bvB{k} \bvs[b]{1,k}[l] + \he{\bvf{k}}[l] \bvr{k}[l] + \he{\bvf{k}}[l] \bvq{k}[l]
\end{equation}
From this, we easily see that to achieve a distortionless response from the desired signal, we must have that $\he{\bvf{k}}[l] \bvA{k} \bvx[b]{1,k}[l] = X_{1,k}[l]$, and therefore the distortionless constraint is given by
\begin{equation}
	\he{\bvf{k}}[l] \bvA{k} = \tr{\bvi{\Delta}}
\end{equation}
where $\bvi{\Delta}$ is a $\sz{L}{1}$ vector of zeroes, except for the $\Delta$-th entry which is a $1$. For this to be true, we assume that $\bvq{k}[l]$ is uncorrelated to $\bvx[b]{1,k}[l]$, which may not be true depending on the values of $\Delta$ and $L_A$ chosen, and the characteristics of the environment.

To minimize the variance of the output signal while obeying the distortionless constraint, a Minimum-Power Distortionless Response (MPDR) beamformer will be used, it being defined as
\begin{equation}
	\label{eq:sec3:minimization_problem_mpdr}
	\bvf{\mpdr;k}[l] = \min_{\bvf{k}[l]} \he{\bvf{k}}[l] \Corr{\bvy{k}}[l] \bvf{k}[l]~\text{s.t.}~\he{\bvf{k}}[l] \bvA{k} = \tr{\bvi{\Delta}}
\end{equation}
where $\Corr{\bvy{k}}[l]$ is the correlation matrix of the observed signal $\bvy{k}[l]$. The solution to this minimization problem 
\begin{equation}
	\label{eq:sec3:solution_mpdr_beamformer}
	\bvf{\mpdr;k}[l] = \iCorr{\bvy{k}}[l] \bvA{k} \inv{\bts{ \he{\bvA{k}} \iCorr{\bvy{k}}[l] \bvA{k} }} \bvi{\Delta}
\end{equation}