\section{Introduction}
\label{sec:introduction}

Beamformers are an important tool for signal enhancement, having a plethora of applications from hearing aids \cite{lobato_worst-case-optimization_2020} to source localization \cite{chen_source_2002} to imaging \cite{lu_biomedical_1994,nguyen_minimum_2017}. One of the possible ways to use beamformers is in the time-frequency domain \cite{benesty_fundamentals_2017}, allowing the exploitation of frequency-related information while also dynamically adapting to signal changes over time. While the Short-Time Fourier Transform (STFT) is widely used for time-frequency analysis \cite{kiymik_comparison_2005,pan_microphone_2021}, alternative transforms \cite{chen_wavelet-based_2018,yang_general_2014,almeida_fractional_1994} can also be employed, offering unique perspectives on signal analysis.

Among these, the Single-Sideband Transform (SSBT) \cite{crochiere_multirate_1983,oyzerman_speech_2012} stands out for its real-valued frequency spectrum. It has been shown that the SSBT works particularly well with short analysis windows \cite{crochiere_multirate_1983}, lending itself useful when working with the convolutive transfer function (CTF) model \cite{talmon_relative_2009} and filter-banks \cite{kumatani_filter_2008,gopinath_tutorial_1993} for signal analysis.

Two of the most important goals in beamforming are output noise minimization and the distortionless-ness of the desired signal, both being achieved by the Minimum-Variance Distortionless-Response (MVDR) beamformer \cite{capon_high-resolution_1969,erdogan_improved_2016}. As the MVDR beamformer can be used on the time-frequency domain without restrictions on the transform chosen, it is of interest to explore and compare the performance of this filter, when designing it through different time-frequency transforms.

Motivated by this, our paper explores the SSB transform and its application on the subject of beamforming within the context of the CTF model. We propose an approach for the CTF that allows the separation of desired and undesired speech components for reverberant environments, and employ this approach for designing the MVDR beamformer. We also explore the traits and limitations of the SSBT, and how to properly adapt the MVDR beamformer to this new transform's constraints. We show that a beamformer designed using the SSBT has a similar performance as an STFT one, while also being able to obey the distortionless constraint.

We organized the paper as follows:
in \cref{sec:stft_and_ssbt} we introduce the proposed time-frequency transforms, how they're related and what are their relevant properties;
\cref{sec:signal_model} the considered signal model in the time domain is presented, and how it is transferred into the time-frequency domain;
and in \cref{sec:true_mvdr_ssbt} we develop a true-MVDR beamformer with the SSBT, taking into account its features.
In \cref{sec:results} we present and discuss the results, comparing the studied methods and beamformers obtained.
\cref{sec:conclusion} concludes this paper.