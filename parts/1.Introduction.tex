\section{Introduction}
\label{sec:introduction}

% ###############

Filtering is one of the essential instruments available in the modern world, being necessary in fields as communications \cite{chen_source_2002}, biomedical applications \cite{lobato_worst-case-optimization_2020,lu_biomedical_1994,nguyen_minimum_2017}, system control \cite{han_comparative_2016, hagglund_signal_2012}, among other areas \cite{hathcock_noise_2016,lee_general_2002,shi_new_2004}. At its core, filtering mitigates the effects of undesired sources and can use temporal and spatial information regarding the signals, sources, sensors, and environment to enhance the signals. Given time samples, these filtering processes can be implemented in the time-, frequency-, and time-frequency domains \cite{benesty_fundamentals_2017}, each offering distinct advantages. In particular, time-frequency methods exploit frequency-related information while dynamically adapting to signal and environmental changes in time, providing a tradeoff between strictly time or frequency information. 
Generically, transforms are the primary tool for achieving desired time-frequency domain data, with the Short-Time Fourier Transform (STFT) \cite{kiymik_comparison_2005,pan_microphone_2021} being prominent in the literature due to its widespread use.
However, alternative transforms can also be employed \cite{chen_wavelet-based_2018,yang_general_2014,almeida_fractional_1994}, each offering a unique perspective and insight regarding the signal, leading to different performances depending on specific application requirements.

% ###############

Among these alternatives, an approach based on single-sideband modulation has occasionally been proposed \cite{crochiere_multirate_1983} for various applications, from acoustic echo cancellation \cite{chin_subband_2001} to speech dereverberation \cite{oyzerman_system_2012} and machine learning signal enhancement \cite{okamoto_subband_2017}, showing varying levels of success. The Single-Sideband Transform (SSBT), characterized by its real-valued representation, holds promise for more straightforward hardware implementations, potentially leading to more cost-efficient devices. Under a convolutive transfer function (CTF) model for the time-frequency transform \cite{talmon_relative_2009}, the SSBT shows potential for spatiotemporal multisensor beamforming. However, understanding its properties is crucial to ensure accurate and meaningful outputs. The SSBT lacks comprehensive examination in the literature, resulting in a limited understanding of its features and limitations compared to other methodologies. This knowledge gap complicates the appropriate application and utilization of this transform.

% ###############

We begin by examining a continuous-frequency version of the SSBT, comparing their properties to those of the Fourier Transform (FT) and the STFT. We study how these properties impact basic beamforming concepts such as the convolution theorem and relative frequency response estimation for the SSBT. While the SSBT is more error-prone and restrictive than the STFT regarding beamforming design, we also demonstrate a bijective interchangeability between them. This allows their joint usage, potentially enhancing beamformer design within the SSBT domain by converting into the STFT and back without depending on inverse transforms, which are computationally intensive. We evaluate beamformers through the two transforms in a real-life-like reverberant scenario.
Our theoretical findings matched the experimental results: the beamformers based on SSBT were slightly less effective than the ones based on STFT in ideal conditions, and significantly less effective in non-ideal scenarios. This emphasizes the limitations of SSBT compared to STFT in practical beamforming.

% ###############

The remainder of this paper is organized as follows: In \cref{sec:stft_and_ssbt}, we introduce the proposed frequency and time-frequency transforms, explain their relationship, and elaborate on their relevant properties. These properties are fully developed in \cref{app:properties_rft}, when proofs are necessary.
\cref{sec:signal_model} presents the considered signal model and how to incorporate the desired constraints while considering the time-frequency transforms at hand, considering their features and peculiarities.
In \cref{sec:results}, we present and discuss the results, comparing the studied filtering approaches for various metrics and situations, each exposing different information regarding their performances.
Finally, \cref{sec:conclusion} concludes this paper.

