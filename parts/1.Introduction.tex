\section{Introduction}
\label{sec:introduction}

Filtering is one of the most important instruments available on the modern world, being necessary for things as communications \cite{chen_source_2002}, signal enhancement \cite{lobato_worst-case-optimization_2020,lu_biomedical_1994,nguyen_minimum_2017}, [...]. At its core, a filter is a mechanism through which one can mitigate the effects of some undesired sources that are contaminating a desired one. This can be implemented by utilizing either temporal information regarding the signals, but also spatial information (also called beamforming) about the signals and also the sources, sensors, and the environment in which these are located. Such filtering processes can be implemented in time-, frequency-, and time-frequency domains \cite{benesty_fundamentals_2017}, each of those having different advantages and use-cases in which they thrive. In particular, time-frequency techniques allow the exploitation of frequency-related information about the signals in question, while also dynamically adapting to signal and environmental changes over time, providing a good tradeoff between the advantages of using strictly time or frequency information. Two important goals in filtering are the distortionless-ness of the desired signal, and the minimization of the noise in the output signal (or the minimization of the output as a whole, assuming non-distortion of the desired portion). Both can be achieved by the Minimum-Power Distortionless-Response (MPDR) beamformer \cite{capon_high-resolution_1969,erdogan_improved_2016}. In order to extract the desired data from time-domain signals usually generated by sensors, the most widely used instruments for this time-frequency analysis are transforms, from which the Short-Time Fourier Transform (STFT) \cite{kiymik_comparison_2005,pan_microphone_2021} stands out in the literature in terms of spread and commonness. However, alternative transforms can also be employed \cite{chen_wavelet-based_2018,yang_general_2014,almeida_fractional_1994}, each having its specific use instances and attributes, and offering unique perspective and information regarding the signal, possibly leading to different performances when fared for a common goal.

%Beamformers are an important tool for signal enhancement, being employed in a plethora of applications from hearing aids \cite{lobato_worst-case-optimization_2020} to source localization \cite{chen_source_2002} to imaging \cite{lu_biomedical_1994,nguyen_minimum_2017}. Among the possible ways to use such devices is to implement them the time-frequency domain \cite{benesty_fundamentals_2017}, which allows the exploitation of frequency-related information while also dynamically adapting to signal changes over time. The most widely used instruments for time-frequency analysis are transforms, from which the Short-Time Fourier Transform (STFT) \cite{kiymik_comparison_2005,pan_microphone_2021} stands out in terms of spread and commonness. However, alternative transforms can also be employed implemented \cite{chen_wavelet-based_2018,yang_general_2014,almeida_fractional_1994}, each offering unique perspective and information regarding the signal, possibly leading to different outputs.

Among these alternatives, an approach based on the Single-Sideband (SSB) modulation \cite{chen_hybrid_2021,xing_single_2017}, fittingly titled Single-Sideband Transform (SSBT), has been occasionally proposed \cite{crochiere_multirate_1983,oyzerman_speech_2012,oyzerman_system_2012}, being employed for different applications from acoustic echo cancellation \cite{chin_subband_2001} to machine learning signal enhancement \cite{okamoto_subband_2017}, with varying levels of success. This transform is of great interest given its real-valued frequency spectrum, which favors the implementation in simpler hardware, possibly leading to cheaper and more cost-effective devices. By using a convolutive transfer function (CTF) model \cite{talmon_relative_2009} to study the desired signal model, the SSBT could lend itself useful in a spatio-temporal multisensor beamforming filtering. However, it is important to be aware of the limitations of the transform, in order to properly utilize it to try and achieve better outputs. This transform lacks a deeper examination, its properties have not been thoroughly studied and scrutinized, and its features and limitations aren't currently known and available on the same level as others in the literature are, thus complicating its appropriate application and utilization, with present knowledge about it.

%Two of the most important goals in beamforming are the minimization of noise in the output signal, and the distortionless-ness of the desired signal, both being achieved by the Minimum-Variance Distortionless-Response (MVDR) beamformer \cite{capon_high-resolution_1969,erdogan_improved_2016}. As the MVDR beamformer can be used on the time-frequency domain without restrictions on the transform chosen, it is possible to explore and compare the performance of this filter, when designing it through different time-frequency transforms.

We start this paper by exploring the SSB transform, analyzing a continuous-frequency analogue to it in order to better explore its properties and drawing comparisons to those of the Fourier Transform (FT), which is the continuous-frequency parallel to the STFT. Equipped with a proper mathematical foundation to this new transform, we then approach the problem at hand, this being the design of an MPDR filter with the CTF model under the time-frequency domain on an echoic environment, doing so for both the STFT and SSBT transforms, and again drawing mathematical similarities between both of them. In order to fare them, we use a real-life-like scenario in which these filters are to be employed, and compare them through metrics that extract relevant information either regarding the beamforming design chosen, but also the environment and signals at hand. Our theoretical results show that the SSBT beamformer has a higher degree of mathematical complexity, needing more constraints to achieve the same desired effects, and being more susceptible to modeling errors and statistical imperfections. Furthermore, our practical simulation results are in agreement with these theoretical statements, consolidating its drawbacks when compared to the widely known STFT. In a perfect and ideal case the SSBT is only slightly worse than the STFT, where one could argue that its advantage of being real-valued compensates for a marginally worse performance, but on a scenario closer to reality the SSBT's performance plummets, while the STFT's results are still acceptable.

We organized the paper as follows:
in \cref{sec:stft_and_ssbt} we introduce the proposed time-frequency transforms, how they're related and what are their relevant properties (which are expanded and fully developed in \cref{app:properties_rft});
\cref{sec:signal_model} the considered signal model is presented, and how to write the desired constraints in considering the time-frequency transforms at hand, taking into account their features and peculiarities.
In \cref{sec:results} we present and discuss the results, comparing the studied methods and beamformers obtained for a varied range of metrics, each exposing different information regarding their performances.
\cref{sec:conclusion} concludes this paper.