\section{Introduction}
\label{sec:introduction}

Beamformers are an important tool for signal enhancement, being employed in a plethora of applications from hearing aids \cite{lobato_worst-case-optimization_2020} to source localization \cite{chen_source_2002} to imaging \cite{lu_biomedical_1994,nguyen_minimum_2017}. Among the possible ways to use such devices is to implement them the time-frequency domain \cite{benesty_fundamentals_2017}, which allows the exploitation of frequency-related information while also dynamically adapting to signal changes over time. The most widely used instruments for time-frequency analysis are transforms, from which the Short-Time Fourier Transform (STFT) \cite{kiymik_comparison_2005,pan_microphone_2021} stands out in terms of spread and commonness. However, alternative transforms can also be employed implemented \cite{chen_wavelet-based_2018,yang_general_2014,almeida_fractional_1994}, each offering unique perspective and information regarding the signal, possibly leading to different outputs.

Among these alternatives, the Single-Sideband Transform (SSBT) \cite{crochiere_multirate_1983,oyzerman_speech_2012} is of great interest, given its real-valued frequency spectrum. It has been shown that the SSBT works particularly well with short analysis windows \cite{crochiere_multirate_1983}. Therefore, if we use the convolutive transfer function (CTF) model \cite{talmon_relative_2009} to study the desired signal model, the SSBT can lend itself to be useful, if we think about the beamforming process through the lenses of filter-banks \cite{kumatani_filter_2008,gopinath_tutorial_1993}. Thus, by applying this transform within this context it is possible to pull off superior performances than only with the STFT. However, it is important to be aware of the limitations of the transform, in order to properly utilize it to try and achieve better outputs.

Two of the most important goals in beamforming are the minimization of noise in the output signal, and the distortionless-ness of the desired signal, both being achieved by the Minimum-Variance Distortionless-Response (MVDR) beamformer \cite{capon_high-resolution_1969,erdogan_improved_2016}. As the MVDR beamformer can be used on the time-frequency domain without restrictions on the transform chosen, it is possible to explore and compare the performance of this filter, when designing it through different time-frequency transforms.

Motivated by this, our paper explores the SSB transform and its application on the subject of beamforming within the context of the CTF model. We propose an approach for the CTF that allows the separation of desired and undesired speech components for reverberant environments, and employ this approach for designing the MVDR beamformer. We also explore the traits and limitations of the SSBT, and how to properly adapt the MVDR beamformer to this new transform's constraints. We show that a beamformer designed using the SSBT can surpass the STFT one, while also conforming to the distortionless constraint.

We organized the paper as follows:
in \cref{sec:stft_and_ssbt} we introduce the proposed time-frequency transforms, how they're related and what are their relevant properties;
\cref{sec:signal_model} the considered signal model in the time domain is presented, and how it is transferred into the time-frequency domain;
and in \cref{sec:true_mvdr_ssbt} we develop a true-MVDR beamformer with the SSBT, taking into account its features.
In \cref{sec:results} we present and discuss the results, comparing the studied methods and beamformers obtained.
\cref{sec:conclusion} concludes this paper.