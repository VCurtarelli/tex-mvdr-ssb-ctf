\section{Introduction}
\label{sec:introduction}

% ###############

Filtering is one of the essential instruments available in the modern world, being necessary in fields as communications \cite{chen_source_2002}, biomedical \cite{lobato_worst-case-optimization_2020,lu_biomedical_1994,nguyen_minimum_2017}, system control \cite{han_comparative_2016, hagglund_signal_2012}, among other areas \cite{hathcock_noise_2016,lee_general_2002,shi_new_2004}. At its core, filtering mitigates the effects of undesired sources, and can use temporal and/or spatial information (usually called beamforming \cite{dougherty_functional_2014}) regarding the signals, sources, sensors, and environment to enhance the signals. Filtering processes can be implemented in time-, frequency-, and time-frequency domains \cite{benesty_fundamentals_2017}, each offering distinct advantages. In particular, time-frequency methods exploit frequency-related signal information while dynamically adapting to signal and environmental changes, providing a tradeoff between strictly time or frequency information. Transforms are the most used tool to achieve time desired time-domain data, with the Short-Time Fourier Transform (STFT) \cite{kiymik_comparison_2005,pan_microphone_2021} standing out in the literature in terms of spread and commonness. However, alternative transforms can also be employed \cite{chen_wavelet-based_2018,yang_general_2014,almeida_fractional_1994}, each offering unique perspective and insight regarding the signal, leading to different performances depending on specific application requirements.

% ###############

Among these alternatives, an approach based on the Single-Sideband (SSB) modulation has been occasionally proposed  \cite{crochiere_multirate_1983} for various applications from acoustic echo cancelation \cite{chin_subband_2001} to speech dereverberation \cite{oyzerman_system_2012} and machine learning signal enhancement \cite{okamoto_subband_2017}, demonstratring varying levels of success. The Single-Sideband Transform (SSBT), characterized by its real-valued frequency spectrum, holds promise for simpler hardware implementations, potentially leading to more cost-efficient devices. Under a convolutive transfer function (CTF) model for the time-frequency transform \cite{talmon_relative_2009}, the SSBT shows potential for spatiotemporal multisensor beamforming. However, a deeper understanding of its properties is crucial to ensure accurate and meaningful outputs. Presently the SSBT lacks comprehensive scrutiny in the literature, its features and limitations not being known on the same level as others in the literature, complicating its appropriate application and utilization.

% ###############

We begin by examining a continous-frequency analogue to the SSBT, analyzing its properties, and comparing them to those of the Fourier Transform (FT) and of the Short-Time Fourier Transform (STFT). We study how these properties impact basic beamforming concepts such as the convolution theorem and relative frequency response estimation. While we find that the SSBT is more error-prone and restrictive than the STFT in terms of beamforming design, we also demonstrate their bijective interchangeability. This allows their joint usage, potentially enhancing beamformer design even within the SSBT domain. We fare the two transforms in a real-life-like reverberant scenario, where our theoretical insights align with experimental results: the SSBT-based beamformers are marginally outperformed by the STFT-based ones under ideal conditions, and significantly outclassed in non-ideal scenarios. This highlights the inherent SSBT drawbacks in practical beamforming, when compared to the STFT.

% ###############

We organized the paper as follows:
In \cref{sec:stft_and_ssbt}, we introduce the proposed frequency and time-frequency transforms, how they're related, and what their relevant properties are (which are expanded and fully developed in \cref{app:properties_rft}, in the cases where their proof is necessary);
\cref{sec:signal_model} presents the considered signal model, and how to compile the desired constraints considering the time-frequency transforms at hand, taking into account their features and peculiarities.
In \cref{sec:results}, we present and discuss the results, comparing the studied filtering approaches for a varied range of metrics and situations, each exposing different information regarding their performances.
\cref{sec:conclusion} concludes this paper.