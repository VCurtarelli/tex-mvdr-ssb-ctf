\section{Introduction}
\label{sec:introduction}

Beamformers play a crucial role in diverse fields, such as telecommunications \cite{viswanath_opportunistic_2002}, acoustics \cite{herbordt_joint_2005,chiariotti_acoustic_2019}, hearing aids \cite{haykin_handbook_2009}, and others \cite{van_veen_beamforming_1988,liu_wideband_2010,huang_energy_2020,elbir_twenty-five_2023}. Among the array configurations used for beamforming, rectangular arrays are an interesting option to be explored \cite{gu_efficient_2019,zhang_two-dimensional_2019,lin_secrecy-energy_2021} as they offer distinct advantages over linear arrays, providing enhanced spatial information regarding impinging sources \cite{heidenreich_joint_2012,ioannides_uniform_2005} and reduced redundancy due to their asymmetry \cite{singh_minimal_2021}.

The development of robust adaptive beamformers with frequency-invariant characteristics has been a significant point of interest since, in this case, frequency does not affect the behavior of the beamformer. Some desired features are constant-beamwidth \cite{goodwin_constant_1993} and null steering \cite{zarifi_collaborative_2010}. One approach for null steering is using the linearly constrained minimum variance (LCMV) beamformer \cite{frost_algorithm_1972,buckley_spatialspectral_1987,souden_study_2010}, which cancels interfering signals from given directions and steers the main beam toward the desired signal. However, it lacks a robust mechanism for maintaining a constant beamwidth. On the other hand, constant-beamwidth (CB) beamforming \cite{hixson_widebandwidth_1970,goodwin_constant_1993,wang_constant-beamwidth_2004} can be accomplished by using window-based beamforming techniques \cite{long_window-based_2019}, but this method cannot incorporate directional restrictions. CB-LCMV beamformers have been explored recently \cite{frank_constant-beamwidth_2022-1}, however only in the context of linear sensor arrays, leaving space for their exploration in the context of different array configurations.

The delay-and-sum (DS) and superdirective (SD) beamformers {\cite{benesty_microphone_2008}} can be used to increase white noise gain or directivity factor, respectively, by maximizing the desired metric. Another quality that is often required for designing a beamformer is a distortionless response to the desired source or to the desired-source direction. This ensures that the desired signal is unaltered by the filtering processes.

While constructing a beamformer with multiple beamforming features is non-trivial, efforts have been made to combine multiple beamforming techniques for a single array of sensors. Two notable approaches are the Kronecker-product (KP) method \cite{abramovich_iterative_2010,werner_estimation_2008} and the linear convolutional Kronecker product (LCKP) method \cite{frank_constant-beamwidth_2022-1}. The LCKP method is valid only for linear arrays. However, it allows for the virtual utilization of more sensors than are physically available \cite{frank_constant-beamwidth_2022-1}. Meanwhile, the KP method can be used in linear or rectangular arrays, but it does not increase the number of sensors available for beamforming. For both combination methods (KP and LCKP), beampattern features and distortionless constraints are conserved in the combined beamformers. Therefore, by using beamformers with desired beampattern characteristics that respect the distortionless constraint, these are maintained in the end result.

In this paper, we propose a novel approach to constructing a CB-LCMV beamformer for rectangular arrays.
For such, we generalize the LCKP beamforming technique to the case of rectangular arrays. We synthesize beamformers for virtual subarrays and, with our proposed generalized technique, apply them to a full array to achieve the desired beamwidth and null placement. This is achieved without sacrificing white noise gain or directivity factor. The performance of the proposed method is compared against beamformers obtained through the KP and LCKP methods. Our results demonstrate superior performance in terms of beamwidth, white noise gain, and directivity for the beamformers obtained using the proposed method when compared to the literature.

This paper is organized as follows: \cref{sec:signal_model} presents the array and signal model considered for the problem;
%\cref{sec:kp_beamformers} is an overview of the literature methods for array analysis with the Kronecker product;
\cref{sec:conv_beamformers} shows the traditional beamforming techniques and methods that will be used further.
In \cref{sec:cblcmv_beamformer}, the newly proposed method for array analysis for rectangular arrays is introduced and detailed, as well as its proposed usage for solving the problem at hand. In \cref{sec:simulations}, we present the simulations realized and discuss the results, comparing them to the literature. Finally, \cref{sec:conclusions} concludes this paper, overviewing the main contributions.