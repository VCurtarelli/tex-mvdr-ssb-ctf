\section{Introduction}
\label{sec:introduction}

Filtering is one of the essential instruments available in the modern world, being necessary in fields as communications \cite{chen_source_2002}, biomedical \cite{lobato_worst-case-optimization_2020,lu_biomedical_1994,nguyen_minimum_2017}, system control \cite{han_comparative_2016, hagglund_signal_2012}, among other areas. At its core, a filter is a mechanism through which one can mitigate the effects of some undesired sources contaminating a desired one. This can be implemented by utilizing temporal and spatial information (usually called beamforming) about the signals and the sources, sensors, and environment in which these are located. Such filtering processes can be implemented in time-, frequency-, and time-frequency domains \cite{benesty_fundamentals_2017}, each with different advantages and use cases in which they thrive. In particular, time-frequency techniques allow the exploitation of frequency-related information about the signals in question while also dynamically adapting to signal and environmental changes over time, providing a good tradeoff between the advantages of strictly time or frequency information. Two crucial goals in filtering are avoiding distortion of the desired signal and minimizing the noise in the output signal (or the minimization of the output as a whole, assuming non-distortion of the desired portion). Both can be achieved by the Minimum-Power Distortionless-Response (MPDR) beamformer \cite{capon_high-resolution_1969,erdogan_improved_2016}. To extract the desired data from time-domain signals usually generated by sensors, the most widely used instruments for this time-frequency analysis are transforms, from which the Short-Time Fourier Transform (STFT) \cite{kiymik_comparison_2005,pan_microphone_2021} stands out in the literature in terms of spread and commonness. However, alternative transforms can also be employed \cite{chen_wavelet-based_2018,yang_general_2014,almeida_fractional_1994}, each having its specific use instances and attributes and offering unique perspective and information regarding the signal, possibly leading to different performances when fared for a common goal.

Among these alternatives, an approach based on the Single-Sideband (SSB) modulation \cite{chen_hybrid_2021,xing_single_2017}, fittingly titled Single-Sideband Transform (SSBT), has been occasionally proposed \cite{crochiere_multirate_1983,oyzerman_system_2012}, being employed for different applications from acoustic echo cancellation \cite{chin_subband_2001} to machine learning signal enhancement \cite{okamoto_subband_2017}, with varying levels of success. Given its real-valued frequency spectrum, this transform is of great interest, which favors simpler hardware, possibly leading to cheaper and more cost-efficient devices. Using a convolutive transfer function (CTF) model \cite{talmon_relative_2009} to study the desired signal model, the SSBT could be helpful for spatiotemporal multisensor beamforming filtering. However, it is essential to be aware of the transform's limitations in order properly achieve meaningful and correct outputs by utilizing it. Presently this transform lacks a more profound examination, its properties not having been thoroughly studied and scrutinized and its features and limitations not being known on the same level as others in the literature, thus complicating its appropriate application and utilization.

We start this paper by exploring the SSB transform, analyzing a continuous-frequency analogue to better explore its properties, and drawing comparisons to those of the Fourier Transform (FT), which is the continuous-frequency parallel to the STFT. Equipped with a proper mathematical foundation for this new transform we then approach the problem at hand, this being the design of a multi-sensor Minimum Power Distortionless Response (MPDR) filter with the CTF model under the time-frequency domain on an echoic environment, doing so for both the STFT and SSBT transforms, and again drawing mathematical similarities between both of them. To fare them, we use a real-life-like scenario in which these filters are to be employed, and compare them through metrics that extract relevant information regarding the chosen beamforming design and the environment and signals at hand. Our theoretical results show that the SSBT beamformer has a higher degree of mathematical complexity, needs more constraints to achieve the same desired effects, and is more susceptible to modeling errors and statistical imperfections. Furthermore, our practical simulation results agree with these theoretical statements, consolidating its drawbacks compared to the widely known STFT. In a perfect and ideal case, the SSBT is only slightly worse than the STFT, where one could argue that its real-valued-spectrum advantage compensates for a marginally worse performance, but on a scenario closer to reality, the SSBT's performance plummets. In this same scenario, the STFT's results are still acceptable.

We organized the paper as follows:
In \cref{sec:stft_and_ssbt}, we introduce the proposed frequency and time-frequency transforms, how they're related, and what their relevant properties are (which are expanded and fully developed in \cref{app:properties_rft}, in the cases where their proof is necessary);
\cref{sec:signal_model} presents the considered signal model, and how to compile the desired constraints considering the time-frequency transforms at hand, taking into account their features and peculiarities.
In \cref{sec:results}, we present and discuss the results, comparing the studied filtering approaches for a varied range of metrics and situations, each exposing different information regarding their performances.
\cref{sec:conclusion} concludes this paper.