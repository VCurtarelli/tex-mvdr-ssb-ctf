\section{Conventional Beamformers}
\label{sec:conv_beamformers}
This section briefly overviews the beamforming techniques used to construct the proposed beamformer, as well as the different methods for beamformer synthesis.

\subsection{LCMV beamformer}
\label{ssec:defs:lcmv_beamformer}

Assuming the presence of undesired sources in known directions, the linearly constrained minimum variance (LCMV) beamformer \cite{frost_algorithm_1972,buckley_spatialspectral_1987,souden_study_2010} is useful to position nulls of the beampattern in those undesired directions. For such, we assume the existence of $N$ (with $N < M$) uncorrelated interfering sources in the far-field, each coming from a (different) direction $\t_i$ ($i \in \set{1 ,, N}$) that we want to cancel. We write $\bvv$ as
\begin{equation}
    \bvv = \sum_{n=1}^{N} \bvd{\t_n} \nu_{n} + \bvu \eqc
\end{equation}
where $\nu_{n}$ is the noise signal for the $n$-th undesired direction, and $\bvu$ is the portion of the noise signal not coming from the $N$ undesired directions, also accounting for acoustic uncorrelated noise. We assume that $\expec{\he{\bvu} \pts{\bvd{\t_n}\nu_{n}}} = 0$. We then use $N+1$ linear constraints, representing the distortionless constraint for the desired signal plus the canceling of the $N$ undesired directions.

The LCMV constraint is written in matrix form as
\begin{subgather}{eq:def_constraint_Ch_q}
    \bv{C} = \tup{ \bvd, \bvd{\t_1} ,, \bvd{\t_{N}} } \eqc \\
    \bv{q} = \tr{ \tup{ 1 , 0 ,, 0 } } \eqc \\
    \he{\bv{C}}\bvh = \bv{q} \eqc
\end{subgather}
where $\bv{C}(\w)$ is an $\sz{M}{(N+1)}$ matrix,  $\bv{q}$ is an $\sz{(N+1)}{1}$ vector, and the superscript $\tr{}$ denotes the transpose operator. The LCMV beamformer is obtained by minimizing the variance of residual noise $\bvv{\rn}$ in the beamformer output (from \cref{eq:Zw_beamformer_output}), given the constraints in \eqref{eq:def_constraint_Ch_q}, which translates to a minimization problem given as
\begin{equation}\label{eq:argmin_lcmv}
    \bvh{\lcmv} = \argmin{ \he{\bvh} \Corr{\bvv} \bvh }{ \bvh }{\he{\bv{C}} \bvh = \bv{q} } \eqc
\end{equation}
where $\Corr{\bvv}(\w)=\expec{\bvv(\w) \he{\bvv}(\w)}$ is the correlation matrix of $\bvv$, assumed to be a full-rank invertible matrix. As the undesired directions will be canceled (assuming an anechoic environment \cite{markovich-golan_combined_2017}), this minimization process minimizes $\bvu$, the noise portion that is uncorrelated to the $N$ undesired directions. The solution to this minimization problem is
\begin{equation}\label{eq:solution_lcmv_MP_inverse}
    \bvh{\lcmv} = \inv{\Corr{\bvv}} \bv{C} \inv*{ \he{\bv{C}} \inv{\Corr{\bvv}} \bv{C} } \bv{q} \eqp
\end{equation}
To ensure the existence of a solution, the number of sensors should be larger than or equal to the number of constraints, i.e., $M \geq N+1$. For $N=0$, only the distortionless constraint remains, and the LCMV beamformer reduces to the minimum variance distortionless response (MVDR) beamformer \cite{erdogan_improved_2016}, which is given by
\begin{equation}
    \bvh{\mvdr} = \frac{\inv{\Corr{\bvv}} \bvd}{\he{\bvd}\inv{\Corr{\bvv}} \bvd } \eqp
\end{equation}

It is possible to show that the LCMV and MVDR beamformers are also defined in terms of the observed signal correlation matrix $\Corr{\bvy}$ \cite{benesty_microphone_2008}, being defined as
\begin{subgather}
    \bvh{\lcmv} = \inv{\Corr{\bvy}} \bv{C} \inv*{ \he{\bv{C}} \inv{\Corr{\bvy}} \bv{C} } \bv{q} \label{eq:solution_lcmv_Corry} \\
    \bvh{\mvdr} = \frac{\inv{\Corr{\bvy}} \bvd}{\he{\bvd}\inv{\Corr{\bvy}} \bvd } \eqp
\end{subgather}
This formulation depends only on the statistics of the observed signal, which are easier to compute than those of the noise signal.

%The resulting beamformer has at least $N$ nulls in the directions $\t_i$ and at most $M-1$ nulls. In order to not have unaccounted nulls, we set $M = N+1$ to guarantee the existence of (and only of) $N$ nulls in the LCMV beamformer's beampattern.

\subsection{CB beamformer}
\label{ssec:defs:cb_beamformer}

Constant-Beamwidth (CB) beamformers guarantee a certain beamwidth around the desired direction that is constant over frequency. This is important to ensure the correct receiving of the desired signal, even if $\td$ is not precisely calibrated.

We define $\tB$ as the first-null beamwidth (FNBW), such that $\abs{\beam{\bvh,\bvd}} > 0$ if $\abs{\t - \td} < \nfrac{\tB\,}{\,2}$.  That is, $\nfrac{\tB\,}{\,2}$ is the first angle in which a null of the beampattern occurs. A constant-beamwidth beamformer can be achieved using a window-based design technique \cite{long_window-based_2019}. Here, the Kaiser window is used \cite{kaiser_use_1980}, which can be written as
\begin{equation}
    \el{\bvw}[m] = \frac{J_0\pts{ \beta\Sqrt{ 1 - \bts{ \frac{2m}{M-1} - 1 }^2 } }}{ J_0(\beta) } \eqc
    \label{eq:calcCB}
\end{equation}
where $J_0(\cdot)$ is the zero-order modified Bessel function of the first kind, and $\beta(\w)$ \eref{long_window-based_2019}{sec3.3.2} is frequency-dependent to maintain $\tB$ constant. This technique requires that the desired source signal is impinging on the array from the broadside direction \cite{long_window-based_2019}. To satisfy the distortionless constraint, we normalize $\bvw$, obtaining
\begin{equation}
    \bvh{\cb} =  \frac{\bvw}{\sum_{m=0}^{M-1}\el{\bvw}[m]}
    \eqp
\end{equation}

\subsection{SD and DS beamformers}
\label{ssec:defs:sd_ds_beamformers}

The superdirective (SD) and delay-and-sum (DS) beamformers are obtained by maximizing the DF and the WNG, respectively \cite{brandstein_microphone_2001,erdogan_improved_2016}, both subject to the distortionless constraint. The solutions to these minimization problems are respectively given by
\begin{subgather}
    \bvh{\sd} = \frac{ \inv{\bvG} \bvd }{ \he{\bvd} \inv{\bvG} \bvd } \label{eq:solution_sd} \eqc \\
    \bvh{\ds} = \frac{\bvd}{M} \label{eq:solution_ds} \eqp
\end{subgather}

\subsection{Kronecker-product beamforming}
\label{ssec:kp_beamformers}

Designing a single beamformer with different features is highly important, allowing different effects to be employed over a single array of sensors. Two methods to accomplish such task are the Kronecker product (KP) \cite{abramovich_iterative_2010,werner_estimation_2008} and linear convolutional Kronecker product (LCKP) methods \cite{frank_constant-beamwidth_2022-1}. In these processes, the sensor array is split into subarrays for which we design separate beamformers, and use the chosen technique to synthesize the whole sensor array (or full-array) beamformer.

% \subsection{KP beamforming}
% 
The KP beamforming process is as follows: given a steering vector $\bvd{\t}$, we decompose it into two parts (namely $\bvd{1;\t}$ and $\bvd{2;\t}$) satisfying the relation $\bvd{\t}= \bvd{1;\t} \kp \bvd{2;\t}$, where $\kp$ represents the Kronecker product. By designing beamformers $\bvh{1}$ and $\bvh{2}$ for $\bvd{1}$ and $\bvd{2}$, respectively, we obtain the beamformer for the full-array as $\bvh = \bvh{1} \kp \bvh{2}$ \cite{huang_robust_2020}.

% \subsection{LCKP beamforming}
% 
LCKP beamforming is achieved similarly: given a uniform linear array's (ULA) steering vector $\bvd{\t}$ of length $M$, we define $\bvd{1;\t}$ as the $M_1$-th first elements of $\bvd{\t}$, and similarly $\bvd{2;\t}$ with $M_2$ elements; respecting $M_1 + M_2 - 1 = M$. By designing beamformers $\bvh{1}$ and $\bvh{2}$ for each subarray, the full-array's beamformer is $\bvh = \bvh{1} \conv \bvh{2}$ \cite{frank_constant-beamwidth_2022-1}, where $\conv$ denotes the linear convolution.

% \subsection{Properties}
% 
For both the KP and LCKP methods, we have  the following properties:
\begin{subgather}{eq:properties_beam_dsdi}
	\label{eq:separation_beampattern}%
    \beam{\bvh,\bvd{\t}} = \beam{\bvh{1},\bvd{1;\t}} \beam{\bvh{2},\bvd{2;\t}} \eqc \\
	\label{eq:ineq_separation_dsdi}%
    \dsdi{\bvh,\bvd} \leq \bts{1 + \dsdi{\bvh{1}, \bvd{1}}} \bts{1 + \dsdi{\bvh{2}, \bvd{2}}} - 1 \eqp
\end{subgather}
The first one shows that the beampattern of $\bvh$ is the combination of the beampatterns for the subarrays. Through the second one, we can see that if $\bvh{1}$ and $\bvh{2}$ are distortionless beamformers, so will $\bvh$ be. From these properties, we can see that the beampattern properties (including the distortionless feature) from $\bvh{1}$ and $\bvh{2}$ are maintained for $\bvh$.