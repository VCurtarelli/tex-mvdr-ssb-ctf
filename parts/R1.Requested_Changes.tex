%TC:ignore
\newpage
\section*{Requested Changes - Review 1}
\begin{reviews}
    \footnotesize
    \newreviewer
    \review{11/37 are self references. Please, reduce it.}{Done, reduced to 3 references.}
    %
    \review{You haven't explained heatmap consequences.}{I didn't understand this comment.}
    %
    \review{Figures 1 and 2 are too far from the comments in text. That makes it hard to follow.}{Done, reorganized figures.}
    %
    \review{Some abbreviations are not listed in the "Abbreviations", such as ULA.}{Done, added missing abbreviations.}
    %
    \review{You should improve presentation of your method, maybe with some flow chart.}{There are algorithms in Appendix B for both the proposed method and proposed beamformer.}
    
    \newreviewer
    %
    \review{[...] Explain how to avoid that the other three virtual beampatterns destroy the constant-beamwidth [...]}{Done, added context and information about this.}
    %
    \review{[...] Should also show and discuss some results for the elevation angle.}{We assumed the elevation to be $0\dg$ as it is unusual, in speech processing, for the desired and undesired sources to have the same azimuth, case in which the elevation would be relevant.}
    %
    \review{An explanation about the 2D beampattern design (azimuth and elevation, jointly) would also be useful.}{Done, added some minimal context about the elevation angle.}
    %
    \review{Since the central frequency is several MHz or GHz, the bandwidth should be much larger, I think, at least 10 or 100 times larger.}{The only beamformer that has restrictions on the frequency is the CB beamformer, for which the minimum and maximum frequencies, for Condition [A], are \cite{long_window-based_2019} $f_L = 4.2$ kHz and $F_H = 11.4$ kHz. We used $4$ kHz and $8$ kHz to keep in line with \cite{frank_constant-beamwidth_2022-1}.}
    %
    \review{``kHz", not ``k Hz".}{Done, fixed units notation.}
    
    \newreviewer
    %
    \review{Move fig. 2 before the Conclusions section.}{Done, reorganized figures.}
    %
    \review{In the simulation you consider frequency range 4-8KHz. Can you give a good reason for this choice? [...]}{This range was chosen to keep in line with the literature, more specifically \cite{long_window-based_2019}.}
    %
    \review{In Introduction section you write: “frequency does not affect the behavior of the beamformer” (line 18-20). In this case considering a wide frequency range in simulation would be benefic.}{The range was chosen to satisfy the conditions for the CB beamformer \cite{long_window-based_2019}.}
    %
    \review{``kHz", not ``k Hz".}{Done, fixed units notation.}
\end{reviews}
%TC:endignore