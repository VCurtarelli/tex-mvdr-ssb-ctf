\section{Correct calculation of steering vectors in SSBT}
\DeclareDocumentCommand{\E}{s m}{%
	\mathrm{E}_{#2}\IfBooleanTF{#1}{'}{^{}}
}
\def\ar{\alpha_{\re}}
\def\ai{\alpha_{\im}}
\def\br{\beta_{\re}}
\def\bi{\beta_{\im}}
Assume we have two LTI systems with impulse responses $h_m(t)$, with a singular input $x(t)$. In the RFT domain,
\begin{equation}
	X_{m\R}(f) = X_{\R}(f) \Re{H_{m\F}}(f) + X_{\R}(\-f) \Im{H_{m\F}}(f)
\end{equation}

Through the properties of the RFT exposed in \cref{app:properties_rft}, we can build the following system of equations:
\begin{subalign}{eqs:appB:system_equations}
	\expec{X_{1\R}(f) X_{1\R}(f)}   & = \pts{  {\Re{H_{1\F}}}^2(f)             + {\Im{H_{1\F}}}^2(f)         } \sigma_{X}^2(f) \label{eq:appB:system_equations:subeq1} \\
	\expec{X_{1\R}(f) X_{2\R}(f)}   & = \pts{   \Re{H_{1\F}}(f) H_{2\F}^\re(f) +  \Im{H_{1\F}} H_{2\F}^\im(f)} \sigma_{X}^2(f) \label{eq:appB:system_equations:subeq2} \\
	\expec{X_{1\R}(f) X_{2\R}(\-f)} & = \pts{\- \Re{H_{1\F}}(f) H_{2\F}^\im(f) +  \Im{H_{1\F}} H_{2\F}^\re(f)} \sigma_{X}^2(f) \label{eq:appB:system_equations:subeq3} \\
	\expec{X_{2\R}(f) X_{2\R}(f)}	& = \pts{  {\Re{H_{2\F}}}^2(f)             + {\Im{H_{2\F}}}^2(f)         } \sigma_{X}^2(f) \label{eq:appB:system_equations:subeq4}
\end{subalign}
where $H_{m\F}(f) = \Re{H_{m\F}}(f) + \j\Im{H_{m\F}}(f)$ is the FT of $h_m(t)$, and $\sigma_X^2(f) = \expec{X_{\R}(f)^2}$.

Note that, we have 4 equations and 5 variables. Our objective is to find $A_{m,\R}'(f)$ and $A_{m,\R}''(f)$ such that
\begin{subgather}{eqs:appB:rtfs_in_rft_domain}
	A_{m,\R}'(f) = \frac{\Re{H_{m\F}}(f)}{\Re{H_{1\F}}(f)} \\
	A_{m,\R}''(f) = \frac{\Im{H_{m\F}}(f)}{\Im{H_{1\F}}(f)}
\end{subgather}
where we're not worried about negative frequencies, since from \cref{prop:rtfs_are_even-func_of_frequency} we know that $A_{m,\R}'(f) = A_{m,\R}'(\-f)$ and $A_{m,\R}''(f) = A_{m,\R}''(\-f)$. We rewrite the equations in \cref{eqs:appB:system_equations} by defining
\begin{equations}
	\ar & 	= \Re{H_{1\F}}(f) \sigma_X(f) \\
	\ai & 	= \Im{H_{1\F}}(f) \sigma_X(f) \\
	\br & 	= \Re{H_{2\F}}(f) \sigma_X(f) \\
	\bi & 	= \Im{H_{2\F}}(f) \sigma_X(f)
\end{equations}
\begin{equations}
	\E{11} 	& 	= \expec{X_{1\R}(f) X_{1\R}(f)} \\
	\E{12}	& 	= \expec{X_{1\R}(f) X_{2\R}(f)} \\
	\E*{12} &   = \expec{X_{1\R}(f) X_{2\R}(\-f)} \\
	\E{22}	&	= \expec{X_{2\R}(f) X_{2\R}(f)}
\end{equations}

Where we will be omitting the $(f)$ index for clarity in these newly defined variables, however they are frequency-dependent. With this, \cref{eqs:appB:system_equations} becomes
\begin{subalign}
	\ar^2 + \ai^2 		& = \E{11} \label{eq:appB:system_equations_simp:subeq1} \\
	\ar \br + \ai \bi 	& = \E{12}\label{eq:appB:system_equations_simp:subeq2} \\
	-\ar \bi + \ai \br 	& = \E*{12} \label{eq:appB:system_equations_simp:subeq3} \\
	\br^2 + \bi^2 		& = \E{22}\label{eq:appB:system_equations_simp:subeq4}
\end{subalign}
and \cref{eqs:appB:rtfs_in_rft_domain} is
\begin{subgather}
	A_{m,\R}'(f) = \frac{\br}{\ar} \\
	A_{m,\R}'(f) = \frac{\bi}{\ai}
\end{subgather}

With this we simplified the problem and got rid of one of the variables, now having a system of four equations and four variables, which is solvable. However, it is a non-linear system, which makes the solution harder to find.

\begin{thought}
	By multiplying \cref{eq:appB:system_equations_simp:subeq2} by $\ar$ and \cref{eq:appB:system_equations_simp:subeq3} by $\ai$ and adding, we get
	\begin{equation}
		\ar^2\br + \ar\ai\bi - \ai\ar\bi + \ai^2\br = \E{12}\ar + \E*{12} \ai
	\end{equation}
	and therefore
	\begin{equation}
		\br = \frac{\E{12}\ar + \E*{12}\ai}{\E{11}}
	\end{equation}
	Similarly, by multiplying \cref{eq:appB:system_equations_simp:subeq2} by $\ai$ and \cref{eq:appB:system_equations_simp:subeq3} by $-\ar$ and adding them, we get
	\begin{equation}
		\ai\ar\br + \ai^2\bi + \ar^2\bi - \ar\ai\br = \E{12}\ai - \E*{12}\ar
	\end{equation}
	and thus
	\begin{equation}
		\bi = \frac{\E{12}\ai - \E*{12}\ar}{\E{11}}
	\end{equation}
	
	However, for this we would have to know $\ar$ and $\ai$, which aren't available.

    Doing the same idea but around (to find $\ar$ and $\ai$ in terms of $\br$ and $\bi$), we find...
    \begin{equation}
        \cdots
    \end{equation}
    With this, we have four linear equations, which can formulate a system as ...
    \begin{equation}
        \cdots
    \end{equation}
    which is of the form $Ax = b$, and therefore the solution is of the form $x = \inv{A} b$. Thus...
\end{thought}