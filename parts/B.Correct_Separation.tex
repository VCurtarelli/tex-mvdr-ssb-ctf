\section{Correct separation of desired signal}
\label{app:derivation_correct_sep}
\def\tn{\tilde{n}}
\def\tnu{\tilde{\nu}}
\def\bn{\bar{n}}
\NewDocumentCommand{\expec}{m}{%
	\textsf{E}\cts{#1}%
}

Let $X_m[l,k]$ be such that
\begin{equations}
	X_m[l,k]
	& = A_m[l,k] \ast X_1[l,k] \\
	& = \tr{\bva{m}}[k] \bvx{1}[l,k]	
\end{equations}
as in \cref{eq:sec3:convert_convol_matmult,eq:sec3:time-freq_model_conv}. We can separate $X_m[l,k]$ as
\begin{equation}
	X_m[l,k] = d_{m}[l,k] X_1[l,k] + X'[l,k]
\end{equation}
where $d_{m}[l,k]$ is the steering vector for the desired speech portion $X_1[l,k]$, and $X'[l,k]$ is the undesired speech component, in such a way that $X_1[l,k]$ and $X'[l,k]$ are uncorrelated.

This seems trivial in a first glance, by adopting $d_m[k]$ as the $0$-th element of $\bva{m}[k]$, and $X'[l,k]$ as the rest of the summation. However, this doesn't take into account that $X[l,k]$ and $X[l',k]$ may be correlated if $\abs{l - l'} \leq S$, where $S = \floor{\nicefrac{(K-1)}{\O}}$. That is, if $l$-th and $l'$-th transform window share samples, then there is some correlation between them.

We define
\begin{equations}{eq:def_rho}
	d_m[l,k]
	& = \frac{\expec{X_1^*[l,k] X_m[l,k]}}{\expec{\abs{X_1[l,k]}^2}} \\
	& = \frac{\sum_{i} A_m[i,k] \expec{X_1^*[l,k] X_1[l-i,k]}}{\expec{\abs{X_1[l,k]}^2}}
\end{equations}

Focusing on each expectation in the numerator,
\begin{equation}
	E_i = \expec{X_1^*[l,k] X_1[l-i,k]}
\end{equation}
Through the definition of the STFT,
\begin{equations}
	E_i
	& = \expec{\sum_{n=0}^{K-1} w(n) x_1(n+l\O) e^{\j2\pi\frac{k}{K}(n+l\O)} \sum_{\nu=0}^{K-1} w(\nu) x_1(\nu+(l-i)\O) e^{-\j2\pi\frac{k}{K}(\nu+(l-i)\O)} } \\
	& = \sum_{n=0}^{K-1} \sum_{\nu=0}^{K-1} w(n) w(\nu) e^{-\j2\pi\frac{k}{K}(\nu - n + i\O)} \expec{x_1(n+l\O) x_1(\nu+(l-i)\O)}
\end{equations}
Using the substitutions $\tn = n+l\O$ and $\tnu = \nu + (l-i)\O$,
\begin{equation}
	E_i = \sum_{\tnu=l\O}^{K-1+l\O} \sum_{\tn=(l-i)\O}^{K-1+(l-i)\O} w(\tn - l\O) w(\tnu - (l-i)\O) \expec{x_1(\tn) x_1(\tnu)} e^{-\j2\pi\frac{k}{K}(\tnu - \tn)}
\end{equation}

We now \emph{assume} $x_1(n)$ is the result of a zero-mean white process, and therefore $x_1(\tn)$ and $x_1(\tnu)$ are independent if $\tn \neq \tnu$ (which isn't true, but will allow us to continue the derivations). Then
\begin{equation}
	E_i = \sum_{\tn} w(\tn - l\O) w(\tn - (l-i)\O) \expec{x_1(\tn)^2}
\end{equation}
Rolling back the substitution $n = \tn - l\O$,
\begin{equation}
	E_i = \sum_{n=0}^{K - 1 -\abs{i\O}} w(n) w(n + i\O) \expec{x_1(n + l\O)^2}
\end{equation}
where the summation takes into account that both windows are finite. Now we at last \emph{assume} that $\expec{x_1(n+l\O)^2} \approx \expec{x_1(l\O)^2}$ for small values of $n$ (such as within at most one window), such that
\begin{equation}
	E_i = \var{x_1}(l\O) \sum_{n=0}^{K - 1 -\abs{i\O}} w(n) w(n + i\O)
\end{equation}

Going back to \cref{eq:def_rho}, and applying the derivations also to the numerator (with $i = 0$),
\begin{equations}
	d_m[l,k]
	& = \frac{\sum_{i} A_m[i,k] \var{x_1}(l\O) \sum_{n=0}^{K - 1 -\abs{i\O}} w(n) w(n + \abs{i\O})}{\var{x_1}(l\O) \sum_{n'=0}^{K - 1} w(n')^2} \\
	& = \frac{\sum_{i} \sum_{n=0}^{K - 1 -\abs{i\O}} A_m[i,k] w(n) w(n + \abs{i\O})}{\sum_{n'=0}^{K - 1} w(n')^2}
\end{equations}
This way, we see that $d_m[l,k] \equiv d_m[k]$ since it doesn't depend on $l$. However, since for the summation over $n$ we need that $K - 1 - \abs{i\O} \geq 0$, this means that $-S \leq i \leq S$, and thus our desired speech signal $X_1[l,k]$ depends on up to the previous (or next) $S = \floor{\nicefrac{(K-1)}{\O}}$ windows.