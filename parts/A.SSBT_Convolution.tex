\section{Properties of the Real Fourier Transform}
\label{app:properties_rft}

The Fourier Transform (FT) is defined as
\begin{equation}
	\label{eq:appA:def_fourier_transform}
	X_{\F}(f) = \int\limits_{-\infty}^{\infty} x(t) e^{-\j 2\pi f t}\dd t
\end{equation}
with an inverse
\begin{equation}
	\label{eq:appA:def_inverse_fourier_transform}
	x(t) = \int\limits_{-\infty}^{\infty} X_{\F}(f) e^{\j 2\pi f t} \dd t
\end{equation}

The Real Fourier Transform (RFT) is defined as
\begin{equation}
	\label{eq:appA:def_real_fourier_transform}
	X_{\R}(f) = \Sqrt{2} \real{\int\limits_{-\infty}^{\infty} x(t) e^{-\j 2\pi f t + \j\frac{3\pi}{4}} \dd t}
\end{equation}
with an inverse
\begin{equation}
	\label{eq:appA:def_inverse_real_fourier_transform}
	x(t) = \Sqrt{2} \real{\int\limits_{-\infty}^{\infty} X_{\R}(f) e^{\j 2\pi f t - \j\frac{3\pi}{4}} \dd f}
\end{equation}

\begin{Property}{The FT and RFT are bijective transformations of one-another.}[\label{prop:FT_RFT_equivalence}]	
	By manipulating \cref{eq:appA:def_real_fourier_transform} we can get
	\begin{equation}\label{eq:propI:propI:manip_Xrf}
		X_{\R}(f) = \Sqrt{2} \real{\pts{\int\limits_{-\infty}^{\infty} x(t) e^{-\j 2\pi f t} \dd t} e^{\j\frac{3\pi}{4}}}
	\end{equation}
	where the term in parenthesis is trivially the FT of $x(t)$ from \cref{eq:appA:def_fourier_transform}. Thus, we get that
	\begin{equations}{eq:propI:equiv_FT_to_RFT}
		X_{\R}(f)
		& = \real{\pts{\Re{X}_{\F}(f) + \j\Im{X}_{\F}(f)} \cdot(-1 + \j)} \\
		& = - \Re{X}_{\F}(f) - \Im{X}_{\F}(f)
	\end{equations}
	Likewise, using that the FT of a real signal is complex-conjugate, such that $X_{\F}(f) = X_{\F}^*(\-f)$, it is easy to see that
	\begin{equation}
		\label{eq:propI:equiv_FT_to_RFT_neg}
		X_{\R}(\-f) = -\Re{X}_{\F}(f) + \Im{X}_{\F}(f)
	\end{equation}
	With this, we have that
	\begin{equations}
		\Sqrt{2}X_{\R}(f) e^{-\j\frac{3\pi}{4}} & = \Re{X}_{\F}(f) + \Im{X}_{\F}(f) + \j\Re{X}_{\F}(f) + \j\Re{X}_{\F}(f) \\
		\Sqrt{2}X_{\R}(\-f) e^{\j\frac{3\pi}{4}} & = \Re{X}_{\F}(f) - \Im{X}_{\F}(f) - \j\Re{X}_{\F}(f) + \j\Re{X}_{\F}(f)
	\end{equations}
	and therefore
	\begin{equations}
		\frac{X_{\R}(f) e^{-\j\frac{3\pi}{4}} + X_{\R}(\-f) e^{\j\frac{3\pi}{4}}}{\Sqrt{2}}
		& = \Re{X}_{\F}(f) + \j\Re{X}_{\F}(f) \\
		& = X_{\F}(f)
	\end{equations}
	Similarly, using \cref{eq:propI:propI:manip_Xrf} we have that
	\begin{equation}
		X_{\R}(f) = \Sqrt{2}\real{X_{\F}(f) e^{\j\frac{3\pi}{4}}}
	\end{equation}
	Using the property of complex numbers that $\real{a} = \tfrac{a+a^*}{2}$, then
	\begin{equation}
		X_{\R}(f) = \frac{X_{\F}(f) e^{\j\frac{3\pi}{4}} + X_{\F}(\-f) e^{-\j\frac{3\pi}{4}}}{\Sqrt{2}}
	\end{equation}
	Since for each $X_{\F}(f)$ there exists one and only one $X_{\R}(f)$, and vice-versa, it is possible to define a bijective transformation $T$ such that
	\begin{equation}
		X_{\F}(f) \hrel{T} X_{\R}(f)
	\end{equation}
\end{Property}

\begin{Property}{The IRFT is the inverse of the RFT.}[\label{prop:RFT_IRFT_inverses}]
	From \cref{prop:FT_RFT_equivalence}, we have that
	\begin{equation}
		\IRFT{X_{\R}(f)} = \frac{X_{\F}(f) e^{\j\frac{3\pi}{4}} + X_{\F}(\-f) e^{-\j\frac{3\pi}{4}}}{\Sqrt{2}}
	\end{equation}
	Substituting this in \cref{eq:appA:def_inverse_real_fourier_transform}, we have
	\begin{equations}
		\IRFT{X_{\R}(f)}
		& = \Sqrt{2} \real{\int\limits_{-\infty}^{\infty} \pts{\frac{X_{\F}(f) e^{\j\frac{3\pi}{4}} + X_{\F}(\-f) e^{-\j\frac{3\pi}{4}}}{\Sqrt{2}}} e^{\j 2\pi f t - \j\frac{3\pi}{4}} \dd f} \\
		& = \real{\int\limits_{-\infty}^{\infty} \pts{X_{\F}(f) e^{\j 2\pi f t} + X_{\F}(\-f) e^{\j 2\pi f t - \j\frac{3\pi}{2}}} \dd f}
	\end{equations}
	The first term expands to the inverse Fourier transform of $X_{\F}(f)$, which is trivially $x(t)$; the second term is the inverse Fourier transform of $X_{\F}(\-f)$, which is from the time reversal property is $x(-t)$. Therefore,
	\begin{equations}
		\IRFT{X_{\R}(f)}
		& = \real{x(t) + x(\-t) e^{-\j\frac{3\pi}{2}}} \\
		& = \real{x(t) + \j x(\-t)} \\
		& = x(t)
	\end{equations}
	thus concluding the proof.
\end{Property}

\begin{Property}{The convolution theorem doesn't apply for the RFT.}[\label{prop:conv_theorem_RFT}]
	Let $h(t)$ be the impulse response of an LIT system, with input $x(t)$. It is trivial that the system's output, $y(t)$, is given by
	\begin{equation}
		y(t) = h(t) \ast x(t)
	\end{equation}
	with $\ast$ being the convolution operator. For the Fourier transform, through the convolution theorem it is trivial that
	\begin{equation}
		Y_{\F}(f) = H_{\F}(f) X_{\F}(f)
	\end{equation}
	Expanding these in terms of real and imaginary parts (omitting the frequency index for clarity),
	\begin{equation}
		\label{eq:propIII:YF_expanded_real_imag}
		Y_{\F} = \Re{H_{\F}} \Re{X_{\F}} + \j \Re{H_{\F}} \Im{X_{\F}} + \j \Im{H_{\F}} \Re{X_{\F}} - \Im{H_{\F}} \Im{X_{\F}}
	\end{equation}
	Now in the RFT domain, with \cref{eq:propI:equiv_FT_to_RFT} we have that
	\begin{equations}
		X_{\R}(f) & = -\Re{X_{\F}}(f) - \Im{X_{\F}}(f) \\
		H_{\R}(f) & = -\Re{H_{\F}}(f) - \Im{H_{\F}}(f)
	\end{equations}
	Assuming the convolution theorem true for the RFT,
	\begin{equation}
		\label{eq:propIII:res_product_of_RFTs}
		Y_{\R} = \Re{H_{\F}} \Re{X_{\F}} + \Re{H_{\F}} \Im{X_{\F}} + \Im{H_{\F}} \Re{X_{\F}} + \Im{H_{\F}} \Im{X_{\F}}
	\end{equation}
	Now, by applying \cref{eq:propI:equiv_FT_to_RFT} on \cref{eq:propIII:YF_expanded_real_imag}, we have
	\begin{equation}
		\label{eq:propIII:res_RFT_of_convol}
		\tilde{Y}_{\R} = - \Re{H_{\F}} \Re{X_{\F}} - \Re{H_{\F}} \Im{X_{\F}} - \Im{H_{\F}} \Re{X_{\F}} + \Im{H_{\F}} \Im{X_{\F}}
	\end{equation}
	where it is explicit that $Y_{\R}(f) \neq \tilde{Y}_{\R}(f)$. Therefore, the RFT of the convolution (\cref{eq:propIII:res_RFT_of_convol}) is not the product of the RFT's of the signals (\cref{eq:propIII:res_product_of_RFTs}), and thus the convolution theorem doesn't hold for the RFT.
\end{Property}

\begin{Property}{There is an equivalent of the convolution theorem for the RFT.}[\label{prop:equivalent_conv_theorem_RFT}]
	From \cref{eq:propIII:res_RFT_of_convol}, we have our objective for the ``convolution theorem''-equivalent for the RFT. From both \cref{eq:propI:equiv_FT_to_RFT,eq:propI:equiv_FT_to_RFT_neg}, we have
	\begin{equations}{eq:propIV:equiv_FT_to_RFT_posneg_XH}
		X_{\R}(f) & = -\Re{X_{\F}}(f) - \Im{X_{\F}}(f) \\
		X_{\R}(\-f) & = -\Re{X_{\F}}(f) + \Im{X_{\F}}(f) \\
		H_{\R}(f) & = -\Re{H_{\F}}(f) - \Im{H_{\F}}(f) \\
		H_{\R}(\-f) & = -\Re{H_{\F}}(f) + \Im{H_{\F}}(f)
	\end{equations}
	We will omit the frequency dependency in the FT values. Taking the possible combinations, we have
	\begin{subalign}
		X_{\R}(f) H_{\R}(f) 	& = \Re{H_{\F}} \Re{X_{\F}} + \Re{H_{\F}} \Im{X_{\F}} + \Im{H_{\F}} \Re{X_{\F}} + \Im{H_{\F}} \Im{X_{\F}} \label{subeq:propIV:mult_XR_HR:++} \\
		X_{\R}(f) H_{\R}(\-f) 	& = \Re{H_{\F}} \Re{X_{\F}} + \Re{H_{\F}} \Im{X_{\F}} - \Im{H_{\F}} \Re{X_{\F}} - \Im{H_{\F}} \Im{X_{\F}} \label{subeq:propIV:mult_XR_HR:+-} \\
		X_{\R}(\-f) H_{\R}(f) 	& = \Re{H_{\F}} \Re{X_{\F}} - \Re{H_{\F}} \Im{X_{\F}} + \Im{H_{\F}} \Re{X_{\F}} - \Im{H_{\F}} \Im{X_{\F}} \label{subeq:propIV:mult_XR_HR:-+} \\
		X_{\R}(\-f) H_{\R}(\-f) & = \Re{H_{\F}} \Re{X_{\F}} - \Re{H_{\F}} \Im{X_{\F}} - \Im{H_{\F}} \Re{X_{\F}} + \Im{H_{\F}} \Im{X_{\F}} \label{subeq:propIV:mult_XR_HR:--}
	\end{subalign}
	Taking the difference between \cref{subeq:propIV:mult_XR_HR:++} and \cref{subeq:propIV:mult_XR_HR:--}, and the sum of \cref{subeq:propIV:mult_XR_HR:+-} and \cref{subeq:propIV:mult_XR_HR:-+}, we have
	\begin{subalign}
		X_{\R}(f) H_{\R}(f) - X_{\R}(\-f) H_{\R}(\-f) & = 2\pts{\Re{H_{\F}} \Im{X_{\F}} + \Im{H_{\F}} \Re{X_{\F}}} \\
		X_{\R}(f) H_{\R}(\-f) + X_{\R}(\-f) H_{\R}(f) & = 2\pts{\Re{H_{\F}} \Re{X_{\F}} - \Im{H_{\F}} \Im{X_{\F}}}
	\end{subalign}
	and therefore, to achieve \cref{eq:propIII:res_RFT_of_convol}, we let
	\begin{equations}
		Y_{\R}(f)
		& = \frac{- X_{\R}(f) H_{\R}(f) + X_{\R}(\-f) H_{\R}(\-f) - X_{\R}(f) H_{\R}(\-f) - X_{\R}(\-f) H_{\R}(f)}{2} \\
		& = X_{\R}(f) \frac{-H_{\R}(f) - H_{\R}(\-f)}{2} + X_{\R}(\-f) \frac{-H_{\R}(f) + H_{\R}(\-f)}{2}
	\end{equations}
	Finally, from \cref{eq:propIV:equiv_FT_to_RFT_posneg_XH}, we achieve
	\begin{equation}
		\label{eq:propIV:convolution_theorem_RFT_result}
		Y_{\R}(f) = X_{\R}(f) \Re{H_{\F}}(f) + X_{\R}(\-f) \Im{H_{\F}}(f)
	\end{equation}
	and, for its conjugate frequency (that is, for $\-f$), we have
	\begin{equations}{eq:propIV:convolution_theorem_RFT_result_neg}
		Y_{\R}(\-f)
		& = X_{\R}(\-f) \Re{H_{\F}}(\-f) + X_{\R}(f) \Im{H_{\F}}(\-f) \\
		& = X_{\R}(\-f) \Re{H_{\F}}(f) - X_{\R}(f) \Im{H_{\F}}(f)
	\end{equations}
\end{Property}

\begin{Property}{Frequencies in the RFT have the same variance as their FT counterpart.}[\label{prop:RFT_same-variance_FT}]
	We now assume that $X_{\F}(f)$ is the transform of a random process, such that its real and imaginary parts are independent and identically distributed with zero mean. Taking the complex correlation of a given frequency in the FT domain,
	\begin{equation}
		\expec{X_{\F}(f) X_{\F}^*(f)} = \expec{\Re{X_{\F}}(f)^2 + \Im{X_{\F}}(f)^2}
	\end{equation}
	Since they are identically distributed, we denote
	\begin{equation}
		\label{eq:propV:same_variance_FT_RI}
		\expec{\Re{X_{\F}}(f)^2} = \expec{\Im{X_{\F}}(f)^2} = \sigma_f^2
	\end{equation}
	and therefore
	\begin{equation}
		\expec{X_{\F}(f) X_{\F}^*(f)} = 2\sigma_f^2
	\end{equation}
	Now in the RFT domain, we take the correlation of a given frequency \cref{eq:propI:equiv_FT_to_RFT},
	\begin{equation}
		\expec{X_{\R}(f)^2} = \expec{\Re{X_{\F}}(f)^2 + 2 \Re{X_{\F}}(f) \Im{X_{\F}}(f) + \Im{X_{\F}}(f)^2}
	\end{equation}
	Using that $\Re{X_{\F}}(f)$ and $\Im{X_{\F}}(f)$ are i.i.d. and zero-mean, the cross terms are zero, and thus
	\begin{equations}
		\expec{X_{\R}(f)^2} & = 2\sigma_f^2 \\
		& = \expec{X_{\F}(f) X_{\F}^*(f)}
	\end{equations}
	It is trivial to see that the same applies for $X_{\R}(\-f)$.
\end{Property}

\begin{Property}{Conjugate frequencies in the RFT domain are independent.}[\label{prop:conjugate-freqs_independent_RFT}]
	We take the same assumptions as those in \cref{prop:RFT_same-variance_FT}. Taking the complex correlation between the two conjugate frequencies, 
	\begin{equation}
		\expec{X_{\F}(f) X_{\F}^*(\-f)} = \expec{\Re{X_{\F}}(f)^2 + 2 \j\Re{X_{\F}}(f) \Im{X_{\F}}(f) - \Im{X_{\F}}(f)^2}
	\end{equation}
	Using that the $\Re{X_{\F}}(f)$ and $\Im{X_{\F}}(f)$ are independent and zero-mean, the cross terms are zero, and with \cref{eq:propV:same_variance_FT_RI} then
	\begin{equation}
		\expec{X_{\F}(f) X_{\F}^*(\-f)} = 0
	\end{equation}
	This result is known, but nonetheless it is useful to show it, since this same procedure will be used for the RFT.
	
	\noindent We now consider \cref{eq:propI:equiv_FT_to_RFT,eq:propI:equiv_FT_to_RFT_neg}. Taking the correlation between the two conjugate frequencies yields
	\begin{equation}
		\expec{X_{\R}(f) X_{\R}(\-f)} = \expec{\Re{X_{\F}}(f)^2 - \Im{X_{\F}}(f)^2}
	\end{equation}
	Under the same assumptions that the real and imaginary parts of $X_{\F}(f)$ are identically distributed, we get the same result as before, where
	\begin{equation}
		\expec{X_{\R}(f) X_{\R}(\-f)} = 0
	\end{equation}
	Note that, with the RFT, we didn't use the complex correlation, since it is real-valued.
	
	\noindent Lastly, we take the correlation between conjugate frequencies of the output of a system according to \cref{eq:propIV:convolution_theorem_RFT_result,eq:propIV:convolution_theorem_RFT_result_neg},
	\begin{equations}
		\expec{Y_{\R}(f) Y_{\R}(\-f)} 
		& = \Re{H_{\F}}(f)^2 \expec{X_{\R}(f) X_{\R}(\-f)} - \Re{H_{\F}}(f) \Im{H_{\F}}(f) \expec{X_{\R}(f)^2} \\
		& + \Re{H_{\F}}(f) \Im{H_{\F}}(f) \expec{X_{\R}(\-f)^2} + \Im{H_{\F}}(f)^2 \expec{X_{\R}(f) X_{\R}(\-f)}
	\end{equations}
	Since $X_{\R}(f)$ and $X_{\R}(\-f)$ are independent and zero-mean, the first and lest terms are zero, and also the other two cancel each other out since $\expec{X_{\R}(f)^2} = \expec{X_{\R}(\-f)^2} = 2\sigma_f^2$. Therefore,
	\begin{equation}
		\expec{Y_{\R}(f) Y_{\R}(\-f)} = 0
	\end{equation}
\end{Property}

\begin{Property}{Relative transfer functions with the RFT are even for frequency-to-frequency and odd for conjugate-frequency.}[\label{prop:rtfs_are_even-odd_on_frequency}]
	First, given two real systems with a shared input $x(t)$, each with an impulse response $h_m(t)$ and an output $y_m(t)$, such that
	\begin{equations}
		Y_{1}(f) = X(f) \Re{H_{1,\F}}(f) + X(\-f) \Im{H_{1,\F}}(f) \\
		Y_{2}(f) = X(f) \Re{H_{2,\F}}(f) + X(\-f) \Im{H_{2,\F}}(f)
	\end{equations}
	and, for their conjugate frequencies,
	\begin{equations}
		Y_{1}(\-f) = X(\-f) \Re{H_{1,\F}}(\-f) + X(f) \Im{H_{1,\F}}(\-f) \\
		Y_{2}(\-f) = X(\-f) \Re{H_{2,\F}}(\-f) + X(f) \Im{H_{2,\F}}(\-f)
	\end{equations}
	where we are omitting the transform index when it is an RFT's domain value, for compactness. If we define $X_1'(f)$ and $X_1''(f)$ as
	\begin{equations}
		X_1'(f) = Y_1(f) \\
		X_1''(f) = Y_1(\-f)
	\end{equations}
	then
	\begin{equations}
		Y_m(f) = A_m'(f) X_1'(f) + A_m''(f) X_1''(f) \\
		Y_m(\-f) =  A_m'(\-f) X_1'(\-f) + A_m''(\-f) X_1''(\-f)
	\end{equations}
	where $A_m'(f)$ and $A_m''(f)$ are given by
	\begin{equations}
		A_m'(f)  = \frac{H_1'(f)H_m'(f) + H_1''(f)H_m''(f)}{{H_1'}^2(f) + {H_1''}^2(f)} \\
		A_m''(f) = \frac{H_1'(f)H_m''(f) - H_1''(f)H_m'(f)}{{H_1'}^2(f) + {H_1''}^2(f)}
	\end{equations}
	and, for their conjugate frequencies,
	\begin{equations}
		A_m'(\-f)  = \frac{H_1'(\-f)H_m'(\-f) + H_1''(\-f)H_m''(\-f)}{{H_1'}^2(\-f) + {H_1''}^2(\-f)} \\
		A_m''(\-f) = \frac{H_1'(\-f)H_m''(\-f) - H_1''(\-f)H_m'(\-f)}{{H_1'}^2(\-f) + {H_1''}^2(\-f)}
	\end{equations}
	Assuming that $h(t)$ is real-valued, then $H_{m,\F}(f) = H_{m,\F}^*(\-f)$, and therefore $H_m'(f) = H_m'(\-f)$, and $H_m''(f) = -H_m''(\-f)$. With this,
	\begin{equations}
			A_m'(\-f)
			& = \frac{H_1'(f) H_m'(f) + \bts{-H_1''(f)}\bts{-H_m''(f)}}{{H_1'}^2(f) + \bts{-H_1''}^2(f)} \\
			& = \frac{H_1'(f)H_m'(f) + H_1''(f)H_m''(f)}{{H_1'}^2(f) + {H_1''}^2(f)} \\
			& = A_m'(f)\\[0.5cm]
			A_m''(\-f)
			& = \frac{H_1'(f) \bts{-H_m''(f)} - \bts{-H_1''(f)}H_m'(f)}{{H_1'}^2(f) + \bts{-H_1''}^2(f)} \\
			& = \frac{- H_1'(f)H_m''(f) + H_1''(f)H_m'(f)}{{H_1'}^2(f) + {H_1''}^2(f)} \\
			& = -A_m''(f)
	\end{equations}
	Therefore, our frequency-to-frequency RTF $A_m'(f)$ is even, since $A_m'(f) = A_m'(\-f)$, and our conjugate-frequency RTF is odd, since $A_m''(f) = -A_m''(\-f)$.
\end{Property}

%\begin{Property}{Relative transfer functions are even on frequency.}[\label{prop:rtfs_are_even-func_of_frequency}]
%	First, given two real systems with a shared input $x(t)$, each with an impulse response $h_m(t)$ and an output $y_m(t)$, such that
%	\begin{equations}
%		Y_{1}(f) = X(f) \Re{H_{1,\F}}(f) + X(\-f) \Im{H_{1,\F}}(f) \\
%		Y_{2}(f) = X(f) \Re{H_{2,\F}}(f) + X(\-f) \Im{H_{2,\F}}(f)
%	\end{equations}
%	and, for their conjugate frequencies,
%	\begin{equations}
%		Y_{1}(\-f) = X(\-f) \Re{H_{1,\F}}(\-f) + X(f) \Im{H_{1,\F}}(\-f) \\
%		Y_{2}(\-f) = X(\-f) \Re{H_{2,\F}}(\-f) + X(f) \Im{H_{2,\F}}(\-f)
%	\end{equations}
%	where we are omitting the transform index when it is an RFT's domain value, for compactness. If we define $X_1'(f)$ and $X_1''(f)$ as
%	\begin{equations}
%		X_1'(f) = \Re{H_{1,\F}}(f) X(f) \\
%		X_1''(f) = \Im{H_{1,\F}}(f) X(\-f)
%	\end{equations}
%	then
%	\begin{equations}
%		Y_m(f) = \frac{\Re{H_{m,\F}}(f)}{\Re{H_{1,\F}}(f)}X_1'(f) + \frac{\Im{H_{m,\F}}(f)}{\Im{H_{1,\F}}(f)} X_1''(f) \\
%		Y_m(\-f) = \frac{\Re{H_{m,\F}}(\-f)}{\Re{H_{1,\F}}(\-f)}X_1'(\-f) + \frac{\Im{H_{m,\F}}(\-f)}{\Im{H_{1,\F}}(\-f)} X_1''(\-f)
%	\end{equations}
%	By defining $A_m'(f)$ and $A_m''(f)$ as
%	\begin{equations}
%		A_m'(f) = \frac{\Re{H_{m,\F}}(f)}{\Re{H_{1,\F}}(f)} \\
%		A_m''(f) = \frac{\Im{H_{m,\F}}(f)}{\Im{H_{1,\F}}(f)}
%	\end{equations}
%	then, by analyzing these for negative frequencies, we have
%	\begin{equations}
%		A_m'(\-f)
%		& = \frac{\Re{H_{m,\F}}(\-f)}{\Re{H_{1,\F}}(\-f)} = \frac{\Re{H_{m,\F}}(f)}{\Re{H_{1,\F}}(f)} \\
%		& = A_m'(f)
%	\end{equations}
%	\begin{equations}
%	A_m''(\-f)
%	& = \frac{\Im{H_{m,\F}}(\-f)}{\Im{H_{1,\F}}(\-f)} = \frac{-\Im{H_{m,\F}}(f)}{-\Im{H_{1,\F}}(f)} \\
%	& = A_m''(f)
%	\end{equations}
%	where we used that $H_{m,\F}(f) = H_{m,\F}^*(\-f)$, if $h_m(t)$ is real-valued.
%	
%	If we instead define one relative transfer function $A_m(f)$ such that
%	\begin{equations}{eq:propVII:new_bva_vector_element_def}
%		A_m(f) = \frac{\Re{H_{m,\F}}(f) \Re{H_{1,\F}}(f) + \Im{H_{m,\F}}(f) \Im{H_{1,\F}}(f)}{{\Re{H_{1,\F}}}^2(f) + {\Im{H_{1,\F}}}^2(f)}
%	\end{equations}
%	it is easy to see that
%	\begin{equations}
%		A_m(\-f)
%		& = \frac{\Re{H_{m,\F}}(f) \Re{H_{1,\F}}(f) + \bts{- \Im{H_{m,\F}}(f)} \bts{-\Im{H_{1,\F}}(f)}}{{\Re{H_{1,\F}}}^2(f) + \bts{-\Im{H_{1,\F}}}^2(f)} \\
%		& = \frac{\Re{H_{m,\F}}(f) \Re{H_{1,\F}}(f) + \Im{H_{m,\F}}(f) \Im{H_{1,\F}}(f)}{{\Re{H_{1,\F}}}^2(f) + {\Im{H_{1,\F}}}^2(f)} \\
%		& = A_m(f)
%	\end{equations}
%	
%	With this, $A_m'(f) = A_m'(\-f)$ and $A_m''(f) = A_m''(\-f)$, and also $A_m(f) = A_m(\-f)$, and therefore all defined relative transfer functions are even in terms of conjugate frequencies.
%	
%\end{Property}