\section{SSBT Convolution}
\label{app:ssbt_convolution}

\def\xfr{\Re{X_{\F}}}
\def\xfi{\Im{X_{\F}}}
\def\hfr{\Re{H_{\F}}}
\def\hfi{\Im{H_{\F}}}

Let $x[n]$ be a time domain signal, with $X_{\F}[l,k]$ being its STFT equivalent, and $X_{\S}[l,k]$ its SSBT equivalent. We here assume that both the STFT and the SSBT have $K$ frequency bins. $X_{\S}[l,k]$ can be obtained using $X_{\F}[l,k]$, through
\begin{equation}
    \label{eq:appA:equiv_ssbt_stft}
    X_{\S}[l,k] = - \xfr[l,k] + \xfi[l,k]
\end{equation}
win which $\Re{(\cdot)}$ and $\Im{(\cdot)}$ represent the real and imaginary components of their argument, respectively.

It is easy to see that
\begin{equation}
    \label{eq:appA:equiv_stft_ssbt}
    X_{\F}[l,k] = \frac{1}{\Sqrt{2}} \pts{ e^{\j\frac{3\pi}{4}} X_{\S}[l,k] + e^{-\j\frac{3\pi}{4}} X_{\S}[l,K-k] }
\end{equation}
As stated before, this formulation isn't valid for $k=0$ and $k = \nicefrac{K}{2}$ if $K$ is even, but in these cases $X_{\F}[l,k] = X_{\S}[l,k]$.

Now, let there also be $h[n]$, $H_{\F}[k]$ and $H_{\S}[k]$, with the same assumptions as before. We define $Y_{\F}[l,k]$ and $Y_{\S}[l,k]$ as the output of an LTI system with impulse response $h[n]$, such that
\begin{subalign}
    Y_{\F}[l,k] & = H_{\F}[k] X_{\F}[l,k] \label{subeq:appA:mtf_stft} \\
    Y_{\S}[l,k] & = H_{\S}[k] X_{\S}[l,k] \label{subeq:appA:mtf_ssbt}
\end{subalign}

We will assume that the MTF model \cite{talmon_relative_2009} correctly models the convolution here. Here we use the MTF model instead of the CTF for simplicity, as these derivations would work exactly the same for the CTF.

Applying \cref{eq:appA:equiv_ssbt_stft} in \cref{subeq:appA:mtf_ssbt}, and knowing that $X_{\F}[l,k] = X_{\F}^{*}[l,K-k]$ (same for $H_{\F}[k]$), with $(\cdot)^*$ representing the complex-conjugate; we get that
\begin{equations}
    Y_{\S}[l,k]
        & = \xfr[l,k] \hfr[l,k] - \xfr[l,k] \hfi[l,k] \\
        & - \xfi[l,k] \hfr[l,k] + \xfi[l,k] \hfi[l,k] \\[0.2cm]
    Y_{\S}[l,K-k] 
        & = \xfr[l,k] \hfr[l,k] + \xfr[l,k] \hfi[l,k] \\
        & + \xfi[l,k] \hfr[l,k] + \xfi[l,k] \hfi[l,k]
\end{equations}

Passing this through \cref{eq:appA:equiv_stft_ssbt},
\begin{equations}{eq:appA:expansion_Yf_ssbt}
    Y_{\F}'[l,k] =
        & - \xfr[l,k] \hfr[l,k] - \j \xfr[l,k] \hfr[l,k] \\
        & - \j \xfi[l,k] \hfr[l,k] - \xfi[l,k] \hfi[l,k]
\end{equations}
where $Y_{\F}'[l,k]$ is the STFT-equivalent of $Y_{\S}[l,k]$.

Expanding \cref{subeq:appA:mtf_stft} in terms of real and imaginary components,
\begin{equations}{eq:appA:expansion_Yf_stft}
    Y_{\F}[l,k] 
        & = \xfr[l,k] \hfr[l,k] + \j \xfr[l,k] \hfi[l,k] \\
        & + \j \xfi[l,k] \hfr[l,k] - \xfi[l,k] \hfi[l,k]
\end{equations}

Comparing \cref{eq:appA:expansion_Yf_ssbt} and \cref{eq:appA:expansion_Yf_stft}, trivially $Y_{\F}'[l,k] \neq Y_{\F}[l,k]$. This proves that the SSBT doesn't appropriately models the convolution, and therefore the convolution theorem doesn't hold when applying this transform.
