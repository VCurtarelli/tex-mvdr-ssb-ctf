\definecolor{ColA}{HTML}{991F3D}
\definecolor{ColB}{HTML}{997A1F}
\definecolor{ColC}{HTML}{3D991F}
\definecolor{ColD}{HTML}{1F997A}
\definecolor{ColE}{HTML}{1F3D99}
\definecolor{ColF}{HTML}{7A1F99}
\section{Simulations}
\label{sec:results}

In the simulations\footnote{The code for the simulations can be found in \url{https://github.com/VCurtarelli/py-ssb-ctf-bf}.}, we will use a sampling frequency $f_s = 16\si{\kilo\hertz}$. The sensor array is assumed an uniform linear array of 10 sensors, with an intersensor distance of $2\si{\cm}$. All RIR's were generated using Habets' room impulse generator \cite{habets_rir-generator}. Signals used were from the SMARD \cite{smard_database} and LINSE \cite{linse_database} databases.

The room's dimensions are $4\m\times6\m\times3\m$, with a reverberation time of $0.11\si{\second}$. The desired source is located at $(2\m,~1\m,~1\m)$, and its signal is a male voice (SMARD, \texttt{50\_male\_speech\_english\_ch8\_OmniPower4296.flac}).

The interfering source is located (simultaneously) at $(0.5\m,~5\m,~[0.3:0.3:2.7]\m)$, emulating an open door, with its signal being a babble sound (LINSE database, \texttt{babble.mat}). The noise signal is a WGN noise (SMARD database, \texttt{wgn\_48kHz\_ch8\_OmniPower4296.flac}). All signals were resampled to the desired sampling frequency. The sensor array is assumed to be an uniform linear array, positioned at $(2\m,~[4.02:0.02:4.2]\m,~1\m)$, whose sensors are assumed to be omnidirectional and with flat frequency response.

We have that the input SNR between desired and interfering signals is of $5\dB$, and the SNR between desired and noise signals is of $30\dB$. The filters were calculated every 25 windows, and consider the previous 25 windows, in order to calculate the correlation matrices.

We will compare the filters obtained through the STFT and SSBT transforms, with T-SSBT denoting the beamformer obtained via the true-distortionless MVDR derived in \cref{sec:true_mvdr_ssbt}, and N-SSBT the naïve approach, in which one would simply use \cref{eq:sec3:solution_mvdr_beamformer} to calculate the SSBT beamformer. The performance analysis was done only on the STFT domain. In all lineplots, the STFT is presented in red, the N-SSBT in green, and the T-SSBT in blue.

\subsection{Results}

In this scenario, we assume that the analysis (and synthesis) windows have 32 samples. \Cref{fig:lineplot_gain_32} shows the window-wise averaged gain in SNR, and \cref{fig:heatmap_gain_32} the gain in SNR for each window (with the windows being represented by the time started, in seconds), for all methods. In \cref{fig:lineplot_dsdi_32} we have the DSDI for all three methods, window-averaged.

Although it isn't as clear from the per-window results of \cref{fig:heatmap_gain_32}, \cref{fig:lineplot_gain_32} shows us that both beamformers obtained from the SSBT led to a better enhancement of the signal, in terms of the output SNR, with the N-SSBT beamformer having a better performance over (almost) all spectrum, and the T-SSBT beamformer being better for lower frequencies, tying with the STFT for higher ones.

Also, \cref{fig:lineplot_dsdi_32} shows that the T-SSBT filter was able to ensure a distortionless response for the desired signal, a feature that wasn't achieved by the N-SSBT beamformer. This is wholly expected, since the later was naively designed with the MVDR in mind, and wasn't fully planned to achieve a distortionless behavior in the STFT (and therefore the time) domain, while the T-SSBT took this into account on its derivation. 

\def\meshcols{316}
\def\meshrows{15}
\def\tmin{0}
\def\tmax{8.3725}
\def\fmin{0.25}
\def\fmax{7.75}

\begin{figure}[H]
	\centering
	%\pgfplotsset{
%    table/myStyleWithMeta/.style={    
%        meta=Label, 
%    }
%}

{\footnotesize%
	\begin{subfigure}{0.48\linewidth}% resA
		%\centering
		\tikzsetnextfilename{heatmap_STFT_32}%
		\begin{tikzpicture}%
			\begin{heatmap}{\meshcols}{Gain (dB)}[colorbar to name={heatmap_STFT_32}, xlabel={Time ($\si{\second}$)}, ylabel={Freq. ($\si{\kilo\hertz}$)}]%
				\addplot3[surf, mesh/cols=\meshcols, mesh/rows=\meshrows, shader=interp] table[y=freq, x=win, z=val, col sep=comma] {figures/io_input/STFT/gain_SINR_lk_STFT_32.csv};%
			\end{heatmap}%
		\end{tikzpicture}%
		\vspace*{-2mm}\caption{STFT}%
		\label{subfig:heatmap_gain_STFT_32}%
		\vspace*{2mm}%
	\end{subfigure}%
}\hspace{0.03\linewidth}
{\footnotesize%
	\begin{subfigure}{0.48\linewidth}% resA
		%\centering
<<<<<<< HEAD
		\tikzsetnextfilename{heatmap_NSSBT32}%
		\begin{tikzpicture}%
			\begin{heatmap}{\meshcols}{Gain (dB)}[colorbar to name={heatmap_NSSBT32}, xlabel={Time ($\si{\second}$)}, ylabel={Freq. ($\si{\kilo\hertz}$)}]%
				\addplot3[surf, mesh/cols=\meshcols, mesh/rows=\meshrows, shader=interp] table[y=freq, x=win, z=val, col sep=comma] {figures/io_input/NSSBT/gain_SINR_lk_NSSBT32.csv};%
			\end{heatmap}%
		\end{tikzpicture}%
		\vspace*{-2mm}\caption{N-SSBT}%
		\label{subfig:heatmap_gain_NSSBT32}%
=======
		\tikzsetnextfilename{heatmap_NSSBT_32}%
		\begin{tikzpicture}%
			\begin{heatmap}{\meshcols}{Gain (dB)}[colorbar to name={heatmap_NSSBT_32}, xlabel={Time ($\si{\second}$)}, ylabel={Freq. ($\si{\kilo\hertz}$)}]%
				\addplot3[surf, mesh/cols=\meshcols, mesh/rows=\meshrows, shader=interp] table[y=freq, x=win, z=val, col sep=comma] {figures/io_input/NSSBT/gain_SINR_lk_NSSBT_32.csv};%
			\end{heatmap}%
		\end{tikzpicture}%
		\vspace*{-2mm}\caption{N-SSBT}%
		\label{subfig:heatmap_gain_NSSBT_32}%
>>>>>>> main
		\vspace*{2mm}%
	\end{subfigure}%
}\newline%
{\footnotesize%
	\begin{subfigure}{0.48\linewidth}% resA
		%\centering
		\tikzsetnextfilename{heatmap_TSSBT_32}%
		\begin{tikzpicture}%
			\begin{heatmap}{\meshcols}{Gain (dB)}[colorbar to name={heatmap_TSSBT_32}, xlabel={Time ($\si{\second}$)}, ylabel={Freq. ($\si{\kilo\hertz}$)}]%
				\addplot3[surf, mesh/cols=\meshcols, mesh/rows=\meshrows, shader=interp] table[y=freq, x=win, z=val, col sep=comma] {figures/io_input/TSSBT/gain_SINR_lk_TSSBT_32.csv};%
			\end{heatmap}%
		\end{tikzpicture}%
		\vspace*{-2mm}\caption{T-SSBT}%
		\label{subfig:heatmap_gain_TSSBT_32}%
		\vspace*{2mm}%
	\end{subfigure}%
}%
	
	\tikzsetnextfilename{heatmap_gain_32}%
	\vspace*{0.4em}
	\ref*{heatmap_SSBT_32}
	\caption{Per-window SNR gain.}
	\label{fig:heatmap_gain_32}
\end{figure}

\begin{figure}[H]
\centering
%% Requires:
% pgfplots.sty
% edit_pgfplots.tex

\pgfplotsset{compat=1.18}
%\begin{subfigure}{\linewidth}
%\centering
\tikzsetnextfilename{gain_lineplot_32}
\begin{tikzpicture}
	\begin{lineplot}{Gain (dB)}[ymin=3, ymax=18]
		\addplot [style=resA]
		table [col sep=comma, y=val] {figures/io_input/STFT/gain_SINR_k_STFT_32.csv};
		
		\addplot [style=resC]
		table [col sep=comma, y=val] {figures/io_input/SSBT/gain_SINR_k_SSBT_32.csv};
		
		\addplot [style=resE]
		table [col sep=comma, y=val] {figures/io_input/SSBT-TRUE/gain_SINR_k_SSBT-TRUE_32.csv};
		
		\addlegendentry{STFT};
		\addlegendentry{N-SSBT};
		\addlegendentry{T-SSBT};
	\end{lineplot}
\end{tikzpicture}
%	\caption{}
%	\label{subfig:1_gain_lineplot}
%\end{subfigure}
\caption{Window-average SNR gain.}
\label{fig:lineplot_gain_32}
\end{figure}

\begin{figure}[H]
	\centering
	%% Requires:
% pgfplots.sty
% edit_pgfplots.tex

\pgfplotsset{compat=1.18}
%\begin{subfigure}{\linewidth}
%\centering
\tikzsetnextfilename{dsdi_lineplot_32}
\begin{tikzpicture}
	\begin{lineplot}{DSDI}%[ymin=-0.1, ymax=3]
		\addplot [style=resA]
		table [col sep=comma, y=val] {figures/io_input/STFT/dsdi_l_STFT_32_p30.csv};
		
		\addplot [style=resC]
		table [col sep=comma, y=val] {figures/io_input/NSSBT/dsdi_l_NSSBT_32_p30.csv};
		
		\addplot [style=resE]
		table [col sep=comma, y=val] {figures/io_input/TSSBT/dsdi_l_TSSBT_32_p30.csv};
		
		\addlegendentry{STFT};
		\addlegendentry{N-SSBT};
		\addlegendentry{T-SSBT};
	\end{lineplot}
\end{tikzpicture}
%	\caption{}
%	\label{subfig:1_gain_lineplot}
%\end{subfigure}
	\caption{Window-average DSDI.}
	\label{fig:lineplot_dsdi_32}
\end{figure}
%
\subsection{Results - 64 samples/window}

In this simulation, we changed the number of samples per window from 32 to 64, keeping everything else the same.

In here, from \cref{fig:lineplot_gain_64,fig:heatmap_gain_64} we see a similar result to that which was obtained previously, with the N-SSBT beamformer having a better performance overall, but causing some distortion in the desired signal; while the T-SSBT beamformer has a slightly better performance than the one obtained through the STFT, while also having a distortionless behavior, as is seen in \cref{fig:lineplot_dsdi_64}.

\def\meshcols{263}
\def\meshrows{31}
\def\tmin{0}
\def\tmax{8.1283125}
\def\fmin{0.125}
\def\fmax{7.875}

\begin{figure}[H]
	\centering
	%\pgfplotsset{
%    table/myStyleWithMeta/.style={    
%        meta=Label, 
%    }
%}

{\footnotesize%
	\begin{subfigure}{0.48\linewidth}% resA
		%\centering
		\tikzsetnextfilename{heatmap_STFT_64}%
		\begin{tikzpicture}%
			\begin{heatmap}{\meshcols}{Gain (dB)}[colorbar to name={heatmap_STFT_64}, xlabel={Time ($\si{\second}$)}, ylabel={Freq. ($\si{\kilo\hertz}$)}]%
				\addplot3[surf, mesh/cols=\meshcols, mesh/rows=\meshrows, shader=interp] table[y=freq, x=win, z=val, col sep=comma] {figures/io_input/STFT/gain_SINR_lk_STFT_64.csv};%
			\end{heatmap}%
		\end{tikzpicture}%
		\vspace*{-2mm}\caption{STFT}%
		\label{subfig:heatmap_gain_STFT_64}%
		\vspace*{2mm}%
	\end{subfigure}%
}\hspace{0.03\linewidth}
{\footnotesize%
	\begin{subfigure}{0.48\linewidth}% resA
		%\centering
		\tikzsetnextfilename{heatmap_NSSBT_64}%
		\begin{tikzpicture}%
			\begin{heatmap}{\meshcols}{Gain (dB)}[colorbar to name={heatmap_NSSBT_64}, xlabel={Time ($\si{\second}$)}, ylabel={Freq. ($\si{\kilo\hertz}$)}]%
				\addplot3[surf, mesh/cols=\meshcols, mesh/rows=\meshrows, shader=interp] table[y=freq, x=win, z=val, col sep=comma] {figures/io_input/NSSBT/gain_SINR_lk_NSSBT_64.csv};%
			\end{heatmap}%
		\end{tikzpicture}%
		\vspace*{-2mm}\caption{N-SSBT}%
		\label{subfig:heatmap_gain_NSSBT_64}%
		\vspace*{2mm}%
	\end{subfigure}%
}\newline%
{\footnotesize%
	\begin{subfigure}{0.48\linewidth}% resA
		%\centering
		\tikzsetnextfilename{heatmap_TSSBT_64}%
		\begin{tikzpicture}%
			\begin{heatmap}{\meshcols}{Gain (dB)}[colorbar to name={heatmap_TSSBT_64}, xlabel={Time ($\si{\second}$)}, ylabel={Freq. ($\si{\kilo\hertz}$)}]%
				\addplot3[surf, mesh/cols=\meshcols, mesh/rows=\meshrows, shader=interp] table[y=freq, x=win, z=val, col sep=comma] {figures/io_input/TSSBT/gain_SINR_lk_TSSBT_64.csv};%
			\end{heatmap}%
		\end{tikzpicture}%
		\vspace*{-2mm}\caption{T-SSBT}%
		\label{subfig:heatmap_gain_TSSBT_64}%
		\vspace*{2mm}%
	\end{subfigure}%
}%
	
	\tikzsetnextfilename{heatmap_gain_64}%
	\vspace*{0.4em}
	\ref*{heatmap_SSBT_64}
	\caption{Per-window SNR gain.}
	\label{fig:heatmap_gain_64}
\end{figure}

\begin{figure}[H]
\centering
%% Requires:
% pgfplots.sty
% edit_pgfplots.tex

\pgfplotsset{compat=1.18}
%\begin{subfigure}{\linewidth}
%\centering
\tikzsetnextfilename{gain_lineplot_64}
\begin{tikzpicture}
	\begin{lineplot}{Gain (dB)}[ymin=0, ymax=18]
		\addplot [style=resA]
		table [col sep=comma, y=val] {figures/io_input/STFT/gain_SINR_k_STFT_64.csv};
		
		\addplot [style=resC]
		table [col sep=comma, y=val] {figures/io_input/SSBT/gain_SINR_k_SSBT_64.csv};
		
		\addplot [style=resE]
		table [col sep=comma, y=val] {figures/io_input/SSBT-TRUE/gain_SINR_k_SSBT-TRUE_64.csv};
		
		\addlegendentry{STFT};
		\addlegendentry{N-SSBT};
		\addlegendentry{T-SSBT};
	\end{lineplot}
\end{tikzpicture}
%	\caption{}
%	\label{subfig:1_gain_lineplot}
%\end{subfigure}
\caption{Window-average SNR gain.}
\label{fig:lineplot_gain_64}
\end{figure}

\begin{figure}[H]
	\centering
	%% Requires:
% pgfplots.sty
% edit_pgfplots.tex

\pgfplotsset{compat=1.18}
%\begin{subfigure}{\linewidth}
%\centering
\tikzsetnextfilename{dsdi_lineplot_64}
\begin{tikzpicture}
	\begin{lineplot}{DSDI}[ymin=-0.1, ymax=1]
		\addplot [style=resA]
		table [col sep=comma, y=val] {figures/io_input/STFT/dsdi_k_STFT_64.csv};
		
		\addplot [style=resC]
		table [col sep=comma, y=val] {figures/io_input/SSBT/dsdi_k_SSBT_64.csv};
		
		\addplot [style=resE]
		table [col sep=comma, y=val] {figures/io_input/SSBT-TRUE/dsdi_k_SSBT-TRUE_64.csv};
		
		\addlegendentry{STFT};
		\addlegendentry{N-SSBT};
		\addlegendentry{T-SSBT};
	\end{lineplot}
\end{tikzpicture}
%	\caption{}
%	\label{subfig:1_gain_lineplot}
%\end{subfigure}
	\caption{Window-average DSDI.}
	\label{fig:lineplot_dsdi_64}
\end{figure}