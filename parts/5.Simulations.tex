\definecolor{ColA}{HTML}{991F3D}
\definecolor{ColB}{HTML}{997A1F}
\definecolor{ColC}{HTML}{3D991F}
\definecolor{ColD}{HTML}{1F997A}
\definecolor{ColE}{HTML}{1F3D99}
\definecolor{ColF}{HTML}{7A1F99}
\section{Simulations}
\label{sec:results}

In the simulations\footnote{The simulation code is available at \url{https://github.com/VCurtarelli/py-ssb-ctf-bf}.}, we employ a sampling frequency of $16\si{\kilo\hertz}$. The sensor array consists of a uniform linear array with 10 sensors spaced at $2\si{\cm}$. Room impulse responses were generated using Habets' RIR generator \cite{habets_rir-generator}, and signals were selected from the SMARD \cite{smard_database} and LINSE \cite{linse_database} databases.

The room's dimensions are $4\m \times 6\m \times 3\m$ (width $\times$ length $\times$ height), with a reverberation time of $0.11\si{\second}$. The desired source is located at $(2\m,~1\m,~1\m)$, it being a male voice (SMARD, \texttt{50\_male\_speech\_english\_ch8\_OmniPower4296.flac}).
%
The interfering source, simulating an open door, is located simultaneously at $(0.5\m,~5\m,~[0.3:0.3:2.7]\m)$, with a babble sound signal (LINSE database, \texttt{babble.mat}). The noise signal is white Gaussian noise (SMARD database, \texttt{wgn\_48kHz\_ch8\_OmniPower4296.flac}). All signals were resampled to the desired frequency.

The uniform linear sensor array is positioned at $(2\m,~[4.02:0.02:4.2]\m,~1\m)$, with omnidirectional sensors of flat frequency response. The input SNR between desired and interfering signals is $5\dB$, and between desired and noise signals is $30\dB$. Filters are calculated every 25 windows, considering the previous 25 windows to calculate correlation matrices.

We compare filters obtained through the STFT and SSBT transforms. T-SSBT will denote the beamformer obtained via the true-distortionless MVDR from \cref{sec:true_mvdr_ssbt}, and N-SSBT uses \cref{eq:sec3:solution_mvdr_beamformer} to (naively) calculate the SSBT beamformer. Performance analysis is conducted via the STFT domain, with the SSBT beamformers being converted into it. In line plots, STFT is presented in red, N-SSBT in green, and T-SSBT in blue.

%%%%%%%%%%%%%%%%%%%%%

\subsection{Results}

In this scenario, we assume that the analysis windows have 32 samples. \cref{fig:heatmap_gain_32} shows the gain in SNR for each window (with the windows being represented by the time started, in seconds), and \cref{fig:lineplot_gain_32} the window-wise averaged gain in SNR, for all methods. In \cref{fig:lineplot_dsdi_32} we have the DSDI for all three methods, window-averaged.

Although it isn't as clear from the per-window results of \cref{fig:heatmap_gain_32}, \cref{fig:lineplot_gain_32} clearly shows that both SSBT beamformers had a better (at most equal) performance than the STFT one, with the N-SSBT beamformer having a better performance over (almost) all spectrum, and the T-SSBT beamformer being better for lower frequencies, tying with the STFT for higher ones.

Also, \cref{fig:lineplot_dsdi_32} shows that the T-SSBT filter was able to ensure a distortionless response for the desired signal, a feature that wasn't achieved by the N-SSBT beamformer. This was wholly expected, since the later was naively designed with the MVDR in mind, and wasn't fully planned to achieve a distortionless behavior in the STFT (and therefore the time) domain, while the T-SSBT took this into account on its derivation.

\def\meshcols{316}
\def\meshrows{15}
\def\tmin{0}
\def\tmax{8.3725}
\def\fmin{0.25}
\def\fmax{7.75}

\begin{figure}[H]
	\centering
	%\pgfplotsset{
%    table/myStyleWithMeta/.style={    
%        meta=Label, 
%    }
%}

{\footnotesize%
	\begin{subfigure}{0.48\linewidth}% resA
		%\centering
		\tikzsetnextfilename{heatmap_STFT_32}%
		\begin{tikzpicture}%
			\begin{heatmap}{\meshcols}{Gain (dB)}[colorbar to name={heatmap_STFT_32}, xlabel={Time ($\si{\second}$)}, ylabel={Freq. ($\si{\kilo\hertz}$)}]%
				\addplot3[surf, mesh/cols=\meshcols, mesh/rows=\meshrows, shader=interp] table[y=freq, x=win, z=val, col sep=comma] {figures/io_input/STFT/gain_SINR_lk_STFT_32.csv};%
			\end{heatmap}%
		\end{tikzpicture}%
		\vspace*{-2mm}\caption{STFT}%
		\label{subfig:heatmap_gain_STFT_32}%
		\vspace*{2mm}%
	\end{subfigure}%
}\hspace{0.03\linewidth}
{\footnotesize%
	\begin{subfigure}{0.48\linewidth}% resA
		%\centering
<<<<<<< HEAD
		\tikzsetnextfilename{heatmap_NSSBT32}%
		\begin{tikzpicture}%
			\begin{heatmap}{\meshcols}{Gain (dB)}[colorbar to name={heatmap_NSSBT32}, xlabel={Time ($\si{\second}$)}, ylabel={Freq. ($\si{\kilo\hertz}$)}]%
				\addplot3[surf, mesh/cols=\meshcols, mesh/rows=\meshrows, shader=interp] table[y=freq, x=win, z=val, col sep=comma] {figures/io_input/NSSBT/gain_SINR_lk_NSSBT32.csv};%
			\end{heatmap}%
		\end{tikzpicture}%
		\vspace*{-2mm}\caption{N-SSBT}%
		\label{subfig:heatmap_gain_NSSBT32}%
=======
		\tikzsetnextfilename{heatmap_NSSBT_32}%
		\begin{tikzpicture}%
			\begin{heatmap}{\meshcols}{Gain (dB)}[colorbar to name={heatmap_NSSBT_32}, xlabel={Time ($\si{\second}$)}, ylabel={Freq. ($\si{\kilo\hertz}$)}]%
				\addplot3[surf, mesh/cols=\meshcols, mesh/rows=\meshrows, shader=interp] table[y=freq, x=win, z=val, col sep=comma] {figures/io_input/NSSBT/gain_SINR_lk_NSSBT_32.csv};%
			\end{heatmap}%
		\end{tikzpicture}%
		\vspace*{-2mm}\caption{N-SSBT}%
		\label{subfig:heatmap_gain_NSSBT_32}%
>>>>>>> main
		\vspace*{2mm}%
	\end{subfigure}%
}\newline%
{\footnotesize%
	\begin{subfigure}{0.48\linewidth}% resA
		%\centering
		\tikzsetnextfilename{heatmap_TSSBT_32}%
		\begin{tikzpicture}%
			\begin{heatmap}{\meshcols}{Gain (dB)}[colorbar to name={heatmap_TSSBT_32}, xlabel={Time ($\si{\second}$)}, ylabel={Freq. ($\si{\kilo\hertz}$)}]%
				\addplot3[surf, mesh/cols=\meshcols, mesh/rows=\meshrows, shader=interp] table[y=freq, x=win, z=val, col sep=comma] {figures/io_input/TSSBT/gain_SINR_lk_TSSBT_32.csv};%
			\end{heatmap}%
		\end{tikzpicture}%
		\vspace*{-2mm}\caption{T-SSBT}%
		\label{subfig:heatmap_gain_TSSBT_32}%
		\vspace*{2mm}%
	\end{subfigure}%
}%
	
	\tikzsetnextfilename{heatmap_gain_32}%
	\vspace*{0.4em}
	\ref*{heatmap_SSBT_32}
	\caption{Per-window SNR gain.}
	\label{fig:heatmap_gain_32}
\end{figure}

\begin{figure}[H]
\centering
%% Requires:
% pgfplots.sty
% edit_pgfplots.tex

\pgfplotsset{compat=1.18}
%\begin{subfigure}{\linewidth}
%\centering
\tikzsetnextfilename{gain_lineplot_32}
\begin{tikzpicture}
	\begin{lineplot}{Gain (dB)}[ymin=3, ymax=18]
		\addplot [style=resA]
		table [col sep=comma, y=val] {figures/io_input/STFT/gain_SINR_k_STFT_32.csv};
		
		\addplot [style=resC]
		table [col sep=comma, y=val] {figures/io_input/SSBT/gain_SINR_k_SSBT_32.csv};
		
		\addplot [style=resE]
		table [col sep=comma, y=val] {figures/io_input/SSBT-TRUE/gain_SINR_k_SSBT-TRUE_32.csv};
		
		\addlegendentry{STFT};
		\addlegendentry{N-SSBT};
		\addlegendentry{T-SSBT};
	\end{lineplot}
\end{tikzpicture}
%	\caption{}
%	\label{subfig:1_gain_lineplot}
%\end{subfigure}
\caption{Window-average SNR gain.}
\label{fig:lineplot_gain_32}
\end{figure}

\begin{figure}[H]
	\centering
	%% Requires:
% pgfplots.sty
% edit_pgfplots.tex

\pgfplotsset{compat=1.18}
%\begin{subfigure}{\linewidth}
%\centering
\tikzsetnextfilename{dsdi_lineplot_32}
\begin{tikzpicture}
	\begin{lineplot}{DSDI}%[ymin=-0.1, ymax=3]
		\addplot [style=resA]
		table [col sep=comma, y=val] {figures/io_input/STFT/dsdi_l_STFT_32_p30.csv};
		
		\addplot [style=resC]
		table [col sep=comma, y=val] {figures/io_input/NSSBT/dsdi_l_NSSBT_32_p30.csv};
		
		\addplot [style=resE]
		table [col sep=comma, y=val] {figures/io_input/TSSBT/dsdi_l_TSSBT_32_p30.csv};
		
		\addlegendentry{STFT};
		\addlegendentry{N-SSBT};
		\addlegendentry{T-SSBT};
	\end{lineplot}
\end{tikzpicture}
%	\caption{}
%	\label{subfig:1_gain_lineplot}
%\end{subfigure}
	\caption{Window-average DSDI.}
	\label{fig:lineplot_dsdi_32}
\end{figure}
%
\subsection{Results - 64 samples/window}

In this simulation, we changed the number of samples per window from 32 to 64, keeping everything else the same.

In here, from \cref{fig:lineplot_gain_64,fig:heatmap_gain_64} we see a similar result to that which was obtained previously, with the N-SSBT beamformer having a better performance overall but causing some distortion in the desired signal; and the T-SSBT beamformer having a slightly better performance STFT one while also having a distortionless behavior, as is seen in \cref{fig:lineplot_dsdi_64}.

\def\meshcols{263}
\def\meshrows{31}
\def\tmin{0}
\def\tmax{8.1283125}
\def\fmin{0.125}
\def\fmax{7.875}

\begin{figure}[H]
	\centering
	%\pgfplotsset{
%    table/myStyleWithMeta/.style={    
%        meta=Label, 
%    }
%}

{\footnotesize%
	\begin{subfigure}{0.48\linewidth}% resA
		%\centering
		\tikzsetnextfilename{heatmap_STFT_64}%
		\begin{tikzpicture}%
			\begin{heatmap}{\meshcols}{Gain (dB)}[colorbar to name={heatmap_STFT_64}, xlabel={Time ($\si{\second}$)}, ylabel={Freq. ($\si{\kilo\hertz}$)}]%
				\addplot3[surf, mesh/cols=\meshcols, mesh/rows=\meshrows, shader=interp] table[y=freq, x=win, z=val, col sep=comma] {figures/io_input/STFT/gain_SINR_lk_STFT_64.csv};%
			\end{heatmap}%
		\end{tikzpicture}%
		\vspace*{-2mm}\caption{STFT}%
		\label{subfig:heatmap_gain_STFT_64}%
		\vspace*{2mm}%
	\end{subfigure}%
}\hspace{0.03\linewidth}
{\footnotesize%
	\begin{subfigure}{0.48\linewidth}% resA
		%\centering
		\tikzsetnextfilename{heatmap_NSSBT_64}%
		\begin{tikzpicture}%
			\begin{heatmap}{\meshcols}{Gain (dB)}[colorbar to name={heatmap_NSSBT_64}, xlabel={Time ($\si{\second}$)}, ylabel={Freq. ($\si{\kilo\hertz}$)}]%
				\addplot3[surf, mesh/cols=\meshcols, mesh/rows=\meshrows, shader=interp] table[y=freq, x=win, z=val, col sep=comma] {figures/io_input/NSSBT/gain_SINR_lk_NSSBT_64.csv};%
			\end{heatmap}%
		\end{tikzpicture}%
		\vspace*{-2mm}\caption{N-SSBT}%
		\label{subfig:heatmap_gain_NSSBT_64}%
		\vspace*{2mm}%
	\end{subfigure}%
}\newline%
{\footnotesize%
	\begin{subfigure}{0.48\linewidth}% resA
		%\centering
		\tikzsetnextfilename{heatmap_TSSBT_64}%
		\begin{tikzpicture}%
			\begin{heatmap}{\meshcols}{Gain (dB)}[colorbar to name={heatmap_TSSBT_64}, xlabel={Time ($\si{\second}$)}, ylabel={Freq. ($\si{\kilo\hertz}$)}]%
				\addplot3[surf, mesh/cols=\meshcols, mesh/rows=\meshrows, shader=interp] table[y=freq, x=win, z=val, col sep=comma] {figures/io_input/TSSBT/gain_SINR_lk_TSSBT_64.csv};%
			\end{heatmap}%
		\end{tikzpicture}%
		\vspace*{-2mm}\caption{T-SSBT}%
		\label{subfig:heatmap_gain_TSSBT_64}%
		\vspace*{2mm}%
	\end{subfigure}%
}%
	
	\tikzsetnextfilename{heatmap_gain_64}%
	\vspace*{0.4em}
	\ref*{heatmap_SSBT_64}
	\caption{Per-window SNR gain.}
	\label{fig:heatmap_gain_64}
\end{figure}

\begin{figure}[H]
\centering
%% Requires:
% pgfplots.sty
% edit_pgfplots.tex

\pgfplotsset{compat=1.18}
%\begin{subfigure}{\linewidth}
%\centering
\tikzsetnextfilename{gain_lineplot_64}
\begin{tikzpicture}
	\begin{lineplot}{Gain (dB)}[ymin=0, ymax=18]
		\addplot [style=resA]
		table [col sep=comma, y=val] {figures/io_input/STFT/gain_SINR_k_STFT_64.csv};
		
		\addplot [style=resC]
		table [col sep=comma, y=val] {figures/io_input/SSBT/gain_SINR_k_SSBT_64.csv};
		
		\addplot [style=resE]
		table [col sep=comma, y=val] {figures/io_input/SSBT-TRUE/gain_SINR_k_SSBT-TRUE_64.csv};
		
		\addlegendentry{STFT};
		\addlegendentry{N-SSBT};
		\addlegendentry{T-SSBT};
	\end{lineplot}
\end{tikzpicture}
%	\caption{}
%	\label{subfig:1_gain_lineplot}
%\end{subfigure}
\caption{Window-average SNR gain.}
\label{fig:lineplot_gain_64}
\end{figure}

\begin{figure}[H]
	\centering
	%% Requires:
% pgfplots.sty
% edit_pgfplots.tex

\pgfplotsset{compat=1.18}
%\begin{subfigure}{\linewidth}
%\centering
\tikzsetnextfilename{dsdi_lineplot_64}
\begin{tikzpicture}
	\begin{lineplot}{DSDI}[ymin=-0.1, ymax=1]
		\addplot [style=resA]
		table [col sep=comma, y=val] {figures/io_input/STFT/dsdi_k_STFT_64.csv};
		
		\addplot [style=resC]
		table [col sep=comma, y=val] {figures/io_input/SSBT/dsdi_k_SSBT_64.csv};
		
		\addplot [style=resE]
		table [col sep=comma, y=val] {figures/io_input/SSBT-TRUE/dsdi_k_SSBT-TRUE_64.csv};
		
		\addlegendentry{STFT};
		\addlegendentry{N-SSBT};
		\addlegendentry{T-SSBT};
	\end{lineplot}
\end{tikzpicture}
%	\caption{}
%	\label{subfig:1_gain_lineplot}
%\end{subfigure}
	\caption{Window-average DSDI.}
	\label{fig:lineplot_dsdi_64}
\end{figure}