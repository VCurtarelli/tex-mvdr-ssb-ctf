\section{Signal and Array Model}
\label{sec:signal_model}

Let $S$ be a uniform rectangular array (URA) of sensors over the ${\x}-{\y}$ plane in an anechoic environment with desired and undesired sources. The URA comprises $\Mx$ sensors spaced $\dx$ apart along the $\x$-axis and $\My$ sensors spaced $\dy$ apart along the $\y$-axis, resulting in a total of $M = \Mx\My$ sensors. Assume a source in the far-field on the same plane as the sensor array (that is, with elevation $\phi = 0\dg$), impinging on it from an azimuth angle $\t$. As it is unusual, in speech enhancement, for the desired and undesired sources to have the same azimuth, differing only by elevation, we assume the elevation to be $0\dg$. This constraint can be easily removed without affecting the mathematical framework developed.

Let $\bvD(\w,\t)$ denote the steering matrix of size $\sz{\Mx}{\My}$ with elements $\{m_x,m_y\}$ given by
\begin{equation}
	\el{\bvD(\w,\t)}[m_x,m_y] = \exp*{-\j \tfrac{\w}{c}\R[m_x,m_y] \cos(\t-\p[m_x,m_y])} \eqc
\end{equation}
where $c=340~$m/s is the speed of sound,
$(\R[m_x,m_y],\p[m_x,m_y])$ are the polar  coordinates of the sensor at $(m_x\dx,m_y\dy)$, $\w = 2\pi f$ is the angular frequency, 
$f$ is the temporal frequency, and $\j=\Sqrt{-1}$ is the imaginary unit. We denote by $\bvd(\w,\t) = \vect{\bvD(\w,\t)}$ the $\sz{M}{1}$ steering vector, with $M=M_xM_y$, and $\vect{\,\cdot\,}$ being the vectorization operation; and let $\bvD(\w,\t) = \ivect{\My}{\bvd(\w,\t)}$ denote the inverse vectorization of $\bvd(\w,\t)$.

The observed signal vector $\bvy(\w)$, of size $M\times 1$, for all sensors in the frequency domain can be written as
\begin{equation}
	\bvy(\w) = \bvd(\w,\td) X(\w) + \bvv(\w) \eqc
\end{equation}
where $X(\w)$ is the desired signal at the reference sensor, $\bvd(\w,\td)$ is the steering vector of the desired source from the direction $\td$, and $\bvv(\w)$ is the additive noise signal vector. All signals are assumed to be zero-mean and uncorrelated. We can estimate the desires signal $X(\w)$ as $Z(\w)$ using a beamformer $\bvh(\w)$ (assumed a 2-D beamformer), through the linear filtering
\begin{equation}
	Z(\w) = \he{\bvh(\w)} \bvy(\w) \eqc
\end{equation}
where the superscript $\he{}$ denotes the conjugate-transpose operator. The beamformer is called distortionless if it satisfies $\he{\bvh(\w)} \bvd(\w,\td) = 1$ for all $\w$. In that case, 
\begin{equation}
    \label{eq:Zw_beamformer_output}
	Z(\w) = X(\w) + \bvv{\rn}(\w) \eqc
\end{equation}
where $\bvv{\rn}(\w)=\he{\bvh(\w)} \bvv(\w)$ is the residual noise at the beamformer's output. This constraint guarantees the beamformer to not affect the desired signal, only altering the undesired noise signal.

From here on, $\w$ will be omitted unless in definitions, and $\t$ in the steering vectors will appear in subscripts where necessary. When no angle is shown, $\bvd=\bvd(\w,\td)$ is assumed to be the desired-signal steering vector for conciseness.

\subsection{Beamformer metrics}

The beampattern $\beam$, as a function of the beamformer $\bvh$ and the direction $\t$ (through the steering vector $\bvd{\t}$) is given by
\begin{equation}
    \beam{\bvh,\bvd{\t}} = \he{\bvh} \bvd{\t} \eqp
\end{equation}

Given the desired-signal steering vector $\bvd$, the white noise gain (WNG), desired signal distortion index (DSDI), and directivity factor (DF) are, respectively
\begin{subgather}
	\wng{\bvh,\bvd} = \frac{ \abs{ \he{\bvh}\bvd }^2 }{ \he{\bvh}\bvh } \eqc \\
	\dsdi{\bvh,\bvd} = \abs{ \he{\bvh} \bvd - 1 }^2 \eqc \\
	\df{\bvh,\bvd} = \frac{ \abs{ \he{\bvh} \bvd }^2 }{ \he{\bvh} \bv{\Gamma} \bvh } \label{subeq:def_df} \eqc
\end{subgather}
where $\bv{\Gamma}(\w)$ is the spherical isotropic noise field coherence matrix \cite{habets_generating_2007}. Using the DSDI, the distortionless constraint can also be written as $\dsdi{\bvh,\bvd} = 0$. 