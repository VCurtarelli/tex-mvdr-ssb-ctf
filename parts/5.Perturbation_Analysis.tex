\section{Perturbation Robustness Analysis}
\label{sec:perturbation_analysis}
Until now, we assumed an appropriate knowledge of all signals and their impulse responses. However, in a real application these would be estimated, and thus prone to error. Given our beamformers from \cref{eq:sec3:solution_mpdr_beamformer,eq:sec4:solution_mpdr_beamformer_tssbt} and their dependence on $\bvd[k]$, they are directly influenced by impulse response estimation errors.

%\begin{changingfrom}
%	We can write
%	\begin{equation}
%		\bvd{x}[k] = \bvd[b]{x}[k] + \bvde{x}[k]
%	\end{equation}
%	where $\bvd[b]{x}[k]$ is the accurate steering vector (SV), $\bvd{x}[k]$ is the measured SV, $\bvde{x}[k]$ is a perturbation (or error) on the SV measurement. With this, the MPDR beamformer with the STFT (assuming the knowledge of $\bvd{x}[k]$) from \cref{eq:sec3:solution_mpdr_beamformer} is
%	\begin{equations}
%		\bvf{\mpdr}
%		& = \frac{\iCorr{\bvy} \bvd{x}}{\pts{\he{\bvd{x}} \iCorr{\bvy} \bvd{x}}} \\
%		& = \frac{\iCorr{\bvy} \pts{\bvd[b]{x} + \bvde{x}}}{\pts{\he{\bvd[b]{x}} + \he{\bvde{x}}} \iCorr{\bvy} \pts{\bvd[b]{x} + \bvde{x}}} \\
%		& = \frac{\iCorr{\bvy} \bvd[b]{x} + \iCorr{\bvy} \bvde{x} }{ \he{\bvd[b]{x}} \iCorr{\bvy} \bvd[b]{x} + \he{\bvd[b]{x}} \iCorr{\bvy} \bvde{x} + \he{\bvde{x}} \iCorr{\bvy} \bvd[b]{x} + \he{\bvde{x}} \iCorr{\bvy} \bvde{x} } \\
%		& = \frac{ \iCorr{\bvy} \bvd[b]{x}}{ \he{\bvd[b]{x}} \iCorr{\bvy} \bvd[b]{x} + e} + \frac{ \iCorr{\bvy} \bvde{x} }{ \he{\bvd[b]{x}} \iCorr{\bvy} \bvd[b]{x} + e}
%	\end{equations}
%	\begin{equation}
%		\label{eq:sec5:beamformer_approx_sv}
%		\bvf{\mpdr}[l,k] = g_e[l,k] \bvfb{\mpdr}[l,k] + \bvf{\bvde}[l,k]
%	\end{equation}
%	in which $\bvfb{\mpdr}[l,k]$ is the beamformer for the accurate steering vector $\bvd[b]{x}[k]$, and $g_e[l,k]$ and $\bvf{\bvde}[l,k]$ are
%	\begin{subgather}
%		g_e = \frac{ \he{\bvd[b]{x}} \iCorr{\bvy} \bvd[b]{x} }{ \he{\bvd[b]{x}} \iCorr{\bvy} \bvd[b]{x} + e} \\
%		\bvf{\bvde} = \frac{ \iCorr{\bvy} \bvde{x} }{ \he{\bvd[b]{x}} \iCorr{\bvy} \bvd[b]{x} + e}
%	\end{subgather}
%	
%	Thus, the beamformer with an estimated steering vector is an affine transformation of $\bvfb{\mpdr}[l,k]$. It is trivial that $\bvde{x} \to \mathbf{0} \implies g_e = 1,~\bvf{\bvde} = \mathbf{0}$.
%	
%	Applying the same procedure to the MPDR beamformer with the SSBT from \cref{eq:sec4:solution_mpdr_beamformer_tssbt}, we get a similar result as obtained in \cref{eq:sec5:beamformer_approx_sv}, but in which $e$, $g_e$, and $\bvf{\bvde}$ now are
%	\begin{subgather}
%		e = \tr{\bvd{x;\re}} \,\bvPsi\, \bvde{x;\im} + \tr{\bvde{x;\re}} \,\bvPsi\, \bvd{x;\im} + \tr{\bvde{x;\re}} \,\bvPsi\, \bvde{x;\im} \\
%		g_e = \frac{ \tr{\bvd{x;\re}} \,\bvPsi\, \bvd{x;\im} }{ \tr{\bvd{x;\re}} \,\bvPsi\, \bvd{x;\im} + e } \\
%		\bvf{\bvde} = \frac{\bvPsi\, \bvde{x;\im}}{ \tr{\bvd{x;\re}} \,\bvPsi\, \bvd{x;\im} + e }
%	\end{subgather}
%	where $\bvde{x;\re}$ and $\bvde{x;\im}$ are the perturbation vectors for $\bvd{x;\re}$ and $\bvd{x;\im}$ respectively.
%\end{changingfrom}
%\begin{changingto}
Going back to the time domain in \cref{eq:sec3:time_model_basic}, we can write the observed/measured room impulse response $h_m[n]$ for each sensor as
\begin{equation}
	h_m[n] = h_m^\star[n] + \dlt h_m[n]
\end{equation}
with $h_m^\star$ being the accurate room impulse response, and $\dlt h_m[n]$ is a perturbation (or error) on the measurement. Through the same derivations that were presented in \cref{sec:signal_model}, we find that
\begin{equation}
	\bvA{k} = \bvA{k}^\star + \dlt \bvA{k}
\end{equation}

This will also have an effect on the beamformers designed with an inaccurate steering matrix $\bvA{k}$, leading to errors in the signal estimation and beamforming output. Note that $\dlt\bvA{k}$ will depend on both $h_m^\star[n]$ and $\dlt h_m[n]$ given the deconvolution process necessary to obtain the relative transfer functions. From now on we will omit the $[l]$ dependence on the variables, unless strictly necessary.

The solutions obtained at the end of both \cref{sec:signal_model,sec:true_mpdr_ssbt} have the same form. Expanding the inverse in the beamformer's solution from \cref{eq:sec3:solution_mpdr_beamformer}, we get
\begin{equations}
	\inv*{\he{\bvA{k}} \iCorr{\bvy;k} \bvA{k}}
	& = \inv*{ \he{\pts{\bvA{k}^\star + \dlt\bvA{k}}} \iCorr{\bvy;k} \he{\pts{\bvA{k}^\star + \dlt\bvA{k}}} } \\
	& = \inv*{ \he{{\bvA{k}^\star}} \iCorr{\bvy;k} \bvA{k}^\star + \he{\dlt\bvA{k}} \iCorr{\bvy;k} \bvA{k}^\star + \he{{\bvA{k}^\star}} \iCorr{\bvy;k} \dlt\bvA{k} + \he{\dlt\bvA{k}} \iCorr{\bvy;k} \dlt\bvA{k} }
\end{equations}
We will denote $\bvOm = \he{{\bvA{k}^\star}} \iCorr{\bvy;k} \bvA{k}^\star$, and $\bvPsi$ as the rest. Using the Woodbury identity, we have that
\begin{equation}
\inv*{ \bvOm + \bvPsi} = \Inv{\bvOm} - \Inv{\bvOm}\Inv*{\Inv{\bvOm} + \Inv{\bvPsi}} \Inv{\bvOm}
\end{equation}

Going back to \cref{eq:sec3:solution_mpdr_beamformer}, we have that
\begin{equations}
	\bvf{\mpdr;k}
	& = \iCorr{\bvy;k} \pts{\bvA{k}^\star + \dlt\bvA{k}} \pts{\Inv{\bvOm} - \Inv{\bvOm}\Inv*{\Inv{\bvOm} + \Inv{\bvPsi}} \Inv{\bvOm}} \bvi{\Delta} \\
	& = \bvf{\mpdr;k}^\star + \bvf{\delta;k}
\end{equations}
in which
\begin{subgather}
	\bvf{\mpdr;k}^\star = \iCorr{\bvy;k} \bvA{k}^\star \inv*{\he{{\bvA{k}^\star}} \iCorr{\bvy;k} \bvA{k}^\star} \bvi{\Delta} \\
	\bvf{\delta;k} = \iCorr{\bvy;k} \bts{ \dlt\bvA{k}\Inv{\bvOm} - \pts{\bvA{k}^\star + \dlt\bvA{k}} \pts{\Inv{\bvOm}\Inv*{\Inv{\bvOm} + \Inv{\bvPsi}} \Inv{\bvOm}} }\bvi{\Delta} 
\end{subgather}

Thus, the resulting beamformer is composed of the accurate beamformer $\bvf{\mpdr;k}^\star$ and one that is dependent on both the accurate and the perturbation steering matrices, $\bvf{\delta;k}$, both being dependent on $[l]$. Of course, when $\dlt\bvA{k} \to \bv{0}$, then $\bvf{\delta;k} \to \bv{0}$, and $\bvf{\mpdr;k} \to \bvf{\mpdr;k}^\star$.

This same process is valid for the beamformer in \cref{eq:sec4:solution_mpdr_beamformer_tssbt}, with $\bvA[u]{k}$, $\Corr{\bvy';k}$ and $\bvi[u]{\Delta}$ instead of $\bvA{k}$, $\Corr{\bvy;k}$ and $\bvi{\Delta}$, respectively, and also taking the transpose instead of the hermitian-transpose.
%\end{changingto}
%\subsection{Tests}
%
%In these tests, we repeated the simulation as before, but now with a slightly miscalculated steering vector. This was done by simulating $\bvde{x}[k]$ as a zero-mean random additive noise, whose variance is the controllable variable. In the plots of \cref{fig:lineplots_32_n15_est_var_v8}, the x-axis indicates the standard deviation of $\bvde{x}[k]$ relative to that of $\bvd{x}[k]$. Trivially, the case where the error is $0\%$ is when the steering vector is accurately evaluated. The input SER in this scenario is of $-15\dB$.
%
%We are still interested in the same three results as before, but given the (possible) desired signal distortion, the Signal-to-Echo Ratio gain (gSER) and Signal-to-Noise Ratio gain (gSNR) will be used. This choice normalizes the metrics by the signal reduction. These are defined as
%\begin{subgather}{eq:sec5:time-average_metrics}
%	\gser = \frac{\erle}{\dsrf} \\
%	\gsnr = \frac{\nsrf}{\dsrf}
%\end{subgather}

%Given that, in practice, the ratio between desired and undesired signals is of uttermost interest, these better reflect the quality of the output, when there is distortion.
%
%\begin{figure}[!t]
%	\centering
%	\begin{subfigure}{\textwidth}
%		\centering
%		\input{figures/lineplot_dsrf_32_n15_est_var_v8.tex}
%		\caption{Time-average broadband DSRF.}
%		\label{subfig:lineplot_dsrf_32_n15_var_est_v8}
%	\end{subfigure}\\[1em]
%	\begin{subfigure}{\textwidth}
%		\centering
%		\input{figures/lineplot_gser_32_n15_est_var_v8.tex}
%		\caption{Time-average broadband gSER.}
%		\label{subfig:lineplot_gser_32_n15_var_est_v8}
%	\end{subfigure}\\[1em]
%	\begin{subfigure}{\textwidth}
%		\centering
%		\input{figures/lineplot_gsnr_32_n15_est_var_v8.tex}
%		\caption{Time-average broadband gSNR.}
%		\label{subfig:lineplot_gsnr_32_n15_est_var_v8}
%	\end{subfigure}\\[1em]
%	\ref*{timeplot_legend}
%	\caption{Output metrics for the beamformers.}
%	\label{fig:lineplots_32_n15_est_var_v8}
%\end{figure}
%
%Analysing the results, we see that all methods follow a similar pattern, in all observed metrics. In the DSRF, it is noticeable that the signal distortion increases as the error in the steering vector increases. Interestingly, the gain in SER increases slightly as the error increases, however in general the overall gain in SNR decreases, so even though we reduce more of the echo signal, we would end up increasing the other undesired sources.
%
%In both \cref{subfig:lineplot_dsrf_32_n15_var_est_v8,subfig:lineplot_gsnr_32_n15_est_var_v8} we see that the STFT beamformer was the more robust one, followed by the DF-SSBT and then the SF-SSBT. The results in \cref{subfig:lineplot_gser_32_n15_var_est_v8} differ from this, with the DF-SSBT marginally outperforming the STFT beamformer.