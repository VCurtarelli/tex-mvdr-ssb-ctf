\section{Perturbation Robustness Analysis}

Until now, we assumed an appropriate knowledge of all signals and their impulse responses. However, in a real application these would be estimated, and thus prone to error. Given our beamformers from \cref{eq:sec3:solution_mpdr_beamformer,eq:sec4:solution_mpdr_beamformer_tssbt} and their dependence on $\bvd[k]$, they are directly influenced by impulse response estimation errors.

We can write
\begin{equation}
	\bvd{x}[k] = \bvd[b]{x}[k] + \bvde{x}[k]
\end{equation}
where $\bvd[b]{x}[k]$ is the accurate steering vector (SV), $\bvd{x}[k]$ is the measured SV, $\bvde{x}[k]$ is a perturbation (or error) on the SV measurement. With this, the MPDR beamformer with the STFT (assuming the knowledge of $\bvd{x}[k]$) from \cref{eq:sec3:solution_mpdr_beamformer} is
\begin{equations}
	\bvf{\mpdr}
	& = \frac{\iCorr{\bvy} \bvd{x}}{\pts{\he{\bvd{x}} \iCorr{\bvy} \bvd{x}}} \\
	& = \frac{\iCorr{\bvy} \pts{\bvd[b]{x} + \bvde{x}}}{\pts{\he{\bvd[b]{x}} + \he{\bvde{x}}} \iCorr{\bvy} \pts{\bvd[b]{x} + \bvde{x}}} \\
	& = \frac{\iCorr{\bvy} \bvd[b]{x} + \iCorr{\bvy} \bvde{x} }{ \he{\bvd[b]{x}} \iCorr{\bvy} \bvd[b]{x} + \he{\bvd[b]{x}} \iCorr{\bvy} \bvde{x} + \he{\bvde{x}} \iCorr{\bvy} \bvd[b]{x} + \he{\bvde{x}} \iCorr{\bvy} \bvde{x} } \\
	& = \frac{ \iCorr{\bvy} \bvd[b]{x}}{ \he{\bvd[b]{x}} \iCorr{\bvy} \bvd[b]{x} + e} + \frac{ \iCorr{\bvy} \bvde{x} }{ \he{\bvd[b]{x}} \iCorr{\bvy} \bvd[b]{x} + e}
\end{equations}
\begin{equation}
	\label{eq:sec5:beamformer_approx_sv}
	\bvf{\mpdr}[l,k] = g_e[l,k] \bvfb{\mpdr}[l,k] + \bvf{\bvde}[l,k]
\end{equation}
in which $\bvfb{\mpdr}[l,k]$ is the beamformer for the accurate steering vector $\bvd[b]{x}[k]$, and $g_e[l,k]$ and $\bvf{\bvde}[l,k]$ are
\begin{subgather}
	g_e = \frac{ \he{\bvd[b]{x}} \iCorr{\bvy} \bvd[b]{x} }{ \he{\bvd[b]{x}} \iCorr{\bvy} \bvd[b]{x} + e} \\
	\bvf{\bvde} = \frac{ \iCorr{\bvy} \bvde{x} }{ \he{\bvd[b]{x}} \iCorr{\bvy} \bvd[b]{x} + e}
\end{subgather}

Thus, the beamformer with an estimated steering vector is an affine transformation of $\bvfb{\mpdr}[l,k]$. It is trivial that $\bvde{x} \to \mathbf{0} \implies g_e = 1,~\bvf{\bvde} = \mathbf{0}$.

Applying the same procedure to the MPDR beamformer with the SSBT from \cref{eq:sec4:cf_solution_mpdr_beamformer_tssbt}, we get a similar result as obtained in \cref{eq:sec5:beamformer_approx_sv}, but in which $e$, $g_e$, and $\bvf{\bvde}$ now are
\begin{subgather}
	e = \tr{\bvd{x;\re}} \,\bvOm\, \bvde{x;\im} + \tr{\bvde{x;\re}} \,\bvOm\, \bvd{x;\im} + \tr{\bvde{x;\re}} \,\bvOm\, \bvde{x;\im} \\
	g_e = \frac{ \tr{\bvd{x;\re}} \,\bvOm\, \bvd{x;\im} }{ \tr{\bvd{x;\re}} \,\bvOm\, \bvd{x;\im} + e } \\
	\bvf{\bvde} = \frac{\bvOm\, \bvde{x;\im}}{ \tr{\bvd{x;\re}} \,\bvOm\, \bvd{x;\im} + e }
\end{subgather}
where $\bvde{x;\re}$ and $\bvde{x;\im}$ are the perturbation vectors for $\bvd{x;\re}$ and $\bvd{x;\im}$ respectively.