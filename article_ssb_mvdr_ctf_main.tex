%  LaTeX support: latex@mdpi.com 
%  For support, please attach all files needed for compiling as well as the log file, and specify your operating system, LaTeX version, and LaTeX editor.

%=================================================================
\documentclass[algorithms ,article,submit,moreauthors]{Definitions/mdpi} 

%BEGIN_FOLD
%--------------------
% Class Options:
%--------------------
%----------
% journal
%----------
% Choose between the following MDPI journals:
% acoustics, actuators, addictions, admsci, adolescents, aerobiology, aerospace, agriculture, agriengineering, agrochemicals, agronomy, ai, air, algorithms, allergies, alloys, analytica, analytics, anatomia, animals, antibiotics, antibodies, antioxidants, applbiosci, appliedchem, appliedmath, applmech, applmicrobiol, applnano, applsci, aquacj, architecture, arm, arthropoda, arts, asc, asi, astronomy, atmosphere, atoms, audiolres, automation, axioms, bacteria, batteries, bdcc, behavsci, beverages, biochem, bioengineering, biologics, biology, biomass, biomechanics, biomed, biomedicines, biomedinformatics, biomimetics, biomolecules, biophysica, biosensors, biotech, birds, bloods, blsf, brainsci, breath, buildings, businesses, cancers, carbon, cardiogenetics, catalysts, cells, ceramics, challenges, chemengineering, chemistry, chemosensors, chemproc, children, chips, cimb, civileng, cleantechnol, climate, clinpract, clockssleep, cmd, coasts, coatings, colloids, colorants, commodities, compounds, computation, computers, condensedmatter, conservation, constrmater, cosmetics, covid, crops, cryptography, crystals, csmf, ctn, curroncol, cyber, dairy, data, ddc, dentistry, dermato, dermatopathology, designs, devices, diabetology, diagnostics, dietetics, digital, disabilities, diseases, diversity, dna, drones, dynamics, earth, ebj, ecologies, econometrics, economies, education, ejihpe, electricity, electrochem, electronicmat, electronics, encyclopedia, endocrines, energies, eng, engproc, entomology, entropy, environments, environsciproc, epidemiologia, epigenomes, est, fermentation, fibers, fintech, fire, fishes, fluids, foods, forecasting, forensicsci, forests, foundations, fractalfract, fuels, future, futureinternet, futurepharmacol, futurephys, futuretransp, galaxies, games, gases, gastroent, gastrointestdisord, gels, genealogy, genes, geographies, geohazards, geomatics, geosciences, geotechnics, geriatrics, grasses, gucdd, hazardousmatters, healthcare, hearts, hemato, hematolrep, heritage, higheredu, highthroughput, histories, horticulturae, hospitals, humanities, humans, hydrobiology, hydrogen, hydrology, hygiene, idr, ijerph, ijfs, ijgi, ijms, ijns, ijpb, ijtm, ijtpp, ime, immuno, informatics, information, infrastructures, inorganics, insects, instruments, inventions, iot, j, jal, jcdd, jcm, jcp, jcs, jcto, jdb, jeta, jfb, jfmk, jimaging, jintelligence, jlpea, jmmp, jmp, jmse, jne, jnt, jof, joitmc, jor, journalmedia, jox, jpm, jrfm, jsan, jtaer, jvd, jzbg, kidneydial, kinasesphosphatases, knowledge, land, languages, laws, life, liquids, literature, livers, logics, logistics, lubricants, lymphatics, machines, macromol, magnetism, magnetochemistry, make, marinedrugs, materials, materproc, mathematics, mca, measurements, medicina, medicines, medsci, membranes, merits, metabolites, metals, meteorology, methane, metrology, micro, microarrays, microbiolres, micromachines, microorganisms, microplastics, minerals, mining, modelling, molbank, molecules, mps, msf, mti, muscles, nanoenergyadv, nanomanufacturing,\gdef\@continuouspages{yes}} nanomaterials, ncrna, ndt, network, neuroglia, neurolint, neurosci, nitrogen, notspecified, %%nri, nursrep, nutraceuticals, nutrients, obesities, oceans, ohbm, onco, %oncopathology, optics, oral, organics, organoids, osteology, oxygen, parasites, parasitologia, particles, pathogens, pathophysiology, pediatrrep, pharmaceuticals, pharmaceutics, pharmacoepidemiology,\gdef\@ISSN{2813-0618}\gdef\@continuous pharmacy, philosophies, photochem, photonics, phycology, physchem, physics, physiologia, plants, plasma, platforms, pollutants, polymers, polysaccharides, poultry, powders, preprints, proceedings, processes, prosthesis, proteomes, psf, psych, psychiatryint, psychoactives, publications, quantumrep, quaternary, qubs, radiation, reactions, receptors, recycling, regeneration, religions, remotesensing, reports, reprodmed, resources, rheumato, risks, robotics, ruminants, safety, sci, scipharm, sclerosis, seeds, sensors, separations, sexes, signals, sinusitis, skins, smartcities, sna, societies, socsci, software, soilsystems, solar, solids, spectroscj, sports, standards, stats, std, stresses, surfaces, surgeries, suschem, sustainability, symmetry, synbio, systems, targets, taxonomy, technologies, telecom, test, textiles, thalassrep, thermo, tomography, tourismhosp, toxics, toxins, transplantology, transportation, traumacare, traumas, tropicalmed, universe, urbansci, uro, vaccines, vehicles, venereology, vetsci, vibration, virtualworlds, viruses, vision, waste, water, wem, wevj, wind, women, world, youth, zoonoticdis 
% For posting an early version of this manuscript as a preprint, you may use "preprints" as the journal. Changing "submit" to "accept" before posting will remove line numbers.

%---------
% article
%---------
% The default type of manuscript is "article", but can be replaced by: 
% abstract, addendum, article, book, bookreview, briefreport, casereport, comment, commentary, communication, conferenceproceedings, correction, conferencereport, entry, expressionofconcern, extendedabstract, datadescriptor, editorial, essay, erratum, hypothesis, interestingimage, obituary, opinion, projectreport, reply, retraction, review, perspective, protocol, shortnote, studyprotocol, systematicreview, supfile, technicalnote, viewpoint, guidelines, registeredreport, tutorial
% supfile = supplementary materials

%----------
% submit
%----------
% The class option "submit" will be changed to "accept" by the Editorial Office when the paper is accepted. This will only make changes to the frontpage (e.g., the logo of the journal will get visible), the headings, and the copyright information. Also, line numbering will be removed. Journal info and pagination for accepted papers will also be assigned by the Editorial Office.

%------------------
% moreauthors
%------------------
% If there is only one author the class option oneauthor should be used. Otherwise use the class option moreauthors.

%---------
% pdftex
%---------
% The option pdftex is for use with pdfLaTeX. Remove "pdftex" for (1) compiling with LaTeX & dvi2pdf (if eps figures are used) or for (2) compiling with XeLaTeX.
%END_FOLD
%=================================================================
% MDPI internal commands - do not modify
\firstpage{1} 
\makeatletter 
\setcounter{page}{\@firstpage} 
\makeatother
\pubvolume{1}
\issuenum{1}
\articlenumber{0}
\pubyear{2023}
\copyrightyear{2023}
%\externaleditor{Academic Editor: Firstname Lastname}
\datereceived{ } 
\daterevised{ } % Comment out if no revised date
\dateaccepted{ } 
\datepublished{ } 
%\datecorrected{} % For corrected papers: "Corrected: XXX" date in the original paper.
%\dateretracted{} % For corrected papers: "Retracted: XXX" date in the original paper.
\hreflink{https://doi.org/} % If needed use \linebreak
%\doinum{}
%\pdfoutput=1 % Uncommented for upload to arXiv.org

%=================================================================
% Add packages and commands here. The following packages are loaded in our class file: fontenc, inputenc, calc, indentfirst, fancyhdr, graphicx, epstopdf, lastpage, ifthen, float, amsmath, amssymb, lineno, setspace, enumitem, mathpazo, booktabs, titlesec, etoolbox, tabto, xcolor, colortbl, soul, multirow, microtype, tikz, totcount, changepage, attrib, upgreek, array, tabularx, pbox, ragged2e, tocloft, marginnote, marginfix, enotez, amsthm, natbib, hyperref, cleveref, scrextend, url, geometry, newfloat, caption, draftwatermark, seqsplit
% cleveref: load \crefname definitions after \begin{document}
% %% Requires:
% Fontspec
% Unicode-math

%% Fontspec fonts
\setmainfont{CMU Serif}
\DeclareMicrotypeAlias{CMU Serif}{TU-basic}
\setmonofont{Cousine}
%
\newfontfamily\tgpfont{TeX Gyre Pagella}
\newcommand\tgp[1]{{\tgpfont\symbol{#1}}}
\newfontfamily\arialfont{Arial}
\newcommand\org{\arialfont\color{DarkGreen}}

%% Unicode-math fonts
\setmathfont{Asana Math}
\setmathfont{Latin Modern Math}[range={cal,bfcal,\ast,\circledast}]
\setmathfont{TeX Gyre DejaVu Math}[range={scr,bfscr, sfit->it,\prime}]
%
\usepackage{algorithm}
\usepackage{algpseudocode}
\usepackage{algorithmicx}
\usepackage{array}
%\usepackage[caption=false,font=normalsize,labelfont=sf,textfont=sf]{subfig}
\usepackage{textcomp}

% \usepackage{cite}
%\hyphenation{op-tical net-works semi-conduc-tor IEEE-Xplore}
%% updated with editorial comments 8/9/2021

% Packages
%\usepackage[svgnames]{xcolor}
%\usepackage[hidelinks]{hyperref}
%\usepackage[nameinlink]{cleveref}
\usepackage{lipsum}
\usepackage{xparse}
\usepackage[math]{cellspace}
\usepackage{nicefrac}
\usepackage{letltxmacro}
%\usepackage[
%	backend=biber,
%	style=ieee,
%	doi=false,
%	isbn=false,
%	url=false,]{biblatex}
\usepackage{relsize}
\usepackage{tikz}
\usepackage{pgfplots}
\usepackage{caption}
\usepackage{subcaption}
\usepackage{titlesec}

\usepackage{accents}
\usepackage{booktabs}
\usepackage{siunitx}
\usepackage{amsfonts}
% \usepackage{showframe}
% %% Requires:
% Fontspec
% Unicode-math

%% Inputenc fonts
\DeclareMathAlphabet{\mathnfrak}{U}{esstixfrak}{m}{n}

% Common commands
%% Requires:
% xparse
% nicefrac
% xcolor

\NewDocumentCommand{\IfEmpty}{m m G{}}{\ifthenelse{\equal{#1}{}}{#2}{#3}}%
\let\ABD\AtBeginDocument
\let\nfrac\nicefrac
\NewDocumentCommand{\MakeSymbol}{m m}{
	\ABD{\DeclareDocumentCommand{#1}{}{#2}}
}
%% Formating
\setlength{\parskip}{1em}
% \setlength{\textheight}{26.0cm}

\definecolor{TodoBlue}{HTML}{50B0D0}
\DeclareDocumentCommand{\todo}{m}{%
	{\texttt{\textcolor{TodoBlue}{-TODO-:} [ \textcolor{TodoBlue}{\detokenize{#1}}]}}%
}
\DeclareDocumentCommand{\tocite}{m m}{%
	{\todo{\cite{#1} / #2}}%
}

\NewDocumentCommand{\real}{m}{%
	\re\cts{#1}%
}
\NewDocumentCommand{\imag}{m}{%
	\im\cts{#1}%
}

\DeclareDocumentCommand{\re}{}{%
	\mathbb{R}%
}
\DeclareDocumentCommand{\im}{}{%
	\mathbb{I}%
}

\DeclareDocumentCommand{\Re}{m}{%
	#1^{\re}%
}
\DeclareDocumentCommand{\Im}{m}{%
	#1^{\im}%
}

\DeclareMathOperator{\Sign}{sgn}
\NewDocumentCommand{\sgn}{m}{\Sign\pts{#1}}

\DeclareDocumentCommand{\ceil}{m}{%
	\left\lceil #1 \right\rceil
}
\DeclareDocumentCommand{\floor}{m}{%
	\left\lfloor #1 \right\rfloor
}

\NewDocumentCommand{\expec}{m}{%
	\mathbb{E}\cts{#1}%
}

\makeatletter
\DeclareDocumentCommand{\asin}{m}{%
	\sin^{-1}\pts{#1}%
}
\DeclareDocumentCommand{\acos}{m}{%
	\cos^{-1}\pts{#1}%
}
\DeclareDocumentCommand{\atan}{m}{%
	\tan^{-1}\pts{#1}%
}

\makeatother

%\setlength{\headheight}{20pt}
%% Requires:
% Fontspec.sty
% basics.tex
% oset.tex

%% Renew forall symbol
% BEGIN_FOLD%
    % \ABD{
    %     \let\oforall\forall
    %     \def\Sym@fa{\text{$\mathrm{A}$}}
    %     \MakeSymbol{\forall}{
    %         ~\raisebox{\heightof{\Sym@fa}}{\scalebox{0.72}[-0.9]{\Sym@fa}~}
    %     }
    % }
%END_FOLD

%% Renew complex-unity 'j' symbol
% BEGIN_FOLD
    % \newfontfamily\cmusuifont{CMU Serif Upright Italic}
    % \newcommand{\cmusui}[1]{{\cmusuifont#1}}
%
    \MakeSymbol{\j}{%
        \mathrm{j}%
    }
% END_FOLD


\MakeSymbol{\-}{
    \text{-}
}
\MakeSymbol{\dg}{^{\mathrm{o}}}
\MakeSymbol{\eqc}{\text{,}}
\MakeSymbol{\eqp}{\text{.}}
\MakeSymbol{\eqsc}{\text{;}}
\MakeSymbol{\dd}{\mathrm{d}}

%\makeatletter
%\NewDocumentCommand{\SI}{m}{%
%    \text{#1}%
%}
%\NewDocumentCommand{\MakeSI}{m G{#1}}{%
%    \ABD{\@namedef{#1}{\SI{#2}}}
%}
%\makeatother
%\MakeSI{cm}
%\MakeSI{cm}
%\MakeSI{dB}
%\MakeSI{Hz}
%\MakeSI{kilo}{k}


\MakeSymbol{\defas}{%
    \mathbin{\oset{\raisebox{1pt}{\scalebox{0.6}[0.6]{$\triangle$}}}{=}}%
}
\MakeSymbol{\idiv}{%
    \mathbin{//}%
}
\MakeSymbol{\mod}%
}
\MakeSymbol{\kp}{%
    \mathbin{\otimes}%
}
\MakeSymbol{\conv}{%
    \mathbin{\ast}%
}
\MakeSymbol{\bconv}{%
	\mathbin{\circledast}%
}
\MakeSymbol{\vbconv}{%
	\mathbin{\rharpoon{\circledast}}%
}

\MakeSymbol{\PM}{
	\mathbin{\ooalign{\raisebox{1.5pt}{\scalebox{0.8}{$+$}}\cr\hfil\raisebox{-1.5pt}{\scalebox{0.8}{$-$}}\hfil}}
}

\MakeSymbol{\bsquare}{\fcolorbox{black}{black}{\null}}

\NewDocumentCommand{\vs}{}{%
	\textit{vs}
}

\makeatletter
\AtBeginDocument{
\def\-{\dabar@\!}
}
\makeatother
\DeclareDocumentCommand{\review}{m g}{
    \item \IfValueTF{#2}{
         {\color{LightGray}\soutit{#1}} - #2
    }{
        {\color{Gray}\textit{#1}}
    }
}

\newcounter{reviewer}
\setcounter{reviewer}{0}
\def\newreviewer{\stepcounter{reviewer}\setcounter{enumi}{0}}

\DeclareDocumentEnvironment{reviews}{}{
    \begin{enumerate}[label=\thereviewer.{\alph*}]
    \footnotesize
}{
    \end{enumerate}
}
%\Crefname{figure}{Fig.}{Figs.}
%\crefname{section}{sec.}{secs.}
\makeatletter
%\LetLtxMacro{\@cref}{\cref}
\ABD{
	\DeclareDocumentCommand{\mref}{m g}{%
		\IfValueT{#2}{\cite{#2}-}\cref{#1}
	}%
}%
\ABD{
	\DeclareDocumentCommand{\eref}{m m}{%
		\cite[][\cref{#1:#2}]{#1}%
	}%
}%
\makeatother

\DeclareDocumentCommand{\addlblresource}{m}{
	\input{#1}
}

\DeclareDocumentCommand{\extlabel}{m m O{0}m}{%
	% Label | Type | Chap. number | Number
	\newlabel{#1}{{20}{1}{}{#2.#3.#4}{}}
	\newlabel{#1@cref}{{[#2][#4][]#4}{[1][1][]1}}
}

\titleformat{\subsubsection}{\normalfont}{\textit{\Alph{subsection}}.\arabic{subsubsection}.}{1em}{\itshape}
%% Requires:
% PGFPlots

%BEGIN_FOLD

\usepgfplotslibrary{polar}
\newlength{\Yaxisshift}
\setlength{\Yaxisshift}{3.8cm}

\usetikzlibrary{external}
\tikzexternalize
\tikzsetexternalprefix{figures/tikzpics/}

%END_FOLD

%BEGIN_FOLD
\usepgfplotslibrary{colorbrewer}
\NewDocumentEnvironment{heatmap}{m m O{}}{
	\begin{axis}[
		colormap/viridis,
		colorbar horizontal,
		colorbar style={
			colormap/viridis,
			width=0.4\linewidth,
			xlabel = {#2},
			xtick = {0, 5, ..., 15},
			xticklabel style={
				/pgf/number format/.cd,
				fixed,
				%title=,
				precision=0,
				fixed zerofill,
			},
			major tick length=4pt,
			xtick style={black},
			xtick pos=top,
			extra x tick style = {major x tick style={draw=none}},
			point meta min=-0,
			point meta max=15,
		},
		width=1.15\linewidth,
		height=0.7\linewidth,
		point meta min= -0.1,
		point meta max=15.1,
		mesh/cols=#1,
		%		mesh/rows=#2,
		%		xlabel={Angle ($\dg$)},
		%		xlabel style = {anchor=north,%
			%						yshift=0.7\linewidth-1.5em,%
			%						},
		view={0}{90},
		xtick = {0, 1, ..., \tmax},
		xtick style = {black},
		xticklabel style = {yshift=-3pt},
		%		xticklabel={$\pgfmathprintnumber{\tick}\dg$},
		grid style = {draw=none},
		%		ylabel = {$f$ ($\kilo\Hz$)},
		ylabel style={yshift=-1.05\linewidth+0.2em},
		ytick = {0, 2, ..., 8},
		major tick length=2pt,
		ytick style = {black},
		ytick pos = left,
		mesh/ordering=x varies,
		#3
		]
	}{
	\end{axis}
}
%END_FOLD

\newlength{\wdtwenty}
\setlength{\wdtwenty}{\widthof{$-20$}}
%\setlength{\wdtwenty}{\widthof{-20}}
\pgfplotsset{
	every colorbar global/.append style={
		zmin=,zmax=,
	}
}

%BEGIN_FOLD
\NewDocumentEnvironment{lineplot}{m O{}}{
	\begin{axis}[
		width=0.6\linewidth,
		height=0.36\linewidth,
		%
		xtick = {1, 2 ,..., 7},
		xtick pos=bottom,
		%		ytick align=outside,
		ytick pos=left,
		%
		xlabel = {Freq. ($\si{\kilo\hertz}$)},
		ylabel = {#1},
		yticklabel style={text width=\wdtwenty, align=right},
		ylabel style={yshift=-0.6\linewidth+0.3em},
		%
		xmin = \fmin,
		xmax = \fmax,
		%
		%		legend pos={outer south},
		%
		legend cell align={left},
		legend style={
			fill=white,
			%			fill opacity=0.6,
			draw opacity=1,
			text opacity=1,
			at={(0.5, -0.3)},
			anchor=north,
			legend columns=6
		},
		#2
		]
		\draw[ultra thin, dotted] (axis cs:\pgfkeysvalueof{/pgfplots/xmin},0) -- (axis cs:\pgfkeysvalueof{/pgfplots/xmax},0);
	}{
	\end{axis}
}
%END_FOLD

%BEGIN_FOLD
\tikzstyle{axis} = [
black
]
\tikzstyle{resA} = [
thick, ColA, dash pattern = {on 1pt off 0pt},
mark=square*, mark size=1, mark options={
	rotate=0,
	solid,
}
]
\tikzstyle{resB} = [
thick, ColB, dash pattern = {on 2pt off 1pt},
mark=triangle*, mark size=1.5, mark options={
	rotate=0,
	solid,
}
]
\tikzstyle{resC} = [
thick, ColC, dash pattern = {on 2pt off 2pt},
mark=star, mark size=2, mark options={
	rotate=0,
	solid,
	very thick
}
]
\tikzstyle{resD} = [
thick, ColD, dash pattern = {on 2pt off 3pt},
mark=square*, mark size=1.2, mark options={
	rotate=45,
	solid,
}
]
\tikzstyle{resE} = [
thick, ColE, dash pattern = {on 3pt off 1pt},
mark=triangle*, mark size=1.5, mark options={
	rotate=180,
	solid,
}
]
\tikzstyle{resF} = [
thick, ColF, dash pattern = {on 3pt off 2pt},
mark=x, mark size=2, mark options={
	rotate=0,
	solid,
	very thick
}
]
\tikzstyle{resG} = [
thick, ColG, dash pattern = {on 3pt off 4pt},
mark=+, mark size=2, mark options={
	rotate=0,
	solid,
	very thick
}
]
%END_FOLD
%% Requires:
% Algorithm
% Algorithmicx
% Algpseudocode

\algrenewcommand\algorithmicrequire{\textbf{Input:}}
\algrenewcommand\algorithmicensure{\textbf{Output:}}
\algrenewcommand\algorithmicprocedure{\textbf{Procedure:}}
\algrenewcommand\Procedure{\item[\algorithmicprocedure]}
\def\algcommsymb{$\#$ }
\algrenewcommand\algorithmiccomment[1]{ \hfill{\color{Gray} \algcommsymb \textit{#1}}}
\def\att{\leftarrow}
\NewDocumentCommand{\LineComment}{m}{\item[{\color{Gray} \algcommsymb \textit{#1}}]}

\DeclareMathOperator{\Len}{len}
\NewDocumentCommand{\len}{m}{\Len\pts{#1}}
%% Requires:
% siunitx

\DeclareSIUnit{\db}{%
	\text{dB}
}

\def\m{\si{\meter}}
\def\dB{\si{\db}}
%% Requires:
% xparse


\makeatletter
\NewDocumentCommand{\@my}{m m}{
	\@nameuse{@my#1}{#2}
}
% \NewDocumentCommand{\@myh}{m}{
%     \oset{\raisebox{1pt}[3pt][3pt]{\scalebox{1.0}[0.5]{$\wedge$}}}{#1}
% }
% \NewDocumentCommand{\@myt}{m}{
%     \oset{\scalebox{0.9}[0.9]{$\sim$}}{#1}
% }
% \NewDocumentCommand{\@mym}{m}{
%     \oset{\scalebox{1.0}[1.0]{$\minus$}}{#1}
% }
% \newlength{\@barwidth}
% \newlength{\@barheight}
% \NewDocumentCommand{\@myb}{s m}{
%     \IfBooleanTF{#1}{
%     	\oset{\scalebox{1.0}[1.0]{$\minus$}}{#1}
%     }{
%     	\setlength{\@barwidth}{\widthof{$#2$}}
%     	\setlength{\@barheight}{\heightof{$\minus$}}
%         \oset{\resizebox{\@barwidth}{\@barheight}{$\minus$}}{#2}
%    	}
% }
\NewDocumentCommand{\@myh}{m}{%
	\hat{#1}
}
\NewDocumentCommand{\@myt}{m}{%
	\tilde{#1}
}
\NewDocumentCommand{\@myb}{m}{%
	\bar{#1}
}
\NewDocumentCommand{\@myd}{m}{
	\dot{#1}
}
\NewDocumentCommand{\@myn}{m}{#1}
% \let\hat\@myh
% \let\tilde\@myt
% \let\bar\@myb
% \let\est\hat
\makeatother
% Requires:
%   xparse
%   cellspace

\newcommand{\pts}[1]{\udel{#1}}
\newcommand{\bts}[1]{\udel[{[}]{#1}[{]}]}
\newcommand{\cts}[1]{\udel[{\{}]{#1}[{\}}]}
\newcommand{\abs}[1]{\udel[|]{#1}[|]}
\newcommand{\norm}[1]{\udel[\lVert]{#1}[\rVert]}
\newsavebox{\jcsboxA}
\newsavebox{\jcsboxB}
\makeatletter
\newlength{\udel@lenA}
\newlength{\udel@lenB}
\NewDocumentCommand{\udel}{O{(} m O{)}}{%
%	\if@display
%		\left#1 #2 \right#3
%	\else
%        \setlength{\udel@lenA}{\heightof{$#2$}}
%        \setlength{\udel@lenB}{\heightof{$#1 #3$}}
%        \ifdim\udel@lenA>\udel@lenB
%            \big#1 #2 \big#3
%        \else
%            #1 #2 #3
%        \fi
%    \fi
	\left#1 #2 \right#3
}%
\makeatother


\makeatletter
\newcommand{\Sym@tr}{\mathsf{T}\vphantom{\bar{\mathsf{T}}}}
\newcommand{\Sym@he}{\mathsf{H}\vphantom{\bar{\mathsf{H}}}}
\newcommand{\tr}[1]{#1^{\Sym@tr}}
\newcommand{\he}[1]{#1^{\Sym@he}}
%
\NewDocumentCommand{\Sym@inv}{O{1}}{
	\myminus\mkern-2mu \mathsf{#1}
}
\NewDocumentCommand{\inv}{s O{1} m}{
	\IfBooleanTF{#1}{
		\bts{#3}^{\Sym@inv[#2]}
	}{
		#3^{\Sym@inv[#2]}
	}
}
\NewDocumentCommand{\invt}{m}{
	#1^{\Sym@inv[T]}
}
%
\NewDocumentCommand{\Sym@pinv}{!o}{
	\cross[#1]
}
\NewDocumentCommand{\pinv}{o m}{
	#2^{\Sym@pinv[#1]}
}
\NewDocumentCommand{\pinvR}{m}{
	#1^{\Sym@pinv[R]}
}
\NewDocumentCommand{\pinvL}{m}{
	#1^{\Sym@pinv[L]}
}

\LetLtxMacro{\Sqrt}{\sqrt}
\RenewDocumentCommand{\sqrt}{m}{
	#1^{\nfrac{1}{2}}
}
%\NewDocumentCommand{\isqrt}{m}{
%	\inv[\nfrac{1}{2}]{#1}
%}
%
\newcommand{\Sym@spr}{'}
\let\p\Sym@spr
\makeatother

\newcommand{\crossU}{ }%
\NewDocumentCommand{\cross}{!o}{%
	\IfValueT{#1}{\def\crossU{#1}}
	\mathpalette\docross\relax
}%
\newcommand{\docross}[2]{%
	\ooalign{%
   		\raise0.15em\hbox{\scalebox{0.6}[1]{$#1\mathsf{-}$}}\cr
   		\hfil$#1\stretchrel*{\vert}{M}$\hfil\cr
   	}#1_{\scalebox{0.5}{$\mathsf{\crossU}$}}%
}%
\newcommand{\myminus}{\mathpalette\domyminus\relax}%
\newcommand{\domyminus}[2]{%
	\scalebox{0.7}[1]{$#1\,\mathsf{-}$}
}%

\DeclareMathOperator{\Trace}{tr}
\NewDocumentCommand{\trace}{m}{%
\Trace\pts{#1}%
}
\NewDocumentCommand{\SubSize}{m m}{\IfValueT{#1}{_{\scalebox{0.7}{$ \sz{#1\!}{\!\IfValueTF{#2}{#2}{#1}} $}}}}

\newlength{\SMtblimit}
\setlength{\SMtblimit}{2pt}
\NewDocumentEnvironment{smatrix}{O{\SMtblimit}}{
	\if@display
		\setlength{\arraycolsep}{3pt}%
		\setlength{\cellspacetoplimit}{#1}%
		\setlength{\cellspacebottomlimit}{#1}%
	\fi
	~\begin{matrix} }{
	\end{matrix}~
}
\NewDocumentEnvironment{sbmatrix}{O{\SMtblimit}}{
	\setlength{\arraycolsep}{3pt}%
	\setlength{\cellspacetoplimit}{#1}%
	\setlength{\cellspacebottomlimit}{#1}%
	\left[~\begin{matrix} }{
	\end{matrix}~\right]
}
\NewDocumentEnvironment{scmatrix}{O{\SMtblimit}}{
	\setlength{\arraycolsep}{3pt}%
	\setlength{\cellspacetoplimit}{#1}%
	\setlength{\cellspacebottomlimit}{#1}%
	\left\{~\begin{matrix} }{
	\end{matrix}~\right\}
}
\NewDocumentCommand{\otup}{O{}m o m o o}{%
	{%
		\begin{sbmatrix}%
			#2#1 & \cdots#1 & \IfValueT{#3}{#3#1 & \cdots#1 & } #4
		\end{sbmatrix}\,%
	}\SubSize{#5}{#6}%
}%
\NewDocumentCommand{\ovtup}{m o m o o}{
	{%
		\begin{sbmatrix}%
			#1 \\
			\vdots \\
			\IfValueT{#2}{%
				#2 \\
				\vdots \\
			}%
			#3
		\end{sbmatrix}\,%
	}\SubSize{#4}{#5}%
}%
\NewDocumentCommand{\pair}{m m o o}{%
	{%
		\begin{sbmatrix}%
			#1 & #2
		\end{sbmatrix}\,%
	}\SubSize{#3}{#4}%
}%
%
\NewDocumentCommand{\vpair}{m m o o}{%
	{%
		\begin{sbmatrix}%
			#1 \\[0.5em]
			#2
		\end{sbmatrix}\,%
	}\SubSize{#3}{#4}%
}%
\def\setsep{}
\newcommand{\setprocessor}[1]{
	\setsep\IfEmpty{#1}{\cdots}{#1}
	\global\def\setsep{\,,&}
}
\newcommand{\vsetprocessor}[1]{
	\setsep\IfEmpty{#1}{\vdots}{#1}
	\global\def\setsep{\\}
}
\NewDocumentCommand{\set}{ >{\SplitList{,}}m }{
    \global\def\setsep{}
	\begin{scmatrix}%
		\ProcessList{#1}{\setprocessor}
	\end{scmatrix}%
	\global\def\setsep{}
}
\NewDocumentCommand{\tup}{ >{\SplitList{,}}m o o }{
    \global\def\setsep{}
	\begin{sbmatrix}%
		\ProcessList{#1}{\setprocessor}
	\end{sbmatrix}\SubSize{#2}{#3}%%
	\global\def\setsep{}
}
\NewDocumentCommand{\vtup}{ >{\SplitList{,}}m o o }{
    \global\def\setsep{}
	\begin{sbmatrix}%
		\ProcessList{#1}{\vsetprocessor}
	\end{sbmatrix}\SubSize{#2}{#3}%%
	\global\def\setsep{}
}
%
\newcommand{\rangeprocessor}[2]{
	\pyif{r"\infty" in r"\string#1"}{
		\def\dell{(}
	}{
		\def\dell{[}
	}
	\pyif{r"\string\infty" in r"\string#2"}{
		\def\delr{)}
	}{
		\def\delr{]}
	}
%	\left\dell #1,#2 \right\delr
	\udel[\dell]{#1,#2}[\delr]
}
\NewDocumentCommand{\range}{ >{\SplitArgument{1}{,}}m }{
	\rangeprocessor#1
}
\DeclareDocumentCommand{\added}{o m}{%
    {%TC:ignore
    \color{LimeGreen} \IfValueT{#1}{\textit{[#1]} }%
    %TC:endignore
    #2}%
}
\DeclareDocumentCommand{\removed}{o m}{%
    {%TC:ignore
    \color{Red} \IfValueT{#1}{\textit{[#1]} }%
    %TC:endignore
    \sout{#2}}%
}
\DeclareDocumentCommand{\changed}{o m m}{%
    {%TC:ignore
    \color{Orange} \IfValueT{#1}{\textit{[#1]} }\textit{\sout{#2}}%
    %TC:endignore
    #3}%
}

% \DeclareDocumentCommand{\added}{o m}{%
%     #2%
% }
% \DeclareDocumentCommand{\removed}{o m}{%
%     #2%
% }
% \DeclareDocumentCommand{\changed}{o m m}{%
%     #3%
% }
\makeatletter
\newcommand{\oset}[2]{{\mathpalette\o@set{{#1}{#2}}}}
\newcommand{\o@set}[2]{\o@@set{#1}#2}
\newcommand{\o@@set}[3]{%
  \vbox{\offinterlineskip
    \ialign{\hfil##\hfil\cr
      $\m@th\o@set@demote{#1}#2$\cr
      \noalign{\vskip0.2pt}
      $\m@th#1#3$\cr
    }%
  }%
}
\newcommand{\o@set@demote}[1]{%
  \ifx#1\displaystyle\scriptstyle\else
  \ifx#1\textstyle\scriptstyle\else
  \scriptscriptstyle\fi\fi
}
\makeatother

% Symbols | Operators | Environments
%% Requires:
% xparse

%% Sensor array symbols
\MakeSymbol{\w}{\omega}

\makeatletter
% BEGIN_FOLD
\NewDocumentCommand{\@OneIdx}{m m O{;} m}{% Name | Optional | Name
    #1_{\IfValueT{#2}{#2#3} {#4}}
}
\NewDocumentCommand{\@TwoIdx}{s m m m m}{
    #2_{\IfValueT{#3}{#3;}(#4\IfValueT{#3}{_{#3}}\IfBooleanT{#1}{'},#5\IfValueT{#3}{_{#3}}\IfBooleanT{#1}{'})}
}
\NewDocumentCommand{\@FourIdx}{m m m m m m}{
    #1_{\IfValueT{#2}{#2;}(#3\IfValueT{#2}{_{#2}},#4\IfValueT{#2}{_{#2}});(#5\IfValueT{#2}{_{#2}},#6\IfValueT{#2}{_{#2}})}
}
% END_FOLD
\makeatother

%% Bold symbols
% BEGIN_FOLD
\makeatletter
\DeclareDocumentCommand{\bv}{m O{n} g}{
	\ensuremath{\@my{#2}{\mathbf{#1}}\IfValueT{#3}{_{#3}}}
}
\DeclareDocumentCommand{\BV}{m g}{
    \mathbf{#1}\IfValueT{#2}{_{#2}}
}
\DeclareDocumentCommand{\newbv}{s o m}{
    \IfBooleanTF{#1}{
        \IfValueTF{#2}{
            \@namedef{bv#2}{\BV{#3}}
        }{
            \@namedef{bv#3}{\BV{#3}}
        }
    }{
        \IfValueTF{#2}{
            \@namedef{bv#2}{\bv{#3}}
        }{
            \@namedef{bv#3}{\bv{#3}}
        }
    }
}
\@namedef{:}{\scalebox{1.01}{\textbf{:}}}
\makeatother
% END_FOLD

\newbv*{b}
\newbv*{x}
\newbv*{c}
\newbv*{v}
\newbv*{y}
\newbv*{B}
\newbv*{C}
\newbv*{r}
\newbv{d}
\newbv*{p}
\newbv*{q}
\newbv*{w}
\newbv*{f}
\newbv{i}
\newbv*{I}
\let\id\bvI
<<<<<<< HEAD
\newbv*{A}
%\def\bvAh{\bvA[h]}
\def\bvfh{\hat{\bvf}}
\newbv*{D}
\def\bvBh{\hat{\bvB}}
\def\bvbe{\boldsymbol{\beta}}
=======
\newbv{A}
\def\bvfh{\hat{\bvf}}
\newbv*{Q}
\newbv*{i}
>>>>>>> main
%\newbv[Ga]{\Gamma}
%\newbv{I}
%\let\id\bvI
%\newbv{i}
%\newbv{u}
%\newbv*{H}
%\newbv*{B}
%\newbv*{G}
%\newbv*{y}
%\newbv*{r}
%\newbv*{z}
%\newbv*{f}
%\newbv*{d}
%\newbv*{w}
%\newbv*{C}
%\newbv*{p}
%\newbv*{q}
%\newbv*{F}
%\newbv*{P}
%\newbv*{A}
%\newbv*{D}
%\newbv*{Q}
% \def\bvL{*************}

\def\mvdr{\text{mvdr}}

%% Requires:
% xparse

\let\mrm\symup
\def\mtimes{\mathbin{\scalebox{0.7}{$\times$}}}
\NewDocumentCommand{\sz}{s m m}{%
	\IfBooleanTF{#1}{%
		\bts{{#2} \mtimes {#3}}%
	}{%
		{#2} \mtimes {#3}%
	}%
}

\NewDocumentCommand{\el}{m g O{m} o}{%
	\bts{#1}_{%
		\IfValueTF{#4}{%
			(#3\IfValueT{#2}{_{#2}},#4\IfValueT{#2}{_{#2}})%
		}{%
			#3\IfValueT{#2}{_{#2}}%
		}        
	}
}

\NewDocumentCommand{\FT}{m O{(\w)}}{%
	\mathcal{F}\cts{#1}#2%
}
\NewDocumentCommand{\IFT}{m O{(t)}}{%
	\mathcal{F}^{-1}\cts{#1}#2%
}
\DeclareDocumentCommand{\F}{}{%
	\mathcal{F}
}
%
\NewDocumentCommand{\RFT}{m O{(\w)}}{%
	\mathcal{S}\cts{#1}#2%
}
\NewDocumentCommand{\IRFT}{m O{(t)}}{%
	\mathcal{S}^{-1}\cts{#1}#2%
}
\DeclareDocumentCommand{\S}{}{%
	\mathcal{S}
}

\NewDocumentCommand{\STFT}{m O{[l,k]}}{%
	\symbb{F}\cts{#1}#2%
}
\NewDocumentCommand{\ISTFT}{m O{[n]}}{%
	\symbb{F}^{-1}\cts{#1}#2%
}
\DeclareDocumentCommand{\sF}{}{%
	\symbb{F}
}
%
\NewDocumentCommand{\SSBT}{m O{[l,k]}}{%
	\symbb{S}\cts{#1}#2%
}
\NewDocumentCommand{\ISSBT}{m O{[n]}}{%
	\symbb{S}^{-1}\cts{#1}#2%
}
\DeclareDocumentCommand{\sS}{}{%
	\symbb{S}
}


%% Mathematical symbols
\NewDocumentCommand{\var}{m}{%
	\phi_{#1}%
}
\NewDocumentCommand{\Corr}{m}{%
	\bv{\Phi}_{#1}%
}
\NewDocumentCommand{\dsdi}{}{%
	\upsilon%
}
\NewDocumentCommand{\dsrf}{}{%
	\xi_{\mathrm{d}}
}
\NewDocumentCommand{\snr}{}{%
	\text{SNR}%
}
\NewDocumentCommand{\gsnr}{}{%
\text{gSNR}%
}
\NewDocumentCommand{\gsrr}{}{%
\text{gSRR}%
}

%% Require:
% xparse

\NewDocumentEnvironment{subalign}{g}{%
	\subequations%
		\IfValueT{#1}{\label{#1}}%
		\allowdisplaybreaks%
		\align%
	}{%
		\endalign%
	\endsubequations%
}%
\NewDocumentEnvironment{subgather}{g}{%
	\subequations%
		\IfValueT{#1}{\label{#1}}%
		\allowdisplaybreaks%
		\gather%
	}{%
		\endgather%
	\endsubequations%
}%
\NewDocumentEnvironment{equations}{g}{%
	\equation%
		\IfValueT{#1}{\label{#1}}%
		\allowdisplaybreaks%
		\aligned%
	}{%
		\endaligned%
	\endequation%
}

\NewDocumentEnvironment{proposition}{m}{%
	\noindent\textbf{Proposition:} {\itshape #1}
	
	\noindent\textbf{Proof:}\itshape
}{

	\noindent\bsquare
}

\setcounter{property}{0}
\DeclareDocumentEnvironment{Property}{s m O{}}{%
	\begingroup
	\IfBooleanF{#1}{\refstepcounter{property}}%
	\IfValueT{#3}{#3}%
	\noindent\textbf{Property \theproperty:} {\itshape #2}%
	
	\noindent\textbf{Proof:}%
	\begin{subequations}%
		\renewcommand{\theequation}{\roman{property}.\arabic{equation}}%
}{
	\end{subequations}
	
	\vspace*{-1em}
	\noindent\rule{0.5\textwidth}{0.4pt}
	\vspace*{1em}
	\endgroup
}

\NewDocumentEnvironment{example}{}{%
	\noindent\textbf{Example:}
}{

	\noindent\bsquare
}

\newcounter{smethod}
\setcounter{smethod}{0}
\newcounter{method}
\setcounter{method}{0}

\NewDocumentEnvironment{method}{s}{%
\IfBooleanT{#1}{\stepcounter{method}\setcounter{smethod}{0}}%
\stepcounter{smethod}%
{\noindent\textbf{Method \Alph{method}.\arabic{smethod}:}}

}{
\vspace{-0.5em}\newline%
\noindent\bsquare%

}

\NewDocumentEnvironment{changingfrom}{}{
	\begin{tcolorbox}[enhanced, breakable, colback=red!5!white, colframe=red!50!black]
	}{
	\end{tcolorbox}
}

\NewDocumentEnvironment{changingto}{}{
\begin{tcolorbox}[enhanced, breakable, colback=green!5!white, colframe=green!50!black]
}{
\end{tcolorbox}
}

%% References
\addlblresource{setup/d2.ref_labels.tex}
%
\NewDocumentCommand{\bsy}{m}{\mathscr{S}\pts{#1}}



%=================================================================
% Please use the following mathematics environments: Theorem, Lemma, Corollary, Proposition, Characterization, Property, Problem, Example, ExamplesandDefinitions, Hypothesis, Remark, Definition, Notation, Assumption
%% For proofs, please use the proof environment (the amsthm package is loaded by the MDPI class).

%=================================================================
% Full title of the paper (Capitalized)
\Title{On the Single-Sideband Transform for MVDR Beamformers}

% MDPI internal command: Title for citation in the left column
\TitleCitation{On the Single-Sideband Transform for MVDR Beamformers}

% Author Orchid ID: enter ID or remove command
\newcommand{\orcidauthorA}{0009-0009-3996-5452} % Add \orcidA{} behind the author's name
\newcommand\autAFN{Vitor Probst}
\newcommand\autALN{Curtarelli}
\newcommand\autAfn{V. P.}
\newcommand{\orcidauthorB}{0000-0002-2556-3972} % Add \orcidB{} behind the author's name
\newcommand\autBFN{Israel}
\newcommand\autBLN{Cohen}
\newcommand\autBfn{I.}

% Authors, for the paper (add full first names)
\Author{\autAFN{} \autALN $^{1,\ast}$\orcidA{}, \autBFN{} \autBLN $^{1}$\orcidB{}}

%\longauthorlist{yes}

% MDPI internal command: Authors, for metadata in PDF
\AuthorNames{\autAFN{} \autALN, \autBFN{} \autBLN}

% MDPI internal command: Authors, for citation in the left column
\AuthorCitation{\autALN, \autAfn; \autBLN, \autBfn}
% If this is a Chicago style journal: Lastname, Firstname, Firstname Lastname, and Firstname Lastname.

% Affiliations / Addresses (Add [1] after \address if there is only one affiliation.)
\address{%\\
$^{1}$ \quad Andrew and Erna Viterbi Faculty of Electrical and Computer Engineering, Technion--Israel Institute of Technology, Technion City, Haifa 3200003, Israel
}

% Contact information of the corresponding author
\corres{Correspondence: vitor.c@campus.technion.ac.il%; Tel.: (optional; include country code; if there are multiple corresponding authors, add author initials) +xx-xxxx-xxx-xxxx (F.L.)%
}

% Current address and/or shared authorship
%\firstnote{These authors contributed equally to this work.}
% The commands \thirdnote{} till \eighthnote{} are available for further notes

% Abstract (Do not insert blank lines, i.e. \\) 
\abstract{%
	In order to explore different beamforming applicaitons, this paper investigates the application of the Single-Sideband Transform (SSBT) for constructing a Minimum-Variance Distortionless-Response (MVDR) beamformer in the context of the convolutive transfer function (CTF) model for short-window time-frequency transforms by making use of filter-banks and their properties. Our study aims to optimize the appropriate utilization of SSBT in this endeavor, by examining its characteristics and traits. We address a reverberant scenario with multiple noise sources, aiming to minimize both undesired interference and reverberation in the output. Through simulations reflecting real-life scenarios, we show that employing the SSBT correctly leads to a beamformer that outperforms the one obtained when via the Short-Time Fourier Transform (STFT), while exploiting the SSBT's property of it being real-valued. Two approaches were developed with the SSBT, one naive and one refined, with the later being able to ensure the desired distortionless behavior, which is not achieved by the former.
}

% Keywords
\keyword{Single-sideband transform; MVDR beamformer; Filter-banks; Array signal processing; Signal enhancement.} 

% The fields PACS, MSC, and JEL may be left empty or commented out if not applicable
%\PACS{J0101}
%\MSC{}
%\JEL{}

\begin{document}
% \setcounter{footnote}{-1}

%%%%%%%%%%%%%%%%%%%%%%%%%%%%%%%%%%%%%%%%%%

%TODO: Rewrite Abstract, Sec1, Sec5.3 and Sec.6
%\Title{On Beamforming with the Single-Sideband Transform}

% MDPI internal command: Title for citation in the left column
\TitleCitation{On Beamforming with the Single-Sideband Transform}

% Author Orchid ID: enter ID or remove command
\newcommand{\orcidauthorA}{0009-0009-3996-5452} % Add \orcidA{} behind the author's name
\newcommand\autAFN{Vitor Probst}
\newcommand\autALN{Curtarelli}
\newcommand\autAfn{V. P.}
\newcommand{\orcidauthorB}{0000-0002-2556-3972} % Add \orcidB{} behind the author's name
\newcommand\autBFN{Israel}
\newcommand\autBLN{Cohen}
\newcommand\autBfn{I.}

% Authors, for the paper (add full first names)
\Author{\autAFN{} \autALN $^{1,\ast}$\orcidA{}, \autBFN{} \autBLN $^{1}$\orcidB{}}

%\longauthorlist{yes}

% MDPI internal command: Authors, for metadata in PDF
\AuthorNames{\autAFN{} \autALN, \autBFN{} \autBLN}

% MDPI internal command: Authors, for citation in the left column
\AuthorCitation{\autALN, \autAfn; \autBLN, \autBfn}


\address{%\\
$^{1}$ \quad Andrew and Erna Viterbi Faculty of Electrical and Computer Engineering, Technion--Israel Institute of Technology, Technion City, Haifa 3200003, Israel
}

% Contact information of the corresponding author
\corres{Correspondence: vitor.c@campus.technion.ac.il
}
\abstract{%
    In this paper, we explore the application of the Single-Sideband Transform for convolutive beamformers. We investigate the unique properties of the SSBT and their implications for beamformer design. Despite its advantageous real coefficients, we find that the transform's handling of convolution poses challenges, impacting fundamental beamforming premises. Compared to the Short-Time Fourier Transform, the SSBT exhibits lower robustness, particularly in scenarios involving mismatch and modeling noise. We validate our theoretical findings through realistic simulations. These simulations demonstrate that while the real-valued SSBT performs marginally worse than its complex-valued counterpart under ideal conditions, its performance deteriorates significantly in non-ideal scenarios. Notably, we establish a direct equivalence between SSBT and STFT under identical parameters, facilitating seamless interchangeability and allowing their joint application. Our study underscores the nuanced trade-offs between SSBT and STFT in beamforming applications, providing insights into their respective strengths and limitations.
}

% Keywords
\keyword{Single-sideband transform; Time-frequency transforms; Convolutive beamforming; Array signal processing; Signal enhancement.} 

\section{Introduction}
\label{sec:introduction}

Beamformers play a crucial role in diverse fields, such as telecommunications \cite{viswanath_opportunistic_2002}, acoustics \cite{herbordt_joint_2005,chiariotti_acoustic_2019}, hearing aids \cite{haykin_handbook_2009}, and others \cite{van_veen_beamforming_1988,liu_wideband_2010,huang_energy_2020,elbir_twenty-five_2023}. Among the array configurations used for beamforming, rectangular arrays are an interesting option to be explored \cite{gu_efficient_2019,zhang_two-dimensional_2019,lin_secrecy-energy_2021} as they offer distinct advantages over linear arrays, providing enhanced spatial information regarding impinging sources \cite{heidenreich_joint_2012,ioannides_uniform_2005} and reduced redundancy due to their asymmetry \cite{singh_minimal_2021}.

The development of robust adaptive beamformers with frequency-invariant characteristics has been a significant point of interest since, in this case, frequency does not affect the behavior of the beamformer. Some desired features are constant-beamwidth \cite{goodwin_constant_1993} and null steering \cite{zarifi_collaborative_2010}. One approach for null steering is using the linearly constrained minimum variance (LCMV) beamformer \cite{frost_algorithm_1972,buckley_spatialspectral_1987,souden_study_2010}, which cancels interfering signals from given directions and steers the main beam toward the desired signal. However, it lacks a robust mechanism for maintaining a constant beamwidth. On the other hand, constant-beamwidth (CB) beamforming \cite{hixson_widebandwidth_1970,goodwin_constant_1993,wang_constant-beamwidth_2004} can be accomplished by using window-based beamforming techniques \cite{long_window-based_2019}, but this method cannot incorporate directional restrictions. CB-LCMV beamformers have been explored recently \cite{frank_constant-beamwidth_2022-1}, however only in the context of linear sensor arrays, leaving space for their exploration in the context of different array configurations.

The delay-and-sum (DS) and superdirective (SD) beamformers {\cite{benesty_microphone_2008}} can be used to increase white noise gain or directivity factor, respectively, by maximizing the desired metric. Another quality that is often required for designing a beamformer is a distortionless response to the desired source or to the desired-source direction. This ensures that the desired signal is unaltered by the filtering processes.

While constructing a beamformer with multiple beamforming features is non-trivial, efforts have been made to combine multiple beamforming techniques for a single array of sensors. Two notable approaches are the Kronecker-product (KP) method \cite{abramovich_iterative_2010,werner_estimation_2008} and the linear convolutional Kronecker product (LCKP) method \cite{frank_constant-beamwidth_2022-1}. The LCKP method is valid only for linear arrays. However, it allows for the virtual utilization of more sensors than are physically available \cite{frank_constant-beamwidth_2022-1}. Meanwhile, the KP method can be used in linear or rectangular arrays, but it does not increase the number of sensors available for beamforming. For both combination methods (KP and LCKP), beampattern features and distortionless constraints are conserved in the combined beamformers. Therefore, by using beamformers with desired beampattern characteristics that respect the distortionless constraint, these are maintained in the end result.

In this paper, we propose a novel approach to constructing a CB-LCMV beamformer for rectangular arrays.
For such, we generalize the LCKP beamforming technique to the case of rectangular arrays. We synthesize beamformers for virtual subarrays and, with our proposed generalized technique, apply them to a full array to achieve the desired beamwidth and null placement. This is achieved without sacrificing white noise gain or directivity factor. The performance of the proposed method is compared against beamformers obtained through the KP and LCKP methods. Our results demonstrate superior performance in terms of beamwidth, white noise gain, and directivity for the beamformers obtained using the proposed method when compared to the literature.

This paper is organized as follows: \cref{sec:signal_model} presents the array and signal model considered for the problem;
%\cref{sec:kp_beamformers} is an overview of the literature methods for array analysis with the Kronecker product;
\cref{sec:conv_beamformers} shows the traditional beamforming techniques and methods that will be used further.
In \cref{sec:cblcmv_beamformer}, the newly proposed method for array analysis for rectangular arrays is introduced and detailed, as well as its proposed usage for solving the problem at hand. In \cref{sec:simulations}, we present the simulations realized and discuss the results, comparing them to the literature. Finally, \cref{sec:conclusions} concludes this paper, overviewing the main contributions.
%
\section{STFT and Single-Sideband Transform}
\label{sec:stft_and_ssbt}

When studying signals and systems, often frequency and time-frequency transforms are used in order to change the signal domain \cite{demuth_frequency_1977}, allowing the exploitation of different patterns and informations that are inherent to the signal.

Given a time-domain signal $x[n]$, its Short-time Fourier Transform (STFT) \cite{kiymik_comparison_2005,pan_microphone_2021} is given by
\begin{equation}
	X_{\F}[l,k] = \sum_{n=0}^{K-1} w[n] x[n - l\cdot O] e^{-\j 2\pi k \frac{(n - l\cdot O)}{K}}
\end{equation}
where $w[n]$ is an analysis window of length $K$; and $O$ is the overlap between windows of the transform, usually $O = \floor{\nicefrac{K}{2}}$. Even though the STFT is the most traditionally used time-frequency transform, it isn't the only one available. Thus, exploring different possibilities for such an operation can be useful and lead to interesting results.

The Single-Sideband Transform (SSBT) \cite{crochiere_multirate_1983} is one such alternative, in which the frequency values are cleverly calculated such that its spectrum is real-valued, without loss of information. The SSB transform of $x[n]$ is defined as
\begin{equation}
	\label{eq:sec2:def_ssbt_xn}
	X_{\S}[l,k] = \Sqrt{2} \real{\sum_{n=0}^{L-1} w[n] x[n-l\cdot O] e^{\j 2\pi k \frac{(n - l\cdot O)}{K} + \j\frac{3\pi}{4} } }
\end{equation}

Assuming that $x[n]$ is real-valued, one advantage of using the STFT is that we only need to work with $\floor{\frac{K+1}{2}}+1$ frequency bins, given its complex-conjugate behavior. Meanwhile, the SSBT needs to use all $K$ possible bins to correctly capture all information of $x[n]$, however it is real-valued.

From \cref{eq:sec2:def_ssbt_xn} it's easy to see that
\begin{equations}
	\label{eq:sec2:equivalence_stft_ssbt}
	X_{\S}[l,k]
	& = \Sqrt{2} \real{X_{\F}[l,k] e^{\j\frac{3\pi}{4}}} \\
	& = - \real{X_{\F}[l,k]} + \imag{X_{\F}[l,k]}
\end{equations}
assuming that all $K$ bins of the STFT are available.

It is possible to show that, unlike with the STFT, the convolution theorem does not hold when employing the SSBT. In other words, if $y[n] = h[n] \ast x[n]$, then $Y_{\F}[l,k] = H_{\F}[l,k] \ast X_{\F}[l,k]$, but $Y_{\S}[l,k] \neq H_{\S}[l,k] \ast X_{\S}[l,k]$. Nonetheless, by first converting any result into the STFT domain (using \cref{eq:sec2:equivalence_stft_ssbt}) before utilization, it remains feasible to employ the obtained values for estimating matrices and signals.
%
\section{Signal Model and Beamforming}
\label{sec:signal_model}

Let there be a generic sensor array within a reverberant environment, it being comprised of $M$ sensors. In this setting there also are a desired and an interfering sources (namely $x[n]$ and $v[n]$), and also uncorrelated noise $r_m[n]$ at each sensor, all traveling with a speed $c$. We assume that the sources are spatially stationary, and all discrete signals were sampled with the same sampling frequency $f_s$.

We denote $h_m[n]$ as the room impulse response between the desired source and the $m$-th sensor ($1 \leq m \leq M$). We similarly define $g_m[n]$ for the interfering source. From this, we write $y_m[n]$ as the observed signal at the $m$-th sensor as
\begin{equation}
	\label{eq:sec3:time_model_basic}
	y_m[n] = h_m[n] \ast x[n] + g_m[n] \ast v[n] + r_m[n]
\end{equation}
We let $m'$ be the reference sensor's index, for simplicity assume $m'=1$. We let $x_1[n] = h_1[n] \ast x[n]$ (and similarly for $v_1[n]$). $b_m[n]$ is the \textit{relative} impulse response between the desired signal (at the reference sensor) and the $m$-th sensor, define such that
\begin{equation}
	b_m[n] \ast x_1[n] = h_m[n] \ast x[n]
\end{equation}
We similarly define $c_m[n]$ such that $c_m[n] \ast v_1[n] = g_m[n] \ast v[n]$. Therefore, \cref{eq:sec3:time_model_basic} becomes
\begin{equation}
	\label{eq:sec3:time_model_relative}
	y_m[n] = b_m[n] \ast x_1[n] + c_m[n] \ast v_1[n] + r_m[n]
\end{equation}
Here, the impulse responses $b_m[n]$ and $c_m[n]$ can be non-causal, depending on the direction of arrival and features of the reverberant environment.

We can use a time-frequency transform (here the STFT or the SSBT, both exposed in \cref{sec:stft_and_ssbt}) with the CTF model \cite{talmon_relative_2009} to get our time-frequency signal model,
\begin{equation}
	\label{eq:sec3:time-freq_model_conv}
	Y_m[l,k] = B_m[l,k] \ast X_1[l,k] + C_m[l,k] \ast V_1[l,k] + R_m[l,k]
\end{equation}
where $Y_m[l,k]$ is the transform of $y_m[n]$ (resp. all other signals); $l$ is the window index, and $k$ the bin index, with $0 \leq k \leq K-1$; and the convolution is in the window-index axis.

Using that $B_m[l,k]$ is a finite (possibly truncated) response with $L_B$ windows, then
\begin{equation}
	B_m[l,k] \ast X_1[l,k] = \tr{\bvb{m}}[k] \bvx{1}[l,k]
\end{equation}
in which
\begin{subalign}
	\bvb{m}[k] & = \tr{\tup{ {B_m[-\Delta,k]} ,, {B_m[0,k]} , , {B_m[L_B-\Delta-1,k]} }} \\
	\bvx{1}[l,k] & = \tr{\tup{ {X_1[l+\Delta,k]} , , { X_1[l,k]} , , {X_1[l-L_B+\Delta+1,k]} }}
\end{subalign}
and in the same way we define $\bvc{m}[k]$ and $\bvv{1}[l,k]$. Note that $\bvb{m}[k]$ and $\bvc{m}[k]$ don't depend on the index $l$, since neither the environment nor the sources' positions change over time. With this, \cref{eq:sec3:time-freq_model_conv} becomes
\begin{equation}
	\label{eq:sec3:time-freq_model_mult1}
	Y_m[l,k] = \tr{\bvb{m}}[k] \bvx{1}[l,k] + \tr{\bvc{m}}[k] \bvv{1}[l,k] + R_m[l,k]
\end{equation}

Vectorizing the signals sensor-wise, we finally get
\begin{equation}
	\label{eq:sec3:bvylk_vectorized}
	\bvy[l,k] = \tr{\bvB}[k] \bvx{1}[l,k] + \tr{\bvC}[k] \bvv{1}[l,k] + \bvr[l,k]
\end{equation}
where
\begin{equation}
	\bvy[l,k] = \tr{\tup{ {y_1[l,k]},,{y_M[l,k]} }}
\end{equation}
and similarly for the other variables. In this situation, $\bvB[k]$ and $\bvC[k]$ are $\sz{L_B}{M}$ and $\sz{L_C}{M}$ matrices respectively; $\bvx{1}[l,k]$ and $\bvv{1}[l,k]$ are $\sz{L_B}{1}$ and $\sz{L_C}{1}$ vectors respectively; and $\bvy[l,k]$ and $\bvr[l,k]$ are $\sz{M}{1}$ vectors.

\subsection{Reverb-aware formulation}
Let the $0$-th window of $\bvB[k]$ be the desired-speech frequency response (named $\bvd{x}[k]$), with the rest comprising an undesired component. We therefore write
\begin{equation}
	\label{eq:sec3:bvBk_bvx1lk_separated}
	\tr{\bvB}[k] \bvx{1}[l,k] = \bvd{x}[k] X_1[l,k] + \sum_{\substack{l'=-\Delta \\ l' \neq 0}}^{L_B-\Delta-1} \bvp{B,l'}[k] X_1[l-l',k]
\end{equation}
where $\bvp{B,l'}[k]$ is the $l'$-th row of $\bvB[k]$. With this, $\bvd{x}[k] X_1[l,k]$ is the desired speech component of $\tr{\bvB}[k] \bvx{1}[l,k]$, and the summation over $l'$ is the undesired component. We will call $\bvd{x}[k]$ the desired-speech frequency response.

It's important to have in mind the sensor delay and window length. If the time for the signal to travel from the reference to the farthest sensor exceeds the window length (in seconds), multiple windows may represent the desired speech. This isn't a problem if $\frac{\delta}{c} < \frac{K}{f_s}$, where $\delta$ is the biggest reference-to-sensor distance, and $K$ is the window length.

Using \cref{eq:sec3:bvBk_bvx1lk_separated} we define $\bvw[l,k]$ as the undesired signal (undesired speech components + interfering source + noise),
\begin{equation}
	\label{eq:sec3:def_undes_sig_bvwlk}
	\bvw[l,k] = \sum_{\substack{l'=-\Delta \\ l' \neq 0}}^{L_B-\Delta-1} \bvp{B,l'}[k] X_1[l-l',k] + \tr{\bvC}[k] \bvv{1}[l,k] + \bvr[l,k]
\end{equation}
and therefore
\begin{equation}
    \label{eq:sec3:time-freq_model_final}
	\bvy[l,k] = \bvd{x}[k] X_1[l,k] + \bvw[l,k]
\end{equation}

%Note that in \cref{eq:sec3:time-freq_model_final,eq:sec3:def_undes_sig_bvwlk} we can treat each window of the incoming signals as a different source, with its own frequency response. This facilitates the use of known methods for signal enhancement.

For simplicity we assume that all windows of $X_{1}[l,k]$ are independent of one-another, and that $\bvw[l,k]$ is independent of  $\bvd{x}[k] X_1[l,k]$. This isn't strictly true, given both the time-frequency windowing process and the reverberant behavior of the environment.

\subsection{MVDR beamformer}

We estimate the desired signal at reference through a filter $\bvf[l,k]$, such that
\begin{equations}
	Z[l,k]
	& = \he{\bvf}[l,k] \bvy[l,k] \\
	& \approx X_1[l,k]
\end{equations}
with $\he{(\cdot)}$ being the transposed-complex-conjugate operator. Since the source signals' properties can vary over time, so can the filter, in order to adapt to the environment.

In order to minimize $\bvw[l,k]$ the MVDR beamformer \cite{erdogan_improved_2016} will be used, being given by
\begin{equation}
	\label{eq:sec3:minimization_problem_mvdr}
	\bvf_{\mvdr}[l,k] = \min_{\bvf[l,k]} \he{\bvf[l,k]} \Corr{\bvy}[l,k] \bvf[l,k]~\text{s.t.}~\he{\bvf}[l,k] \bvd{x}[k] = 1
\end{equation}
in which $\he{\bvf}[l,k] \bvd{x}[k] = 1$ is the distortionless constraint, and $\Corr{\bvy}[l,k]$ is the correlation matrix of the observed signal $\bvy[l,k]$. This formulation is preferred over the one using $\Corr{\bvw}[l,k]$ given the observed signal's availability.

The solution to \cref{eq:sec3:minimization_problem_mvdr} is
\begin{equation}
	\label{eq:sec3:solution_mvdr_beamformer}
	\bvf{\mvdr}[l,k] = \frac{ \inv{\Corr{\bvy}}[l,k] \bvd{x}[k] }{ \he{\bvd{x}}[k] \inv{\Corr{\bvy}}[l,k] \bvd{x}[k] }
\end{equation}

\subsection{Beamformer metrics}

Considering the problem, the relevant metrics are the narrowband gain in signal-to-noise ratio (SNR) and narrowband desired signal distortion index (DSDI), respectively given by
\begin{equation}
	\Delta\snr[l,k] = \var{V_1}[l,k] \frac{ \abs{\he{\bvf}[l,k] \bvd{x}[k]}^2}{ \he{\bvf}[l,k] \Corr{\bvw}[l,k] \bvf[l,k] }
\end{equation}
\begin{equation}
	\dsdi[l,k] = \abs{\he{\bvf}[l,k] \bvd{x}[k] - 1}^2
\end{equation}
We can also define the window-averaged gain in SNR and DSDI as
\begin{equation}
	\Delta\snr[k] = \frac{1}{L_Z}\sum_{l=0}^{L_Z-1} \text{gSNR}[l,k]
\end{equation}
\begin{equation}
	\dsdi[k] = \frac{1}{L_Z}\sum_{l=0}^{L_Z-1} \dsdi[l,k]
\end{equation}
with $L_Z$ being the number of windows of $Z[l,k]$.
%
\section{True-MVDR with the Single-Sideband Transform}
\label{sec:true_mvdr_ssbt}

When carelessly using any of the established methods with the SSBT, the distortionless constraint ensures that the beamformer avoids causing distortion exclusively within the SSBT domain. However, as explained in \cref{sec:stft_and_ssbt} the SSBT beamformer must be carefully constructed to achieve the desired effects, such as the distortionless constraint.

We thus propose a framework for the SSBT in which we consider both bins $k$ and $K-k$ simultaneously, since from \cref{eq:sec2:equivalence_stft_ssbt} they aren't independent. We define $\bvy'[l,k]$ as
\begin{equation}
	\bvy'[l,k] = \vtup{ {\bvy[l,k]} , {\bvy[l, K-k]} }\SubSize{2M}{1}
\end{equation}
from which we define $\Corr{\bvy'}[l,k]$ as its correlation matrix. Under this idea, our filter $\bvf'[l,k]$ is a $\sz{2M}{1}$ vector, with the first $M$ values being for the $k$-th bin, and the last $M$ values for the $[K-k]$-th bin. We let the STFT-equivalent filter for the SSBT beamformer $\bvf'[l,k]$ be $\bvf{\F}'[l,k]$, given by
\begin{equation}
	\label{eq:sec4:conversion_beamformer_ssbt_to_stft}
	\bvf{\F}'[l,k] = \bvA \bvf'[l,k]
\end{equation}
in which
\begin{equation}
	\bvA = \frac{1}{\Sqrt{2}} \begin{bmatrix}
		e^{\j\frac{3\pi}{4}} & 0 & \cdots & 0  & e^{-\j\frac{3\pi}{4}} & 0 & \cdots & 0\\
		0 & e^{\j\frac{3\pi}{4}} & \cdots & 0  & 0 & e^{-\j\frac{3\pi}{4}} & \cdots & 0\\
		\vdots & \vdots & \ddots & \vdots & \vdots & \ddots & \vdots  \\
		0 & 0 & \cdots & e^{\j\frac{3\pi}{4}} & 0 & 0 & \cdots & e^{-\j\frac{3\pi}{4}}
	\end{bmatrix}\SubSize{M}{2M}
\end{equation}

\subsection{Reverb-rejecting SSBT formulation}\label{subsec:sec4:reverb-rejecting_formulation}

Continuing the results of \cref{subsec:sec3:reverb-rejecting_formulation}, from \cref{eq:sec4:conversion_beamformer_ssbt_to_stft} the distortionless constraint for the STFT, within the SSBT domain, is
\begin{subgather}
	\label{eq:sec4:distortionless_constraint_in_ssbt}
	\he{\bvf'}[l,k] \bvd[h]{x}[k] = 1 \\
    \bvd[h]{x}[k] = \he{\bvA} \bvd{\F;x}[k]
\end{subgather}
where $\bvd{\F;x}[l,k]$ is the desired-speech frequency response in the STFT domain. In this scheme, our minimization problem becomes
\begin{equation}
	\label{eq:true-mvdr_ssbt_beamformer}
	\bvf'_{\mvdr}[l,k] = \min_{\bvf'[l,k]} \he{\bvf'}[l,k] \Corr{\bvy'}[l,k] \bvf'[l,k]~\text{s.t.}~\he{\bvf'}[l,k] \bvd[h]{x}[k] = 1
\end{equation}

Although $\Corr{\bvy'}[l,k]$ is a matrix with real entries, $\bvd[h]{x}$ is complex-valued, and thus is the solution to \cref{eq:true-mvdr_ssbt_beamformer}, contradicting the purpose of utilizing the SSBT.

\subsection{Real-valued true-MVDR beamformer with SSBT}

To ensure the desired behavior of $\bvf'[l,k]$ being real-valued, an additional constraint is necessary. By forcing $\bvf'[l,k]$ to have real entries, from \cref{eq:sec4:distortionless_constraint_in_ssbt} we trivially have that
\begin{subalign}
	\tr{\bvf'}[l,k] \real{\bvd[h]{x}[k]} & = 1 \\
	\tr{\bvf'}[l,k] \imag{\bvd[h]{x}[k]} & = 0	
\end{subalign}
which can be put in matricial form as $\tr{\bvf'}[l,k] \bvD{x}[k] = \tr{\bvi}$, with
\begin{subgather}
	\bvD{x}[k] = \tup{ \real{\bvd[h]{x}[k]} , \imag{\bvd[h]{x}[k]} }\SubSize{2M}{2} \\
	\bvi = \vtup{1 , 0}
\end{subgather}

Therefore, the minimization problem from \cref{eq:true-mvdr_ssbt_beamformer} becomes
\begin{equation}
	\label{eq:true-mvdr_real-ssbt_beamformer}
	\bvf'_{\mvdr}[l,k] = \min_{\bvf'[l,k]} \tr{\bvf'}[l,k] \Corr{\bvy'}[l,k] \bvf'[l,k]~\text{s.t.}~\tr{\bvf'}[l,k] \bvD{x}[k] = \tr{\bvi}
\end{equation}
whose formulation is again similar to that of the LCMV beamformer, and therefore
\begin{equation}
	\label{eq:sec4:mvdr_reverb-reject_formulation}
	\bvf'_{\mvdr}[l,k] = \inv{\Corr{\bvy'}}[l,k] \bvD{x}[k] \inv{\pts{ \tr{\bvD{x}}[k] \inv{\Corr{\bvy'}}[k] \bvD{x}[k] }} \bvi
\end{equation}
Using \cref{eq:sec4:conversion_beamformer_ssbt_to_stft}, we can obtain the desired beamformer $\bvf{\F;\mvdr}'[l,k]$, in the STFT domain.

%
\definecolor{ColA}{HTML}{991F3D}
\definecolor{ColB}{HTML}{997A1F}
\definecolor{ColC}{HTML}{3D991F}
\definecolor{ColD}{HTML}{1F997A}
\definecolor{ColE}{HTML}{1F3D99}
\definecolor{ColF}{HTML}{7A1F99}
\section{Simulations}
\label{sec:results}

In the simulations\footnote{Code is available at \url{https://github.com/VCurtarelli/py-ssb-ctf-bf}.}, we employ a sampling frequency of $16\si{\kilo\hertz}$. The sensor array consists of a uniform linear array with 10 sensors spaced at $2\si{\cm}$. Room impulse responses were generated using Habets' RIR generator \cite{habets_rir-generator}, and signals were selected from the SMARD \cite{smard_database} and LINSE \cite{linse_database} databases.

The room's dimensions are $4\m \times 6\m \times 3\m$ (width $\times$ length $\times$ height), with a reverberation time of $0.3\si{\second}$. The desired source is located at $(2\m,~1\m,~1\m)$, it being a male voice (SMARD, \texttt{50\_male\_speech\_english\_ch8\_OmniPower4296.flac}).
%
The device composed of the loudspeaker + sensors is centered at $(2\m,~2\m,~1\m)$. It is comprised of 10 radially-symmetric sensors located $8\si{\centi\meter}$ away from the center, all omnidirectional and of flat frequency response. In the center is the loudspeaker source, whose signal is a music noise (SMARD database, \texttt{69\_abba\_ch8\_OmniPower4296.flac}). The interfering source, simulating an open door, is located simultaneously at $(0.5\m,~5\m,~[0.3:0.3:2.7]\m)$, with a babble sound signal (LINSE database, \texttt{babble.mat}). The noise signal is white Gaussian noise (SMARD database, \texttt{wgn\_48kHz\_ch8\_OmniPower4296.flac}). All signals were resampled to the desired frequency.

At the reference sensor, the SNR for the loudspeaker's signal, interfering signal and noise are, respectively, of $-15\dB$, $10\dB$ and $30\dB$. The beamformers were calculated every 200 windows, and used up to the previous 1000 samples to estimate the correlation matrices.

We compare filters obtained through the STFT and SSBT transforms. N-SSBT uses \cref{eq:sec3:solution_lcmv_beamformer} to (naively) calculate the SSBT beamformer, and T-SSBT will denote the beamformer obtained via the true-distortionless MVDR from \cref{sec:true_mvdr_ssbt}. Performance analysis is conducted via the STFT domain, with the SSBT beamformers being converted into it. In line plots, STFT is presented in red, N-SSBT in green, and T-SSBT in blue. The output metrics were averaged over 20 windows, to facilitate visualization.

\def\meshcols{316}
\def\meshrows{15}
\def\tmin{0}
\def\tmax{8.3725}
\def\fmin{0.25}
\def\fmax{7.75}
\begin{figure}[H]
	\centering
	%% Requires:
% pgfplots.sty
% edit_pgfplots.tex

\pgfplotsset{compat=1.18}
%\begin{subfigure}{\linewidth}
%\centering
\tikzsetnextfilename{gain_lineplot_32}
\begin{tikzpicture}
	\begin{lineplot}{Gain (dB)}[ymin=3, ymax=18]
		\addplot [style=resA]
		table [col sep=comma, y=val] {figures/io_input/STFT/gain_SINR_k_STFT_32.csv};
		
		\addplot [style=resC]
		table [col sep=comma, y=val] {figures/io_input/SSBT/gain_SINR_k_SSBT_32.csv};
		
		\addplot [style=resE]
		table [col sep=comma, y=val] {figures/io_input/SSBT-TRUE/gain_SINR_k_SSBT-TRUE_32.csv};
		
		\addlegendentry{STFT};
		\addlegendentry{N-SSBT};
		\addlegendentry{T-SSBT};
	\end{lineplot}
\end{tikzpicture}
%	\caption{}
%	\label{subfig:1_gain_lineplot}
%\end{subfigure}
	\caption{Per-window broadband gain for $K = 32$.}
	\label{fig:lineplot_gain_32}
\end{figure}
\begin{figure}[H]
	\centering
	%% Requires:
% pgfplots.sty
% edit_pgfplots.tex

\pgfplotsset{compat=1.18}
%\begin{subfigure}{\linewidth}
%\centering
\tikzsetnextfilename{dsdi_lineplot_32}
\begin{tikzpicture}
	\begin{lineplot}{DSDI}%[ymin=-0.1, ymax=3]
		\addplot [style=resA]
		table [col sep=comma, y=val] {figures/io_input/STFT/dsdi_l_STFT_32_p30.csv};
		
		\addplot [style=resC]
		table [col sep=comma, y=val] {figures/io_input/NSSBT/dsdi_l_NSSBT_32_p30.csv};
		
		\addplot [style=resE]
		table [col sep=comma, y=val] {figures/io_input/TSSBT/dsdi_l_TSSBT_32_p30.csv};
		
		\addlegendentry{STFT};
		\addlegendentry{N-SSBT};
		\addlegendentry{T-SSBT};
	\end{lineplot}
\end{tikzpicture}
%	\caption{}
%	\label{subfig:1_gain_lineplot}
%\end{subfigure}
	\caption{Per-window broadband DSDI gain for $K = 32$.}
	\label{fig:lineplot_dsdi_32}
\end{figure}
\begin{figure}[H]
	\centering
	%% Requires:
% pgfplots.sty
% edit_pgfplots.tex

\pgfplotsset{compat=1.18}
%\begin{subfigure}{\linewidth}
%\centering
\tikzsetnextfilename{erle_lineplot_32}
\begin{tikzpicture}
	\begin{lineplot}{ERLE (dB)}[ymin=-0.1, ymax=50]
		\addplot [style=resA, mark=none]
		table [col sep=comma, y=val] {figures/io_input/STFT/erle_l_STFT_32_p30.csv};
		
		\addplot [style=resC, mark=none]
		table [col sep=comma, y=val] {figures/io_input/NSSBT/erle_l_NSSBT_32_p30.csv};
		
		\addplot [style=resE, mark=none]
		table [col sep=comma, y=val] {figures/io_input/TSSBT/erle_l_TSSBT_32_p30.csv};
		
		\addlegendentry{STFT};
		\addlegendentry{N-SSBT};
		\addlegendentry{T-SSBT};
	\end{lineplot}
\end{tikzpicture}
%	\caption{}
%	\label{subfig:1_gain_lineplot}
%\end{subfigure}
	\caption{Per-window broadband ERLE gain for $K = 32$.}
	\label{fig:lineplot_erle_32}
\end{figure}

From \cref{fig:lineplot_gain_32}, all beamformers achieved a similar gain in SNR over time, with some fluctuations but no distinguishable advantage. From \cref{fig:lineplot_dsdi_32}, both the STFT and T-SSBT beamformers achieved the desired null distortion on the desired signal, while the N-SSBT one caused some minimal distortion, although present.

In \cref{fig:lineplot_erle_32}, we see that the STFT beamformer clearly outperformed both other beamformers in terms of ERLE, which would mean a better reduction of the loudspeaker's signal, the main objective.

%%%%%%%%%%%%%%%%%%%%%

%\subsection{Simulations with \emph{iSIR = 5dB}}
%
%In this scenario, we assume that the input SIR is $5\dB$, and we use analysis windows with: 32 samples in \cref{fig:lineplot_dsrf_32_p5,fig:lineplot_gain_32_p5}; or 64 samples in \cref{fig:lineplot_dsrf_64_p5,fig:lineplot_gain_64_p5}. \cref{fig:lineplot_gain_32_p5,fig:lineplot_gain_64_p5} show the window-wise averaged narrowband gain in SNR, and \cref{fig:lineplot_dsrf_32_p5,fig:lineplot_dsrf_64_p5} the window-averaged narrowband DSRF, for all presented methods.
%
%From \cref{fig:lineplot_gain_32_p5,fig:lineplot_gain_64_p5} it is clear that both beamformers derived through the SSBT outperformed the STFT beamformer in terms of SNR gain, with the N-SSBT one having a slightly better yield than the T-SSBT one. Also, both \cref{fig:lineplot_dsrf_32_p5,fig:lineplot_dsrf_64_p5} show that the T-SSBT and the STFT beamformers ensured a distortionless response, a feature that wasn't achieved by the N-SSBT beamformer. This was expected, since the T-SSBT was appropriately designed to achieve this quality, while the N-SSBT wasn't.
%
%\def\meshcols{316}
\def\meshrows{15}
\def\tmin{0}
\def\tmax{8.3725}
\def\fmin{0.25}
\def\fmax{7.75}
%\begin{figure}[H]
%\centering
%%% Requires:
% pgfplots.sty
% edit_pgfplots.tex

\pgfplotsset{compat=1.18}
%\begin{subfigure}{\linewidth}
%\centering
\tikzsetnextfilename{gain_lineplot_32_p5}
\begin{tikzpicture}
	\begin{lineplot}{Gain (dB)}%[ymin=-0.5, ymax=12]
		\addplot [style=resA]
		table [col sep=comma, y=val] {figures/io_input/STFT/gSINR_k_STFT_32_p5.csv};
		
		\addplot [style=resC]
		table [col sep=comma, y=val] {figures/io_input/NSSBT/gSINR_k_NSSBT_32_p5.csv};
		
		\addplot [style=resE]
		table [col sep=comma, y=val] {figures/io_input/TSSBT/gSINR_k_TSSBT_32_p5.csv};
		
		\addlegendentry{STFT};
		\addlegendentry{N-SSBT};
		\addlegendentry{T-SSBT};
	\end{lineplot}
\end{tikzpicture}
%	\caption{}
%	\label{subfig:1_gain_lineplot}
%\end{subfigure}
%\caption{Window-average SNR gain for $K = 32$.}
%\label{fig:lineplot_gain_32_p5}
%\end{figure}
%\begin{figure}[H]
%	\centering
%	%% Requires:
% pgfplots.sty
% edit_pgfplots.tex

\pgfplotsset{compat=1.18}
%\begin{subfigure}{\linewidth}
%\centering
\tikzsetnextfilename{dsrf_lineplot_32_p5}
\begin{tikzpicture}
	\begin{lineplot}{DSRF (dB)}%[ymin=-0.1, ymax=3]
		\addplot [style=resA]
		table [col sep=comma, y=val] {figures/io_input/STFT/dsrf_k_STFT_32_p5.csv};
		
		\addplot [style=resC]
		table [col sep=comma, y=val] {figures/io_input/NSSBT/dsrf_k_NSSBT_32_p5.csv};
		
		\addplot [style=resE]
		table [col sep=comma, y=val] {figures/io_input/TSSBT/dsrf_k_TSSBT_32_p5.csv};
		
		\addlegendentry{STFT};
		\addlegendentry{N-SSBT};
		\addlegendentry{T-SSBT};
	\end{lineplot}
\end{tikzpicture}
%	\caption{}
%	\label{subfig:1_gain_lineplot}
%\end{subfigure}
%	\caption{Window-average DSRF for $K = 32$.}
%	\label{fig:lineplot_dsrf_32_p5}
%\end{figure}
%%\begin{figure}[H]
%%	\centering
%%	%% Requires:
% pgfplots.sty
% edit_pgfplots.tex

\pgfplotsset{compat=1.18}
%\begin{subfigure}{\linewidth}
%\centering
\tikzsetnextfilename{gsrr_lineplot_32_p5}
\begin{tikzpicture}
	\begin{lineplot}{SRR gain (dB)}[ymin=-0.5, ymax=12]
		\addplot [style=resA]
		table [col sep=comma, y=val] {figures/io_input/STFT/gSRR_k_STFT_32_p5.csv};
		
		\addplot [style=resC]
		table [col sep=comma, y=val] {figures/io_input/NSSBT/gSRR_k_NSSBT_32_p5.csv};
		
		\addplot [style=resE]
		table [col sep=comma, y=val] {figures/io_input/TSSBT/gSRR_k_TSSBT_32_p5.csv};
		
		\addlegendentry{STFT};
		\addlegendentry{N-SSBT};
		\addlegendentry{T-SSBT};
	\end{lineplot}
\end{tikzpicture}
%	\caption{}
%	\label{subfig:1_gain_lineplot}
%\end{subfigure}
%%	\caption{Window-average SRR for $K = 32$.}
%%	\label{fig:lineplot_gsrr_32_p5}
%%\end{figure}
%
%\def\meshcols{263}
\def\meshrows{31}
\def\tmin{0}
\def\tmax{8.1283125}
\def\fmin{0.125}
\def\fmax{7.875}
%\begin{figure}[H]
%\centering
%%% Requires:
% pgfplots.sty
% edit_pgfplots.tex

\pgfplotsset{compat=1.18}
%\begin{subfigure}{\linewidth}
%\centering
\tikzsetnextfilename{gain_lineplot_64_p5}
\begin{tikzpicture}
	\begin{lineplot}{Gain (dB)}[ymin=-5, ymax=15]
		\addplot [style=resA]
		table [col sep=comma, y=val] {figures/io_input/STFT/gSINR_k_STFT_64_p5.csv};
		
		\addplot [style=resC]
		table [col sep=comma, y=val] {figures/io_input/NSSBT/gSINR_k_NSSBT_64_p5.csv};
		
		\addplot [style=resE]
		table [col sep=comma, y=val] {figures/io_input/TSSBT/gSINR_k_TSSBT_64_p5.csv};
		
		\addlegendentry{STFT};
		\addlegendentry{N-SSBT};
		\addlegendentry{T-SSBT};
	\end{lineplot}
\end{tikzpicture}
%	\caption{}
%	\label{subfig:1_gain_lineplot}
%\end{subfigure}
%\caption{Window-average SNR gain for $K = 64$.}
%\label{fig:lineplot_gain_64_p5}
%\end{figure}
%\begin{figure}[H]
%	\centering
%	%% Requires:
% pgfplots.sty
% edit_pgfplots.tex

\pgfplotsset{compat=1.18}
%\begin{subfigure}{\linewidth}
%\centering
\tikzsetnextfilename{dsrf_lineplot_64_p5}
\begin{tikzpicture}
	\begin{lineplot}{DSRF (dB)}%[ymin=-0.1, ymax=3]
		\addplot [style=resA]
		table [col sep=comma, y=val] {figures/io_input/STFT/dsrf_k_STFT_64_p5.csv};
		
		\addplot [style=resC]
		table [col sep=comma, y=val] {figures/io_input/NSSBT/dsrf_k_NSSBT_64_p5.csv};
		
		\addplot [style=resE]
		table [col sep=comma, y=val] {figures/io_input/TSSBT/dsrf_k_TSSBT_64_p5.csv};
		
		\addlegendentry{STFT};
		\addlegendentry{N-SSBT};
		\addlegendentry{T-SSBT};
	\end{lineplot}
\end{tikzpicture}
%	\caption{}
%	\label{subfig:1_gain_lineplot}
%\end{subfigure}
%	\caption{Window-average DSRF for $K = 64$.}
%	\label{fig:lineplot_dsrf_64_p5}
%\end{figure}
%
%\subsection{Average results per iSIR}
%
%Here, we now take the broadband metrics for the beamformers, allowing us to compare them for different values of input SIR. We again test for both $K = 32$ and $K = 64$.
%
%From \cref{fig:lineplot_dsrf_32_av,fig:lineplot_dsrf_64_av}, as expected we see that the STFT and T-SSBT beamformers didn't cause distortion on the desired signal, while the N-SSBT did, due to its nature. In terms of SNR gain (\cref{fig:lineplot_gain_32_av,fig:lineplot_gain_64_av}), we again see that the SSBT beamformers led to a strictly better result than the STFT one, the later being 6-7 dB worse than the former ones for all input SIRs.
%
%\begin{figure}[H]
%	\centering
%	%% Requires:
% pgfplots.sty
% edit_pgfplots.tex

\pgfplotsset{compat=1.18}
%\begin{subfigure}{\linewidth}
%\centering
\tikzsetnextfilename{dsrf_lineplot_32}
\begin{tikzpicture}
	\begin{lineplot}{DSRF (dB)}[xtick={-20, -10, ..., 20}, xlabel={iSIR (dB)}, xmin=-21, xmax=+21]
		\addplot [style=resA]
		table [col sep=comma, y=val] {figures/io_input/STFT/DSRF_STFT_32.csv};
		
		\addplot [style=resC]
		table [col sep=comma, y=val] {figures/io_input/NSSBT/DSRF_NSSBT_32.csv};
		
		\addplot [style=resE]
		table [col sep=comma, y=val] {figures/io_input/TSSBT/DSRF_TSSBT_32.csv};
		
		\addlegendentry{STFT};
		\addlegendentry{N-SSBT};
		\addlegendentry{T-SSBT};
	\end{lineplot}
\end{tikzpicture}
%	\caption{}
%	\label{subfig:1_gain_lineplot}
%\end{subfigure}
%	\caption{Broadband DSRF for $K = 32$.}
%	\label{fig:lineplot_dsrf_32_av}
%\end{figure}
%
%\begin{figure}[H]
%	\centering
%	%% Requires:
% pgfplots.sty
% edit_pgfplots.tex

\pgfplotsset{compat=1.18}
%\begin{subfigure}{\linewidth}
%\centering
\tikzsetnextfilename{dsrf_lineplot_64}
\begin{tikzpicture}
	\begin{lineplot}{DSRF (dB)}[xtick={-20, -10, ..., 20}, xlabel={iSIR (dB)}, xmin=-21, xmax=+21]
		\addplot [style=resA]
		table [col sep=comma, y=val] {figures/io_input/STFT/DSRF_STFT_64.csv};
		
		\addplot [style=resC]
		table [col sep=comma, y=val] {figures/io_input/NSSBT/DSRF_NSSBT_64.csv};
		
		\addplot [style=resE]
		table [col sep=comma, y=val] {figures/io_input/TSSBT/DSRF_TSSBT_64.csv};
		
		\addlegendentry{STFT};
		\addlegendentry{N-SSBT};
		\addlegendentry{T-SSBT};
	\end{lineplot}
\end{tikzpicture}
%	\caption{}
%	\label{subfig:1_gain_lineplot}
%\end{subfigure}
%	\caption{Broadband DSRF for $K = 64$.}
%	\label{fig:lineplot_dsrf_64_av}
%\end{figure}
%
%\begin{figure}[H]
%	\centering
%	%% Requires:
% pgfplots.sty
% edit_pgfplots.tex

\pgfplotsset{compat=1.18}
%\begin{subfigure}{\linewidth}
%\centering
\tikzsetnextfilename{gain_lineplot_32}
\begin{tikzpicture}
	\begin{lineplot}{Gain (dB)}[
			xtick={-20, -10, ..., 20}, xlabel={iSIR (dB)}, xmin=-21, xmax=+21,
%			ymin=-0.5, ymax=12,
			]
		\addplot [style=resA]
		table [col sep=comma, y=val] {figures/io_input/STFT/gSINR_STFT_32.csv};
		
		\addplot [style=resC]
		table [col sep=comma, y=val] {figures/io_input/NSSBT/gSINR_NSSBT_32.csv};
		
		\addplot [style=resE]
		table [col sep=comma, y=val] {figures/io_input/TSSBT/gSINR_TSSBT_32.csv};
		
		\addlegendentry{STFT};
		\addlegendentry{N-SSBT};
		\addlegendentry{T-SSBT};
	\end{lineplot}
\end{tikzpicture}
%	\caption{}
%	\label{subfig:1_gain_lineplot}
%\end{subfigure}
%	\caption{Broadband SNR gain for $K = 32$.}
%	\label{fig:lineplot_gain_32_av}
%\end{figure}
%
%\begin{figure}[H]
%	\centering
%	%% Requires:
% pgfplots.sty
% edit_pgfplots.tex

\pgfplotsset{compat=1.18}
%\begin{subfigure}{\linewidth}
%\centering
\tikzsetnextfilename{gain_lineplot_64}
\begin{tikzpicture}
	\begin{lineplot}{Gain (dB)}[ymin=-0.5, ymax=12, xtick={-20, -10, ..., 20}, xlabel={iSIR (dB)}, xmin=-21, xmax=+21]
		\addplot [style=resA]
		table [col sep=comma, y=val] {figures/io_input/STFT/gSINR_STFT_64.csv};
		
		\addplot [style=resC]
		table [col sep=comma, y=val] {figures/io_input/NSSBT/gSINR_NSSBT_64.csv};
		
		\addplot [style=resE]
		table [col sep=comma, y=val] {figures/io_input/TSSBT/gSINR_TSSBT_64.csv};
		
		\addlegendentry{STFT};
		\addlegendentry{N-SSBT};
		\addlegendentry{T-SSBT};
	\end{lineplot}
\end{tikzpicture}
%	\caption{}
%	\label{subfig:1_gain_lineplot}
%\end{subfigure}
%	\caption{Broadband SNR gain for $K = 64$.}
%	\label{fig:lineplot_gain_64_av}
%	\end{figure}
%
%\subsection{Overall result}
%
%In all situations evaluated, the proposed true-MVDR SSBT beamformer consistently outperformed the beamformer through the traditional STFT in terms of SNR gain, both achieving the appropriate DSRF necessary for the desired distortionless behavior of the MVDR filter. Meanwhile, the naive SSBT beamformer had a (slightly) better overall than that of the T-SSBT beamformer for both evaluated $K$, while also incurring some undesired distortion on the desired signal seen via the DSRF. It is interesting to note that for all beamformers the gain in SNR decreases as the input SIR increases, which was expected as in this scenario the weight of both the undesired speech components and uncorrelated noise over the input SNR increase.
%
%\section{Conclusion}
\label{sec:conclusion}

In this study, we investigated the application of the Single-Sideband Transform in beamforming within a reverberant environment, utilizing the convolutive transfer function model for filter bank (i.e., the beamformer) estimation. We implemented a Minimum-Variance Distortionless-Response beamformer to enhance signals in a real-life-like scenario, elucidating the process to achieve a truly-distortionless MVDR beamformer when employing the SSB transform.

The true-MVDR SSBT beamformer outperformed (at most matched) the beamformer obtained with the traditional STFT, in terms of SNR gain and distortionless response. The naive-MVDR SSBT beamformer's SNR gain performance varied with the number of samples per window and input signal-to-noise ratio, and in all situations it caused distortion in the desired signal.

Future research avenues may explore the integration of this transform into different beamformers, or undertake further comparisons of the proposed SSBT beamformer (following the considerations exposed in here) against the established and reliable STFT methodology.
%
%%%%%%%%%%%%%%%%%%%%%%%%%%%%%%%%%%%%%%%%%%
%% optional
%\supplementary{The following supporting information can be downloaded at:  \linksupplementary{s1}, Figure S1: title; Table S1: title; Video S1: title.}

%%%%%%%%%%%%%%%%%%%%%%%%%%%%%%%%%%%%%%%%%%
%\authorcontributions{For research articles with several authors, a short paragraph specifying their individual contributions must be provided. The following statements should be used ``Conceptualization, X.X. and Y.Y.; methodology, X.X.; software, X.X.; validation, X.X., Y.Y. and Z.Z.; formal analysis, X.X.; investigation, X.X.; resources, X.X.; data curation, X.X.; writing---original draft preparation, X.X.; writing---review and editing, X.X.; visualization, X.X.; supervision, X.X.; project administration, X.X.; funding acquisition, Y.Y. All authors have read and agreed to the published version of the manuscript.'', please turn to the  \href{http://img.mdpi.org/data/contributor-role-instruction.pdf}{CRediT taxonomy} for the term explanation. Authorship must be limited to those who have contributed substantially to the work~reported.} %TODO: Contributions
\authorcontributions{Conceptualization, I. Cohen and V. Curtarelli; Methodology, V. Curtarelli; Software, V. Curtarelli; Writing---original draft: V. Curtarelli; Writing---review and editing, I. Cohen and V. Curtarelli; Supervision, V. Curtarelli. All authors have read and agreed to the published version of the manuscript.} %TODO: Fix contributions.

\funding{This research was supported by the Pazy Research Foundation, and the Israel Science Foundation (grant no. 1449/23).} %TODO: Fix funding.

\dataavailability{The source-code for the simulations developed here is available at \url{https://github.com/VCurtarelli/py-ssb-ctf-bf}.} %%TODO: Source-code availability

\conflictsofinterest{The authors declare no conflict of interest.} 

%%%%%%%%%%%%%%%%%%%%%%%%%%%%%%%%%%%%%%%%%%
\abbreviations{Abbreviations}{
The following abbreviations are used in this manuscript:

\noindent 
\begin{tabular}{@{}ll} %%TODO: Abbreviations used
%CB & Constant-beamwidth \\
CTF & Convolutive Transfer Function \\
DSRF & Desired Signal Reduction Factor \\
LCMV & Linearly-Constrained Minimum-Variance \\
MVDR & Minimum-Variance Distortionless-Response \\
MTF & Multiplicative Transfer Function \\
SNR & Signal-to-Noise Ratio \\
SSBT & Single-Sideband Transform \\
STFT & Short-Time Fourier Transform
\end{tabular}
}

%%%%%%%%%%%%%%%%%%%%%%%%%%%%%%%%%%%%%%%%%%%
% Optional
\appendixtitles{yes} % Leave argument "no" if all appendix headings stay EMPTY (then no dot is printed after "Appendix A"). If the appendix sections contain a heading then change the argument to "yes".
\appendixstart
\appendix
\renewcommand{\thealgorithm}{\arabic{algorithm}}

\section{Properties of the Real Fourier Transform}
\label{app:properties_rft}

The Fourier Transform (FT) is defined as
\begin{equation}
	\label{eq:appA:def_fourier_transform}
	X_{\F}(f) = \int\limits_{-\infty}^{\infty} x(t) e^{-\j 2\pi f t}\dd t
\end{equation}
with an inverse
\begin{equation}
	\label{eq:appA:def_inverse_fourier_transform}
	x(t) = \int\limits_{-\infty}^{\infty} X_{\F}(f) e^{\j 2\pi f t} \dd t
\end{equation}

The Real Fourier Transform (RFT) is defined as
\begin{equation}
	\label{eq:appA:def_real_fourier_transform}
	X_{\R}(f) = \Sqrt{2} \real{\int\limits_{-\infty}^{\infty} x(t) e^{-\j 2\pi f t + \j\frac{3\pi}{4}} \dd t}
\end{equation}
with an inverse
\begin{equation}
	\label{eq:appA:def_inverse_real_fourier_transform}
	x(t) = \Sqrt{2} \real{\int\limits_{-\infty}^{\infty} X_{\R}(f) e^{\j 2\pi f t - \j\frac{3\pi}{4}} \dd f}
\end{equation}

\begin{Property}{The FT and RFT are bijective transformations of one-another.}[\label{prop:FT_RFT_equivalence}]	
	By manipulating \cref{eq:appA:def_real_fourier_transform} we can get
	\begin{equation}\label{eq:propI:propI:manip_Xrf}
		X_{\R}(f) = \Sqrt{2} \real{\pts{\int\limits_{-\infty}^{\infty} x(t) e^{-\j 2\pi f t} \dd t} e^{\j\frac{3\pi}{4}}}
	\end{equation}
	where the term in parenthesis is trivially the FT of $x(t)$ from \cref{eq:appA:def_fourier_transform}. Thus, we get that
	\begin{equations}{eq:propI:equiv_FT_to_RFT}
		X_{\R}(f)
		& = \real{\pts{\Re{X}_{\F}(f) + \j\Im{X}_{\F}(f)} \cdot(-1 + \j)} \\
		& = - \Re{X}_{\F}(f) - \Im{X}_{\F}(f)
	\end{equations}
	Likewise, using that the FT of a real signal is complex-conjugate, such that $X_{\F}(f) = X_{\F}^*(\-f)$, it is easy to see that
	\begin{equation}
		\label{eq:propI:equiv_FT_to_RFT_neg}
		X_{\R}(\-f) = -\Re{X}_{\F}(f) + \Im{X}_{\F}(f)
	\end{equation}
	With this, we have that
	\begin{equations}
		\Sqrt{2}X_{\R}(f) e^{-\j\frac{3\pi}{4}} & = \Re{X}_{\F}(f) + \Im{X}_{\F}(f) + \j\Re{X}_{\F}(f) + \j\Re{X}_{\F}(f) \\
		\Sqrt{2}X_{\R}(\-f) e^{\j\frac{3\pi}{4}} & = \Re{X}_{\F}(f) - \Im{X}_{\F}(f) - \j\Re{X}_{\F}(f) + \j\Re{X}_{\F}(f)
	\end{equations}
	and therefore
	\begin{equations}
		\frac{X_{\R}(f) e^{-\j\frac{3\pi}{4}} + X_{\R}(\-f) e^{\j\frac{3\pi}{4}}}{\Sqrt{2}}
		& = \Re{X}_{\F}(f) + \j\Re{X}_{\F}(f) \\
		& = X_{\F}(f)
	\end{equations}
	Similarly, using \cref{eq:propI:propI:manip_Xrf} we have that
	\begin{equation}
		X_{\R}(f) = \Sqrt{2}\real{X_{\F}(f) e^{\j\frac{3\pi}{4}}}
	\end{equation}
	Using the property of complex numbers that $\real{a} = \tfrac{a+a^*}{2}$, then
	\begin{equation}
		X_{\R}(f) = \frac{X_{\F}(f) e^{\j\frac{3\pi}{4}} + X_{\F}(\-f) e^{-\j\frac{3\pi}{4}}}{\Sqrt{2}}
	\end{equation}
	Since for each $X_{\F}(f)$ there exists one and only one $X_{\R}(f)$, and vice-versa, it is possible to define a bijective transformation $T$ such that
	\begin{equation}
		X_{\F}(f) \hrel{T} X_{\R}(f)
	\end{equation}
\end{Property}

\begin{Property}{The IRFT is the inverse of the RFT.}[\label{prop:RFT_IRFT_inverses}]
	From \cref{prop:FT_RFT_equivalence}, we have that
	\begin{equation}
		\IRFT{X_{\R}(f)} = \frac{X_{\F}(f) e^{\j\frac{3\pi}{4}} + X_{\F}(\-f) e^{-\j\frac{3\pi}{4}}}{\Sqrt{2}}
	\end{equation}
	Substituting this in \cref{eq:appA:def_inverse_real_fourier_transform}, we have
	\begin{equations}
		\IRFT{X_{\R}(f)}
		& = \Sqrt{2} \real{\int\limits_{-\infty}^{\infty} \pts{\frac{X_{\F}(f) e^{\j\frac{3\pi}{4}} + X_{\F}(\-f) e^{-\j\frac{3\pi}{4}}}{\Sqrt{2}}} e^{\j 2\pi f t - \j\frac{3\pi}{4}} \dd f} \\
		& = \real{\int\limits_{-\infty}^{\infty} \pts{X_{\F}(f) e^{\j 2\pi f t} + X_{\F}(\-f) e^{\j 2\pi f t - \j\frac{3\pi}{2}}} \dd f}
	\end{equations}
	The first term expands to the inverse Fourier transform of $X_{\F}(f)$, which is trivially $x(t)$; the second term is the inverse Fourier transform of $X_{\F}(\-f)$, which is from the time reversal property is $x(-t)$. Therefore,
	\begin{equations}
		\IRFT{X_{\R}(f)}
		& = \real{x(t) + x(\-t) e^{-\j\frac{3\pi}{2}}} \\
		& = \real{x(t) + \j x(\-t)} \\
		& = x(t)
	\end{equations}
	thus concluding the proof.
\end{Property}

\begin{Property}{The convolution theorem doesn't apply for the RFT.}[\label{prop:conv_theorem_RFT}]
	Let $h(t)$ be the impulse response of an LIT system, with input $x(t)$. It is trivial that the system's output, $y(t)$, is given by
	\begin{equation}
		y(t) = h(t) \ast x(t)
	\end{equation}
	with $\ast$ being the convolution operator. For the Fourier transform, through the convolution theorem it is trivial that
	\begin{equation}
		Y_{\F}(f) = H_{\F}(f) X_{\F}(f)
	\end{equation}
	Expanding these in terms of real and imaginary parts (omitting the frequency index for clarity),
	\begin{equation}
		\label{eq:propIII:YF_expanded_real_imag}
		Y_{\F} = \Re{H_{\F}} \Re{X_{\F}} + \j \Re{H_{\F}} \Im{X_{\F}} + \j \Im{H_{\F}} \Re{X_{\F}} - \Im{H_{\F}} \Im{X_{\F}}
	\end{equation}
	Now in the RFT domain, with \cref{eq:propI:equiv_FT_to_RFT} we have that
	\begin{equations}
		X_{\R}(f) & = -\Re{X_{\F}}(f) - \Im{X_{\F}}(f) \\
		H_{\R}(f) & = -\Re{H_{\F}}(f) - \Im{H_{\F}}(f)
	\end{equations}
	Assuming the convolution theorem true for the RFT,
	\begin{equation}
		\label{eq:propIII:res_product_of_RFTs}
		Y_{\R} = \Re{H_{\F}} \Re{X_{\F}} + \Re{H_{\F}} \Im{X_{\F}} + \Im{H_{\F}} \Re{X_{\F}} + \Im{H_{\F}} \Im{X_{\F}}
	\end{equation}
	Now, by applying \cref{eq:propI:equiv_FT_to_RFT} on \cref{eq:propIII:YF_expanded_real_imag}, we have
	\begin{equation}
		\label{eq:propIII:res_RFT_of_convol}
		\tilde{Y}_{\R} = - \Re{H_{\F}} \Re{X_{\F}} - \Re{H_{\F}} \Im{X_{\F}} - \Im{H_{\F}} \Re{X_{\F}} + \Im{H_{\F}} \Im{X_{\F}}
	\end{equation}
	where it is explicit that $Y_{\R}(f) \neq \tilde{Y}_{\R}(f)$. Therefore, the RFT of the convolution (\cref{eq:propIII:res_RFT_of_convol}) is not the product of the RFT's of the signals (\cref{eq:propIII:res_product_of_RFTs}), and thus the convolution theorem doesn't hold for the RFT.
\end{Property}

\begin{Property}{There is an equivalent of the convolution theorem for the RFT.}[\label{prop:equivalent_conv_theorem_RFT}]
	From \cref{eq:propIII:res_RFT_of_convol}, we have our objective for the ``convolution theorem''-equivalent for the RFT. From both \cref{eq:propI:equiv_FT_to_RFT,eq:propI:equiv_FT_to_RFT_neg}, we have
	\begin{equations}{eq:propIV:equiv_FT_to_RFT_posneg_XH}
		X_{\R}(f) & = -\Re{X_{\F}}(f) - \Im{X_{\F}}(f) \\
		X_{\R}(\-f) & = -\Re{X_{\F}}(f) + \Im{X_{\F}}(f) \\
		H_{\R}(f) & = -\Re{H_{\F}}(f) - \Im{H_{\F}}(f) \\
		H_{\R}(\-f) & = -\Re{H_{\F}}(f) + \Im{H_{\F}}(f)
	\end{equations}
	We will omit the frequency dependency in the FT values. Taking the possible combinations, we have
	\begin{subalign}
		X_{\R}(f) H_{\R}(f) 	& = \Re{H_{\F}} \Re{X_{\F}} + \Re{H_{\F}} \Im{X_{\F}} + \Im{H_{\F}} \Re{X_{\F}} + \Im{H_{\F}} \Im{X_{\F}} \label{subeq:propIV:mult_XR_HR:++} \\
		X_{\R}(f) H_{\R}(\-f) 	& = \Re{H_{\F}} \Re{X_{\F}} + \Re{H_{\F}} \Im{X_{\F}} - \Im{H_{\F}} \Re{X_{\F}} - \Im{H_{\F}} \Im{X_{\F}} \label{subeq:propIV:mult_XR_HR:+-} \\
		X_{\R}(\-f) H_{\R}(f) 	& = \Re{H_{\F}} \Re{X_{\F}} - \Re{H_{\F}} \Im{X_{\F}} + \Im{H_{\F}} \Re{X_{\F}} - \Im{H_{\F}} \Im{X_{\F}} \label{subeq:propIV:mult_XR_HR:-+} \\
		X_{\R}(\-f) H_{\R}(\-f) & = \Re{H_{\F}} \Re{X_{\F}} - \Re{H_{\F}} \Im{X_{\F}} - \Im{H_{\F}} \Re{X_{\F}} + \Im{H_{\F}} \Im{X_{\F}} \label{subeq:propIV:mult_XR_HR:--}
	\end{subalign}
	Taking the difference between \cref{subeq:propIV:mult_XR_HR:++} and \cref{subeq:propIV:mult_XR_HR:--}, and the sum of \cref{subeq:propIV:mult_XR_HR:+-} and \cref{subeq:propIV:mult_XR_HR:-+}, we have
	\begin{subalign}
		X_{\R}(f) H_{\R}(f) - X_{\R}(\-f) H_{\R}(\-f) & = 2\pts{\Re{H_{\F}} \Im{X_{\F}} + \Im{H_{\F}} \Re{X_{\F}}} \\
		X_{\R}(f) H_{\R}(\-f) + X_{\R}(\-f) H_{\R}(f) & = 2\pts{\Re{H_{\F}} \Re{X_{\F}} - \Im{H_{\F}} \Im{X_{\F}}}
	\end{subalign}
	and therefore, to achieve \cref{eq:propIII:res_RFT_of_convol}, we let
	\begin{equations}
		Y_{\R}(f)
		& = \frac{- X_{\R}(f) H_{\R}(f) + X_{\R}(\-f) H_{\R}(\-f) - X_{\R}(f) H_{\R}(\-f) - X_{\R}(\-f) H_{\R}(f)}{2} \\
		& = X_{\R}(f) \frac{-H_{\R}(f) - H_{\R}(\-f)}{2} + X_{\R}(\-f) \frac{-H_{\R}(f) + H_{\R}(\-f)}{2}
	\end{equations}
	Finally, from \cref{eq:propIV:equiv_FT_to_RFT_posneg_XH}, we achieve
	\begin{equation}
		\label{eq:propIV:convolution_theorem_RFT_result}
		Y_{\R}(f) = X_{\R}(f) \Re{H_{\F}}(f) + X_{\R}(\-f) \Im{H_{\F}}(f)
	\end{equation}
	and, for its conjugate frequency (that is, for $\-f$), we have
	\begin{equations}{eq:propIV:convolution_theorem_RFT_result_neg}
		Y_{\R}(\-f)
		& = X_{\R}(\-f) \Re{H_{\F}}(\-f) + X_{\R}(f) \Im{H_{\F}}(\-f) \\
		& = X_{\R}(\-f) \Re{H_{\F}}(f) - X_{\R}(f) \Im{H_{\F}}(f)
	\end{equations}
\end{Property}

\begin{Property}{Frequencies in the RFT have the same variance as their FT counterpart.}[\label{prop:RFT_same-variance_FT}]
	We now assume that $X_{\F}(f)$ is the transform of a random process, such that its real and imaginary parts are independent and identically distributed with zero mean. Taking the complex correlation of a given frequency in the FT domain,
	\begin{equation}
		\expec{X_{\F}(f) X_{\F}^*(f)} = \expec{\Re{X_{\F}}(f)^2 + \Im{X_{\F}}(f)^2}
	\end{equation}
	Since they are identically distributed, we denote
	\begin{equation}
		\label{eq:propV:same_variance_FT_RI}
		\expec{\Re{X_{\F}}(f)^2} = \expec{\Im{X_{\F}}(f)^2} = \sigma_f^2
	\end{equation}
	and therefore
	\begin{equation}
		\expec{X_{\F}(f) X_{\F}^*(f)} = 2\sigma_f^2
	\end{equation}
	Now in the RFT domain, we take the correlation of a given frequency \cref{eq:propI:equiv_FT_to_RFT},
	\begin{equation}
		\expec{X_{\R}(f)^2} = \expec{\Re{X_{\F}}(f)^2 + 2 \Re{X_{\F}}(f) \Im{X_{\F}}(f) + \Im{X_{\F}}(f)^2}
	\end{equation}
	Using that $\Re{X_{\F}}(f)$ and $\Im{X_{\F}}(f)$ are i.i.d. and zero-mean, the cross terms are zero, and thus
	\begin{equations}
		\expec{X_{\R}(f)^2} & = 2\sigma_f^2 \\
		& = \expec{X_{\F}(f) X_{\F}^*(f)}
	\end{equations}
	It is trivial to see that the same applies for $X_{\R}(\-f)$.
\end{Property}

\begin{Property}{Conjugate frequencies in the RFT domain are independent.}[\label{prop:conjugate-freqs_independent_RFT}]
	We take the same assumptions as those in \cref{prop:RFT_same-variance_FT}. Taking the complex correlation between the two conjugate frequencies, 
	\begin{equation}
		\expec{X_{\F}(f) X_{\F}^*(\-f)} = \expec{\Re{X_{\F}}(f)^2 + 2 \j\Re{X_{\F}}(f) \Im{X_{\F}}(f) - \Im{X_{\F}}(f)^2}
	\end{equation}
	Using that the $\Re{X_{\F}}(f)$ and $\Im{X_{\F}}(f)$ are independent and zero-mean, the cross terms are zero, and with \cref{eq:propV:same_variance_FT_RI} then
	\begin{equation}
		\expec{X_{\F}(f) X_{\F}^*(\-f)} = 0
	\end{equation}
	This result is known, but nonetheless it is useful to show it, since this same procedure will be used for the RFT.
	
	\noindent We now consider \cref{eq:propI:equiv_FT_to_RFT,eq:propI:equiv_FT_to_RFT_neg}. Taking the correlation between the two conjugate frequencies yields
	\begin{equation}
		\expec{X_{\R}(f) X_{\R}(\-f)} = \expec{\Re{X_{\F}}(f)^2 - \Im{X_{\F}}(f)^2}
	\end{equation}
	Under the same assumptions that the real and imaginary parts of $X_{\F}(f)$ are identically distributed, we get the same result as before, where
	\begin{equation}
		\expec{X_{\R}(f) X_{\R}(\-f)} = 0
	\end{equation}
	Note that, with the RFT, we didn't use the complex correlation, since it is real-valued.
	
	\noindent Lastly, we take the correlation between conjugate frequencies of the output of a system according to \cref{eq:propIV:convolution_theorem_RFT_result,eq:propIV:convolution_theorem_RFT_result_neg},
	\begin{equations}
		\expec{Y_{\R}(f) Y_{\R}(\-f)} 
		& = \Re{H_{\F}}(f)^2 \expec{X_{\R}(f) X_{\R}(\-f)} - \Re{H_{\F}}(f) \Im{H_{\F}}(f) \expec{X_{\R}(f)^2} \\
		& + \Re{H_{\F}}(f) \Im{H_{\F}}(f) \expec{X_{\R}(\-f)^2} + \Im{H_{\F}}(f)^2 \expec{X_{\R}(f) X_{\R}(\-f)}
	\end{equations}
	Since $X_{\R}(f)$ and $X_{\R}(\-f)$ are independent and zero-mean, the first and lest terms are zero, and also the other two cancel each other out since $\expec{X_{\R}(f)^2} = \expec{X_{\R}(\-f)^2} = 2\sigma_f^2$. Therefore,
	\begin{equation}
		\expec{Y_{\R}(f) Y_{\R}(\-f)} = 0
	\end{equation}
\end{Property}

\begin{Property}{Relative transfer functions with the RFT are even for frequency-to-frequency and odd for conjugate-frequency.}[\label{prop:rtfs_are_even-odd_on_frequency}]
	First, given two real systems with a shared input $x(t)$, each with an impulse response $h_m(t)$ and an output $y_m(t)$, such that
	\begin{equations}
		Y_{1}(f) = X(f) \Re{H_{1,\F}}(f) + X(\-f) \Im{H_{1,\F}}(f) \\
		Y_{2}(f) = X(f) \Re{H_{2,\F}}(f) + X(\-f) \Im{H_{2,\F}}(f)
	\end{equations}
	and, for their conjugate frequencies,
	\begin{equations}
		Y_{1}(\-f) = X(\-f) \Re{H_{1,\F}}(\-f) + X(f) \Im{H_{1,\F}}(\-f) \\
		Y_{2}(\-f) = X(\-f) \Re{H_{2,\F}}(\-f) + X(f) \Im{H_{2,\F}}(\-f)
	\end{equations}
	where we are omitting the transform index when it is an RFT's domain value, for compactness. If we define $X_1'(f)$ and $X_1''(f)$ as
	\begin{equations}
		X_1'(f) = Y_1(f) \\
		X_1''(f) = Y_1(\-f)
	\end{equations}
	then
	\begin{equations}
		Y_m(f) = A_m'(f) X_1'(f) + A_m''(f) X_1''(f) \\
		Y_m(\-f) =  A_m'(\-f) X_1'(\-f) + A_m''(\-f) X_1''(\-f)
	\end{equations}
	where $A_m'(f)$ and $A_m''(f)$ are given by
	\begin{equations}
		A_m'(f)  = \frac{H_1'(f)H_m'(f) + H_1''(f)H_m''(f)}{{H_1'}^2(f) + {H_1''}^2(f)} \\
		A_m''(f) = \frac{H_1'(f)H_m''(f) - H_1''(f)H_m'(f)}{{H_1'}^2(f) + {H_1''}^2(f)}
	\end{equations}
	and, for their conjugate frequencies,
	\begin{equations}
		A_m'(\-f)  = \frac{H_1'(\-f)H_m'(\-f) + H_1''(\-f)H_m''(\-f)}{{H_1'}^2(\-f) + {H_1''}^2(\-f)} \\
		A_m''(\-f) = \frac{H_1'(\-f)H_m''(\-f) - H_1''(\-f)H_m'(\-f)}{{H_1'}^2(\-f) + {H_1''}^2(\-f)}
	\end{equations}
	Assuming that $h(t)$ is real-valued, then $H_{m,\F}(f) = H_{m,\F}^*(\-f)$, and therefore $H_m'(f) = H_m'(\-f)$, and $H_m''(f) = -H_m''(\-f)$. With this,
	\begin{equations}
		\begin{split}
			A_m'(\-f)
			& = \frac{H_1'(f) H_m'(f) + \bts{-H_1''(f)}\bts{-H_m''(f)}}{{H_1'}^2(f) + \bts{-H_1''}^2(f)} \\
			& = \frac{H_1'(f)H_m'(f) + H_1''(f)H_m''(f)}{{H_1'}^2(f) + {H_1''}^2(f)} \\
			& = A_m'(f)
		\end{split}\\[0.5cm]
		%
		\begin{split}
			A_m''(\-f)
			& = \frac{H_1'(f) \bts{-H_m''(f)} - \bts{-H_1''(f)}H_m'(f)}{{H_1'}^2(f) + \bts{-H_1''}^2(f)} \\
			& = \frac{- H_1'(f)H_m''(f) + H_1''(f)H_m'(f)}{{H_1'}^2(f) + {H_1''}^2(f)} \\
			& = -A_m''(f)
		\end{split}\\
	\end{equations}
	Therefore, our frequency-to-frequency RTF $A_m'(f)$ is even, since $A_m'(f) = A_m'(\-f)$, and our conjugate-frequency RTF is odd, since $A_m''(f) = -A_m''(\-f)$.
\end{Property}

%\begin{Property}{Relative transfer functions are even on frequency.}[\label{prop:rtfs_are_even-func_of_frequency}]
%	First, given two real systems with a shared input $x(t)$, each with an impulse response $h_m(t)$ and an output $y_m(t)$, such that
%	\begin{equations}
%		Y_{1}(f) = X(f) \Re{H_{1,\F}}(f) + X(\-f) \Im{H_{1,\F}}(f) \\
%		Y_{2}(f) = X(f) \Re{H_{2,\F}}(f) + X(\-f) \Im{H_{2,\F}}(f)
%	\end{equations}
%	and, for their conjugate frequencies,
%	\begin{equations}
%		Y_{1}(\-f) = X(\-f) \Re{H_{1,\F}}(\-f) + X(f) \Im{H_{1,\F}}(\-f) \\
%		Y_{2}(\-f) = X(\-f) \Re{H_{2,\F}}(\-f) + X(f) \Im{H_{2,\F}}(\-f)
%	\end{equations}
%	where we are omitting the transform index when it is an RFT's domain value, for compactness. If we define $X_1'(f)$ and $X_1''(f)$ as
%	\begin{equations}
%		X_1'(f) = \Re{H_{1,\F}}(f) X(f) \\
%		X_1''(f) = \Im{H_{1,\F}}(f) X(\-f)
%	\end{equations}
%	then
%	\begin{equations}
%		Y_m(f) = \frac{\Re{H_{m,\F}}(f)}{\Re{H_{1,\F}}(f)}X_1'(f) + \frac{\Im{H_{m,\F}}(f)}{\Im{H_{1,\F}}(f)} X_1''(f) \\
%		Y_m(\-f) = \frac{\Re{H_{m,\F}}(\-f)}{\Re{H_{1,\F}}(\-f)}X_1'(\-f) + \frac{\Im{H_{m,\F}}(\-f)}{\Im{H_{1,\F}}(\-f)} X_1''(\-f)
%	\end{equations}
%	By defining $A_m'(f)$ and $A_m''(f)$ as
%	\begin{equations}
%		A_m'(f) = \frac{\Re{H_{m,\F}}(f)}{\Re{H_{1,\F}}(f)} \\
%		A_m''(f) = \frac{\Im{H_{m,\F}}(f)}{\Im{H_{1,\F}}(f)}
%	\end{equations}
%	then, by analyzing these for negative frequencies, we have
%	\begin{equations}
%		A_m'(\-f)
%		& = \frac{\Re{H_{m,\F}}(\-f)}{\Re{H_{1,\F}}(\-f)} = \frac{\Re{H_{m,\F}}(f)}{\Re{H_{1,\F}}(f)} \\
%		& = A_m'(f)
%	\end{equations}
%	\begin{equations}
%	A_m''(\-f)
%	& = \frac{\Im{H_{m,\F}}(\-f)}{\Im{H_{1,\F}}(\-f)} = \frac{-\Im{H_{m,\F}}(f)}{-\Im{H_{1,\F}}(f)} \\
%	& = A_m''(f)
%	\end{equations}
%	where we used that $H_{m,\F}(f) = H_{m,\F}^*(\-f)$, if $h_m(t)$ is real-valued.
%	
%	If we instead define one relative transfer function $A_m(f)$ such that
%	\begin{equations}{eq:propVII:new_bva_vector_element_def}
%		A_m(f) = \frac{\Re{H_{m,\F}}(f) \Re{H_{1,\F}}(f) + \Im{H_{m,\F}}(f) \Im{H_{1,\F}}(f)}{{\Re{H_{1,\F}}}^2(f) + {\Im{H_{1,\F}}}^2(f)}
%	\end{equations}
%	it is easy to see that
%	\begin{equations}
%		A_m(\-f)
%		& = \frac{\Re{H_{m,\F}}(f) \Re{H_{1,\F}}(f) + \bts{- \Im{H_{m,\F}}(f)} \bts{-\Im{H_{1,\F}}(f)}}{{\Re{H_{1,\F}}}^2(f) + \bts{-\Im{H_{1,\F}}}^2(f)} \\
%		& = \frac{\Re{H_{m,\F}}(f) \Re{H_{1,\F}}(f) + \Im{H_{m,\F}}(f) \Im{H_{1,\F}}(f)}{{\Re{H_{1,\F}}}^2(f) + {\Im{H_{1,\F}}}^2(f)} \\
%		& = A_m(f)
%	\end{equations}
%	
%	With this, $A_m'(f) = A_m'(\-f)$ and $A_m''(f) = A_m''(\-f)$, and also $A_m(f) = A_m(\-f)$, and therefore all defined relative transfer functions are even in terms of conjugate frequencies.
%	
%\end{Property}

\section{Correct calculation of steering vectors in SSBT}
\DeclareDocumentCommand{\E}{s m}{%
	\mathrm{E}_{#2}\IfBooleanTF{#1}{'}{^{}}
}
\def\ar{\alpha_{\re}}
\def\ai{\alpha_{\im}}
\def\br{\beta_{\re}}
\def\bi{\beta_{\im}}
Assume we have two LTI systems with impulse responses $h_m(t)$, with a singular input $x(t)$. In the RFT domain,
\begin{equation}
	X_{m\R}(f) = X_{\R}(f) \Re{H_{m\F}}(f) + X_{\R}(\-f) \Im{H_{m\F}}(f)
\end{equation}

Through the properties of the RFT exposed in \cref{app:properties_rft}, we can build the following system of equations:
\begin{subalign}{eqs:appB:system_equations}
	\expec{X_{1\R}(f) X_{1\R}(f)}   & = \pts{  {\Re{H_{1\F}}}^2(f)             + {\Im{H_{1\F}}}^2(f)         } \sigma_{X}^2(f) \label{eq:appB:system_equations:subeq1} \\
	\expec{X_{1\R}(f) X_{2\R}(f)}   & = \pts{   \Re{H_{1\F}}(f) H_{2\F}^\re(f) +  \Im{H_{1\F}} H_{2\F}^\im(f)} \sigma_{X}^2(f) \label{eq:appB:system_equations:subeq2} \\
	\expec{X_{1\R}(f) X_{2\R}(\-f)} & = \pts{\- \Re{H_{1\F}}(f) H_{2\F}^\im(f) +  \Im{H_{1\F}} H_{2\F}^\re(f)} \sigma_{X}^2(f) \label{eq:appB:system_equations:subeq3} \\
	\expec{X_{2\R}(f) X_{2\R}(f)}	& = \pts{  {\Re{H_{2\F}}}^2(f)             + {\Im{H_{2\F}}}^2(f)         } \sigma_{X}^2(f) \label{eq:appB:system_equations:subeq4}
\end{subalign}
where $H_{m\F}(f) = \Re{H_{m\F}}(f) + \j\Im{H_{m\F}}(f)$ is the FT of $h_m(t)$, and $\sigma_X^2(f) = \expec{X_{\R}(f)^2}$.

Note that, we have 4 equations and 5 variables. Our objective is to find $A_{m,\R}'(f)$ and $A_{m,\R}''(f)$ such that
\begin{subgather}{eqs:appB:rtfs_in_rft_domain}
	A_{m,\R}'(f) = \frac{\Re{H_{m\F}}(f)}{\Re{H_{1\F}}(f)} \\
	A_{m,\R}''(f) = \frac{\Im{H_{m\F}}(f)}{\Im{H_{1\F}}(f)}
\end{subgather}
where we're not worried about negative frequencies, since from \cref{prop:rtfs_are_even-func_of_frequency} we know that $A_{m,\R}'(f) = A_{m,\R}'(\-f)$ and $A_{m,\R}''(f) = A_{m,\R}''(\-f)$. We rewrite the equations in \cref{eqs:appB:system_equations} by defining
\begin{equations}
	\ar & 	= \Re{H_{1\F}}(f) \sigma_X(f) \\
	\ai & 	= \Im{H_{1\F}}(f) \sigma_X(f) \\
	\br & 	= \Re{H_{2\F}}(f) \sigma_X(f) \\
	\bi & 	= \Im{H_{2\F}}(f) \sigma_X(f)
\end{equations}
\begin{equations}
	\E{11} 	& 	= \expec{X_{1\R}(f) X_{1\R}(f)} \\
	\E{12}	& 	= \expec{X_{1\R}(f) X_{2\R}(f)} \\
	\E*{12} &   = \expec{X_{1\R}(f) X_{2\R}(\-f)} \\
	\E{22}	&	= \expec{X_{2\R}(f) X_{2\R}(f)}
\end{equations}

Where we will be omitting the $(f)$ index for clarity in these newly defined variables, however they are frequency-dependent. With this, \cref{eqs:appB:system_equations} becomes
\begin{subalign}
	\ar^2 + \ai^2 		& = \E{11} \label{eq:appB:system_equations_simp:subeq1} \\
	\ar \br + \ai \bi 	& = \E{12}\label{eq:appB:system_equations_simp:subeq2} \\
	-\ar \bi + \ai \br 	& = \E*{12} \label{eq:appB:system_equations_simp:subeq3} \\
	\br^2 + \bi^2 		& = \E{22}\label{eq:appB:system_equations_simp:subeq4}
\end{subalign}
and \cref{eqs:appB:rtfs_in_rft_domain} is
\begin{subgather}
	A_{m,\R}'(f) = \frac{\br}{\ar} \\
	A_{m,\R}'(f) = \frac{\bi}{\ai}
\end{subgather}

With this we simplified the problem and got rid of one of the variables, now having a system of four equations and four variables, which is solvable. However, it is a non-linear system, which makes the solution harder to find.

\begin{thought}
	By multiplying \cref{eq:appB:system_equations_simp:subeq2} by $\ar$ and \cref{eq:appB:system_equations_simp:subeq3} by $\ai$ and adding, we get
	\begin{equation}
		\ar^2\br + \ar\ai\bi - \ai\ar\bi + \ai^2\br = \E{12}\ar + \E*{12} \ai
	\end{equation}
	and therefore
	\begin{equation}
		\br = \frac{\E\ar + \E*{12}\ai}{\E{11}}
	\end{equation}
	Similarly, by multiplying \cref{eq:appB:system_equations_simp:subeq2} by $\ai$ and \cref{eq:appB:system_equations_simp:subeq3} by $-\ar$ and adding them, we get
	\begin{equation}
		\ai\ar\br + \ai^2\bi + \ar^2\bi - \ar\ai\br = \E\ai - \E*{12}\ar
	\end{equation}
	and thus
	\begin{equation}
		\bi = \frac{\E\ai - \E*{12}\ar}{\E{11}}
	\end{equation}
	
	However, for this we would have to know $\ar$ and $\ai$, which aren't available.
\end{thought}

\reftitle{References}
%\UseRawInputEncoding
\bibliography{setup/d0.references.bib,setup/d1.ext_references.bib} %TODO: Increase references?

% If authors have biography, please use the format below
%\section*{Short Biography of Authors}
%\bio
%{\raisebox{-0.35cm}{\includegraphics[width=3.5cm,height=5.3cm,clip,keepaspectratio]{Definitions/author1.pdf}}}
%{\textbf{Firstname Lastname} Biography of first author}
%
%\bio
%{\raisebox{-0.35cm}{\includegraphics[width=3.5cm,height=5.3cm,clip,keepaspectratio]{Definitions/author2.jpg}}}
%{\textbf{Firstname Lastname} Biography of second author}

% For the MDPI journals use author-date citation, please follow the formatting guidelines on http://www.mdpi.com/authors/references
% To cite two works by the same author: \citeauthor{ref-journal-1a} (\citeyear{ref-journal-1a}, \citeyear{ref-journal-1b}). This produces: Whittaker (1967, 1975)
% To cite two works by the same author with specific pages: \citeauthor{ref-journal-3a} (\citeyear{ref-journal-3a}, p. 328; \citeyear{ref-journal-3b}, p.475). This produces: Wong (1999, p. 328; 2000, p. 475)

\PublishersNote{}

% %TC:ignore
\newpage
\section*{Requested Changes - Review 1}
\begin{reviews}
    \footnotesize
    \newreviewer
    \review{11/37 are self references. Please, reduce it.}{Done, reduced to 3 references.}
    %
    \review{You haven't explained heatmap consequences.}{I didn't understand this comment.}
    %
    \review{Figures 1 and 2 are too far from the comments in text. That makes it hard to follow.}{Done, reorganized figures.}
    %
    \review{Some abbreviations are not listed in the "Abbreviations", such as ULA.}{Done, added missing abbreviations.}
    %
    \review{You should improve presentation of your method, maybe with some flow chart.}{There are algorithms in Appendix B for both the proposed method and proposed beamformer.}
    
    \newreviewer
    %
    \review{[...] Explain how to avoid that the other three virtual beampatterns destroy the constant-beamwidth [...]}{Done, added context and information about this.}
    %
    \review{[...] Should also show and discuss some results for the elevation angle.}{We assumed the elevation to be $0\dg$ as it is unusual, in speech processing, for the desired and undesired sources to have the same azimuth, case in which the elevation would be relevant.}
    %
    \review{An explanation about the 2D beampattern design (azimuth and elevation, jointly) would also be useful.}{Done, added some minimal context about the elevation angle.}
    %
    \review{Since the central frequency is several MHz or GHz, the bandwidth should be much larger, I think, at least 10 or 100 times larger.}{The only beamformer that has restrictions on the frequency is the CB beamformer, for which the minimum and maximum frequencies, for Condition [A], are \cite{long_window-based_2019} $f_L = 4.2$ kHz and $F_H = 11.4$ kHz. We used $4$ kHz and $8$ kHz to keep in line with \cite{frank_constant-beamwidth_2022-1}.}
    %
    \review{``kHz", not ``k Hz".}{Done, fixed units notation.}
    
    \newreviewer
    %
    \review{Move fig. 2 before the Conclusions section.}{Done, reorganized figures.}
    %
    \review{In the simulation you consider frequency range 4-8KHz. Can you give a good reason for this choice? [...]}{This range was chosen to keep in line with the literature, more specifically \cite{long_window-based_2019}.}
    %
    \review{In Introduction section you write: “frequency does not affect the behavior of the beamformer” (line 18-20). In this case considering a wide frequency range in simulation would be benefic.}{The range was chosen to satisfy the conditions for the CB beamformer \cite{long_window-based_2019}.}
    %
    \review{``kHz", not ``k Hz".}{Done, fixed units notation.}
\end{reviews}
%TC:endignore
%
% %TC:ignore
\newpage
\section*{Requested Changes - Review 2}
\setcounter{reviewer}{1}
\begin{reviews}
    \footnotesize
    \newreviewer
    \review{A discussion about the elevation angle}{Done, added explanation for irrelevancy of elevation angle (see par. 1 in Section 2).}
    %
    \review{A discussion about [...] the motivation of rectangular arrays when the elevation angle is not of interest.}{The motivation for rectangular arrays when the elevation angle is not of interest is already present in the text (for example, see: par. 1 in Section 1; par. 1 in Section 4; and par. 2 in Section 4.2).}
    %
    \review{The minimum and maximum frequencies mentioned in the answer to the comments of the reviewers.}{Done, added information regarding the choice of frequencies (see par. 2 in Section 5).}
\end{reviews}
%TC:endignore
\end{document}

